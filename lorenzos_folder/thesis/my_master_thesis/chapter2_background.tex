\chapter{Background}

This chapter aims to provide an overview on the fundamentals of Space Flight Dynamics that constitute the theoretical basis of this thesis, with a specific focus on Earth Observation applications in Low Earth Orbit (LEO), as well as a literature review on orbit management methods addressed by this work.

\section{Space Flight Dynamics Overview}

Orbital Mechanics, Astrodynamics, Astronautics and Space Flight Dynamics are all titles of university courses whose principal topic is two-body orbital motion, that involves orbit determination, orbital flight time, and orbital maneuvers \cite{kluever2018space}.
In this context, a proper definition of the subject can be the following: the study of the motion of man-made objects in space, subject to both natural and artificially induced forces \cite{griffin2004space}.

\subsection{Satellite State Representations}

% There are a number of independent parameters describing the size, shape, and spatial position of an orbit.
% Six of these have become the parameters of choice to define and describe an orbit.
To define the \textit{state} of a satellite in space six quantities are required, and they may take on many equivalent forms.
Whatever the form, the collection is called either a \textit{state vector}, usually associated with position and velocity vectors, or an \textit{element set}, typically used with scalar magnitude and angular representations of the orbit called \textit{orbital elements}.
Either set of quantities completely specify the two-body orbit and provide a complete set of initial conditions for solving an initial value problem class of differential equations.
Time is always associated with a state vector and is often considered a seventh component.
State vectors and element sets are referenced to a particular coordinate frame \cite{vallado2013fundamentals}.

This thesis will use both state vectors and a specific element set.
The latter, which is also the most common one in this field, is represented by the \textit{\textbf{classical orbital elements}}:
\begin{itemize}
    \item Semimajor axis, \textit{a}: the orbit size is defined by one half of the major axis dimension
    \item Eccentricity, \textit{e}: the ratio of minor to major dimensions of an orbit defines the shape
    \item Inclination, \textit{i}: the angle between the orbit plane and the reference plane or the angle between the normals to the two planes
    \item Longitude of the ascending node or Right Ascension of the Ascending Node (RAAN), $\Omega$: the angle between the vernal equinox vector and the ascending node measured in the reference plane in a couterclockwise direction as viewed from the northern hemisphere
    \item Argument of periapsis, $\omega$: the angle from the ascending node to the periapsis, measured in the orbital plane in the direction of spacecraft motion. The ascending node is the point where the spacecraft crosses the reference plane headed from south to north. the line of nodes is the line formed by the intersection of the orbit plane and the reference plane
    \item True anomaly, $\nu$: the sixth element locates the spacecraft position on the orbit
\end{itemize}
\cite{brown1998spacecraft}
%---------IMMAGINI---------

This thesis work makes also use of the \textit{argument of latitude, u}, which is the angle measured between the ascending node and the satellite's position vector in the direction of satellite motion.
A relation that is always valid exists between classical orbital elements and the argument of latitude \cite{vallado2013fundamentals}:
\begin{equation} \label{eq:argl}
    u = \omega + \nu
\end{equation}
One more element that is often used is the \textit{mean anomaly, M}, which is the angle between the periapsis of an orbit and the position of an imaginary body that orbits in the same period as the real one but at a constant angular speed (circular orbit).
The angular speed assigned to the imaginary body is the satellite's average angular velocity over one orbit, and is called \textit{mean motion, n} \cite{ridpath2012dictionary}.

This research utilizes also another element set which results necessary to address case studies of past missions: the \textit{Two-line element set} (TLE).
A two-line element consists of a satellite identifier, an epoch, six orbital elements and a \textit{B*} term related to the ballistic coefficient \cite{riesing2015orbit}.
These elsets are available to the general public through Air Force Space Command (AFSPC) \cite{vallado2013fundamentals}.

The elements of a TLE are shown in equation \ref{eq:tle} \cite{vallado2013fundamentals}, where:
\begin{itemize}
    \item[-] the bars on  the mean motion and semimajor axis denote \textit{Kozai mean values} 
    \item[-] the numerators of the first two elements of the second line represent mean motion rate and acceleration
    \item[-] $\frac{c_D A}{m}$ corresponds to the inverse of the \textit{ballistic coefficient, BC}
    \item[-] $\rho_0$ is the atmospheric density at perigee of the orbit
    \item[-] $R_\mathTerra$ defines an Earth Radius of 6378.135 km
    \item[-] the epoch is expressed in Coordinate Universal Time (UTC)
\end{itemize}

\begin{equation} \label{eq:tle}
    \begin{split}
        \bar{n}=\sqrt{\frac{\mu}{\bar{a}^{3}}} \;\;\;\;\;\;\;\;\;\; e \;\;\;\;\;\;\;\;\;\; i \;\;\;\;\;\;\;\;\;\; \Omega \;\;\;\;\;\;\;\;\;\; \omega \;\;\;\;\;\;\;\;\;\; M \\
        \frac{\dot{n}}{2} \;\;\;\;\;\;\;\;\;\;\;\; \frac{\ddot{n}}{6} \;\;\;\;\;\;\;\;\;\;\;\; B^* = \frac{1}{2} \frac{c_D A}{m} \rho_0 R_\mathTerra \;\;\;\;\;\;\;\;\;\;\;\; UTC
    \end{split}
\end{equation}

\subsection{Reference Frames}
Two types of reference frames are adopted by this research: the geocentric-inertial coordinate system and the geographic-body-fixed system.
The origin of both systems is the center of mass of the central body, which in all case studies of this work is the Earth.
They will therefore be labeled Earth-centered inertial (ECI) and Eart-centered Earth-fixed (ECEF) coordinate frames, respectively.

\subsubsection{Earth-centered inertial}
The ECI system is shown in FIGURE???.
The equatorial plane is the reference plane.
The X axis is the vernal equinox vector, and the Z axis is the spin axis of the Earth; north is positive.
The axes are fixed in inertial space or fixed with respect to the stars \cite{brown1998spacecraft}.

\subsubsection{Earth-centered Earth-fixed}
FIGURE shows a representation of the ECEF reference frame.
The system is Earth-centered and fixed to the rotating Earth \cite{vallado2013fundamentals}.
Considering the ECI, the Z axis is the same, while the X axis always points towards the Greenwich meridian.
Satellite ground track is commonly plotted in this coordinate system \cite{brown1998spacecraft}.

\subsection{Orbital Perturbations} \label{perturbations}
Orbital perturbations are deviations from a normal, idealized, or undisturbed motion.
Introducing an alteration from two-body problem assumptions, the actual motion will vary due to perturbations caused by other bodies, and additional forces not considered in Keplerian motion \cite{vallado2013fundamentals}.
This subsection provides an overview of the main perturbations for an Earth orbiting spacecraft.

\subsubsection{Earth's Gravity Field}
Spinning celestial bodies are nor perfect spheres, but they are much more similar to oblate spheroids.
For such a planet, the spin axis can be considered as the axis of rotational symmetry and the gravitational field will vary with the latitude as well as radius.
The planet's oblateness provides a rotationally symmetric perturbation $\Phi$, which does not depend on the longitude \cite{curtis2020orbital}.

In particular, $\Phi$ is given by an infinite series characterized by the so-called \textit{zonal harmonics} $J_k$ of the planet of reference.
Considering a spherical coordinate system for convenience, with origin at the planet's center of mass and third axis as the axis of rotational symmetry,
in this series, shown in equation \ref{eq:harmonics_series}, $r$ and $\phi$ are the distance from the origin and the polar angle respectively, $\mu$ is the Earth's standard gravitational parameter, $R$ the equatorial radius and $P_k$ represent the Legendre polynomials \cite{curtis2020orbital}.
\begin{equation} \label{eq:harmonics_series}
    \Phi (r,\phi) = \frac{\mu}{r} \sum_{k=2}^{\infty} J_k \left(\frac{R}{r} \right)^k P_k (\cos{\phi})
\end{equation}
The zonal harmonics are non-dimensional quantities which are evaluated from satellite observation mission around the planet.
The summation starts from $k = 2$, and the Earth's set of zonal harmonics is highly dominated by $J_2$.
Taking into account only $J_2$ and starting from equation \ref{eq:harmonics_series}, it is possible to derivate the perturbing gravitational acceleration due to the respective zonal harmonic \cite{curtis2020orbital}:
\begin{equation} \label{eq:j2_acc}
    \vec{a}_{J_2} = \frac{3}{2} \frac{J_2 \mu R^2}{r^4} \left[\frac{x}{r}\left(5 \frac{z^2}{r^2} - 1 \right)\hat{\textbf{i}} + \frac{y}{r}\left(5 \frac{z^2}{r^2} - 1\right)\hat{\textbf{j}} + \frac{z}{r}\left(5 \frac{z^2}{r^2} - 3\right)\hat{\textbf{k}} \right]
\end{equation}

The right ascension $\Omega$ and the argument of perigee $\omega$ are significantly affected by oblateness \cite{curtis2020orbital}.
% Their variation in time is described by equations \ref{eq:5} and \ref{eq:6} respectively \cite{curtis2020orbital}:
% \begin{equation} \label{eq:5}
%     \dot{\Omega} = - \left[\frac{3}{2} \frac{\sqrt{\mu} J_2 R^2}{(1-e^2)^2 a^{7/2}}\right]\cos{i}
% \end{equation}
% \begin{equation} \label{eq:6}
%     \dot{\omega} = - \left[\frac{3}{2} \frac{\sqrt{\mu} J_2 R^2}{(1-e^2)^2 a^{7/2}}\right] \left(\frac{5}{2} \sin^2{i} - 2\right)
% \end{equation}

It is necessary to underline that the Earth's gravitational field vary not only with latitude but also with longitude due to irregularities in geometry and mass distribution \cite{curtis2020orbital}.
However, with negligible approximation, it is definitely reasonable to consider only oblateness with $J_2$ effects in LEO missions, which is the major force for this range of altitudes, second only to Earth's gravity \cite{brown1998spacecraft}.
In light of the above, this thesis will take into account only $J_2$ forces with respect to the Earth's gravity field perturbations.

\subsubsection{Atmospheric Drag}
The residual atmosphere present at a few hundred kilometers of altitude strongly influences the motion of satellites in LEO.
The basic equation of the perturbing specific force (force per unit mass) due to drag is the following \cite{vallado2013fundamentals}:
\begin{equation} \label{eq:drag_acc}
    \vec{a}_{drag} = - \frac{1}{2} \frac{C_D A}{m} \rho V_{rel}^{2} \frac{\vec{V_{rel}}}{|\vec{V_{rel}}|}
\end{equation}
In equation \ref{eq:drag_acc}, $\frac{m}{C_D A}$ is again the ballistic coefficient, that already appeared in the TLE representation (equation \ref{eq:tle}):
$m$ is the mass of the spacecraft, $A$ represents the cross-sectional area of the satellite with respect to the atmosphere and $C_D$ is the drag coefficient, a dimensionless parameter which takes into account every aerodynamic configuration aspect of the body with respect to the drag forces \cite{sadraey2009drag}.
The value of the drag coefficient for satellites in the upper atmosphere is generally around 2.2 \cite{vallado2013fundamentals};
this number has been considered for all the case studies of the thesis.
Finally, the velocity vector $\vec{V_{rel}}$ is relative to the atmosphere, as well as the cross section.

The main effects provided by aerodynamic drag are changes in the semimajor axis and eccentricity of the orbit and an accurate description of the atmospheric properties is crucial for the evaluation of drag on satellites \cite{vallado2013fundamentals}.
However, uncertainties in the time variance of upper atmosphere make perfect prediction of spacecraft drag impossible \cite{brown1998spacecraft}.
More in depth, the density changes because of a complex interaction between three factors: the nature of the atmosphere's molecular structure, the incident solar flux, and geomagnetic interactions.
Several atmospheric models can be found in literature, either static or time-varying.
The static ones are less accurate, but definitely simpler than the time-varying models, thanks to the assumption of all constant parameters \cite{vallado2013fundamentals}.
This thesis exploits three different models, depending on the applications addressed in the case studies.

The first model considered by this research is static: the exponential model, valid in the range of altitudes between 0 and 1000 km.
It assumes a spherically symmetrical distribution of particles of the atmosphere, in which the density decreases exponentially with increasing the altitude according to equation \ref{eq:exp_model} \cite{vallado2013fundamentals}
\begin{equation} \label{eq:exp_model}
    \rho = \rho_0 \exp{\left[- \frac{h - h_0}{H}\right]}
\end{equation}
where $\rho_0$ and $h_0$ represent a reference density and altitude respectively and $H$ is the scale height.

The second atmospheric model, still static, is the Standard Atmosphere published in 1976 by the U.S. Committee on Extension to the Standard Atmosphere (COESA), valid drom 0 to 1000 km of altitude.
It is an ideal, steady-state model of the Earth's atmosphere at a latitude of $45^{\circ}$N in moderate solar activity conditions \cite{vallado2013fundamentals}.

Finally, Jacchia-Bowman 2008 (JB2008) is an empirical time-varying atmospheric density model. It revises and improves the earlier Jacchia-Bowman 2006 which is based on the diffusion equations of the Jacchia 71 model.
JB2008 takes into account factors like solar irradiances, computed from driving solar indices based on orbit-based sensor data.
Density variations are described by semiannual density equations based on 81-day average solar indices, as well as from temperature equations that include corrections for diurnal and latitudinal effects.
Geomagnetic effects are modeled too.
The model is validated in the altitude range of 175 to 1000 km through comparisons with accurate daily density drag data previously collected for numerous satellites, existing atmospheric models and other measurements from several Earth orbiting satellite missions \cite{bowman2008jb2006,bowman2008new}.

\subsubsection{Third-Body Perturbations}
All the orbital elements are periodically affected by the gravitational forces of the Sun and the Moon.
Similarly to what is triggered by the Earth's equatorial bulge, they apply an external torque to the orbits and cause the angular momentum to rotate.
This perturbation is highly negligible in LEO, where the main effects are provided by $J_2$ and aerodynamic drag \cite{wertz2009orbit}.
However, low Earth orbits characterized by a constant geometry with respect to the perturbing body can be affected by emphasized long-period effects.
This is the case of Sun-synchronous orbits (SSO), for which the constant pattern with the Sun causes a long-term phenomenon of variation of the inclination (around 0.05 degrees per year) \cite{giacaglia1994long}.
Although this number might appear meaningless, inclination is critical to SSO, as will be explained in paragraph \ref{sso_paragraph}.

\subsubsection{Solar-Radiation Pressure}
The last perturbation treated by this thesis is the solar-radiation pressure (SRP), which induces periodic variations in all the orbital elements.
SRP generally becomes significant above 800 km of altitude, as drag becomes less important. Below this threshold, it might be neglected \cite{wertz2009orbit}.
The perturbing acceleration can be approximated by the following equation \cite{vallado2013fundamentals}
\begin{equation} \label{eq:srp_acc}
    \vec{a}_{SRP} = - \frac{p_{SRP} c_{R} A_{\mathSun}}{m_{sat}} \frac{\vec{r}_{\mathTerra \mathSun}}{|\vec{r}_{\mathTerra \mathSun}|}
\end{equation}
$p_{SRP}$ is the solar pressure, which is a quantity representing the change in momentum per unit area, derived by the ratio between the solar flux and the speed of light.
The reflectivity, $c_R$, a value between 0.0 and 2.0, models the type of interaction between radiation and surface exposed to the Sun, $A_{\mathSun}$ \cite{vallado2013fundamentals}.
$m_{sat}$ is trivially the mass of the satellite.

\subsection{Mean Orbital Elements}
The previous subsection has described how orbital perturbations generate continuous variations and oscillations in the orbit elements.
The values of the elements at a single point in time, which are periodically and secularly affected by the perturbing forces, are called \textit{osculating elements}.
On the other side, it is possible to define the \textit{mean orbit elements}, which represent the average motion over a span of time \cite{wertz2009orbit}.
In this way it is possible to obtain a representation of the orbital motion removing short and long-periodic effects induced by perturbations.
And for most operational purposes, conversion from osculating to mean elements is indispensable \cite{walter1967conversion}.
Indeed, applications addressed by this work, like station-keeping and differential drag, require the secular behavior of the satellite to be implemented.

The oblateness of the Earth is the main guilty of periodic effects, provoking variations in all osculating elements, as well as all the other zonal harmonics.
On the other hand, only even zonal gravitational harmonics and atmospheric drag give raise to secular effects in osculating elements, which are constant or non-periodic.
Aerodynamic drag secular perturbation plays a crucial role in LEO \cite{der1996conversion}.

Figure %\ref{fig:osc_vs_mean_sma} shows the comparison between osculating and mean semi-major axis of a LEO satellite motion over the course of four months.
The orbital decay induced by the atmospheric drag secular effect is evident, as well as the periodic consequences of $J_2$ perturbation.
\begin{figure}[h] \label{fig:osc_vs_mean_sma}
    \begin{center}
        \textbf{Semi-Major Axis}\par\medskip
        %%% Creator: Matplotlib, PGF backend
%%
%% To include the figure in your LaTeX document, write
%%   \input{<filename>.pgf}
%%
%% Make sure the required packages are loaded in your preamble
%%   \usepackage{pgf}
%%
%% Also ensure that all the required font packages are loaded; for instance,
%% the lmodern package is sometimes necessary when using math font.
%%   \usepackage{lmodern}
%%
%% Figures using additional raster images can only be included by \input if
%% they are in the same directory as the main LaTeX file. For loading figures
%% from other directories you can use the `import` package
%%   \usepackage{import}
%%
%% and then include the figures with
%%   \import{<path to file>}{<filename>.pgf}
%%
%% Matplotlib used the following preamble
%%   \def\mathdefault#1{#1}
%%   \everymath=\expandafter{\the\everymath\displaystyle}
%%   
%%   \makeatletter\@ifpackageloaded{underscore}{}{\usepackage[strings]{underscore}}\makeatother
%%
\begingroup%
\makeatletter%
\begin{pgfpicture}%
\pgfpathrectangle{\pgfpointorigin}{\pgfqpoint{4.774700in}{3.500000in}}%
\pgfusepath{use as bounding box, clip}%
\begin{pgfscope}%
\pgfsetbuttcap%
\pgfsetmiterjoin%
\definecolor{currentfill}{rgb}{1.000000,1.000000,1.000000}%
\pgfsetfillcolor{currentfill}%
\pgfsetlinewidth{0.000000pt}%
\definecolor{currentstroke}{rgb}{1.000000,1.000000,1.000000}%
\pgfsetstrokecolor{currentstroke}%
\pgfsetdash{}{0pt}%
\pgfpathmoveto{\pgfqpoint{0.000000in}{0.000000in}}%
\pgfpathlineto{\pgfqpoint{4.774700in}{0.000000in}}%
\pgfpathlineto{\pgfqpoint{4.774700in}{3.500000in}}%
\pgfpathlineto{\pgfqpoint{0.000000in}{3.500000in}}%
\pgfpathlineto{\pgfqpoint{0.000000in}{0.000000in}}%
\pgfpathclose%
\pgfusepath{fill}%
\end{pgfscope}%
\begin{pgfscope}%
\pgfsetbuttcap%
\pgfsetmiterjoin%
\definecolor{currentfill}{rgb}{1.000000,1.000000,1.000000}%
\pgfsetfillcolor{currentfill}%
\pgfsetlinewidth{0.000000pt}%
\definecolor{currentstroke}{rgb}{0.000000,0.000000,0.000000}%
\pgfsetstrokecolor{currentstroke}%
\pgfsetstrokeopacity{0.000000}%
\pgfsetdash{}{0pt}%
\pgfpathmoveto{\pgfqpoint{0.450320in}{0.472202in}}%
\pgfpathlineto{\pgfqpoint{4.640412in}{0.472202in}}%
\pgfpathlineto{\pgfqpoint{4.640412in}{3.352990in}}%
\pgfpathlineto{\pgfqpoint{0.450320in}{3.352990in}}%
\pgfpathlineto{\pgfqpoint{0.450320in}{0.472202in}}%
\pgfpathclose%
\pgfusepath{fill}%
\end{pgfscope}%
\begin{pgfscope}%
\pgfsetbuttcap%
\pgfsetroundjoin%
\definecolor{currentfill}{rgb}{0.000000,0.000000,0.000000}%
\pgfsetfillcolor{currentfill}%
\pgfsetlinewidth{0.803000pt}%
\definecolor{currentstroke}{rgb}{0.000000,0.000000,0.000000}%
\pgfsetstrokecolor{currentstroke}%
\pgfsetdash{}{0pt}%
\pgfsys@defobject{currentmarker}{\pgfqpoint{0.000000in}{-0.048611in}}{\pgfqpoint{0.000000in}{0.000000in}}{%
\pgfpathmoveto{\pgfqpoint{0.000000in}{0.000000in}}%
\pgfpathlineto{\pgfqpoint{0.000000in}{-0.048611in}}%
\pgfusepath{stroke,fill}%
}%
\begin{pgfscope}%
\pgfsys@transformshift{0.640461in}{0.472202in}%
\pgfsys@useobject{currentmarker}{}%
\end{pgfscope}%
\end{pgfscope}%
\begin{pgfscope}%
\definecolor{textcolor}{rgb}{0.000000,0.000000,0.000000}%
\pgfsetstrokecolor{textcolor}%
\pgfsetfillcolor{textcolor}%
\pgftext[x=0.640461in,y=0.374980in,,top]{\color{textcolor}{\rmfamily\fontsize{12.000000}{14.400000}\selectfont\catcode`\^=\active\def^{\ifmmode\sp\else\^{}\fi}\catcode`\%=\active\def%{\%}$\mathdefault{0}$}}%
\end{pgfscope}%
\begin{pgfscope}%
\pgfsetbuttcap%
\pgfsetroundjoin%
\definecolor{currentfill}{rgb}{0.000000,0.000000,0.000000}%
\pgfsetfillcolor{currentfill}%
\pgfsetlinewidth{0.803000pt}%
\definecolor{currentstroke}{rgb}{0.000000,0.000000,0.000000}%
\pgfsetstrokecolor{currentstroke}%
\pgfsetdash{}{0pt}%
\pgfsys@defobject{currentmarker}{\pgfqpoint{0.000000in}{-0.048611in}}{\pgfqpoint{0.000000in}{0.000000in}}{%
\pgfpathmoveto{\pgfqpoint{0.000000in}{0.000000in}}%
\pgfpathlineto{\pgfqpoint{0.000000in}{-0.048611in}}%
\pgfusepath{stroke,fill}%
}%
\begin{pgfscope}%
\pgfsys@transformshift{1.275376in}{0.472202in}%
\pgfsys@useobject{currentmarker}{}%
\end{pgfscope}%
\end{pgfscope}%
\begin{pgfscope}%
\definecolor{textcolor}{rgb}{0.000000,0.000000,0.000000}%
\pgfsetstrokecolor{textcolor}%
\pgfsetfillcolor{textcolor}%
\pgftext[x=1.275376in,y=0.374980in,,top]{\color{textcolor}{\rmfamily\fontsize{12.000000}{14.400000}\selectfont\catcode`\^=\active\def^{\ifmmode\sp\else\^{}\fi}\catcode`\%=\active\def%{\%}$\mathdefault{20}$}}%
\end{pgfscope}%
\begin{pgfscope}%
\pgfsetbuttcap%
\pgfsetroundjoin%
\definecolor{currentfill}{rgb}{0.000000,0.000000,0.000000}%
\pgfsetfillcolor{currentfill}%
\pgfsetlinewidth{0.803000pt}%
\definecolor{currentstroke}{rgb}{0.000000,0.000000,0.000000}%
\pgfsetstrokecolor{currentstroke}%
\pgfsetdash{}{0pt}%
\pgfsys@defobject{currentmarker}{\pgfqpoint{0.000000in}{-0.048611in}}{\pgfqpoint{0.000000in}{0.000000in}}{%
\pgfpathmoveto{\pgfqpoint{0.000000in}{0.000000in}}%
\pgfpathlineto{\pgfqpoint{0.000000in}{-0.048611in}}%
\pgfusepath{stroke,fill}%
}%
\begin{pgfscope}%
\pgfsys@transformshift{1.910292in}{0.472202in}%
\pgfsys@useobject{currentmarker}{}%
\end{pgfscope}%
\end{pgfscope}%
\begin{pgfscope}%
\definecolor{textcolor}{rgb}{0.000000,0.000000,0.000000}%
\pgfsetstrokecolor{textcolor}%
\pgfsetfillcolor{textcolor}%
\pgftext[x=1.910292in,y=0.374980in,,top]{\color{textcolor}{\rmfamily\fontsize{12.000000}{14.400000}\selectfont\catcode`\^=\active\def^{\ifmmode\sp\else\^{}\fi}\catcode`\%=\active\def%{\%}$\mathdefault{40}$}}%
\end{pgfscope}%
\begin{pgfscope}%
\pgfsetbuttcap%
\pgfsetroundjoin%
\definecolor{currentfill}{rgb}{0.000000,0.000000,0.000000}%
\pgfsetfillcolor{currentfill}%
\pgfsetlinewidth{0.803000pt}%
\definecolor{currentstroke}{rgb}{0.000000,0.000000,0.000000}%
\pgfsetstrokecolor{currentstroke}%
\pgfsetdash{}{0pt}%
\pgfsys@defobject{currentmarker}{\pgfqpoint{0.000000in}{-0.048611in}}{\pgfqpoint{0.000000in}{0.000000in}}{%
\pgfpathmoveto{\pgfqpoint{0.000000in}{0.000000in}}%
\pgfpathlineto{\pgfqpoint{0.000000in}{-0.048611in}}%
\pgfusepath{stroke,fill}%
}%
\begin{pgfscope}%
\pgfsys@transformshift{2.545207in}{0.472202in}%
\pgfsys@useobject{currentmarker}{}%
\end{pgfscope}%
\end{pgfscope}%
\begin{pgfscope}%
\definecolor{textcolor}{rgb}{0.000000,0.000000,0.000000}%
\pgfsetstrokecolor{textcolor}%
\pgfsetfillcolor{textcolor}%
\pgftext[x=2.545207in,y=0.374980in,,top]{\color{textcolor}{\rmfamily\fontsize{12.000000}{14.400000}\selectfont\catcode`\^=\active\def^{\ifmmode\sp\else\^{}\fi}\catcode`\%=\active\def%{\%}$\mathdefault{60}$}}%
\end{pgfscope}%
\begin{pgfscope}%
\pgfsetbuttcap%
\pgfsetroundjoin%
\definecolor{currentfill}{rgb}{0.000000,0.000000,0.000000}%
\pgfsetfillcolor{currentfill}%
\pgfsetlinewidth{0.803000pt}%
\definecolor{currentstroke}{rgb}{0.000000,0.000000,0.000000}%
\pgfsetstrokecolor{currentstroke}%
\pgfsetdash{}{0pt}%
\pgfsys@defobject{currentmarker}{\pgfqpoint{0.000000in}{-0.048611in}}{\pgfqpoint{0.000000in}{0.000000in}}{%
\pgfpathmoveto{\pgfqpoint{0.000000in}{0.000000in}}%
\pgfpathlineto{\pgfqpoint{0.000000in}{-0.048611in}}%
\pgfusepath{stroke,fill}%
}%
\begin{pgfscope}%
\pgfsys@transformshift{3.180122in}{0.472202in}%
\pgfsys@useobject{currentmarker}{}%
\end{pgfscope}%
\end{pgfscope}%
\begin{pgfscope}%
\definecolor{textcolor}{rgb}{0.000000,0.000000,0.000000}%
\pgfsetstrokecolor{textcolor}%
\pgfsetfillcolor{textcolor}%
\pgftext[x=3.180122in,y=0.374980in,,top]{\color{textcolor}{\rmfamily\fontsize{12.000000}{14.400000}\selectfont\catcode`\^=\active\def^{\ifmmode\sp\else\^{}\fi}\catcode`\%=\active\def%{\%}$\mathdefault{80}$}}%
\end{pgfscope}%
\begin{pgfscope}%
\pgfsetbuttcap%
\pgfsetroundjoin%
\definecolor{currentfill}{rgb}{0.000000,0.000000,0.000000}%
\pgfsetfillcolor{currentfill}%
\pgfsetlinewidth{0.803000pt}%
\definecolor{currentstroke}{rgb}{0.000000,0.000000,0.000000}%
\pgfsetstrokecolor{currentstroke}%
\pgfsetdash{}{0pt}%
\pgfsys@defobject{currentmarker}{\pgfqpoint{0.000000in}{-0.048611in}}{\pgfqpoint{0.000000in}{0.000000in}}{%
\pgfpathmoveto{\pgfqpoint{0.000000in}{0.000000in}}%
\pgfpathlineto{\pgfqpoint{0.000000in}{-0.048611in}}%
\pgfusepath{stroke,fill}%
}%
\begin{pgfscope}%
\pgfsys@transformshift{3.815038in}{0.472202in}%
\pgfsys@useobject{currentmarker}{}%
\end{pgfscope}%
\end{pgfscope}%
\begin{pgfscope}%
\definecolor{textcolor}{rgb}{0.000000,0.000000,0.000000}%
\pgfsetstrokecolor{textcolor}%
\pgfsetfillcolor{textcolor}%
\pgftext[x=3.815038in,y=0.374980in,,top]{\color{textcolor}{\rmfamily\fontsize{12.000000}{14.400000}\selectfont\catcode`\^=\active\def^{\ifmmode\sp\else\^{}\fi}\catcode`\%=\active\def%{\%}$\mathdefault{100}$}}%
\end{pgfscope}%
\begin{pgfscope}%
\pgfsetbuttcap%
\pgfsetroundjoin%
\definecolor{currentfill}{rgb}{0.000000,0.000000,0.000000}%
\pgfsetfillcolor{currentfill}%
\pgfsetlinewidth{0.803000pt}%
\definecolor{currentstroke}{rgb}{0.000000,0.000000,0.000000}%
\pgfsetstrokecolor{currentstroke}%
\pgfsetdash{}{0pt}%
\pgfsys@defobject{currentmarker}{\pgfqpoint{0.000000in}{-0.048611in}}{\pgfqpoint{0.000000in}{0.000000in}}{%
\pgfpathmoveto{\pgfqpoint{0.000000in}{0.000000in}}%
\pgfpathlineto{\pgfqpoint{0.000000in}{-0.048611in}}%
\pgfusepath{stroke,fill}%
}%
\begin{pgfscope}%
\pgfsys@transformshift{4.449953in}{0.472202in}%
\pgfsys@useobject{currentmarker}{}%
\end{pgfscope}%
\end{pgfscope}%
\begin{pgfscope}%
\definecolor{textcolor}{rgb}{0.000000,0.000000,0.000000}%
\pgfsetstrokecolor{textcolor}%
\pgfsetfillcolor{textcolor}%
\pgftext[x=4.449953in,y=0.374980in,,top]{\color{textcolor}{\rmfamily\fontsize{12.000000}{14.400000}\selectfont\catcode`\^=\active\def^{\ifmmode\sp\else\^{}\fi}\catcode`\%=\active\def%{\%}$\mathdefault{120}$}}%
\end{pgfscope}%
\begin{pgfscope}%
\definecolor{textcolor}{rgb}{0.000000,0.000000,0.000000}%
\pgfsetstrokecolor{textcolor}%
\pgfsetfillcolor{textcolor}%
\pgftext[x=2.545366in,y=0.171277in,,top]{\color{textcolor}{\rmfamily\fontsize{12.000000}{14.400000}\selectfont\catcode`\^=\active\def^{\ifmmode\sp\else\^{}\fi}\catcode`\%=\active\def%{\%}Time [days]}}%
\end{pgfscope}%
\begin{pgfscope}%
\pgfsetbuttcap%
\pgfsetroundjoin%
\definecolor{currentfill}{rgb}{0.000000,0.000000,0.000000}%
\pgfsetfillcolor{currentfill}%
\pgfsetlinewidth{0.803000pt}%
\definecolor{currentstroke}{rgb}{0.000000,0.000000,0.000000}%
\pgfsetstrokecolor{currentstroke}%
\pgfsetdash{}{0pt}%
\pgfsys@defobject{currentmarker}{\pgfqpoint{-0.048611in}{0.000000in}}{\pgfqpoint{-0.000000in}{0.000000in}}{%
\pgfpathmoveto{\pgfqpoint{-0.000000in}{0.000000in}}%
\pgfpathlineto{\pgfqpoint{-0.048611in}{0.000000in}}%
\pgfusepath{stroke,fill}%
}%
\begin{pgfscope}%
\pgfsys@transformshift{0.450320in}{0.569413in}%
\pgfsys@useobject{currentmarker}{}%
\end{pgfscope}%
\end{pgfscope}%
\begin{pgfscope}%
\definecolor{textcolor}{rgb}{0.000000,0.000000,0.000000}%
\pgfsetstrokecolor{textcolor}%
\pgfsetfillcolor{textcolor}%
\pgftext[x=0.026712in, y=0.511543in, left, base]{\color{textcolor}{\rmfamily\fontsize{12.000000}{14.400000}\selectfont\catcode`\^=\active\def^{\ifmmode\sp\else\^{}\fi}\catcode`\%=\active\def%{\%}$\mathdefault{6830}$}}%
\end{pgfscope}%
\begin{pgfscope}%
\pgfsetbuttcap%
\pgfsetroundjoin%
\definecolor{currentfill}{rgb}{0.000000,0.000000,0.000000}%
\pgfsetfillcolor{currentfill}%
\pgfsetlinewidth{0.803000pt}%
\definecolor{currentstroke}{rgb}{0.000000,0.000000,0.000000}%
\pgfsetstrokecolor{currentstroke}%
\pgfsetdash{}{0pt}%
\pgfsys@defobject{currentmarker}{\pgfqpoint{-0.048611in}{0.000000in}}{\pgfqpoint{-0.000000in}{0.000000in}}{%
\pgfpathmoveto{\pgfqpoint{-0.000000in}{0.000000in}}%
\pgfpathlineto{\pgfqpoint{-0.048611in}{0.000000in}}%
\pgfusepath{stroke,fill}%
}%
\begin{pgfscope}%
\pgfsys@transformshift{0.450320in}{1.120476in}%
\pgfsys@useobject{currentmarker}{}%
\end{pgfscope}%
\end{pgfscope}%
\begin{pgfscope}%
\definecolor{textcolor}{rgb}{0.000000,0.000000,0.000000}%
\pgfsetstrokecolor{textcolor}%
\pgfsetfillcolor{textcolor}%
\pgftext[x=0.026712in, y=1.062606in, left, base]{\color{textcolor}{\rmfamily\fontsize{12.000000}{14.400000}\selectfont\catcode`\^=\active\def^{\ifmmode\sp\else\^{}\fi}\catcode`\%=\active\def%{\%}$\mathdefault{6840}$}}%
\end{pgfscope}%
\begin{pgfscope}%
\pgfsetbuttcap%
\pgfsetroundjoin%
\definecolor{currentfill}{rgb}{0.000000,0.000000,0.000000}%
\pgfsetfillcolor{currentfill}%
\pgfsetlinewidth{0.803000pt}%
\definecolor{currentstroke}{rgb}{0.000000,0.000000,0.000000}%
\pgfsetstrokecolor{currentstroke}%
\pgfsetdash{}{0pt}%
\pgfsys@defobject{currentmarker}{\pgfqpoint{-0.048611in}{0.000000in}}{\pgfqpoint{-0.000000in}{0.000000in}}{%
\pgfpathmoveto{\pgfqpoint{-0.000000in}{0.000000in}}%
\pgfpathlineto{\pgfqpoint{-0.048611in}{0.000000in}}%
\pgfusepath{stroke,fill}%
}%
\begin{pgfscope}%
\pgfsys@transformshift{0.450320in}{1.671540in}%
\pgfsys@useobject{currentmarker}{}%
\end{pgfscope}%
\end{pgfscope}%
\begin{pgfscope}%
\definecolor{textcolor}{rgb}{0.000000,0.000000,0.000000}%
\pgfsetstrokecolor{textcolor}%
\pgfsetfillcolor{textcolor}%
\pgftext[x=0.026712in, y=1.613670in, left, base]{\color{textcolor}{\rmfamily\fontsize{12.000000}{14.400000}\selectfont\catcode`\^=\active\def^{\ifmmode\sp\else\^{}\fi}\catcode`\%=\active\def%{\%}$\mathdefault{6850}$}}%
\end{pgfscope}%
\begin{pgfscope}%
\pgfsetbuttcap%
\pgfsetroundjoin%
\definecolor{currentfill}{rgb}{0.000000,0.000000,0.000000}%
\pgfsetfillcolor{currentfill}%
\pgfsetlinewidth{0.803000pt}%
\definecolor{currentstroke}{rgb}{0.000000,0.000000,0.000000}%
\pgfsetstrokecolor{currentstroke}%
\pgfsetdash{}{0pt}%
\pgfsys@defobject{currentmarker}{\pgfqpoint{-0.048611in}{0.000000in}}{\pgfqpoint{-0.000000in}{0.000000in}}{%
\pgfpathmoveto{\pgfqpoint{-0.000000in}{0.000000in}}%
\pgfpathlineto{\pgfqpoint{-0.048611in}{0.000000in}}%
\pgfusepath{stroke,fill}%
}%
\begin{pgfscope}%
\pgfsys@transformshift{0.450320in}{2.222603in}%
\pgfsys@useobject{currentmarker}{}%
\end{pgfscope}%
\end{pgfscope}%
\begin{pgfscope}%
\definecolor{textcolor}{rgb}{0.000000,0.000000,0.000000}%
\pgfsetstrokecolor{textcolor}%
\pgfsetfillcolor{textcolor}%
\pgftext[x=0.026712in, y=2.164733in, left, base]{\color{textcolor}{\rmfamily\fontsize{12.000000}{14.400000}\selectfont\catcode`\^=\active\def^{\ifmmode\sp\else\^{}\fi}\catcode`\%=\active\def%{\%}$\mathdefault{6860}$}}%
\end{pgfscope}%
\begin{pgfscope}%
\pgfsetbuttcap%
\pgfsetroundjoin%
\definecolor{currentfill}{rgb}{0.000000,0.000000,0.000000}%
\pgfsetfillcolor{currentfill}%
\pgfsetlinewidth{0.803000pt}%
\definecolor{currentstroke}{rgb}{0.000000,0.000000,0.000000}%
\pgfsetstrokecolor{currentstroke}%
\pgfsetdash{}{0pt}%
\pgfsys@defobject{currentmarker}{\pgfqpoint{-0.048611in}{0.000000in}}{\pgfqpoint{-0.000000in}{0.000000in}}{%
\pgfpathmoveto{\pgfqpoint{-0.000000in}{0.000000in}}%
\pgfpathlineto{\pgfqpoint{-0.048611in}{0.000000in}}%
\pgfusepath{stroke,fill}%
}%
\begin{pgfscope}%
\pgfsys@transformshift{0.450320in}{2.773667in}%
\pgfsys@useobject{currentmarker}{}%
\end{pgfscope}%
\end{pgfscope}%
\begin{pgfscope}%
\definecolor{textcolor}{rgb}{0.000000,0.000000,0.000000}%
\pgfsetstrokecolor{textcolor}%
\pgfsetfillcolor{textcolor}%
\pgftext[x=0.026712in, y=2.715797in, left, base]{\color{textcolor}{\rmfamily\fontsize{12.000000}{14.400000}\selectfont\catcode`\^=\active\def^{\ifmmode\sp\else\^{}\fi}\catcode`\%=\active\def%{\%}$\mathdefault{6870}$}}%
\end{pgfscope}%
\begin{pgfscope}%
\pgfsetbuttcap%
\pgfsetroundjoin%
\definecolor{currentfill}{rgb}{0.000000,0.000000,0.000000}%
\pgfsetfillcolor{currentfill}%
\pgfsetlinewidth{0.803000pt}%
\definecolor{currentstroke}{rgb}{0.000000,0.000000,0.000000}%
\pgfsetstrokecolor{currentstroke}%
\pgfsetdash{}{0pt}%
\pgfsys@defobject{currentmarker}{\pgfqpoint{-0.048611in}{0.000000in}}{\pgfqpoint{-0.000000in}{0.000000in}}{%
\pgfpathmoveto{\pgfqpoint{-0.000000in}{0.000000in}}%
\pgfpathlineto{\pgfqpoint{-0.048611in}{0.000000in}}%
\pgfusepath{stroke,fill}%
}%
\begin{pgfscope}%
\pgfsys@transformshift{0.450320in}{3.324730in}%
\pgfsys@useobject{currentmarker}{}%
\end{pgfscope}%
\end{pgfscope}%
\begin{pgfscope}%
\definecolor{textcolor}{rgb}{0.000000,0.000000,0.000000}%
\pgfsetstrokecolor{textcolor}%
\pgfsetfillcolor{textcolor}%
\pgftext[x=0.026712in, y=3.266860in, left, base]{\color{textcolor}{\rmfamily\fontsize{12.000000}{14.400000}\selectfont\catcode`\^=\active\def^{\ifmmode\sp\else\^{}\fi}\catcode`\%=\active\def%{\%}$\mathdefault{6880}$}}%
\end{pgfscope}%
\begin{pgfscope}%
\pgfpathrectangle{\pgfqpoint{0.450320in}{0.472202in}}{\pgfqpoint{4.190092in}{2.880788in}}%
\pgfusepath{clip}%
\pgfsetrectcap%
\pgfsetroundjoin%
\pgfsetlinewidth{1.505625pt}%
\definecolor{currentstroke}{rgb}{0.121569,0.466667,0.705882}%
\pgfsetstrokecolor{currentstroke}%
\pgfsetdash{}{0pt}%
\pgfpathmoveto{\pgfqpoint{0.640778in}{3.222045in}}%
\pgfpathlineto{\pgfqpoint{0.641413in}{2.298730in}}%
\pgfpathlineto{\pgfqpoint{0.642048in}{2.813038in}}%
\pgfpathlineto{\pgfqpoint{0.642366in}{2.193670in}}%
\pgfpathlineto{\pgfqpoint{0.643001in}{3.061935in}}%
\pgfpathlineto{\pgfqpoint{0.643318in}{2.224898in}}%
\pgfpathlineto{\pgfqpoint{0.643953in}{3.196992in}}%
\pgfpathlineto{\pgfqpoint{0.644270in}{2.388198in}}%
\pgfpathlineto{\pgfqpoint{0.644588in}{2.407918in}}%
\pgfpathlineto{\pgfqpoint{0.644905in}{3.199689in}}%
\pgfpathlineto{\pgfqpoint{0.645540in}{2.226933in}}%
\pgfpathlineto{\pgfqpoint{0.645858in}{3.064842in}}%
\pgfpathlineto{\pgfqpoint{0.646175in}{2.923572in}}%
\pgfpathlineto{\pgfqpoint{0.646493in}{2.182071in}}%
\pgfpathlineto{\pgfqpoint{0.647128in}{3.133248in}}%
\pgfpathlineto{\pgfqpoint{0.647445in}{2.283954in}}%
\pgfpathlineto{\pgfqpoint{0.648080in}{3.219500in}}%
\pgfpathlineto{\pgfqpoint{0.648397in}{2.503477in}}%
\pgfpathlineto{\pgfqpoint{0.648715in}{2.312961in}}%
\pgfpathlineto{\pgfqpoint{0.649032in}{3.159698in}}%
\pgfpathlineto{\pgfqpoint{0.649667in}{2.194241in}}%
\pgfpathlineto{\pgfqpoint{0.649985in}{2.969847in}}%
\pgfpathlineto{\pgfqpoint{0.650302in}{3.038136in}}%
\pgfpathlineto{\pgfqpoint{0.650620in}{2.210629in}}%
\pgfpathlineto{\pgfqpoint{0.651255in}{3.184469in}}%
\pgfpathlineto{\pgfqpoint{0.651572in}{2.369278in}}%
\pgfpathlineto{\pgfqpoint{0.651889in}{2.421480in}}%
\pgfpathlineto{\pgfqpoint{0.652207in}{3.203407in}}%
\pgfpathlineto{\pgfqpoint{0.652842in}{2.235613in}}%
\pgfpathlineto{\pgfqpoint{0.653159in}{3.077956in}}%
\pgfpathlineto{\pgfqpoint{0.653794in}{2.177350in}}%
\pgfpathlineto{\pgfqpoint{0.654112in}{2.840699in}}%
\pgfpathlineto{\pgfqpoint{0.654429in}{3.117775in}}%
\pgfpathlineto{\pgfqpoint{0.654747in}{2.266347in}}%
\pgfpathlineto{\pgfqpoint{0.655381in}{3.215030in}}%
\pgfpathlineto{\pgfqpoint{0.655699in}{2.487134in}}%
\pgfpathlineto{\pgfqpoint{0.656016in}{2.326173in}}%
\pgfpathlineto{\pgfqpoint{0.656334in}{3.170714in}}%
\pgfpathlineto{\pgfqpoint{0.656969in}{2.195567in}}%
\pgfpathlineto{\pgfqpoint{0.657286in}{2.984798in}}%
\pgfpathlineto{\pgfqpoint{0.657604in}{3.014407in}}%
\pgfpathlineto{\pgfqpoint{0.657921in}{2.199608in}}%
\pgfpathlineto{\pgfqpoint{0.658556in}{3.173417in}}%
\pgfpathlineto{\pgfqpoint{0.658874in}{2.347193in}}%
\pgfpathlineto{\pgfqpoint{0.659191in}{2.440333in}}%
\pgfpathlineto{\pgfqpoint{0.659508in}{3.204462in}}%
\pgfpathlineto{\pgfqpoint{0.660143in}{2.242389in}}%
\pgfpathlineto{\pgfqpoint{0.660461in}{3.090396in}}%
\pgfpathlineto{\pgfqpoint{0.661096in}{2.173299in}}%
\pgfpathlineto{\pgfqpoint{0.661413in}{2.860291in}}%
\pgfpathlineto{\pgfqpoint{0.661731in}{3.100948in}}%
\pgfpathlineto{\pgfqpoint{0.662048in}{2.252736in}}%
\pgfpathlineto{\pgfqpoint{0.662683in}{3.215458in}}%
\pgfpathlineto{\pgfqpoint{0.663000in}{2.464220in}}%
\pgfpathlineto{\pgfqpoint{0.663318in}{2.342822in}}%
\pgfpathlineto{\pgfqpoint{0.663635in}{3.178150in}}%
\pgfpathlineto{\pgfqpoint{0.664270in}{2.195325in}}%
\pgfpathlineto{\pgfqpoint{0.664588in}{2.997433in}}%
\pgfpathlineto{\pgfqpoint{0.664905in}{2.988759in}}%
\pgfpathlineto{\pgfqpoint{0.665223in}{2.187954in}}%
\pgfpathlineto{\pgfqpoint{0.665858in}{3.160844in}}%
\pgfpathlineto{\pgfqpoint{0.666175in}{2.329699in}}%
\pgfpathlineto{\pgfqpoint{0.666492in}{2.455910in}}%
\pgfpathlineto{\pgfqpoint{0.666810in}{3.206596in}}%
\pgfpathlineto{\pgfqpoint{0.667445in}{2.252870in}}%
\pgfpathlineto{\pgfqpoint{0.667762in}{3.101514in}}%
\pgfpathlineto{\pgfqpoint{0.668397in}{2.170270in}}%
\pgfpathlineto{\pgfqpoint{0.668715in}{2.877945in}}%
\pgfpathlineto{\pgfqpoint{0.669032in}{3.084348in}}%
\pgfpathlineto{\pgfqpoint{0.669350in}{2.239407in}}%
\pgfpathlineto{\pgfqpoint{0.669985in}{3.212114in}}%
\pgfpathlineto{\pgfqpoint{0.670302in}{2.442926in}}%
\pgfpathlineto{\pgfqpoint{0.670619in}{2.355223in}}%
\pgfpathlineto{\pgfqpoint{0.670937in}{3.183309in}}%
\pgfpathlineto{\pgfqpoint{0.671572in}{2.197308in}}%
\pgfpathlineto{\pgfqpoint{0.671889in}{3.010189in}}%
\pgfpathlineto{\pgfqpoint{0.672207in}{2.964608in}}%
\pgfpathlineto{\pgfqpoint{0.672524in}{2.179173in}}%
\pgfpathlineto{\pgfqpoint{0.673159in}{3.148979in}}%
\pgfpathlineto{\pgfqpoint{0.673477in}{2.309082in}}%
\pgfpathlineto{\pgfqpoint{0.674111in}{3.205528in}}%
\pgfpathlineto{\pgfqpoint{0.674429in}{2.550317in}}%
\pgfpathlineto{\pgfqpoint{0.674746in}{2.261088in}}%
\pgfpathlineto{\pgfqpoint{0.675064in}{3.112582in}}%
\pgfpathlineto{\pgfqpoint{0.675699in}{2.168729in}}%
\pgfpathlineto{\pgfqpoint{0.676016in}{2.897338in}}%
\pgfpathlineto{\pgfqpoint{0.676334in}{3.070573in}}%
\pgfpathlineto{\pgfqpoint{0.676651in}{2.229291in}}%
\pgfpathlineto{\pgfqpoint{0.677286in}{3.206907in}}%
\pgfpathlineto{\pgfqpoint{0.677604in}{2.415793in}}%
\pgfpathlineto{\pgfqpoint{0.677921in}{2.370815in}}%
\pgfpathlineto{\pgfqpoint{0.678238in}{3.184087in}}%
\pgfpathlineto{\pgfqpoint{0.678873in}{2.199066in}}%
\pgfpathlineto{\pgfqpoint{0.679191in}{3.022510in}}%
\pgfpathlineto{\pgfqpoint{0.679508in}{2.939284in}}%
\pgfpathlineto{\pgfqpoint{0.679826in}{2.170917in}}%
\pgfpathlineto{\pgfqpoint{0.680461in}{3.135563in}}%
\pgfpathlineto{\pgfqpoint{0.680778in}{2.292636in}}%
\pgfpathlineto{\pgfqpoint{0.681413in}{3.205257in}}%
\pgfpathlineto{\pgfqpoint{0.681730in}{2.522750in}}%
\pgfpathlineto{\pgfqpoint{0.682048in}{2.273586in}}%
\pgfpathlineto{\pgfqpoint{0.682365in}{3.122104in}}%
\pgfpathlineto{\pgfqpoint{0.683000in}{2.167551in}}%
\pgfpathlineto{\pgfqpoint{0.683318in}{2.915859in}}%
\pgfpathlineto{\pgfqpoint{0.683635in}{3.055989in}}%
\pgfpathlineto{\pgfqpoint{0.683953in}{2.216429in}}%
\pgfpathlineto{\pgfqpoint{0.684588in}{3.195854in}}%
\pgfpathlineto{\pgfqpoint{0.684905in}{2.392217in}}%
\pgfpathlineto{\pgfqpoint{0.685222in}{2.382427in}}%
\pgfpathlineto{\pgfqpoint{0.685540in}{3.185159in}}%
\pgfpathlineto{\pgfqpoint{0.686175in}{2.204694in}}%
\pgfpathlineto{\pgfqpoint{0.686492in}{3.035307in}}%
\pgfpathlineto{\pgfqpoint{0.686810in}{2.916377in}}%
\pgfpathlineto{\pgfqpoint{0.687127in}{2.165000in}}%
\pgfpathlineto{\pgfqpoint{0.687762in}{3.122121in}}%
\pgfpathlineto{\pgfqpoint{0.688080in}{2.273340in}}%
\pgfpathlineto{\pgfqpoint{0.688715in}{3.201652in}}%
\pgfpathlineto{\pgfqpoint{0.689032in}{2.499431in}}%
\pgfpathlineto{\pgfqpoint{0.689349in}{2.283591in}}%
\pgfpathlineto{\pgfqpoint{0.689667in}{3.132041in}}%
\pgfpathlineto{\pgfqpoint{0.690302in}{2.169609in}}%
\pgfpathlineto{\pgfqpoint{0.690619in}{2.936685in}}%
\pgfpathlineto{\pgfqpoint{0.690937in}{3.038106in}}%
\pgfpathlineto{\pgfqpoint{0.691254in}{2.205469in}}%
\pgfpathlineto{\pgfqpoint{0.691889in}{3.183646in}}%
\pgfpathlineto{\pgfqpoint{0.692207in}{2.365098in}}%
\pgfpathlineto{\pgfqpoint{0.692524in}{2.399024in}}%
\pgfpathlineto{\pgfqpoint{0.692841in}{3.184316in}}%
\pgfpathlineto{\pgfqpoint{0.693476in}{2.209629in}}%
\pgfpathlineto{\pgfqpoint{0.693794in}{3.048953in}}%
\pgfpathlineto{\pgfqpoint{0.694111in}{2.891830in}}%
\pgfpathlineto{\pgfqpoint{0.694429in}{2.159942in}}%
\pgfpathlineto{\pgfqpoint{0.695064in}{3.106822in}}%
\pgfpathlineto{\pgfqpoint{0.695381in}{2.257724in}}%
\pgfpathlineto{\pgfqpoint{0.696016in}{3.198694in}}%
\pgfpathlineto{\pgfqpoint{0.696334in}{2.472020in}}%
\pgfpathlineto{\pgfqpoint{0.696651in}{2.298331in}}%
\pgfpathlineto{\pgfqpoint{0.696968in}{3.140020in}}%
\pgfpathlineto{\pgfqpoint{0.697603in}{2.172524in}}%
\pgfpathlineto{\pgfqpoint{0.697921in}{2.956561in}}%
\pgfpathlineto{\pgfqpoint{0.698238in}{3.017102in}}%
\pgfpathlineto{\pgfqpoint{0.698556in}{2.192238in}}%
\pgfpathlineto{\pgfqpoint{0.699191in}{3.168307in}}%
\pgfpathlineto{\pgfqpoint{0.699508in}{2.342966in}}%
\pgfpathlineto{\pgfqpoint{0.699826in}{2.412471in}}%
\pgfpathlineto{\pgfqpoint{0.700143in}{3.185132in}}%
\pgfpathlineto{\pgfqpoint{0.700778in}{2.219176in}}%
\pgfpathlineto{\pgfqpoint{0.701095in}{3.062682in}}%
\pgfpathlineto{\pgfqpoint{0.701730in}{2.155948in}}%
\pgfpathlineto{\pgfqpoint{0.702048in}{2.831216in}}%
\pgfpathlineto{\pgfqpoint{0.702365in}{3.090479in}}%
\pgfpathlineto{\pgfqpoint{0.702683in}{2.239655in}}%
\pgfpathlineto{\pgfqpoint{0.703318in}{3.192564in}}%
\pgfpathlineto{\pgfqpoint{0.703635in}{2.449091in}}%
\pgfpathlineto{\pgfqpoint{0.703952in}{2.310348in}}%
\pgfpathlineto{\pgfqpoint{0.704270in}{3.148579in}}%
\pgfpathlineto{\pgfqpoint{0.704905in}{2.177750in}}%
\pgfpathlineto{\pgfqpoint{0.705222in}{2.975707in}}%
\pgfpathlineto{\pgfqpoint{0.705540in}{2.994052in}}%
\pgfpathlineto{\pgfqpoint{0.705857in}{2.181076in}}%
\pgfpathlineto{\pgfqpoint{0.706492in}{3.153729in}}%
\pgfpathlineto{\pgfqpoint{0.706810in}{2.318330in}}%
\pgfpathlineto{\pgfqpoint{0.707127in}{2.431693in}}%
\pgfpathlineto{\pgfqpoint{0.707445in}{3.184379in}}%
\pgfpathlineto{\pgfqpoint{0.708079in}{2.227191in}}%
\pgfpathlineto{\pgfqpoint{0.708397in}{3.076686in}}%
\pgfpathlineto{\pgfqpoint{0.709032in}{2.153110in}}%
\pgfpathlineto{\pgfqpoint{0.709349in}{2.852722in}}%
\pgfpathlineto{\pgfqpoint{0.709667in}{3.072294in}}%
\pgfpathlineto{\pgfqpoint{0.709984in}{2.224965in}}%
\pgfpathlineto{\pgfqpoint{0.710619in}{3.187148in}}%
\pgfpathlineto{\pgfqpoint{0.710937in}{2.422309in}}%
\pgfpathlineto{\pgfqpoint{0.711254in}{2.327323in}}%
\pgfpathlineto{\pgfqpoint{0.711571in}{3.154996in}}%
\pgfpathlineto{\pgfqpoint{0.712206in}{2.182006in}}%
\pgfpathlineto{\pgfqpoint{0.712524in}{2.993531in}}%
\pgfpathlineto{\pgfqpoint{0.712841in}{2.967644in}}%
\pgfpathlineto{\pgfqpoint{0.713159in}{2.169292in}}%
\pgfpathlineto{\pgfqpoint{0.713794in}{3.137437in}}%
\pgfpathlineto{\pgfqpoint{0.714111in}{2.298511in}}%
\pgfpathlineto{\pgfqpoint{0.714429in}{2.448120in}}%
\pgfpathlineto{\pgfqpoint{0.714746in}{3.185325in}}%
\pgfpathlineto{\pgfqpoint{0.715381in}{2.239835in}}%
\pgfpathlineto{\pgfqpoint{0.715698in}{3.089278in}}%
\pgfpathlineto{\pgfqpoint{0.716333in}{2.150676in}}%
\pgfpathlineto{\pgfqpoint{0.716651in}{2.872131in}}%
\pgfpathlineto{\pgfqpoint{0.716968in}{3.052737in}}%
\pgfpathlineto{\pgfqpoint{0.717286in}{2.208742in}}%
\pgfpathlineto{\pgfqpoint{0.717921in}{3.178975in}}%
\pgfpathlineto{\pgfqpoint{0.718238in}{2.400815in}}%
\pgfpathlineto{\pgfqpoint{0.718556in}{2.342178in}}%
\pgfpathlineto{\pgfqpoint{0.718873in}{3.164252in}}%
\pgfpathlineto{\pgfqpoint{0.719508in}{2.188344in}}%
\pgfpathlineto{\pgfqpoint{0.719825in}{3.009675in}}%
\pgfpathlineto{\pgfqpoint{0.720143in}{2.941464in}}%
\pgfpathlineto{\pgfqpoint{0.720460in}{2.159749in}}%
\pgfpathlineto{\pgfqpoint{0.721095in}{3.121737in}}%
\pgfpathlineto{\pgfqpoint{0.721413in}{2.276572in}}%
\pgfpathlineto{\pgfqpoint{0.722048in}{3.183650in}}%
\pgfpathlineto{\pgfqpoint{0.722365in}{2.513135in}}%
\pgfpathlineto{\pgfqpoint{0.722682in}{2.250101in}}%
\pgfpathlineto{\pgfqpoint{0.723000in}{3.101777in}}%
\pgfpathlineto{\pgfqpoint{0.723635in}{2.150170in}}%
\pgfpathlineto{\pgfqpoint{0.723952in}{2.892459in}}%
\pgfpathlineto{\pgfqpoint{0.724270in}{3.032509in}}%
\pgfpathlineto{\pgfqpoint{0.724587in}{2.195692in}}%
\pgfpathlineto{\pgfqpoint{0.725222in}{3.171263in}}%
\pgfpathlineto{\pgfqpoint{0.725540in}{2.378888in}}%
\pgfpathlineto{\pgfqpoint{0.725857in}{2.363328in}}%
\pgfpathlineto{\pgfqpoint{0.726175in}{3.171617in}}%
\pgfpathlineto{\pgfqpoint{0.726809in}{2.192506in}}%
\pgfpathlineto{\pgfqpoint{0.727127in}{3.024336in}}%
\pgfpathlineto{\pgfqpoint{0.727444in}{2.911770in}}%
\pgfpathlineto{\pgfqpoint{0.727762in}{2.150643in}}%
\pgfpathlineto{\pgfqpoint{0.728397in}{3.104909in}}%
\pgfpathlineto{\pgfqpoint{0.728714in}{2.258909in}}%
\pgfpathlineto{\pgfqpoint{0.729349in}{3.182971in}}%
\pgfpathlineto{\pgfqpoint{0.729667in}{2.484656in}}%
\pgfpathlineto{\pgfqpoint{0.729984in}{2.264951in}}%
\pgfpathlineto{\pgfqpoint{0.730301in}{3.111853in}}%
\pgfpathlineto{\pgfqpoint{0.730936in}{2.149629in}}%
\pgfpathlineto{\pgfqpoint{0.731254in}{2.911436in}}%
\pgfpathlineto{\pgfqpoint{0.731571in}{3.010812in}}%
\pgfpathlineto{\pgfqpoint{0.731889in}{2.181786in}}%
\pgfpathlineto{\pgfqpoint{0.732524in}{3.164444in}}%
\pgfpathlineto{\pgfqpoint{0.732841in}{2.360783in}}%
\pgfpathlineto{\pgfqpoint{0.733159in}{2.381002in}}%
\pgfpathlineto{\pgfqpoint{0.733476in}{3.178614in}}%
\pgfpathlineto{\pgfqpoint{0.734111in}{2.199496in}}%
\pgfpathlineto{\pgfqpoint{0.734428in}{3.037221in}}%
\pgfpathlineto{\pgfqpoint{0.734746in}{2.883969in}}%
\pgfpathlineto{\pgfqpoint{0.735063in}{2.143388in}}%
\pgfpathlineto{\pgfqpoint{0.735698in}{3.088014in}}%
\pgfpathlineto{\pgfqpoint{0.736016in}{2.239285in}}%
\pgfpathlineto{\pgfqpoint{0.736651in}{3.179006in}}%
\pgfpathlineto{\pgfqpoint{0.736968in}{2.460298in}}%
\pgfpathlineto{\pgfqpoint{0.737286in}{2.276959in}}%
\pgfpathlineto{\pgfqpoint{0.737603in}{3.122119in}}%
\pgfpathlineto{\pgfqpoint{0.738238in}{2.151966in}}%
\pgfpathlineto{\pgfqpoint{0.738555in}{2.931014in}}%
\pgfpathlineto{\pgfqpoint{0.738873in}{2.988971in}}%
\pgfpathlineto{\pgfqpoint{0.739190in}{2.172115in}}%
\pgfpathlineto{\pgfqpoint{0.739825in}{3.158509in}}%
\pgfpathlineto{\pgfqpoint{0.740143in}{2.336620in}}%
\pgfpathlineto{\pgfqpoint{0.740460in}{2.402458in}}%
\pgfpathlineto{\pgfqpoint{0.740778in}{3.179852in}}%
\pgfpathlineto{\pgfqpoint{0.741412in}{2.204551in}}%
\pgfpathlineto{\pgfqpoint{0.741730in}{3.049710in}}%
\pgfpathlineto{\pgfqpoint{0.742365in}{2.137822in}}%
\pgfpathlineto{\pgfqpoint{0.742682in}{2.810866in}}%
\pgfpathlineto{\pgfqpoint{0.743000in}{3.070386in}}%
\pgfpathlineto{\pgfqpoint{0.743317in}{2.223422in}}%
\pgfpathlineto{\pgfqpoint{0.743952in}{3.175671in}}%
\pgfpathlineto{\pgfqpoint{0.744270in}{2.431896in}}%
\pgfpathlineto{\pgfqpoint{0.744587in}{2.293853in}}%
\pgfpathlineto{\pgfqpoint{0.744905in}{3.129736in}}%
\pgfpathlineto{\pgfqpoint{0.745539in}{2.153857in}}%
\pgfpathlineto{\pgfqpoint{0.745857in}{2.949780in}}%
\pgfpathlineto{\pgfqpoint{0.746174in}{2.968692in}}%
\pgfpathlineto{\pgfqpoint{0.746492in}{2.163628in}}%
\pgfpathlineto{\pgfqpoint{0.747127in}{3.148007in}}%
\pgfpathlineto{\pgfqpoint{0.747444in}{2.314663in}}%
\pgfpathlineto{\pgfqpoint{0.747762in}{2.419599in}}%
\pgfpathlineto{\pgfqpoint{0.748079in}{3.179702in}}%
\pgfpathlineto{\pgfqpoint{0.748714in}{2.213953in}}%
\pgfpathlineto{\pgfqpoint{0.749031in}{3.060514in}}%
\pgfpathlineto{\pgfqpoint{0.749666in}{2.133576in}}%
\pgfpathlineto{\pgfqpoint{0.749984in}{2.831597in}}%
\pgfpathlineto{\pgfqpoint{0.750301in}{3.051960in}}%
\pgfpathlineto{\pgfqpoint{0.750619in}{2.206076in}}%
\pgfpathlineto{\pgfqpoint{0.751254in}{3.169015in}}%
\pgfpathlineto{\pgfqpoint{0.751571in}{2.407803in}}%
\pgfpathlineto{\pgfqpoint{0.751889in}{2.307805in}}%
\pgfpathlineto{\pgfqpoint{0.752206in}{3.137970in}}%
\pgfpathlineto{\pgfqpoint{0.752841in}{2.159221in}}%
\pgfpathlineto{\pgfqpoint{0.753158in}{2.968265in}}%
\pgfpathlineto{\pgfqpoint{0.753476in}{2.949252in}}%
\pgfpathlineto{\pgfqpoint{0.753793in}{2.156747in}}%
\pgfpathlineto{\pgfqpoint{0.754428in}{3.134525in}}%
\pgfpathlineto{\pgfqpoint{0.754746in}{2.288637in}}%
\pgfpathlineto{\pgfqpoint{0.755063in}{2.440493in}}%
\pgfpathlineto{\pgfqpoint{0.755381in}{3.175385in}}%
\pgfpathlineto{\pgfqpoint{0.756016in}{2.221893in}}%
\pgfpathlineto{\pgfqpoint{0.756333in}{3.072181in}}%
\pgfpathlineto{\pgfqpoint{0.756968in}{2.132105in}}%
\pgfpathlineto{\pgfqpoint{0.757285in}{2.853808in}}%
\pgfpathlineto{\pgfqpoint{0.757603in}{3.032922in}}%
\pgfpathlineto{\pgfqpoint{0.757920in}{2.191868in}}%
\pgfpathlineto{\pgfqpoint{0.758555in}{3.162511in}}%
\pgfpathlineto{\pgfqpoint{0.758873in}{2.380021in}}%
\pgfpathlineto{\pgfqpoint{0.759190in}{2.326884in}}%
\pgfpathlineto{\pgfqpoint{0.759508in}{3.143343in}}%
\pgfpathlineto{\pgfqpoint{0.760142in}{2.163919in}}%
\pgfpathlineto{\pgfqpoint{0.760460in}{2.987770in}}%
\pgfpathlineto{\pgfqpoint{0.760777in}{2.926527in}}%
\pgfpathlineto{\pgfqpoint{0.761095in}{2.148817in}}%
\pgfpathlineto{\pgfqpoint{0.761730in}{3.116752in}}%
\pgfpathlineto{\pgfqpoint{0.762047in}{2.266548in}}%
\pgfpathlineto{\pgfqpoint{0.762682in}{3.171666in}}%
\pgfpathlineto{\pgfqpoint{0.763000in}{2.491321in}}%
\pgfpathlineto{\pgfqpoint{0.763317in}{2.235092in}}%
\pgfpathlineto{\pgfqpoint{0.763635in}{3.082434in}}%
\pgfpathlineto{\pgfqpoint{0.764269in}{2.131116in}}%
\pgfpathlineto{\pgfqpoint{0.764587in}{2.875065in}}%
\pgfpathlineto{\pgfqpoint{0.764904in}{3.012318in}}%
\pgfpathlineto{\pgfqpoint{0.765222in}{2.176772in}}%
\pgfpathlineto{\pgfqpoint{0.765857in}{3.152894in}}%
\pgfpathlineto{\pgfqpoint{0.766174in}{2.356609in}}%
\pgfpathlineto{\pgfqpoint{0.766492in}{2.342940in}}%
\pgfpathlineto{\pgfqpoint{0.766809in}{3.149497in}}%
\pgfpathlineto{\pgfqpoint{0.767444in}{2.173854in}}%
\pgfpathlineto{\pgfqpoint{0.767761in}{3.006609in}}%
\pgfpathlineto{\pgfqpoint{0.768079in}{2.902823in}}%
\pgfpathlineto{\pgfqpoint{0.768396in}{2.141321in}}%
\pgfpathlineto{\pgfqpoint{0.769031in}{3.097599in}}%
\pgfpathlineto{\pgfqpoint{0.769349in}{2.242529in}}%
\pgfpathlineto{\pgfqpoint{0.769984in}{3.165436in}}%
\pgfpathlineto{\pgfqpoint{0.770301in}{2.464377in}}%
\pgfpathlineto{\pgfqpoint{0.770619in}{2.246346in}}%
\pgfpathlineto{\pgfqpoint{0.770936in}{3.094034in}}%
\pgfpathlineto{\pgfqpoint{0.771571in}{2.133210in}}%
\pgfpathlineto{\pgfqpoint{0.771888in}{2.896942in}}%
\pgfpathlineto{\pgfqpoint{0.772206in}{2.991030in}}%
\pgfpathlineto{\pgfqpoint{0.772523in}{2.164112in}}%
\pgfpathlineto{\pgfqpoint{0.773158in}{3.143056in}}%
\pgfpathlineto{\pgfqpoint{0.773476in}{2.330022in}}%
\pgfpathlineto{\pgfqpoint{0.773793in}{2.364215in}}%
\pgfpathlineto{\pgfqpoint{0.774111in}{3.152550in}}%
\pgfpathlineto{\pgfqpoint{0.774746in}{2.182881in}}%
\pgfpathlineto{\pgfqpoint{0.775063in}{3.025502in}}%
\pgfpathlineto{\pgfqpoint{0.775380in}{2.874837in}}%
\pgfpathlineto{\pgfqpoint{0.775698in}{2.134249in}}%
\pgfpathlineto{\pgfqpoint{0.776333in}{3.076557in}}%
\pgfpathlineto{\pgfqpoint{0.776650in}{2.222510in}}%
\pgfpathlineto{\pgfqpoint{0.777285in}{3.160751in}}%
\pgfpathlineto{\pgfqpoint{0.777603in}{2.434398in}}%
\pgfpathlineto{\pgfqpoint{0.777920in}{2.263367in}}%
\pgfpathlineto{\pgfqpoint{0.778238in}{3.103929in}}%
\pgfpathlineto{\pgfqpoint{0.778872in}{2.134784in}}%
\pgfpathlineto{\pgfqpoint{0.779190in}{2.917902in}}%
\pgfpathlineto{\pgfqpoint{0.779507in}{2.967150in}}%
\pgfpathlineto{\pgfqpoint{0.779825in}{2.151298in}}%
\pgfpathlineto{\pgfqpoint{0.780460in}{3.130559in}}%
\pgfpathlineto{\pgfqpoint{0.780777in}{2.307747in}}%
\pgfpathlineto{\pgfqpoint{0.781095in}{2.382438in}}%
\pgfpathlineto{\pgfqpoint{0.781412in}{3.156283in}}%
\pgfpathlineto{\pgfqpoint{0.782047in}{2.195627in}}%
\pgfpathlineto{\pgfqpoint{0.782365in}{3.041586in}}%
\pgfpathlineto{\pgfqpoint{0.782999in}{2.127956in}}%
\pgfpathlineto{\pgfqpoint{0.783317in}{2.805570in}}%
\pgfpathlineto{\pgfqpoint{0.783634in}{3.054875in}}%
\pgfpathlineto{\pgfqpoint{0.783952in}{2.201944in}}%
\pgfpathlineto{\pgfqpoint{0.784587in}{3.154147in}}%
\pgfpathlineto{\pgfqpoint{0.784904in}{2.409048in}}%
\pgfpathlineto{\pgfqpoint{0.785222in}{2.277995in}}%
\pgfpathlineto{\pgfqpoint{0.785539in}{3.114709in}}%
\pgfpathlineto{\pgfqpoint{0.786174in}{2.139761in}}%
\pgfpathlineto{\pgfqpoint{0.786491in}{2.938346in}}%
\pgfpathlineto{\pgfqpoint{0.786809in}{2.943227in}}%
\pgfpathlineto{\pgfqpoint{0.787126in}{2.140513in}}%
\pgfpathlineto{\pgfqpoint{0.787761in}{3.117673in}}%
\pgfpathlineto{\pgfqpoint{0.788079in}{2.282847in}}%
\pgfpathlineto{\pgfqpoint{0.788396in}{2.405660in}}%
\pgfpathlineto{\pgfqpoint{0.788714in}{3.157236in}}%
\pgfpathlineto{\pgfqpoint{0.789349in}{2.206292in}}%
\pgfpathlineto{\pgfqpoint{0.789666in}{3.057524in}}%
\pgfpathlineto{\pgfqpoint{0.790301in}{2.123422in}}%
\pgfpathlineto{\pgfqpoint{0.790618in}{2.827369in}}%
\pgfpathlineto{\pgfqpoint{0.790936in}{3.032418in}}%
\pgfpathlineto{\pgfqpoint{0.791253in}{2.184955in}}%
\pgfpathlineto{\pgfqpoint{0.791888in}{3.148263in}}%
\pgfpathlineto{\pgfqpoint{0.792206in}{2.380488in}}%
\pgfpathlineto{\pgfqpoint{0.792523in}{2.298057in}}%
\pgfpathlineto{\pgfqpoint{0.792841in}{3.122480in}}%
\pgfpathlineto{\pgfqpoint{0.793476in}{2.143597in}}%
\pgfpathlineto{\pgfqpoint{0.793793in}{2.957962in}}%
\pgfpathlineto{\pgfqpoint{0.794110in}{2.916333in}}%
\pgfpathlineto{\pgfqpoint{0.794428in}{2.130596in}}%
\pgfpathlineto{\pgfqpoint{0.795063in}{3.102750in}}%
\pgfpathlineto{\pgfqpoint{0.795380in}{2.262415in}}%
\pgfpathlineto{\pgfqpoint{0.796015in}{3.161034in}}%
\pgfpathlineto{\pgfqpoint{0.796333in}{2.509607in}}%
\pgfpathlineto{\pgfqpoint{0.796650in}{2.220511in}}%
\pgfpathlineto{\pgfqpoint{0.796968in}{3.069924in}}%
\pgfpathlineto{\pgfqpoint{0.797602in}{2.119536in}}%
\pgfpathlineto{\pgfqpoint{0.797920in}{2.848283in}}%
\pgfpathlineto{\pgfqpoint{0.798237in}{3.008909in}}%
\pgfpathlineto{\pgfqpoint{0.798555in}{2.167920in}}%
\pgfpathlineto{\pgfqpoint{0.799190in}{3.139745in}}%
\pgfpathlineto{\pgfqpoint{0.799507in}{2.356150in}}%
\pgfpathlineto{\pgfqpoint{0.799825in}{2.315023in}}%
\pgfpathlineto{\pgfqpoint{0.800142in}{3.130461in}}%
\pgfpathlineto{\pgfqpoint{0.800777in}{2.151477in}}%
\pgfpathlineto{\pgfqpoint{0.801095in}{2.976285in}}%
\pgfpathlineto{\pgfqpoint{0.801412in}{2.890448in}}%
\pgfpathlineto{\pgfqpoint{0.801729in}{2.122284in}}%
\pgfpathlineto{\pgfqpoint{0.802364in}{3.086907in}}%
\pgfpathlineto{\pgfqpoint{0.802682in}{2.242362in}}%
\pgfpathlineto{\pgfqpoint{0.803317in}{3.161345in}}%
\pgfpathlineto{\pgfqpoint{0.803634in}{2.481956in}}%
\pgfpathlineto{\pgfqpoint{0.803952in}{2.231774in}}%
\pgfpathlineto{\pgfqpoint{0.804269in}{3.082519in}}%
\pgfpathlineto{\pgfqpoint{0.804904in}{2.118573in}}%
\pgfpathlineto{\pgfqpoint{0.805221in}{2.869610in}}%
\pgfpathlineto{\pgfqpoint{0.805539in}{2.985500in}}%
\pgfpathlineto{\pgfqpoint{0.805856in}{2.153849in}}%
\pgfpathlineto{\pgfqpoint{0.806491in}{3.131277in}}%
\pgfpathlineto{\pgfqpoint{0.806809in}{2.328689in}}%
\pgfpathlineto{\pgfqpoint{0.807126in}{2.337134in}}%
\pgfpathlineto{\pgfqpoint{0.807444in}{3.134904in}}%
\pgfpathlineto{\pgfqpoint{0.808079in}{2.157825in}}%
\pgfpathlineto{\pgfqpoint{0.808396in}{2.994429in}}%
\pgfpathlineto{\pgfqpoint{0.808714in}{2.861498in}}%
\pgfpathlineto{\pgfqpoint{0.809031in}{2.115698in}}%
\pgfpathlineto{\pgfqpoint{0.809666in}{3.071388in}}%
\pgfpathlineto{\pgfqpoint{0.809983in}{2.226561in}}%
\pgfpathlineto{\pgfqpoint{0.810618in}{3.161518in}}%
\pgfpathlineto{\pgfqpoint{0.810936in}{2.448771in}}%
\pgfpathlineto{\pgfqpoint{0.811253in}{2.247383in}}%
\pgfpathlineto{\pgfqpoint{0.811571in}{3.091385in}}%
\pgfpathlineto{\pgfqpoint{0.812206in}{2.117580in}}%
\pgfpathlineto{\pgfqpoint{0.812523in}{2.890377in}}%
\pgfpathlineto{\pgfqpoint{0.812840in}{2.960353in}}%
\pgfpathlineto{\pgfqpoint{0.813158in}{2.140239in}}%
\pgfpathlineto{\pgfqpoint{0.813793in}{3.120065in}}%
\pgfpathlineto{\pgfqpoint{0.814110in}{2.305224in}}%
\pgfpathlineto{\pgfqpoint{0.814428in}{2.356098in}}%
\pgfpathlineto{\pgfqpoint{0.814745in}{3.139719in}}%
\pgfpathlineto{\pgfqpoint{0.815380in}{2.168971in}}%
\pgfpathlineto{\pgfqpoint{0.815698in}{3.010968in}}%
\pgfpathlineto{\pgfqpoint{0.816333in}{2.111243in}}%
\pgfpathlineto{\pgfqpoint{0.816650in}{2.776273in}}%
\pgfpathlineto{\pgfqpoint{0.816967in}{3.057432in}}%
\pgfpathlineto{\pgfqpoint{0.817285in}{2.207428in}}%
\pgfpathlineto{\pgfqpoint{0.817920in}{3.156344in}}%
\pgfpathlineto{\pgfqpoint{0.818237in}{2.419001in}}%
\pgfpathlineto{\pgfqpoint{0.818555in}{2.260182in}}%
\pgfpathlineto{\pgfqpoint{0.818872in}{3.100432in}}%
\pgfpathlineto{\pgfqpoint{0.819507in}{2.121019in}}%
\pgfpathlineto{\pgfqpoint{0.819825in}{2.911045in}}%
\pgfpathlineto{\pgfqpoint{0.820142in}{2.936096in}}%
\pgfpathlineto{\pgfqpoint{0.820459in}{2.128924in}}%
\pgfpathlineto{\pgfqpoint{0.821094in}{3.108265in}}%
\pgfpathlineto{\pgfqpoint{0.821412in}{2.279103in}}%
\pgfpathlineto{\pgfqpoint{0.821729in}{2.380031in}}%
\pgfpathlineto{\pgfqpoint{0.822047in}{3.140929in}}%
\pgfpathlineto{\pgfqpoint{0.822682in}{2.178292in}}%
\pgfpathlineto{\pgfqpoint{0.822999in}{3.027753in}}%
\pgfpathlineto{\pgfqpoint{0.823634in}{2.111365in}}%
\pgfpathlineto{\pgfqpoint{0.823951in}{2.804157in}}%
\pgfpathlineto{\pgfqpoint{0.824269in}{3.040372in}}%
\pgfpathlineto{\pgfqpoint{0.824586in}{2.189423in}}%
\pgfpathlineto{\pgfqpoint{0.825221in}{3.149660in}}%
\pgfpathlineto{\pgfqpoint{0.825539in}{2.385073in}}%
\pgfpathlineto{\pgfqpoint{0.825856in}{2.278191in}}%
\pgfpathlineto{\pgfqpoint{0.826174in}{3.106281in}}%
\pgfpathlineto{\pgfqpoint{0.826809in}{2.123860in}}%
\pgfpathlineto{\pgfqpoint{0.827126in}{2.931550in}}%
\pgfpathlineto{\pgfqpoint{0.827444in}{2.909353in}}%
\pgfpathlineto{\pgfqpoint{0.827761in}{2.118702in}}%
\pgfpathlineto{\pgfqpoint{0.828396in}{3.093988in}}%
\pgfpathlineto{\pgfqpoint{0.828713in}{2.256800in}}%
\pgfpathlineto{\pgfqpoint{0.829031in}{2.400952in}}%
\pgfpathlineto{\pgfqpoint{0.829348in}{3.142570in}}%
\pgfpathlineto{\pgfqpoint{0.829983in}{2.192752in}}%
\pgfpathlineto{\pgfqpoint{0.830301in}{3.042126in}}%
\pgfpathlineto{\pgfqpoint{0.830936in}{2.110981in}}%
\pgfpathlineto{\pgfqpoint{0.831253in}{2.829967in}}%
\pgfpathlineto{\pgfqpoint{0.831570in}{3.018770in}}%
\pgfpathlineto{\pgfqpoint{0.831888in}{2.170288in}}%
\pgfpathlineto{\pgfqpoint{0.832523in}{3.137841in}}%
\pgfpathlineto{\pgfqpoint{0.832840in}{2.355216in}}%
\pgfpathlineto{\pgfqpoint{0.833158in}{2.293312in}}%
\pgfpathlineto{\pgfqpoint{0.833475in}{3.112536in}}%
\pgfpathlineto{\pgfqpoint{0.834110in}{2.131582in}}%
\pgfpathlineto{\pgfqpoint{0.834428in}{2.951355in}}%
\pgfpathlineto{\pgfqpoint{0.834745in}{2.883887in}}%
\pgfpathlineto{\pgfqpoint{0.835063in}{2.109984in}}%
\pgfpathlineto{\pgfqpoint{0.835697in}{3.078550in}}%
\pgfpathlineto{\pgfqpoint{0.836015in}{2.232565in}}%
\pgfpathlineto{\pgfqpoint{0.836650in}{3.140705in}}%
\pgfpathlineto{\pgfqpoint{0.836967in}{2.466674in}}%
\pgfpathlineto{\pgfqpoint{0.837285in}{2.205123in}}%
\pgfpathlineto{\pgfqpoint{0.837602in}{3.057631in}}%
\pgfpathlineto{\pgfqpoint{0.838237in}{2.112679in}}%
\pgfpathlineto{\pgfqpoint{0.838555in}{2.854698in}}%
\pgfpathlineto{\pgfqpoint{0.838872in}{2.994536in}}%
\pgfpathlineto{\pgfqpoint{0.839189in}{2.152466in}}%
\pgfpathlineto{\pgfqpoint{0.839824in}{3.124869in}}%
\pgfpathlineto{\pgfqpoint{0.840142in}{2.323384in}}%
\pgfpathlineto{\pgfqpoint{0.840459in}{2.314427in}}%
\pgfpathlineto{\pgfqpoint{0.840777in}{3.116193in}}%
\pgfpathlineto{\pgfqpoint{0.841412in}{2.138144in}}%
\pgfpathlineto{\pgfqpoint{0.841729in}{2.971300in}}%
\pgfpathlineto{\pgfqpoint{0.842047in}{2.855025in}}%
\pgfpathlineto{\pgfqpoint{0.842364in}{2.103051in}}%
\pgfpathlineto{\pgfqpoint{0.842999in}{3.061071in}}%
\pgfpathlineto{\pgfqpoint{0.843316in}{2.211897in}}%
\pgfpathlineto{\pgfqpoint{0.843951in}{3.139134in}}%
\pgfpathlineto{\pgfqpoint{0.844269in}{2.434008in}}%
\pgfpathlineto{\pgfqpoint{0.844586in}{2.224081in}}%
\pgfpathlineto{\pgfqpoint{0.844904in}{3.071034in}}%
\pgfpathlineto{\pgfqpoint{0.845539in}{2.112915in}}%
\pgfpathlineto{\pgfqpoint{0.845856in}{2.877764in}}%
\pgfpathlineto{\pgfqpoint{0.846174in}{2.967141in}}%
\pgfpathlineto{\pgfqpoint{0.846491in}{2.135334in}}%
\pgfpathlineto{\pgfqpoint{0.847126in}{3.110678in}}%
\pgfpathlineto{\pgfqpoint{0.847443in}{2.296803in}}%
\pgfpathlineto{\pgfqpoint{0.847761in}{2.333073in}}%
\pgfpathlineto{\pgfqpoint{0.848078in}{3.120911in}}%
\pgfpathlineto{\pgfqpoint{0.848713in}{2.149976in}}%
\pgfpathlineto{\pgfqpoint{0.849031in}{2.989662in}}%
\pgfpathlineto{\pgfqpoint{0.849348in}{2.827744in}}%
\pgfpathlineto{\pgfqpoint{0.849666in}{2.096927in}}%
\pgfpathlineto{\pgfqpoint{0.850300in}{3.041760in}}%
\pgfpathlineto{\pgfqpoint{0.850618in}{2.190200in}}%
\pgfpathlineto{\pgfqpoint{0.851253in}{3.134244in}}%
\pgfpathlineto{\pgfqpoint{0.851570in}{2.407860in}}%
\pgfpathlineto{\pgfqpoint{0.851888in}{2.241836in}}%
\pgfpathlineto{\pgfqpoint{0.852205in}{3.085584in}}%
\pgfpathlineto{\pgfqpoint{0.852840in}{2.115859in}}%
\pgfpathlineto{\pgfqpoint{0.853158in}{2.899158in}}%
\pgfpathlineto{\pgfqpoint{0.853475in}{2.939025in}}%
\pgfpathlineto{\pgfqpoint{0.853793in}{2.120183in}}%
\pgfpathlineto{\pgfqpoint{0.854427in}{3.096116in}}%
\pgfpathlineto{\pgfqpoint{0.854745in}{2.268763in}}%
\pgfpathlineto{\pgfqpoint{0.855062in}{2.357568in}}%
\pgfpathlineto{\pgfqpoint{0.855380in}{3.123193in}}%
\pgfpathlineto{\pgfqpoint{0.856015in}{2.160094in}}%
\pgfpathlineto{\pgfqpoint{0.856332in}{3.008330in}}%
\pgfpathlineto{\pgfqpoint{0.856967in}{2.093448in}}%
\pgfpathlineto{\pgfqpoint{0.857285in}{2.779316in}}%
\pgfpathlineto{\pgfqpoint{0.857602in}{3.021228in}}%
\pgfpathlineto{\pgfqpoint{0.857919in}{2.171692in}}%
\pgfpathlineto{\pgfqpoint{0.858554in}{3.129275in}}%
\pgfpathlineto{\pgfqpoint{0.858872in}{2.379776in}}%
\pgfpathlineto{\pgfqpoint{0.859189in}{2.264154in}}%
\pgfpathlineto{\pgfqpoint{0.859507in}{3.095986in}}%
\pgfpathlineto{\pgfqpoint{0.860142in}{2.117465in}}%
\pgfpathlineto{\pgfqpoint{0.860459in}{2.919903in}}%
\pgfpathlineto{\pgfqpoint{0.860777in}{2.908046in}}%
\pgfpathlineto{\pgfqpoint{0.861094in}{2.107149in}}%
\pgfpathlineto{\pgfqpoint{0.861729in}{3.080485in}}%
\pgfpathlineto{\pgfqpoint{0.862046in}{2.245368in}}%
\pgfpathlineto{\pgfqpoint{0.862364in}{2.379764in}}%
\pgfpathlineto{\pgfqpoint{0.862681in}{3.126340in}}%
\pgfpathlineto{\pgfqpoint{0.863316in}{2.175435in}}%
\pgfpathlineto{\pgfqpoint{0.863634in}{3.024363in}}%
\pgfpathlineto{\pgfqpoint{0.864269in}{2.090110in}}%
\pgfpathlineto{\pgfqpoint{0.864586in}{2.804476in}}%
\pgfpathlineto{\pgfqpoint{0.864904in}{2.998459in}}%
\pgfpathlineto{\pgfqpoint{0.865221in}{2.153062in}}%
\pgfpathlineto{\pgfqpoint{0.865856in}{3.121133in}}%
\pgfpathlineto{\pgfqpoint{0.866173in}{2.355606in}}%
\pgfpathlineto{\pgfqpoint{0.866491in}{2.283332in}}%
\pgfpathlineto{\pgfqpoint{0.866808in}{3.106698in}}%
\pgfpathlineto{\pgfqpoint{0.867443in}{2.123092in}}%
\pgfpathlineto{\pgfqpoint{0.867761in}{2.938953in}}%
\pgfpathlineto{\pgfqpoint{0.868078in}{2.878260in}}%
\pgfpathlineto{\pgfqpoint{0.868396in}{2.096078in}}%
\pgfpathlineto{\pgfqpoint{0.869030in}{3.064140in}}%
\pgfpathlineto{\pgfqpoint{0.869348in}{2.220835in}}%
\pgfpathlineto{\pgfqpoint{0.869983in}{3.125985in}}%
\pgfpathlineto{\pgfqpoint{0.870300in}{2.458309in}}%
\pgfpathlineto{\pgfqpoint{0.870618in}{2.188326in}}%
\pgfpathlineto{\pgfqpoint{0.870935in}{3.040486in}}%
\pgfpathlineto{\pgfqpoint{0.871570in}{2.090306in}}%
\pgfpathlineto{\pgfqpoint{0.871888in}{2.829630in}}%
\pgfpathlineto{\pgfqpoint{0.872205in}{2.975343in}}%
\pgfpathlineto{\pgfqpoint{0.872523in}{2.137033in}}%
\pgfpathlineto{\pgfqpoint{0.873157in}{3.114125in}}%
\pgfpathlineto{\pgfqpoint{0.873475in}{2.327301in}}%
\pgfpathlineto{\pgfqpoint{0.873792in}{2.306728in}}%
\pgfpathlineto{\pgfqpoint{0.874110in}{3.112625in}}%
\pgfpathlineto{\pgfqpoint{0.874745in}{2.127620in}}%
\pgfpathlineto{\pgfqpoint{0.875062in}{2.958171in}}%
\pgfpathlineto{\pgfqpoint{0.875380in}{2.845483in}}%
\pgfpathlineto{\pgfqpoint{0.875697in}{2.087968in}}%
\pgfpathlineto{\pgfqpoint{0.876332in}{3.046663in}}%
\pgfpathlineto{\pgfqpoint{0.876649in}{2.199893in}}%
\pgfpathlineto{\pgfqpoint{0.877284in}{3.125488in}}%
\pgfpathlineto{\pgfqpoint{0.877602in}{2.424781in}}%
\pgfpathlineto{\pgfqpoint{0.877919in}{2.206455in}}%
\pgfpathlineto{\pgfqpoint{0.878237in}{3.053145in}}%
\pgfpathlineto{\pgfqpoint{0.878872in}{2.090213in}}%
\pgfpathlineto{\pgfqpoint{0.879189in}{2.854231in}}%
\pgfpathlineto{\pgfqpoint{0.879507in}{2.949585in}}%
\pgfpathlineto{\pgfqpoint{0.879824in}{2.122735in}}%
\pgfpathlineto{\pgfqpoint{0.880459in}{3.105556in}}%
\pgfpathlineto{\pgfqpoint{0.880776in}{2.301995in}}%
\pgfpathlineto{\pgfqpoint{0.881094in}{2.326948in}}%
\pgfpathlineto{\pgfqpoint{0.881411in}{3.118195in}}%
\pgfpathlineto{\pgfqpoint{0.882046in}{2.137494in}}%
\pgfpathlineto{\pgfqpoint{0.882364in}{2.975416in}}%
\pgfpathlineto{\pgfqpoint{0.882681in}{2.814982in}}%
\pgfpathlineto{\pgfqpoint{0.882999in}{2.081135in}}%
\pgfpathlineto{\pgfqpoint{0.883634in}{3.027592in}}%
\pgfpathlineto{\pgfqpoint{0.883951in}{2.178202in}}%
\pgfpathlineto{\pgfqpoint{0.884586in}{3.121240in}}%
\pgfpathlineto{\pgfqpoint{0.884903in}{2.395296in}}%
\pgfpathlineto{\pgfqpoint{0.885221in}{2.222071in}}%
\pgfpathlineto{\pgfqpoint{0.885538in}{3.066124in}}%
\pgfpathlineto{\pgfqpoint{0.886173in}{2.094566in}}%
\pgfpathlineto{\pgfqpoint{0.886491in}{2.878137in}}%
\pgfpathlineto{\pgfqpoint{0.886808in}{2.924411in}}%
\pgfpathlineto{\pgfqpoint{0.887126in}{2.112504in}}%
\pgfpathlineto{\pgfqpoint{0.887760in}{3.095567in}}%
\pgfpathlineto{\pgfqpoint{0.888078in}{2.272779in}}%
\pgfpathlineto{\pgfqpoint{0.888395in}{2.351467in}}%
\pgfpathlineto{\pgfqpoint{0.888713in}{3.119361in}}%
\pgfpathlineto{\pgfqpoint{0.889348in}{2.146068in}}%
\pgfpathlineto{\pgfqpoint{0.889665in}{2.993282in}}%
\pgfpathlineto{\pgfqpoint{0.890300in}{2.077810in}}%
\pgfpathlineto{\pgfqpoint{0.890618in}{2.762270in}}%
\pgfpathlineto{\pgfqpoint{0.890935in}{3.007522in}}%
\pgfpathlineto{\pgfqpoint{0.891253in}{2.159341in}}%
\pgfpathlineto{\pgfqpoint{0.891887in}{3.116395in}}%
\pgfpathlineto{\pgfqpoint{0.892205in}{2.362477in}}%
\pgfpathlineto{\pgfqpoint{0.892522in}{2.243254in}}%
\pgfpathlineto{\pgfqpoint{0.892840in}{3.075665in}}%
\pgfpathlineto{\pgfqpoint{0.893475in}{2.098368in}}%
\pgfpathlineto{\pgfqpoint{0.893792in}{2.901894in}}%
\pgfpathlineto{\pgfqpoint{0.894110in}{2.899188in}}%
\pgfpathlineto{\pgfqpoint{0.894427in}{2.104226in}}%
\pgfpathlineto{\pgfqpoint{0.895062in}{3.081955in}}%
\pgfpathlineto{\pgfqpoint{0.895379in}{2.246069in}}%
\pgfpathlineto{\pgfqpoint{0.895697in}{2.372814in}}%
\pgfpathlineto{\pgfqpoint{0.896014in}{3.120027in}}%
\pgfpathlineto{\pgfqpoint{0.896649in}{2.160443in}}%
\pgfpathlineto{\pgfqpoint{0.896967in}{3.008711in}}%
\pgfpathlineto{\pgfqpoint{0.897602in}{2.074836in}}%
\pgfpathlineto{\pgfqpoint{0.897919in}{2.789209in}}%
\pgfpathlineto{\pgfqpoint{0.898237in}{2.984979in}}%
\pgfpathlineto{\pgfqpoint{0.898554in}{2.140459in}}%
\pgfpathlineto{\pgfqpoint{0.899189in}{3.108138in}}%
\pgfpathlineto{\pgfqpoint{0.899506in}{2.333867in}}%
\pgfpathlineto{\pgfqpoint{0.899824in}{2.261983in}}%
\pgfpathlineto{\pgfqpoint{0.900141in}{3.085754in}}%
\pgfpathlineto{\pgfqpoint{0.900776in}{2.109640in}}%
\pgfpathlineto{\pgfqpoint{0.901094in}{2.925901in}}%
\pgfpathlineto{\pgfqpoint{0.901411in}{2.875250in}}%
\pgfpathlineto{\pgfqpoint{0.901729in}{2.096071in}}%
\pgfpathlineto{\pgfqpoint{0.902364in}{3.064648in}}%
\pgfpathlineto{\pgfqpoint{0.902681in}{2.216804in}}%
\pgfpathlineto{\pgfqpoint{0.903316in}{3.116754in}}%
\pgfpathlineto{\pgfqpoint{0.903633in}{2.444253in}}%
\pgfpathlineto{\pgfqpoint{0.903951in}{2.172991in}}%
\pgfpathlineto{\pgfqpoint{0.904268in}{3.024778in}}%
\pgfpathlineto{\pgfqpoint{0.904903in}{2.075830in}}%
\pgfpathlineto{\pgfqpoint{0.905221in}{2.815880in}}%
\pgfpathlineto{\pgfqpoint{0.905538in}{2.961788in}}%
\pgfpathlineto{\pgfqpoint{0.905856in}{2.123862in}}%
\pgfpathlineto{\pgfqpoint{0.906490in}{3.098994in}}%
\pgfpathlineto{\pgfqpoint{0.906808in}{2.302746in}}%
\pgfpathlineto{\pgfqpoint{0.907125in}{2.286182in}}%
\pgfpathlineto{\pgfqpoint{0.907443in}{3.092128in}}%
\pgfpathlineto{\pgfqpoint{0.908078in}{2.120060in}}%
\pgfpathlineto{\pgfqpoint{0.908395in}{2.950324in}}%
\pgfpathlineto{\pgfqpoint{0.908713in}{2.845669in}}%
\pgfpathlineto{\pgfqpoint{0.909030in}{2.087571in}}%
\pgfpathlineto{\pgfqpoint{0.909665in}{3.044055in}}%
\pgfpathlineto{\pgfqpoint{0.909983in}{2.191303in}}%
\pgfpathlineto{\pgfqpoint{0.910617in}{3.113023in}}%
\pgfpathlineto{\pgfqpoint{0.910935in}{2.408133in}}%
\pgfpathlineto{\pgfqpoint{0.911252in}{2.191650in}}%
\pgfpathlineto{\pgfqpoint{0.911570in}{3.037956in}}%
\pgfpathlineto{\pgfqpoint{0.912205in}{2.076439in}}%
\pgfpathlineto{\pgfqpoint{0.912522in}{2.842065in}}%
\pgfpathlineto{\pgfqpoint{0.912840in}{2.935390in}}%
\pgfpathlineto{\pgfqpoint{0.913157in}{2.108280in}}%
\pgfpathlineto{\pgfqpoint{0.913792in}{3.087041in}}%
\pgfpathlineto{\pgfqpoint{0.914109in}{2.275715in}}%
\pgfpathlineto{\pgfqpoint{0.914427in}{2.307850in}}%
\pgfpathlineto{\pgfqpoint{0.914744in}{3.098793in}}%
\pgfpathlineto{\pgfqpoint{0.915379in}{2.134272in}}%
\pgfpathlineto{\pgfqpoint{0.915697in}{2.971068in}}%
\pgfpathlineto{\pgfqpoint{0.916014in}{2.815613in}}%
\pgfpathlineto{\pgfqpoint{0.916332in}{2.078643in}}%
\pgfpathlineto{\pgfqpoint{0.916967in}{3.020786in}}%
\pgfpathlineto{\pgfqpoint{0.917284in}{2.165636in}}%
\pgfpathlineto{\pgfqpoint{0.917919in}{3.106477in}}%
\pgfpathlineto{\pgfqpoint{0.918236in}{2.377162in}}%
\pgfpathlineto{\pgfqpoint{0.918554in}{2.208278in}}%
\pgfpathlineto{\pgfqpoint{0.918871in}{3.051803in}}%
\pgfpathlineto{\pgfqpoint{0.919506in}{2.081561in}}%
\pgfpathlineto{\pgfqpoint{0.919824in}{2.866988in}}%
\pgfpathlineto{\pgfqpoint{0.920141in}{2.909133in}}%
\pgfpathlineto{\pgfqpoint{0.920459in}{2.094540in}}%
\pgfpathlineto{\pgfqpoint{0.921094in}{3.073829in}}%
\pgfpathlineto{\pgfqpoint{0.921411in}{2.246839in}}%
\pgfpathlineto{\pgfqpoint{0.921728in}{2.335694in}}%
\pgfpathlineto{\pgfqpoint{0.922046in}{3.102860in}}%
\pgfpathlineto{\pgfqpoint{0.922681in}{2.145683in}}%
\pgfpathlineto{\pgfqpoint{0.922998in}{2.991193in}}%
\pgfpathlineto{\pgfqpoint{0.923633in}{2.072625in}}%
\pgfpathlineto{\pgfqpoint{0.923951in}{2.756667in}}%
\pgfpathlineto{\pgfqpoint{0.924268in}{2.996400in}}%
\pgfpathlineto{\pgfqpoint{0.924586in}{2.143407in}}%
\pgfpathlineto{\pgfqpoint{0.925220in}{3.100125in}}%
\pgfpathlineto{\pgfqpoint{0.925538in}{2.343649in}}%
\pgfpathlineto{\pgfqpoint{0.925855in}{2.230969in}}%
\pgfpathlineto{\pgfqpoint{0.926173in}{3.062360in}}%
\pgfpathlineto{\pgfqpoint{0.926808in}{2.085761in}}%
\pgfpathlineto{\pgfqpoint{0.927125in}{2.891757in}}%
\pgfpathlineto{\pgfqpoint{0.927443in}{2.879223in}}%
\pgfpathlineto{\pgfqpoint{0.927760in}{2.082889in}}%
\pgfpathlineto{\pgfqpoint{0.928395in}{3.058424in}}%
\pgfpathlineto{\pgfqpoint{0.928713in}{2.222069in}}%
\pgfpathlineto{\pgfqpoint{0.929030in}{2.362456in}}%
\pgfpathlineto{\pgfqpoint{0.929347in}{3.108020in}}%
\pgfpathlineto{\pgfqpoint{0.929982in}{2.161108in}}%
\pgfpathlineto{\pgfqpoint{0.930300in}{3.007136in}}%
\pgfpathlineto{\pgfqpoint{0.930935in}{2.066635in}}%
\pgfpathlineto{\pgfqpoint{0.931252in}{2.782583in}}%
\pgfpathlineto{\pgfqpoint{0.931570in}{2.969642in}}%
\pgfpathlineto{\pgfqpoint{0.931887in}{2.122784in}}%
\pgfpathlineto{\pgfqpoint{0.932522in}{3.091391in}}%
\pgfpathlineto{\pgfqpoint{0.932839in}{2.314614in}}%
\pgfpathlineto{\pgfqpoint{0.933157in}{2.251475in}}%
\pgfpathlineto{\pgfqpoint{0.933474in}{3.073442in}}%
\pgfpathlineto{\pgfqpoint{0.934109in}{2.094995in}}%
\pgfpathlineto{\pgfqpoint{0.934427in}{2.914420in}}%
\pgfpathlineto{\pgfqpoint{0.934744in}{2.850293in}}%
\pgfpathlineto{\pgfqpoint{0.935062in}{2.072512in}}%
\pgfpathlineto{\pgfqpoint{0.935697in}{3.041137in}}%
\pgfpathlineto{\pgfqpoint{0.936014in}{2.197853in}}%
\pgfpathlineto{\pgfqpoint{0.936649in}{3.108586in}}%
\pgfpathlineto{\pgfqpoint{0.936966in}{2.440979in}}%
\pgfpathlineto{\pgfqpoint{0.937284in}{2.174129in}}%
\pgfpathlineto{\pgfqpoint{0.937601in}{3.023322in}}%
\pgfpathlineto{\pgfqpoint{0.938236in}{2.065440in}}%
\pgfpathlineto{\pgfqpoint{0.938554in}{2.808388in}}%
\pgfpathlineto{\pgfqpoint{0.938871in}{2.943503in}}%
\pgfpathlineto{\pgfqpoint{0.939189in}{2.105290in}}%
\pgfpathlineto{\pgfqpoint{0.939824in}{3.082026in}}%
\pgfpathlineto{\pgfqpoint{0.940141in}{2.283574in}}%
\pgfpathlineto{\pgfqpoint{0.940458in}{2.277275in}}%
\pgfpathlineto{\pgfqpoint{0.940776in}{3.080384in}}%
\pgfpathlineto{\pgfqpoint{0.941411in}{2.102731in}}%
\pgfpathlineto{\pgfqpoint{0.941728in}{2.937126in}}%
\pgfpathlineto{\pgfqpoint{0.942046in}{2.817360in}}%
\pgfpathlineto{\pgfqpoint{0.942363in}{2.065235in}}%
\pgfpathlineto{\pgfqpoint{0.942998in}{3.022212in}}%
\pgfpathlineto{\pgfqpoint{0.943316in}{2.177848in}}%
\pgfpathlineto{\pgfqpoint{0.943950in}{3.108489in}}%
\pgfpathlineto{\pgfqpoint{0.944268in}{2.404695in}}%
\pgfpathlineto{\pgfqpoint{0.944585in}{2.192310in}}%
\pgfpathlineto{\pgfqpoint{0.944903in}{3.035481in}}%
\pgfpathlineto{\pgfqpoint{0.945538in}{2.064281in}}%
\pgfpathlineto{\pgfqpoint{0.945855in}{2.834285in}}%
\pgfpathlineto{\pgfqpoint{0.946173in}{2.914725in}}%
\pgfpathlineto{\pgfqpoint{0.946490in}{2.090022in}}%
\pgfpathlineto{\pgfqpoint{0.947125in}{3.070195in}}%
\pgfpathlineto{\pgfqpoint{0.947443in}{2.256433in}}%
\pgfpathlineto{\pgfqpoint{0.947760in}{2.300577in}}%
\pgfpathlineto{\pgfqpoint{0.948077in}{3.087308in}}%
\pgfpathlineto{\pgfqpoint{0.948712in}{2.115942in}}%
\pgfpathlineto{\pgfqpoint{0.949030in}{2.957044in}}%
\pgfpathlineto{\pgfqpoint{0.949665in}{2.058604in}}%
\pgfpathlineto{\pgfqpoint{0.949982in}{2.720790in}}%
\pgfpathlineto{\pgfqpoint{0.950300in}{3.000994in}}%
\pgfpathlineto{\pgfqpoint{0.950617in}{2.157235in}}%
\pgfpathlineto{\pgfqpoint{0.951252in}{3.103940in}}%
\pgfpathlineto{\pgfqpoint{0.951569in}{2.371937in}}%
\pgfpathlineto{\pgfqpoint{0.951887in}{2.208392in}}%
\pgfpathlineto{\pgfqpoint{0.952204in}{3.048016in}}%
\pgfpathlineto{\pgfqpoint{0.952839in}{2.068907in}}%
\pgfpathlineto{\pgfqpoint{0.953157in}{2.859067in}}%
\pgfpathlineto{\pgfqpoint{0.953474in}{2.887004in}}%
\pgfpathlineto{\pgfqpoint{0.953792in}{2.076751in}}%
\pgfpathlineto{\pgfqpoint{0.954427in}{3.056866in}}%
\pgfpathlineto{\pgfqpoint{0.954744in}{2.227769in}}%
\pgfpathlineto{\pgfqpoint{0.955061in}{2.328684in}}%
\pgfpathlineto{\pgfqpoint{0.955379in}{3.090035in}}%
\pgfpathlineto{\pgfqpoint{0.956014in}{2.127392in}}%
\pgfpathlineto{\pgfqpoint{0.956331in}{2.977294in}}%
\pgfpathlineto{\pgfqpoint{0.956966in}{2.055904in}}%
\pgfpathlineto{\pgfqpoint{0.957284in}{2.750480in}}%
\pgfpathlineto{\pgfqpoint{0.957601in}{2.980815in}}%
\pgfpathlineto{\pgfqpoint{0.957919in}{2.139149in}}%
\pgfpathlineto{\pgfqpoint{0.958554in}{3.098025in}}%
\pgfpathlineto{\pgfqpoint{0.958871in}{2.335548in}}%
\pgfpathlineto{\pgfqpoint{0.959188in}{2.229834in}}%
\pgfpathlineto{\pgfqpoint{0.959506in}{3.056483in}}%
\pgfpathlineto{\pgfqpoint{0.960141in}{2.072810in}}%
\pgfpathlineto{\pgfqpoint{0.960458in}{2.884094in}}%
\pgfpathlineto{\pgfqpoint{0.960776in}{2.855961in}}%
\pgfpathlineto{\pgfqpoint{0.961093in}{2.066318in}}%
\pgfpathlineto{\pgfqpoint{0.961728in}{3.041334in}}%
\pgfpathlineto{\pgfqpoint{0.962046in}{2.202402in}}%
\pgfpathlineto{\pgfqpoint{0.962363in}{2.354407in}}%
\pgfpathlineto{\pgfqpoint{0.962680in}{3.092408in}}%
\pgfpathlineto{\pgfqpoint{0.963315in}{2.144723in}}%
\pgfpathlineto{\pgfqpoint{0.963633in}{2.994287in}}%
\pgfpathlineto{\pgfqpoint{0.964268in}{2.054773in}}%
\pgfpathlineto{\pgfqpoint{0.964585in}{2.781702in}}%
\pgfpathlineto{\pgfqpoint{0.964903in}{2.958469in}}%
\pgfpathlineto{\pgfqpoint{0.965220in}{2.120018in}}%
\pgfpathlineto{\pgfqpoint{0.965855in}{3.088169in}}%
\pgfpathlineto{\pgfqpoint{0.966173in}{2.303170in}}%
\pgfpathlineto{\pgfqpoint{0.966490in}{2.249382in}}%
\pgfpathlineto{\pgfqpoint{0.966807in}{3.065466in}}%
\pgfpathlineto{\pgfqpoint{0.967442in}{2.082765in}}%
\pgfpathlineto{\pgfqpoint{0.967760in}{2.907146in}}%
\pgfpathlineto{\pgfqpoint{0.968077in}{2.826394in}}%
\pgfpathlineto{\pgfqpoint{0.968395in}{2.056788in}}%
\pgfpathlineto{\pgfqpoint{0.969030in}{3.023354in}}%
\pgfpathlineto{\pgfqpoint{0.969347in}{2.176269in}}%
\pgfpathlineto{\pgfqpoint{0.969982in}{3.090791in}}%
\pgfpathlineto{\pgfqpoint{0.970299in}{2.406772in}}%
\pgfpathlineto{\pgfqpoint{0.970617in}{2.160148in}}%
\pgfpathlineto{\pgfqpoint{0.970934in}{3.011696in}}%
\pgfpathlineto{\pgfqpoint{0.971569in}{2.059729in}}%
\pgfpathlineto{\pgfqpoint{0.971887in}{2.811364in}}%
\pgfpathlineto{\pgfqpoint{0.972204in}{2.934684in}}%
\pgfpathlineto{\pgfqpoint{0.972522in}{2.103299in}}%
\pgfpathlineto{\pgfqpoint{0.973157in}{3.076162in}}%
\pgfpathlineto{\pgfqpoint{0.973474in}{2.268365in}}%
\pgfpathlineto{\pgfqpoint{0.973791in}{2.274355in}}%
\pgfpathlineto{\pgfqpoint{0.974109in}{3.070480in}}%
\pgfpathlineto{\pgfqpoint{0.974744in}{2.091493in}}%
\pgfpathlineto{\pgfqpoint{0.975061in}{2.930586in}}%
\pgfpathlineto{\pgfqpoint{0.975379in}{2.792833in}}%
\pgfpathlineto{\pgfqpoint{0.975696in}{2.050750in}}%
\pgfpathlineto{\pgfqpoint{0.976331in}{3.003713in}}%
\pgfpathlineto{\pgfqpoint{0.976649in}{2.153029in}}%
\pgfpathlineto{\pgfqpoint{0.977284in}{3.088502in}}%
\pgfpathlineto{\pgfqpoint{0.977601in}{2.370500in}}%
\pgfpathlineto{\pgfqpoint{0.977918in}{2.181431in}}%
\pgfpathlineto{\pgfqpoint{0.978236in}{3.025297in}}%
\pgfpathlineto{\pgfqpoint{0.978871in}{2.064631in}}%
\pgfpathlineto{\pgfqpoint{0.979188in}{2.840676in}}%
\pgfpathlineto{\pgfqpoint{0.979506in}{2.906450in}}%
\pgfpathlineto{\pgfqpoint{0.979823in}{2.087314in}}%
\pgfpathlineto{\pgfqpoint{0.980458in}{3.060746in}}%
\pgfpathlineto{\pgfqpoint{0.980776in}{2.237373in}}%
\pgfpathlineto{\pgfqpoint{0.981093in}{2.297332in}}%
\pgfpathlineto{\pgfqpoint{0.981410in}{3.075636in}}%
\pgfpathlineto{\pgfqpoint{0.982045in}{2.106343in}}%
\pgfpathlineto{\pgfqpoint{0.982363in}{2.951114in}}%
\pgfpathlineto{\pgfqpoint{0.982998in}{2.045006in}}%
\pgfpathlineto{\pgfqpoint{0.983315in}{2.720110in}}%
\pgfpathlineto{\pgfqpoint{0.983633in}{2.981007in}}%
\pgfpathlineto{\pgfqpoint{0.983950in}{2.130218in}}%
\pgfpathlineto{\pgfqpoint{0.984585in}{3.082642in}}%
\pgfpathlineto{\pgfqpoint{0.984903in}{2.338352in}}%
\pgfpathlineto{\pgfqpoint{0.985220in}{2.201063in}}%
\pgfpathlineto{\pgfqpoint{0.985537in}{3.039392in}}%
\pgfpathlineto{\pgfqpoint{0.986172in}{2.073387in}}%
\pgfpathlineto{\pgfqpoint{0.986490in}{2.866884in}}%
\pgfpathlineto{\pgfqpoint{0.986807in}{2.877743in}}%
\pgfpathlineto{\pgfqpoint{0.987125in}{2.071946in}}%
\pgfpathlineto{\pgfqpoint{0.987760in}{3.043352in}}%
\pgfpathlineto{\pgfqpoint{0.988077in}{2.205751in}}%
\pgfpathlineto{\pgfqpoint{0.988395in}{2.325485in}}%
\pgfpathlineto{\pgfqpoint{0.988712in}{3.077012in}}%
\pgfpathlineto{\pgfqpoint{0.989347in}{2.119683in}}%
\pgfpathlineto{\pgfqpoint{0.989664in}{2.972122in}}%
\pgfpathlineto{\pgfqpoint{0.990299in}{2.043542in}}%
\pgfpathlineto{\pgfqpoint{0.990617in}{2.750049in}}%
\pgfpathlineto{\pgfqpoint{0.990934in}{2.957438in}}%
\pgfpathlineto{\pgfqpoint{0.991252in}{2.109882in}}%
\pgfpathlineto{\pgfqpoint{0.991887in}{3.075707in}}%
\pgfpathlineto{\pgfqpoint{0.992204in}{2.303860in}}%
\pgfpathlineto{\pgfqpoint{0.992522in}{2.228843in}}%
\pgfpathlineto{\pgfqpoint{0.992839in}{3.050588in}}%
\pgfpathlineto{\pgfqpoint{0.993474in}{2.080139in}}%
\pgfpathlineto{\pgfqpoint{0.993791in}{2.892607in}}%
\pgfpathlineto{\pgfqpoint{0.994109in}{2.844078in}}%
\pgfpathlineto{\pgfqpoint{0.994426in}{2.058859in}}%
\pgfpathlineto{\pgfqpoint{0.995061in}{3.024195in}}%
\pgfpathlineto{\pgfqpoint{0.995379in}{2.177806in}}%
\pgfpathlineto{\pgfqpoint{0.996014in}{3.078404in}}%
\pgfpathlineto{\pgfqpoint{0.996331in}{2.411465in}}%
\pgfpathlineto{\pgfqpoint{0.996648in}{2.139046in}}%
\pgfpathlineto{\pgfqpoint{0.996966in}{2.989448in}}%
\pgfpathlineto{\pgfqpoint{0.997601in}{2.041753in}}%
\pgfpathlineto{\pgfqpoint{0.997918in}{2.779667in}}%
\pgfpathlineto{\pgfqpoint{0.998236in}{2.930351in}}%
\pgfpathlineto{\pgfqpoint{0.998553in}{2.091185in}}%
\pgfpathlineto{\pgfqpoint{0.999188in}{3.065676in}}%
\pgfpathlineto{\pgfqpoint{0.999506in}{2.273657in}}%
\pgfpathlineto{\pgfqpoint{0.999823in}{2.254722in}}%
\pgfpathlineto{\pgfqpoint{1.000140in}{3.062080in}}%
\pgfpathlineto{\pgfqpoint{1.000775in}{2.090894in}}%
\pgfpathlineto{\pgfqpoint{1.001093in}{2.914836in}}%
\pgfpathlineto{\pgfqpoint{1.001410in}{2.811352in}}%
\pgfpathlineto{\pgfqpoint{1.001728in}{2.046898in}}%
\pgfpathlineto{\pgfqpoint{1.002363in}{3.002908in}}%
\pgfpathlineto{\pgfqpoint{1.002680in}{2.150844in}}%
\pgfpathlineto{\pgfqpoint{1.003315in}{3.076269in}}%
\pgfpathlineto{\pgfqpoint{1.003633in}{2.377961in}}%
\pgfpathlineto{\pgfqpoint{1.003950in}{2.156620in}}%
\pgfpathlineto{\pgfqpoint{1.004267in}{3.007144in}}%
\pgfpathlineto{\pgfqpoint{1.004902in}{2.044958in}}%
\pgfpathlineto{\pgfqpoint{1.005220in}{2.807803in}}%
\pgfpathlineto{\pgfqpoint{1.005537in}{2.903304in}}%
\pgfpathlineto{\pgfqpoint{1.005855in}{2.074449in}}%
\pgfpathlineto{\pgfqpoint{1.006490in}{3.053849in}}%
\pgfpathlineto{\pgfqpoint{1.006807in}{2.244199in}}%
\pgfpathlineto{\pgfqpoint{1.007125in}{2.284983in}}%
\pgfpathlineto{\pgfqpoint{1.007442in}{3.069429in}}%
\pgfpathlineto{\pgfqpoint{1.008077in}{2.098935in}}%
\pgfpathlineto{\pgfqpoint{1.008394in}{2.936709in}}%
\pgfpathlineto{\pgfqpoint{1.008712in}{2.773433in}}%
\pgfpathlineto{\pgfqpoint{1.009029in}{2.038639in}}%
\pgfpathlineto{\pgfqpoint{1.009664in}{2.980522in}}%
\pgfpathlineto{\pgfqpoint{1.009982in}{2.127073in}}%
\pgfpathlineto{\pgfqpoint{1.010617in}{3.073418in}}%
\pgfpathlineto{\pgfqpoint{1.010934in}{2.341947in}}%
\pgfpathlineto{\pgfqpoint{1.011252in}{2.179891in}}%
\pgfpathlineto{\pgfqpoint{1.011569in}{3.020580in}}%
\pgfpathlineto{\pgfqpoint{1.012204in}{2.047395in}}%
\pgfpathlineto{\pgfqpoint{1.012521in}{2.836001in}}%
\pgfpathlineto{\pgfqpoint{1.012839in}{2.872387in}}%
\pgfpathlineto{\pgfqpoint{1.013156in}{2.060411in}}%
\pgfpathlineto{\pgfqpoint{1.013791in}{3.040357in}}%
\pgfpathlineto{\pgfqpoint{1.014109in}{2.218599in}}%
\pgfpathlineto{\pgfqpoint{1.014426in}{2.312280in}}%
\pgfpathlineto{\pgfqpoint{1.014744in}{3.075605in}}%
\pgfpathlineto{\pgfqpoint{1.015378in}{2.112189in}}%
\pgfpathlineto{\pgfqpoint{1.015696in}{2.955070in}}%
\pgfpathlineto{\pgfqpoint{1.016331in}{2.031096in}}%
\pgfpathlineto{\pgfqpoint{1.016648in}{2.722512in}}%
\pgfpathlineto{\pgfqpoint{1.016966in}{2.955361in}}%
\pgfpathlineto{\pgfqpoint{1.017283in}{2.105083in}}%
\pgfpathlineto{\pgfqpoint{1.017918in}{3.067071in}}%
\pgfpathlineto{\pgfqpoint{1.018236in}{2.309719in}}%
\pgfpathlineto{\pgfqpoint{1.018553in}{2.200942in}}%
\pgfpathlineto{\pgfqpoint{1.018870in}{3.033957in}}%
\pgfpathlineto{\pgfqpoint{1.019505in}{2.055482in}}%
\pgfpathlineto{\pgfqpoint{1.019823in}{2.861928in}}%
\pgfpathlineto{\pgfqpoint{1.020140in}{2.842448in}}%
\pgfpathlineto{\pgfqpoint{1.020458in}{2.047709in}}%
\pgfpathlineto{\pgfqpoint{1.021093in}{3.026246in}}%
\pgfpathlineto{\pgfqpoint{1.021410in}{2.191894in}}%
\pgfpathlineto{\pgfqpoint{1.021728in}{2.342384in}}%
\pgfpathlineto{\pgfqpoint{1.022045in}{3.076442in}}%
\pgfpathlineto{\pgfqpoint{1.022680in}{2.123109in}}%
\pgfpathlineto{\pgfqpoint{1.022997in}{2.973475in}}%
\pgfpathlineto{\pgfqpoint{1.023632in}{2.029094in}}%
\pgfpathlineto{\pgfqpoint{1.023950in}{2.752541in}}%
\pgfpathlineto{\pgfqpoint{1.024267in}{2.930148in}}%
\pgfpathlineto{\pgfqpoint{1.024585in}{2.085721in}}%
\pgfpathlineto{\pgfqpoint{1.025220in}{3.059216in}}%
\pgfpathlineto{\pgfqpoint{1.025537in}{2.275456in}}%
\pgfpathlineto{\pgfqpoint{1.025855in}{2.227329in}}%
\pgfpathlineto{\pgfqpoint{1.026172in}{3.042913in}}%
\pgfpathlineto{\pgfqpoint{1.026807in}{2.062593in}}%
\pgfpathlineto{\pgfqpoint{1.027124in}{2.888217in}}%
\pgfpathlineto{\pgfqpoint{1.027442in}{2.808187in}}%
\pgfpathlineto{\pgfqpoint{1.027759in}{2.041280in}}%
\pgfpathlineto{\pgfqpoint{1.028394in}{3.010048in}}%
\pgfpathlineto{\pgfqpoint{1.028712in}{2.166542in}}%
\pgfpathlineto{\pgfqpoint{1.029347in}{3.075471in}}%
\pgfpathlineto{\pgfqpoint{1.029664in}{2.388974in}}%
\pgfpathlineto{\pgfqpoint{1.029982in}{2.140620in}}%
\pgfpathlineto{\pgfqpoint{1.030299in}{2.988337in}}%
\pgfpathlineto{\pgfqpoint{1.030934in}{2.027673in}}%
\pgfpathlineto{\pgfqpoint{1.031251in}{2.782854in}}%
\pgfpathlineto{\pgfqpoint{1.031569in}{2.901631in}}%
\pgfpathlineto{\pgfqpoint{1.031886in}{2.068935in}}%
\pgfpathlineto{\pgfqpoint{1.032521in}{3.048241in}}%
\pgfpathlineto{\pgfqpoint{1.032839in}{2.244627in}}%
\pgfpathlineto{\pgfqpoint{1.033156in}{2.251734in}}%
\pgfpathlineto{\pgfqpoint{1.033474in}{3.051666in}}%
\pgfpathlineto{\pgfqpoint{1.034108in}{2.075798in}}%
\pgfpathlineto{\pgfqpoint{1.034426in}{2.911465in}}%
\pgfpathlineto{\pgfqpoint{1.034743in}{2.776902in}}%
\pgfpathlineto{\pgfqpoint{1.035061in}{2.037928in}}%
\pgfpathlineto{\pgfqpoint{1.035696in}{2.989495in}}%
\pgfpathlineto{\pgfqpoint{1.036013in}{2.139250in}}%
\pgfpathlineto{\pgfqpoint{1.036648in}{3.069765in}}%
\pgfpathlineto{\pgfqpoint{1.036966in}{2.351147in}}%
\pgfpathlineto{\pgfqpoint{1.037283in}{2.156864in}}%
\pgfpathlineto{\pgfqpoint{1.037600in}{3.003721in}}%
\pgfpathlineto{\pgfqpoint{1.038235in}{2.032745in}}%
\pgfpathlineto{\pgfqpoint{1.038553in}{2.811672in}}%
\pgfpathlineto{\pgfqpoint{1.038870in}{2.873828in}}%
\pgfpathlineto{\pgfqpoint{1.039188in}{2.053813in}}%
\pgfpathlineto{\pgfqpoint{1.039823in}{3.034904in}}%
\pgfpathlineto{\pgfqpoint{1.040140in}{2.212610in}}%
\pgfpathlineto{\pgfqpoint{1.040458in}{2.281102in}}%
\pgfpathlineto{\pgfqpoint{1.040775in}{3.056117in}}%
\pgfpathlineto{\pgfqpoint{1.041410in}{2.088038in}}%
\pgfpathlineto{\pgfqpoint{1.041727in}{2.936888in}}%
\pgfpathlineto{\pgfqpoint{1.042362in}{2.036862in}}%
\pgfpathlineto{\pgfqpoint{1.042680in}{2.710294in}}%
\pgfpathlineto{\pgfqpoint{1.042997in}{2.965846in}}%
\pgfpathlineto{\pgfqpoint{1.043315in}{2.113783in}}%
\pgfpathlineto{\pgfqpoint{1.043950in}{3.062165in}}%
\pgfpathlineto{\pgfqpoint{1.044267in}{2.311340in}}%
\pgfpathlineto{\pgfqpoint{1.044585in}{2.179962in}}%
\pgfpathlineto{\pgfqpoint{1.044902in}{3.015466in}}%
\pgfpathlineto{\pgfqpoint{1.045537in}{2.037624in}}%
\pgfpathlineto{\pgfqpoint{1.045854in}{2.840914in}}%
\pgfpathlineto{\pgfqpoint{1.046172in}{2.841886in}}%
\pgfpathlineto{\pgfqpoint{1.046489in}{2.041671in}}%
\pgfpathlineto{\pgfqpoint{1.047124in}{3.018995in}}%
\pgfpathlineto{\pgfqpoint{1.047442in}{2.183695in}}%
\pgfpathlineto{\pgfqpoint{1.047759in}{2.308868in}}%
\pgfpathlineto{\pgfqpoint{1.048077in}{3.060060in}}%
\pgfpathlineto{\pgfqpoint{1.048712in}{2.109671in}}%
\pgfpathlineto{\pgfqpoint{1.049029in}{2.959016in}}%
\pgfpathlineto{\pgfqpoint{1.049664in}{2.033860in}}%
\pgfpathlineto{\pgfqpoint{1.049981in}{2.740503in}}%
\pgfpathlineto{\pgfqpoint{1.050299in}{2.937291in}}%
\pgfpathlineto{\pgfqpoint{1.050616in}{2.088880in}}%
\pgfpathlineto{\pgfqpoint{1.051251in}{3.051562in}}%
\pgfpathlineto{\pgfqpoint{1.051569in}{2.276343in}}%
\pgfpathlineto{\pgfqpoint{1.051886in}{2.202295in}}%
\pgfpathlineto{\pgfqpoint{1.052204in}{3.027861in}}%
\pgfpathlineto{\pgfqpoint{1.052838in}{2.048757in}}%
\pgfpathlineto{\pgfqpoint{1.053156in}{2.867253in}}%
\pgfpathlineto{\pgfqpoint{1.053473in}{2.810945in}}%
\pgfpathlineto{\pgfqpoint{1.053791in}{2.030194in}}%
\pgfpathlineto{\pgfqpoint{1.054426in}{3.000046in}}%
\pgfpathlineto{\pgfqpoint{1.054743in}{2.154916in}}%
\pgfpathlineto{\pgfqpoint{1.055378in}{3.059910in}}%
\pgfpathlineto{\pgfqpoint{1.055696in}{2.389554in}}%
\pgfpathlineto{\pgfqpoint{1.056013in}{2.130837in}}%
\pgfpathlineto{\pgfqpoint{1.056330in}{2.980790in}}%
\pgfpathlineto{\pgfqpoint{1.056965in}{2.033350in}}%
\pgfpathlineto{\pgfqpoint{1.057283in}{2.768547in}}%
\pgfpathlineto{\pgfqpoint{1.057600in}{2.908238in}}%
\pgfpathlineto{\pgfqpoint{1.057918in}{2.066252in}}%
\pgfpathlineto{\pgfqpoint{1.058553in}{3.040166in}}%
\pgfpathlineto{\pgfqpoint{1.058870in}{2.240932in}}%
\pgfpathlineto{\pgfqpoint{1.059188in}{2.231109in}}%
\pgfpathlineto{\pgfqpoint{1.059505in}{3.036468in}}%
\pgfpathlineto{\pgfqpoint{1.060140in}{2.058690in}}%
\pgfpathlineto{\pgfqpoint{1.060457in}{2.893767in}}%
\pgfpathlineto{\pgfqpoint{1.060775in}{2.775163in}}%
\pgfpathlineto{\pgfqpoint{1.061092in}{2.022603in}}%
\pgfpathlineto{\pgfqpoint{1.061727in}{2.979459in}}%
\pgfpathlineto{\pgfqpoint{1.062045in}{2.129078in}}%
\pgfpathlineto{\pgfqpoint{1.062680in}{3.059110in}}%
\pgfpathlineto{\pgfqpoint{1.062997in}{2.353869in}}%
\pgfpathlineto{\pgfqpoint{1.063315in}{2.155988in}}%
\pgfpathlineto{\pgfqpoint{1.063632in}{2.996716in}}%
\pgfpathlineto{\pgfqpoint{1.064267in}{2.030893in}}%
\pgfpathlineto{\pgfqpoint{1.064584in}{2.796652in}}%
\pgfpathlineto{\pgfqpoint{1.064902in}{2.875019in}}%
\pgfpathlineto{\pgfqpoint{1.065219in}{2.047276in}}%
\pgfpathlineto{\pgfqpoint{1.065854in}{3.026791in}}%
\pgfpathlineto{\pgfqpoint{1.066172in}{2.209927in}}%
\pgfpathlineto{\pgfqpoint{1.066489in}{2.258688in}}%
\pgfpathlineto{\pgfqpoint{1.066807in}{3.044722in}}%
\pgfpathlineto{\pgfqpoint{1.067442in}{2.074357in}}%
\pgfpathlineto{\pgfqpoint{1.067759in}{2.916383in}}%
\pgfpathlineto{\pgfqpoint{1.068394in}{2.015239in}}%
\pgfpathlineto{\pgfqpoint{1.068711in}{2.681548in}}%
\pgfpathlineto{\pgfqpoint{1.069029in}{2.955531in}}%
\pgfpathlineto{\pgfqpoint{1.069346in}{2.104639in}}%
\pgfpathlineto{\pgfqpoint{1.069981in}{3.056649in}}%
\pgfpathlineto{\pgfqpoint{1.070299in}{2.322295in}}%
\pgfpathlineto{\pgfqpoint{1.070616in}{2.176888in}}%
\pgfpathlineto{\pgfqpoint{1.070934in}{3.011760in}}%
\pgfpathlineto{\pgfqpoint{1.071568in}{2.033235in}}%
\pgfpathlineto{\pgfqpoint{1.071886in}{2.822582in}}%
\pgfpathlineto{\pgfqpoint{1.072203in}{2.842669in}}%
\pgfpathlineto{\pgfqpoint{1.072521in}{2.030595in}}%
\pgfpathlineto{\pgfqpoint{1.073156in}{3.011337in}}%
\pgfpathlineto{\pgfqpoint{1.073473in}{2.179234in}}%
\pgfpathlineto{\pgfqpoint{1.073791in}{2.290735in}}%
\pgfpathlineto{\pgfqpoint{1.074108in}{3.048288in}}%
\pgfpathlineto{\pgfqpoint{1.074743in}{2.088139in}}%
\pgfpathlineto{\pgfqpoint{1.075060in}{2.939141in}}%
\pgfpathlineto{\pgfqpoint{1.075695in}{2.013159in}}%
\pgfpathlineto{\pgfqpoint{1.076013in}{2.714070in}}%
\pgfpathlineto{\pgfqpoint{1.076330in}{2.930887in}}%
\pgfpathlineto{\pgfqpoint{1.076648in}{2.084298in}}%
\pgfpathlineto{\pgfqpoint{1.077283in}{3.052130in}}%
\pgfpathlineto{\pgfqpoint{1.077600in}{2.286861in}}%
\pgfpathlineto{\pgfqpoint{1.077918in}{2.200596in}}%
\pgfpathlineto{\pgfqpoint{1.078235in}{3.020849in}}%
\pgfpathlineto{\pgfqpoint{1.078870in}{2.035325in}}%
\pgfpathlineto{\pgfqpoint{1.079187in}{2.849541in}}%
\pgfpathlineto{\pgfqpoint{1.079505in}{2.806502in}}%
\pgfpathlineto{\pgfqpoint{1.079822in}{2.019292in}}%
\pgfpathlineto{\pgfqpoint{1.080457in}{2.994027in}}%
\pgfpathlineto{\pgfqpoint{1.080775in}{2.151566in}}%
\pgfpathlineto{\pgfqpoint{1.081410in}{3.050546in}}%
\pgfpathlineto{\pgfqpoint{1.081727in}{2.387468in}}%
\pgfpathlineto{\pgfqpoint{1.082045in}{2.108000in}}%
\pgfpathlineto{\pgfqpoint{1.082362in}{2.957866in}}%
\pgfpathlineto{\pgfqpoint{1.082997in}{2.011123in}}%
\pgfpathlineto{\pgfqpoint{1.083314in}{2.750463in}}%
\pgfpathlineto{\pgfqpoint{1.083632in}{2.905412in}}%
\pgfpathlineto{\pgfqpoint{1.083949in}{2.068983in}}%
\pgfpathlineto{\pgfqpoint{1.084584in}{3.042325in}}%
\pgfpathlineto{\pgfqpoint{1.084902in}{2.252378in}}%
\pgfpathlineto{\pgfqpoint{1.085219in}{2.221246in}}%
\pgfpathlineto{\pgfqpoint{1.085537in}{3.029036in}}%
\pgfpathlineto{\pgfqpoint{1.086172in}{2.045087in}}%
\pgfpathlineto{\pgfqpoint{1.086489in}{2.874028in}}%
\pgfpathlineto{\pgfqpoint{1.086806in}{2.772555in}}%
\pgfpathlineto{\pgfqpoint{1.087124in}{2.009609in}}%
\pgfpathlineto{\pgfqpoint{1.087759in}{2.973426in}}%
\pgfpathlineto{\pgfqpoint{1.088076in}{2.124539in}}%
\pgfpathlineto{\pgfqpoint{1.088711in}{3.048478in}}%
\pgfpathlineto{\pgfqpoint{1.089029in}{2.350471in}}%
\pgfpathlineto{\pgfqpoint{1.089346in}{2.126587in}}%
\pgfpathlineto{\pgfqpoint{1.089664in}{2.976869in}}%
\pgfpathlineto{\pgfqpoint{1.090298in}{2.019200in}}%
\pgfpathlineto{\pgfqpoint{1.090616in}{2.785122in}}%
\pgfpathlineto{\pgfqpoint{1.090933in}{2.879896in}}%
\pgfpathlineto{\pgfqpoint{1.091251in}{2.053096in}}%
\pgfpathlineto{\pgfqpoint{1.091886in}{3.027831in}}%
\pgfpathlineto{\pgfqpoint{1.092203in}{2.214247in}}%
\pgfpathlineto{\pgfqpoint{1.092521in}{2.246204in}}%
\pgfpathlineto{\pgfqpoint{1.092838in}{3.032764in}}%
\pgfpathlineto{\pgfqpoint{1.093473in}{2.054942in}}%
\pgfpathlineto{\pgfqpoint{1.093790in}{2.899718in}}%
\pgfpathlineto{\pgfqpoint{1.094425in}{2.005315in}}%
\pgfpathlineto{\pgfqpoint{1.094743in}{2.663343in}}%
\pgfpathlineto{\pgfqpoint{1.095060in}{2.951428in}}%
\pgfpathlineto{\pgfqpoint{1.095378in}{2.099660in}}%
\pgfpathlineto{\pgfqpoint{1.096013in}{3.044760in}}%
\pgfpathlineto{\pgfqpoint{1.096330in}{2.311349in}}%
\pgfpathlineto{\pgfqpoint{1.096648in}{2.151168in}}%
\pgfpathlineto{\pgfqpoint{1.096965in}{2.992433in}}%
\pgfpathlineto{\pgfqpoint{1.097600in}{2.028086in}}%
\pgfpathlineto{\pgfqpoint{1.097917in}{2.817136in}}%
\pgfpathlineto{\pgfqpoint{1.098235in}{2.847013in}}%
\pgfpathlineto{\pgfqpoint{1.098552in}{2.036485in}}%
\pgfpathlineto{\pgfqpoint{1.099187in}{3.008966in}}%
\pgfpathlineto{\pgfqpoint{1.099505in}{2.178669in}}%
\pgfpathlineto{\pgfqpoint{1.099822in}{2.271091in}}%
\pgfpathlineto{\pgfqpoint{1.100140in}{3.036705in}}%
\pgfpathlineto{\pgfqpoint{1.100775in}{2.072785in}}%
\pgfpathlineto{\pgfqpoint{1.101092in}{2.922071in}}%
\pgfpathlineto{\pgfqpoint{1.101727in}{2.001081in}}%
\pgfpathlineto{\pgfqpoint{1.102044in}{2.697862in}}%
\pgfpathlineto{\pgfqpoint{1.102362in}{2.925116in}}%
\pgfpathlineto{\pgfqpoint{1.102679in}{2.076419in}}%
\pgfpathlineto{\pgfqpoint{1.103314in}{3.037490in}}%
\pgfpathlineto{\pgfqpoint{1.103632in}{2.275969in}}%
\pgfpathlineto{\pgfqpoint{1.103949in}{2.178553in}}%
\pgfpathlineto{\pgfqpoint{1.104267in}{3.009993in}}%
\pgfpathlineto{\pgfqpoint{1.104902in}{2.038976in}}%
\pgfpathlineto{\pgfqpoint{1.105219in}{2.842892in}}%
\pgfpathlineto{\pgfqpoint{1.105536in}{2.812694in}}%
\pgfpathlineto{\pgfqpoint{1.105854in}{2.018618in}}%
\pgfpathlineto{\pgfqpoint{1.106489in}{2.987724in}}%
\pgfpathlineto{\pgfqpoint{1.106806in}{2.144962in}}%
\pgfpathlineto{\pgfqpoint{1.107124in}{2.302893in}}%
\pgfpathlineto{\pgfqpoint{1.107441in}{3.037589in}}%
\pgfpathlineto{\pgfqpoint{1.108076in}{2.090460in}}%
\pgfpathlineto{\pgfqpoint{1.108394in}{2.944962in}}%
\pgfpathlineto{\pgfqpoint{1.109028in}{2.002127in}}%
\pgfpathlineto{\pgfqpoint{1.109346in}{2.730400in}}%
\pgfpathlineto{\pgfqpoint{1.109663in}{2.898087in}}%
\pgfpathlineto{\pgfqpoint{1.109981in}{2.054967in}}%
\pgfpathlineto{\pgfqpoint{1.110616in}{3.028043in}}%
\pgfpathlineto{\pgfqpoint{1.110933in}{2.239477in}}%
\pgfpathlineto{\pgfqpoint{1.111251in}{2.212466in}}%
\pgfpathlineto{\pgfqpoint{1.111568in}{3.021954in}}%
\pgfpathlineto{\pgfqpoint{1.112203in}{2.045479in}}%
\pgfpathlineto{\pgfqpoint{1.112520in}{2.867847in}}%
\pgfpathlineto{\pgfqpoint{1.112838in}{2.772678in}}%
\pgfpathlineto{\pgfqpoint{1.113155in}{2.005334in}}%
\pgfpathlineto{\pgfqpoint{1.113790in}{2.966064in}}%
\pgfpathlineto{\pgfqpoint{1.114108in}{2.115910in}}%
\pgfpathlineto{\pgfqpoint{1.114743in}{3.038125in}}%
\pgfpathlineto{\pgfqpoint{1.115060in}{2.343884in}}%
\pgfpathlineto{\pgfqpoint{1.115378in}{2.114478in}}%
\pgfpathlineto{\pgfqpoint{1.115695in}{2.963053in}}%
\pgfpathlineto{\pgfqpoint{1.116330in}{2.002315in}}%
\pgfpathlineto{\pgfqpoint{1.116647in}{2.763107in}}%
\pgfpathlineto{\pgfqpoint{1.116965in}{2.866551in}}%
\pgfpathlineto{\pgfqpoint{1.117282in}{2.036905in}}%
\pgfpathlineto{\pgfqpoint{1.117917in}{3.016284in}}%
\pgfpathlineto{\pgfqpoint{1.118235in}{2.209274in}}%
\pgfpathlineto{\pgfqpoint{1.118552in}{2.243059in}}%
\pgfpathlineto{\pgfqpoint{1.118870in}{3.031613in}}%
\pgfpathlineto{\pgfqpoint{1.119505in}{2.055550in}}%
\pgfpathlineto{\pgfqpoint{1.119822in}{2.889034in}}%
\pgfpathlineto{\pgfqpoint{1.120139in}{2.734324in}}%
\pgfpathlineto{\pgfqpoint{1.120457in}{1.993583in}}%
\pgfpathlineto{\pgfqpoint{1.121092in}{2.941819in}}%
\pgfpathlineto{\pgfqpoint{1.121409in}{2.090131in}}%
\pgfpathlineto{\pgfqpoint{1.122044in}{3.034727in}}%
\pgfpathlineto{\pgfqpoint{1.122362in}{2.307501in}}%
\pgfpathlineto{\pgfqpoint{1.122679in}{2.136210in}}%
\pgfpathlineto{\pgfqpoint{1.122997in}{2.980472in}}%
\pgfpathlineto{\pgfqpoint{1.123632in}{2.008323in}}%
\pgfpathlineto{\pgfqpoint{1.123949in}{2.793246in}}%
\pgfpathlineto{\pgfqpoint{1.124266in}{2.835584in}}%
\pgfpathlineto{\pgfqpoint{1.124584in}{2.020341in}}%
\pgfpathlineto{\pgfqpoint{1.125219in}{3.004060in}}%
\pgfpathlineto{\pgfqpoint{1.125536in}{2.179306in}}%
\pgfpathlineto{\pgfqpoint{1.125854in}{2.274324in}}%
\pgfpathlineto{\pgfqpoint{1.126171in}{3.034437in}}%
\pgfpathlineto{\pgfqpoint{1.126806in}{2.064678in}}%
\pgfpathlineto{\pgfqpoint{1.127124in}{2.911341in}}%
\pgfpathlineto{\pgfqpoint{1.127758in}{1.989339in}}%
\pgfpathlineto{\pgfqpoint{1.128076in}{2.681354in}}%
\pgfpathlineto{\pgfqpoint{1.128393in}{2.917350in}}%
\pgfpathlineto{\pgfqpoint{1.128711in}{2.067048in}}%
\pgfpathlineto{\pgfqpoint{1.129346in}{3.028810in}}%
\pgfpathlineto{\pgfqpoint{1.129663in}{2.268934in}}%
\pgfpathlineto{\pgfqpoint{1.129981in}{2.163000in}}%
\pgfpathlineto{\pgfqpoint{1.130298in}{2.993007in}}%
\pgfpathlineto{\pgfqpoint{1.130933in}{2.013992in}}%
\pgfpathlineto{\pgfqpoint{1.131250in}{2.824754in}}%
\pgfpathlineto{\pgfqpoint{1.131568in}{2.800583in}}%
\pgfpathlineto{\pgfqpoint{1.131885in}{2.010929in}}%
\pgfpathlineto{\pgfqpoint{1.132520in}{2.988554in}}%
\pgfpathlineto{\pgfqpoint{1.132838in}{2.150005in}}%
\pgfpathlineto{\pgfqpoint{1.133155in}{2.301927in}}%
\pgfpathlineto{\pgfqpoint{1.133473in}{3.034722in}}%
\pgfpathlineto{\pgfqpoint{1.134108in}{2.081304in}}%
\pgfpathlineto{\pgfqpoint{1.134425in}{2.930378in}}%
\pgfpathlineto{\pgfqpoint{1.135060in}{1.986594in}}%
\pgfpathlineto{\pgfqpoint{1.135377in}{2.716643in}}%
\pgfpathlineto{\pgfqpoint{1.135695in}{2.888775in}}%
\pgfpathlineto{\pgfqpoint{1.136012in}{2.046784in}}%
\pgfpathlineto{\pgfqpoint{1.136647in}{3.019053in}}%
\pgfpathlineto{\pgfqpoint{1.136965in}{2.233248in}}%
\pgfpathlineto{\pgfqpoint{1.137282in}{2.188694in}}%
\pgfpathlineto{\pgfqpoint{1.137600in}{3.005244in}}%
\pgfpathlineto{\pgfqpoint{1.138235in}{2.028115in}}%
\pgfpathlineto{\pgfqpoint{1.138552in}{2.857302in}}%
\pgfpathlineto{\pgfqpoint{1.138869in}{2.770406in}}%
\pgfpathlineto{\pgfqpoint{1.139187in}{2.005116in}}%
\pgfpathlineto{\pgfqpoint{1.139822in}{2.965850in}}%
\pgfpathlineto{\pgfqpoint{1.140139in}{2.118071in}}%
\pgfpathlineto{\pgfqpoint{1.140774in}{3.030416in}}%
\pgfpathlineto{\pgfqpoint{1.141092in}{2.339476in}}%
\pgfpathlineto{\pgfqpoint{1.141409in}{2.098096in}}%
\pgfpathlineto{\pgfqpoint{1.141727in}{2.950417in}}%
\pgfpathlineto{\pgfqpoint{1.142362in}{1.990742in}}%
\pgfpathlineto{\pgfqpoint{1.142679in}{2.749527in}}%
\pgfpathlineto{\pgfqpoint{1.142996in}{2.859828in}}%
\pgfpathlineto{\pgfqpoint{1.143314in}{2.027492in}}%
\pgfpathlineto{\pgfqpoint{1.143949in}{3.006285in}}%
\pgfpathlineto{\pgfqpoint{1.144266in}{2.197172in}}%
\pgfpathlineto{\pgfqpoint{1.144584in}{2.219633in}}%
\pgfpathlineto{\pgfqpoint{1.144901in}{3.013025in}}%
\pgfpathlineto{\pgfqpoint{1.145536in}{2.046167in}}%
\pgfpathlineto{\pgfqpoint{1.145854in}{2.887346in}}%
\pgfpathlineto{\pgfqpoint{1.146171in}{2.732645in}}%
\pgfpathlineto{\pgfqpoint{1.146488in}{1.998397in}}%
\pgfpathlineto{\pgfqpoint{1.147123in}{2.939723in}}%
\pgfpathlineto{\pgfqpoint{1.147441in}{2.086523in}}%
\pgfpathlineto{\pgfqpoint{1.148076in}{3.024488in}}%
\pgfpathlineto{\pgfqpoint{1.148393in}{2.296530in}}%
\pgfpathlineto{\pgfqpoint{1.148711in}{2.122692in}}%
\pgfpathlineto{\pgfqpoint{1.149028in}{2.966603in}}%
\pgfpathlineto{\pgfqpoint{1.149663in}{1.994822in}}%
\pgfpathlineto{\pgfqpoint{1.149980in}{2.782779in}}%
\pgfpathlineto{\pgfqpoint{1.150298in}{2.825870in}}%
\pgfpathlineto{\pgfqpoint{1.150615in}{2.011876in}}%
\pgfpathlineto{\pgfqpoint{1.151250in}{2.990893in}}%
\pgfpathlineto{\pgfqpoint{1.151568in}{2.164269in}}%
\pgfpathlineto{\pgfqpoint{1.151885in}{2.252791in}}%
\pgfpathlineto{\pgfqpoint{1.152203in}{3.023324in}}%
\pgfpathlineto{\pgfqpoint{1.152838in}{2.067180in}}%
\pgfpathlineto{\pgfqpoint{1.153155in}{2.908785in}}%
\pgfpathlineto{\pgfqpoint{1.153790in}{1.987824in}}%
\pgfpathlineto{\pgfqpoint{1.154107in}{2.673978in}}%
\pgfpathlineto{\pgfqpoint{1.154425in}{2.909012in}}%
\pgfpathlineto{\pgfqpoint{1.154742in}{2.057977in}}%
\pgfpathlineto{\pgfqpoint{1.155377in}{3.016896in}}%
\pgfpathlineto{\pgfqpoint{1.155695in}{2.258596in}}%
\pgfpathlineto{\pgfqpoint{1.156012in}{2.148151in}}%
\pgfpathlineto{\pgfqpoint{1.156330in}{2.983125in}}%
\pgfpathlineto{\pgfqpoint{1.156965in}{2.005251in}}%
\pgfpathlineto{\pgfqpoint{1.157282in}{2.812323in}}%
\pgfpathlineto{\pgfqpoint{1.157599in}{2.792349in}}%
\pgfpathlineto{\pgfqpoint{1.157917in}{1.997027in}}%
\pgfpathlineto{\pgfqpoint{1.158552in}{2.972065in}}%
\pgfpathlineto{\pgfqpoint{1.158869in}{2.132266in}}%
\pgfpathlineto{\pgfqpoint{1.159187in}{2.292738in}}%
\pgfpathlineto{\pgfqpoint{1.159504in}{3.028364in}}%
\pgfpathlineto{\pgfqpoint{1.160139in}{2.082225in}}%
\pgfpathlineto{\pgfqpoint{1.160457in}{2.928516in}}%
\pgfpathlineto{\pgfqpoint{1.161092in}{1.982518in}}%
\pgfpathlineto{\pgfqpoint{1.161409in}{2.705042in}}%
\pgfpathlineto{\pgfqpoint{1.161726in}{2.879256in}}%
\pgfpathlineto{\pgfqpoint{1.162044in}{2.033690in}}%
\pgfpathlineto{\pgfqpoint{1.162679in}{3.007941in}}%
\pgfpathlineto{\pgfqpoint{1.162996in}{2.221440in}}%
\pgfpathlineto{\pgfqpoint{1.163314in}{2.179559in}}%
\pgfpathlineto{\pgfqpoint{1.163631in}{2.994385in}}%
\pgfpathlineto{\pgfqpoint{1.164266in}{2.014270in}}%
\pgfpathlineto{\pgfqpoint{1.164584in}{2.842041in}}%
\pgfpathlineto{\pgfqpoint{1.164901in}{2.753718in}}%
\pgfpathlineto{\pgfqpoint{1.165218in}{1.987374in}}%
\pgfpathlineto{\pgfqpoint{1.165853in}{2.951705in}}%
\pgfpathlineto{\pgfqpoint{1.166171in}{2.104898in}}%
\pgfpathlineto{\pgfqpoint{1.166806in}{3.028987in}}%
\pgfpathlineto{\pgfqpoint{1.167123in}{2.336293in}}%
\pgfpathlineto{\pgfqpoint{1.167441in}{2.101195in}}%
\pgfpathlineto{\pgfqpoint{1.167758in}{2.943757in}}%
\pgfpathlineto{\pgfqpoint{1.168393in}{1.978817in}}%
\pgfpathlineto{\pgfqpoint{1.168711in}{2.738911in}}%
\pgfpathlineto{\pgfqpoint{1.169028in}{2.846399in}}%
\pgfpathlineto{\pgfqpoint{1.169345in}{2.015348in}}%
\pgfpathlineto{\pgfqpoint{1.169980in}{2.996046in}}%
\pgfpathlineto{\pgfqpoint{1.170298in}{2.187276in}}%
\pgfpathlineto{\pgfqpoint{1.170615in}{2.208953in}}%
\pgfpathlineto{\pgfqpoint{1.170933in}{3.003941in}}%
\pgfpathlineto{\pgfqpoint{1.171568in}{2.029371in}}%
\pgfpathlineto{\pgfqpoint{1.171885in}{2.867724in}}%
\pgfpathlineto{\pgfqpoint{1.172203in}{2.716740in}}%
\pgfpathlineto{\pgfqpoint{1.172520in}{1.978467in}}%
\pgfpathlineto{\pgfqpoint{1.173155in}{2.930106in}}%
\pgfpathlineto{\pgfqpoint{1.173472in}{2.081018in}}%
\pgfpathlineto{\pgfqpoint{1.174107in}{3.023459in}}%
\pgfpathlineto{\pgfqpoint{1.174425in}{2.295176in}}%
\pgfpathlineto{\pgfqpoint{1.174742in}{2.118942in}}%
\pgfpathlineto{\pgfqpoint{1.175060in}{2.959672in}}%
\pgfpathlineto{\pgfqpoint{1.175695in}{1.984075in}}%
\pgfpathlineto{\pgfqpoint{1.176012in}{2.770760in}}%
\pgfpathlineto{\pgfqpoint{1.176329in}{2.814741in}}%
\pgfpathlineto{\pgfqpoint{1.176647in}{1.998990in}}%
\pgfpathlineto{\pgfqpoint{1.177282in}{2.980291in}}%
\pgfpathlineto{\pgfqpoint{1.177599in}{2.152948in}}%
\pgfpathlineto{\pgfqpoint{1.177917in}{2.242075in}}%
\pgfpathlineto{\pgfqpoint{1.178234in}{3.008532in}}%
\pgfpathlineto{\pgfqpoint{1.178869in}{2.043994in}}%
\pgfpathlineto{\pgfqpoint{1.179187in}{2.894350in}}%
\pgfpathlineto{\pgfqpoint{1.179822in}{1.978507in}}%
\pgfpathlineto{\pgfqpoint{1.180139in}{2.675362in}}%
\pgfpathlineto{\pgfqpoint{1.180456in}{2.906848in}}%
\pgfpathlineto{\pgfqpoint{1.180774in}{2.057130in}}%
\pgfpathlineto{\pgfqpoint{1.181409in}{3.013912in}}%
\pgfpathlineto{\pgfqpoint{1.181726in}{2.251985in}}%
\pgfpathlineto{\pgfqpoint{1.182044in}{2.143248in}}%
\pgfpathlineto{\pgfqpoint{1.182361in}{2.971647in}}%
\pgfpathlineto{\pgfqpoint{1.182996in}{1.990119in}}%
\pgfpathlineto{\pgfqpoint{1.183314in}{2.803575in}}%
\pgfpathlineto{\pgfqpoint{1.183631in}{2.778128in}}%
\pgfpathlineto{\pgfqpoint{1.183948in}{1.987143in}}%
\pgfpathlineto{\pgfqpoint{1.184583in}{2.961745in}}%
\pgfpathlineto{\pgfqpoint{1.184901in}{2.120718in}}%
\pgfpathlineto{\pgfqpoint{1.185218in}{2.274737in}}%
\pgfpathlineto{\pgfqpoint{1.185536in}{3.011685in}}%
\pgfpathlineto{\pgfqpoint{1.186171in}{2.067012in}}%
\pgfpathlineto{\pgfqpoint{1.186488in}{2.920987in}}%
\pgfpathlineto{\pgfqpoint{1.187123in}{1.981372in}}%
\pgfpathlineto{\pgfqpoint{1.187441in}{2.711436in}}%
\pgfpathlineto{\pgfqpoint{1.187758in}{2.874822in}}%
\pgfpathlineto{\pgfqpoint{1.188075in}{2.033638in}}%
\pgfpathlineto{\pgfqpoint{1.188710in}{2.999944in}}%
\pgfpathlineto{\pgfqpoint{1.189028in}{2.211423in}}%
\pgfpathlineto{\pgfqpoint{1.189345in}{2.167752in}}%
\pgfpathlineto{\pgfqpoint{1.189663in}{2.983709in}}%
\pgfpathlineto{\pgfqpoint{1.190298in}{2.003916in}}%
\pgfpathlineto{\pgfqpoint{1.190615in}{2.832460in}}%
\pgfpathlineto{\pgfqpoint{1.190933in}{2.742445in}}%
\pgfpathlineto{\pgfqpoint{1.191250in}{1.975603in}}%
\pgfpathlineto{\pgfqpoint{1.191885in}{2.938909in}}%
\pgfpathlineto{\pgfqpoint{1.192202in}{2.090379in}}%
\pgfpathlineto{\pgfqpoint{1.192837in}{3.011062in}}%
\pgfpathlineto{\pgfqpoint{1.193155in}{2.318787in}}%
\pgfpathlineto{\pgfqpoint{1.193472in}{2.095040in}}%
\pgfpathlineto{\pgfqpoint{1.193790in}{2.944844in}}%
\pgfpathlineto{\pgfqpoint{1.194425in}{1.984353in}}%
\pgfpathlineto{\pgfqpoint{1.194742in}{2.739820in}}%
\pgfpathlineto{\pgfqpoint{1.195059in}{2.840666in}}%
\pgfpathlineto{\pgfqpoint{1.195377in}{2.008127in}}%
\pgfpathlineto{\pgfqpoint{1.196012in}{2.983790in}}%
\pgfpathlineto{\pgfqpoint{1.196329in}{2.171977in}}%
\pgfpathlineto{\pgfqpoint{1.196647in}{2.199321in}}%
\pgfpathlineto{\pgfqpoint{1.196964in}{2.991950in}}%
\pgfpathlineto{\pgfqpoint{1.197599in}{2.017605in}}%
\pgfpathlineto{\pgfqpoint{1.197917in}{2.861594in}}%
\pgfpathlineto{\pgfqpoint{1.198234in}{2.701495in}}%
\pgfpathlineto{\pgfqpoint{1.198552in}{1.969535in}}%
\pgfpathlineto{\pgfqpoint{1.199186in}{2.914707in}}%
\pgfpathlineto{\pgfqpoint{1.199504in}{2.062371in}}%
\pgfpathlineto{\pgfqpoint{1.200139in}{3.011605in}}%
\pgfpathlineto{\pgfqpoint{1.200456in}{2.280474in}}%
\pgfpathlineto{\pgfqpoint{1.200774in}{2.124001in}}%
\pgfpathlineto{\pgfqpoint{1.201091in}{2.959639in}}%
\pgfpathlineto{\pgfqpoint{1.201726in}{1.982472in}}%
\pgfpathlineto{\pgfqpoint{1.202044in}{2.768613in}}%
\pgfpathlineto{\pgfqpoint{1.202361in}{2.801185in}}%
\pgfpathlineto{\pgfqpoint{1.202678in}{1.988081in}}%
\pgfpathlineto{\pgfqpoint{1.203313in}{2.966986in}}%
\pgfpathlineto{\pgfqpoint{1.203631in}{2.137090in}}%
\pgfpathlineto{\pgfqpoint{1.203948in}{2.232700in}}%
\pgfpathlineto{\pgfqpoint{1.204266in}{2.999367in}}%
\pgfpathlineto{\pgfqpoint{1.204901in}{2.037791in}}%
\pgfpathlineto{\pgfqpoint{1.205218in}{2.885691in}}%
\pgfpathlineto{\pgfqpoint{1.205853in}{1.963489in}}%
\pgfpathlineto{\pgfqpoint{1.206171in}{2.660487in}}%
\pgfpathlineto{\pgfqpoint{1.206488in}{2.885913in}}%
\pgfpathlineto{\pgfqpoint{1.206805in}{2.037620in}}%
\pgfpathlineto{\pgfqpoint{1.207440in}{3.006933in}}%
\pgfpathlineto{\pgfqpoint{1.207758in}{2.243290in}}%
\pgfpathlineto{\pgfqpoint{1.208075in}{2.148187in}}%
\pgfpathlineto{\pgfqpoint{1.208393in}{2.971867in}}%
\pgfpathlineto{\pgfqpoint{1.209028in}{1.987501in}}%
\pgfpathlineto{\pgfqpoint{1.209345in}{2.796133in}}%
\pgfpathlineto{\pgfqpoint{1.209663in}{2.763995in}}%
\pgfpathlineto{\pgfqpoint{1.209980in}{1.971681in}}%
\pgfpathlineto{\pgfqpoint{1.210615in}{2.947376in}}%
\pgfpathlineto{\pgfqpoint{1.210932in}{2.105429in}}%
\pgfpathlineto{\pgfqpoint{1.211567in}{3.001586in}}%
\pgfpathlineto{\pgfqpoint{1.211885in}{2.343718in}}%
\pgfpathlineto{\pgfqpoint{1.212202in}{2.056473in}}%
\pgfpathlineto{\pgfqpoint{1.212520in}{2.909452in}}%
\pgfpathlineto{\pgfqpoint{1.213155in}{1.964047in}}%
\pgfpathlineto{\pgfqpoint{1.213472in}{2.695837in}}%
\pgfpathlineto{\pgfqpoint{1.213789in}{2.856866in}}%
\pgfpathlineto{\pgfqpoint{1.214107in}{2.016016in}}%
\pgfpathlineto{\pgfqpoint{1.214742in}{2.996744in}}%
\pgfpathlineto{\pgfqpoint{1.215059in}{2.203707in}}%
\pgfpathlineto{\pgfqpoint{1.215377in}{2.175570in}}%
\pgfpathlineto{\pgfqpoint{1.215694in}{2.979212in}}%
\pgfpathlineto{\pgfqpoint{1.216329in}{1.994246in}}%
\pgfpathlineto{\pgfqpoint{1.216647in}{2.826105in}}%
\pgfpathlineto{\pgfqpoint{1.216964in}{2.723381in}}%
\pgfpathlineto{\pgfqpoint{1.217282in}{1.963190in}}%
\pgfpathlineto{\pgfqpoint{1.217916in}{2.926043in}}%
\pgfpathlineto{\pgfqpoint{1.218234in}{2.075907in}}%
\pgfpathlineto{\pgfqpoint{1.218869in}{3.000821in}}%
\pgfpathlineto{\pgfqpoint{1.219186in}{2.299864in}}%
\pgfpathlineto{\pgfqpoint{1.219504in}{2.080870in}}%
\pgfpathlineto{\pgfqpoint{1.219821in}{2.928298in}}%
\pgfpathlineto{\pgfqpoint{1.220456in}{1.964859in}}%
\pgfpathlineto{\pgfqpoint{1.220774in}{2.736382in}}%
\pgfpathlineto{\pgfqpoint{1.221091in}{2.826121in}}%
\pgfpathlineto{\pgfqpoint{1.221408in}{2.000625in}}%
\pgfpathlineto{\pgfqpoint{1.222043in}{2.981571in}}%
\pgfpathlineto{\pgfqpoint{1.222361in}{2.165570in}}%
\pgfpathlineto{\pgfqpoint{1.222678in}{2.202092in}}%
\pgfpathlineto{\pgfqpoint{1.222996in}{2.985983in}}%
\pgfpathlineto{\pgfqpoint{1.223631in}{2.009714in}}%
\pgfpathlineto{\pgfqpoint{1.223948in}{2.852724in}}%
\pgfpathlineto{\pgfqpoint{1.224583in}{1.955538in}}%
\pgfpathlineto{\pgfqpoint{1.224901in}{2.620189in}}%
\pgfpathlineto{\pgfqpoint{1.225218in}{2.899596in}}%
\pgfpathlineto{\pgfqpoint{1.225535in}{2.048597in}}%
\pgfpathlineto{\pgfqpoint{1.226170in}{2.995706in}}%
\pgfpathlineto{\pgfqpoint{1.226488in}{2.258908in}}%
\pgfpathlineto{\pgfqpoint{1.226805in}{2.105132in}}%
\pgfpathlineto{\pgfqpoint{1.227123in}{2.947089in}}%
\pgfpathlineto{\pgfqpoint{1.227758in}{1.976012in}}%
\pgfpathlineto{\pgfqpoint{1.228075in}{2.774109in}}%
\pgfpathlineto{\pgfqpoint{1.228393in}{2.794215in}}%
\pgfpathlineto{\pgfqpoint{1.228710in}{1.985735in}}%
\pgfpathlineto{\pgfqpoint{1.229345in}{2.960290in}}%
\pgfpathlineto{\pgfqpoint{1.229662in}{2.127161in}}%
\pgfpathlineto{\pgfqpoint{1.229980in}{2.232863in}}%
\pgfpathlineto{\pgfqpoint{1.230297in}{2.988550in}}%
\pgfpathlineto{\pgfqpoint{1.230932in}{2.025799in}}%
\pgfpathlineto{\pgfqpoint{1.231250in}{2.879910in}}%
\pgfpathlineto{\pgfqpoint{1.231885in}{1.954237in}}%
\pgfpathlineto{\pgfqpoint{1.232202in}{2.657283in}}%
\pgfpathlineto{\pgfqpoint{1.232519in}{2.871746in}}%
\pgfpathlineto{\pgfqpoint{1.232837in}{2.022577in}}%
\pgfpathlineto{\pgfqpoint{1.233472in}{2.987669in}}%
\pgfpathlineto{\pgfqpoint{1.233789in}{2.217760in}}%
\pgfpathlineto{\pgfqpoint{1.234107in}{2.137532in}}%
\pgfpathlineto{\pgfqpoint{1.234424in}{2.965282in}}%
\pgfpathlineto{\pgfqpoint{1.235059in}{1.989338in}}%
\pgfpathlineto{\pgfqpoint{1.235377in}{2.806739in}}%
\pgfpathlineto{\pgfqpoint{1.235694in}{2.753660in}}%
\pgfpathlineto{\pgfqpoint{1.236012in}{1.971850in}}%
\pgfpathlineto{\pgfqpoint{1.236646in}{2.936494in}}%
\pgfpathlineto{\pgfqpoint{1.236964in}{2.090854in}}%
\pgfpathlineto{\pgfqpoint{1.237599in}{2.989932in}}%
\pgfpathlineto{\pgfqpoint{1.237916in}{2.321612in}}%
\pgfpathlineto{\pgfqpoint{1.238234in}{2.049569in}}%
\pgfpathlineto{\pgfqpoint{1.238551in}{2.902161in}}%
\pgfpathlineto{\pgfqpoint{1.239186in}{1.952798in}}%
\pgfpathlineto{\pgfqpoint{1.239504in}{2.694882in}}%
\pgfpathlineto{\pgfqpoint{1.239821in}{2.838633in}}%
\pgfpathlineto{\pgfqpoint{1.240138in}{2.000820in}}%
\pgfpathlineto{\pgfqpoint{1.240773in}{2.976516in}}%
\pgfpathlineto{\pgfqpoint{1.241091in}{2.180376in}}%
\pgfpathlineto{\pgfqpoint{1.241408in}{2.174087in}}%
\pgfpathlineto{\pgfqpoint{1.241726in}{2.980602in}}%
\pgfpathlineto{\pgfqpoint{1.242361in}{2.002860in}}%
\pgfpathlineto{\pgfqpoint{1.242678in}{2.830385in}}%
\pgfpathlineto{\pgfqpoint{1.242996in}{2.712147in}}%
\pgfpathlineto{\pgfqpoint{1.243313in}{1.955223in}}%
\pgfpathlineto{\pgfqpoint{1.243948in}{2.909155in}}%
\pgfpathlineto{\pgfqpoint{1.244265in}{2.058087in}}%
\pgfpathlineto{\pgfqpoint{1.244900in}{2.988114in}}%
\pgfpathlineto{\pgfqpoint{1.245218in}{2.279838in}}%
\pgfpathlineto{\pgfqpoint{1.245535in}{2.073679in}}%
\pgfpathlineto{\pgfqpoint{1.245853in}{2.923964in}}%
\pgfpathlineto{\pgfqpoint{1.246488in}{1.958030in}}%
\pgfpathlineto{\pgfqpoint{1.246805in}{2.729055in}}%
\pgfpathlineto{\pgfqpoint{1.247123in}{2.805480in}}%
\pgfpathlineto{\pgfqpoint{1.247440in}{1.980482in}}%
\pgfpathlineto{\pgfqpoint{1.248075in}{2.964360in}}%
\pgfpathlineto{\pgfqpoint{1.248392in}{2.145960in}}%
\pgfpathlineto{\pgfqpoint{1.248710in}{2.211136in}}%
\pgfpathlineto{\pgfqpoint{1.249027in}{2.986799in}}%
\pgfpathlineto{\pgfqpoint{1.249662in}{2.012314in}}%
\pgfpathlineto{\pgfqpoint{1.249980in}{2.854459in}}%
\pgfpathlineto{\pgfqpoint{1.250615in}{1.945967in}}%
\pgfpathlineto{\pgfqpoint{1.250932in}{2.616376in}}%
\pgfpathlineto{\pgfqpoint{1.251249in}{2.882173in}}%
\pgfpathlineto{\pgfqpoint{1.251567in}{2.029111in}}%
\pgfpathlineto{\pgfqpoint{1.252202in}{2.983892in}}%
\pgfpathlineto{\pgfqpoint{1.252519in}{2.237668in}}%
\pgfpathlineto{\pgfqpoint{1.252837in}{2.103481in}}%
\pgfpathlineto{\pgfqpoint{1.253154in}{2.940022in}}%
\pgfpathlineto{\pgfqpoint{1.253789in}{1.962876in}}%
\pgfpathlineto{\pgfqpoint{1.254107in}{2.764000in}}%
\pgfpathlineto{\pgfqpoint{1.254424in}{2.767102in}}%
\pgfpathlineto{\pgfqpoint{1.254742in}{1.965738in}}%
\pgfpathlineto{\pgfqpoint{1.255376in}{2.949327in}}%
\pgfpathlineto{\pgfqpoint{1.255694in}{2.113121in}}%
\pgfpathlineto{\pgfqpoint{1.256011in}{2.245173in}}%
\pgfpathlineto{\pgfqpoint{1.256329in}{2.988903in}}%
\pgfpathlineto{\pgfqpoint{1.256964in}{2.028765in}}%
\pgfpathlineto{\pgfqpoint{1.257281in}{2.875369in}}%
\pgfpathlineto{\pgfqpoint{1.257916in}{1.939717in}}%
\pgfpathlineto{\pgfqpoint{1.258234in}{2.656083in}}%
\pgfpathlineto{\pgfqpoint{1.258551in}{2.851358in}}%
\pgfpathlineto{\pgfqpoint{1.258868in}{2.005914in}}%
\pgfpathlineto{\pgfqpoint{1.259503in}{2.975434in}}%
\pgfpathlineto{\pgfqpoint{1.259821in}{2.198014in}}%
\pgfpathlineto{\pgfqpoint{1.260138in}{2.132207in}}%
\pgfpathlineto{\pgfqpoint{1.260456in}{2.954573in}}%
\pgfpathlineto{\pgfqpoint{1.261091in}{1.974859in}}%
\pgfpathlineto{\pgfqpoint{1.261408in}{2.794831in}}%
\pgfpathlineto{\pgfqpoint{1.261726in}{2.729510in}}%
\pgfpathlineto{\pgfqpoint{1.262043in}{1.952266in}}%
\pgfpathlineto{\pgfqpoint{1.262678in}{2.927897in}}%
\pgfpathlineto{\pgfqpoint{1.262995in}{2.080815in}}%
\pgfpathlineto{\pgfqpoint{1.263630in}{2.986197in}}%
\pgfpathlineto{\pgfqpoint{1.263948in}{2.304893in}}%
\pgfpathlineto{\pgfqpoint{1.264265in}{2.046362in}}%
\pgfpathlineto{\pgfqpoint{1.264583in}{2.897853in}}%
\pgfpathlineto{\pgfqpoint{1.265218in}{1.942451in}}%
\pgfpathlineto{\pgfqpoint{1.265535in}{2.693088in}}%
\pgfpathlineto{\pgfqpoint{1.265853in}{2.820364in}}%
\pgfpathlineto{\pgfqpoint{1.266170in}{1.984155in}}%
\pgfpathlineto{\pgfqpoint{1.266805in}{2.962479in}}%
\pgfpathlineto{\pgfqpoint{1.267122in}{2.158407in}}%
\pgfpathlineto{\pgfqpoint{1.267440in}{2.165252in}}%
\pgfpathlineto{\pgfqpoint{1.267757in}{2.963931in}}%
\pgfpathlineto{\pgfqpoint{1.268392in}{1.987030in}}%
\pgfpathlineto{\pgfqpoint{1.268710in}{2.828225in}}%
\pgfpathlineto{\pgfqpoint{1.269027in}{2.688979in}}%
\pgfpathlineto{\pgfqpoint{1.269345in}{1.948554in}}%
\pgfpathlineto{\pgfqpoint{1.269979in}{2.902837in}}%
\pgfpathlineto{\pgfqpoint{1.270297in}{2.049063in}}%
\pgfpathlineto{\pgfqpoint{1.270932in}{2.980893in}}%
\pgfpathlineto{\pgfqpoint{1.271249in}{2.258426in}}%
\pgfpathlineto{\pgfqpoint{1.271567in}{2.071548in}}%
\pgfpathlineto{\pgfqpoint{1.271884in}{2.916025in}}%
\pgfpathlineto{\pgfqpoint{1.272519in}{1.945743in}}%
\pgfpathlineto{\pgfqpoint{1.272837in}{2.730724in}}%
\pgfpathlineto{\pgfqpoint{1.273154in}{2.783542in}}%
\pgfpathlineto{\pgfqpoint{1.273472in}{1.966993in}}%
\pgfpathlineto{\pgfqpoint{1.274106in}{2.946236in}}%
\pgfpathlineto{\pgfqpoint{1.274424in}{2.120978in}}%
\pgfpathlineto{\pgfqpoint{1.274741in}{2.198784in}}%
\pgfpathlineto{\pgfqpoint{1.275059in}{2.971505in}}%
\pgfpathlineto{\pgfqpoint{1.275694in}{2.008547in}}%
\pgfpathlineto{\pgfqpoint{1.276011in}{2.860282in}}%
\pgfpathlineto{\pgfqpoint{1.276646in}{1.945139in}}%
\pgfpathlineto{\pgfqpoint{1.276964in}{2.634551in}}%
\pgfpathlineto{\pgfqpoint{1.277281in}{2.870435in}}%
\pgfpathlineto{\pgfqpoint{1.277598in}{2.019758in}}%
\pgfpathlineto{\pgfqpoint{1.278233in}{2.971569in}}%
\pgfpathlineto{\pgfqpoint{1.278551in}{2.214845in}}%
\pgfpathlineto{\pgfqpoint{1.278868in}{2.097679in}}%
\pgfpathlineto{\pgfqpoint{1.279186in}{2.933832in}}%
\pgfpathlineto{\pgfqpoint{1.279821in}{1.956538in}}%
\pgfpathlineto{\pgfqpoint{1.280138in}{2.763912in}}%
\pgfpathlineto{\pgfqpoint{1.280456in}{2.746806in}}%
\pgfpathlineto{\pgfqpoint{1.280773in}{1.950424in}}%
\pgfpathlineto{\pgfqpoint{1.281408in}{2.925496in}}%
\pgfpathlineto{\pgfqpoint{1.281725in}{2.085981in}}%
\pgfpathlineto{\pgfqpoint{1.282043in}{2.236738in}}%
\pgfpathlineto{\pgfqpoint{1.282360in}{2.978087in}}%
\pgfpathlineto{\pgfqpoint{1.282995in}{2.034532in}}%
\pgfpathlineto{\pgfqpoint{1.283313in}{2.887378in}}%
\pgfpathlineto{\pgfqpoint{1.283948in}{1.944211in}}%
\pgfpathlineto{\pgfqpoint{1.284265in}{2.666495in}}%
\pgfpathlineto{\pgfqpoint{1.284583in}{2.836331in}}%
\pgfpathlineto{\pgfqpoint{1.284900in}{1.990850in}}%
\pgfpathlineto{\pgfqpoint{1.285535in}{2.959659in}}%
\pgfpathlineto{\pgfqpoint{1.285852in}{2.172464in}}%
\pgfpathlineto{\pgfqpoint{1.286170in}{2.130495in}}%
\pgfpathlineto{\pgfqpoint{1.286487in}{2.946636in}}%
\pgfpathlineto{\pgfqpoint{1.287122in}{1.967329in}}%
\pgfpathlineto{\pgfqpoint{1.287440in}{2.797253in}}%
\pgfpathlineto{\pgfqpoint{1.287757in}{2.704466in}}%
\pgfpathlineto{\pgfqpoint{1.288075in}{1.940041in}}%
\pgfpathlineto{\pgfqpoint{1.288709in}{2.902750in}}%
\pgfpathlineto{\pgfqpoint{1.289027in}{2.053076in}}%
\pgfpathlineto{\pgfqpoint{1.289662in}{2.982223in}}%
\pgfpathlineto{\pgfqpoint{1.289979in}{2.284897in}}%
\pgfpathlineto{\pgfqpoint{1.290297in}{2.060673in}}%
\pgfpathlineto{\pgfqpoint{1.290614in}{2.904840in}}%
\pgfpathlineto{\pgfqpoint{1.291249in}{1.939456in}}%
\pgfpathlineto{\pgfqpoint{1.291567in}{2.699234in}}%
\pgfpathlineto{\pgfqpoint{1.291884in}{2.797394in}}%
\pgfpathlineto{\pgfqpoint{1.292202in}{1.967671in}}%
\pgfpathlineto{\pgfqpoint{1.292836in}{2.945668in}}%
\pgfpathlineto{\pgfqpoint{1.293154in}{2.133273in}}%
\pgfpathlineto{\pgfqpoint{1.293471in}{2.164570in}}%
\pgfpathlineto{\pgfqpoint{1.293789in}{2.957647in}}%
\pgfpathlineto{\pgfqpoint{1.294424in}{1.984997in}}%
\pgfpathlineto{\pgfqpoint{1.294741in}{2.825471in}}%
\pgfpathlineto{\pgfqpoint{1.295059in}{2.663191in}}%
\pgfpathlineto{\pgfqpoint{1.295376in}{1.930340in}}%
\pgfpathlineto{\pgfqpoint{1.296011in}{2.875244in}}%
\pgfpathlineto{\pgfqpoint{1.296328in}{2.025013in}}%
\pgfpathlineto{\pgfqpoint{1.296963in}{2.978588in}}%
\pgfpathlineto{\pgfqpoint{1.297281in}{2.241624in}}%
\pgfpathlineto{\pgfqpoint{1.297598in}{2.083684in}}%
\pgfpathlineto{\pgfqpoint{1.297916in}{2.920616in}}%
\pgfpathlineto{\pgfqpoint{1.298551in}{1.942543in}}%
\pgfpathlineto{\pgfqpoint{1.298868in}{2.731188in}}%
\pgfpathlineto{\pgfqpoint{1.299186in}{2.759827in}}%
\pgfpathlineto{\pgfqpoint{1.299503in}{1.947109in}}%
\pgfpathlineto{\pgfqpoint{1.300138in}{2.927943in}}%
\pgfpathlineto{\pgfqpoint{1.300455in}{2.096842in}}%
\pgfpathlineto{\pgfqpoint{1.300773in}{2.202774in}}%
\pgfpathlineto{\pgfqpoint{1.301090in}{2.963146in}}%
\pgfpathlineto{\pgfqpoint{1.301725in}{2.002409in}}%
\pgfpathlineto{\pgfqpoint{1.302043in}{2.853585in}}%
\pgfpathlineto{\pgfqpoint{1.302678in}{1.927880in}}%
\pgfpathlineto{\pgfqpoint{1.302995in}{2.629362in}}%
\pgfpathlineto{\pgfqpoint{1.303313in}{2.848510in}}%
\pgfpathlineto{\pgfqpoint{1.303630in}{1.999717in}}%
\pgfpathlineto{\pgfqpoint{1.304265in}{2.969149in}}%
\pgfpathlineto{\pgfqpoint{1.304582in}{2.196976in}}%
\pgfpathlineto{\pgfqpoint{1.304900in}{2.111381in}}%
\pgfpathlineto{\pgfqpoint{1.305217in}{2.931824in}}%
\pgfpathlineto{\pgfqpoint{1.305852in}{1.947975in}}%
\pgfpathlineto{\pgfqpoint{1.306170in}{2.765578in}}%
\pgfpathlineto{\pgfqpoint{1.306487in}{2.718232in}}%
\pgfpathlineto{\pgfqpoint{1.306805in}{1.934926in}}%
\pgfpathlineto{\pgfqpoint{1.307439in}{2.907696in}}%
\pgfpathlineto{\pgfqpoint{1.307757in}{2.062668in}}%
\pgfpathlineto{\pgfqpoint{1.308392in}{2.965441in}}%
\pgfpathlineto{\pgfqpoint{1.308709in}{2.296333in}}%
\pgfpathlineto{\pgfqpoint{1.309027in}{2.026157in}}%
\pgfpathlineto{\pgfqpoint{1.309344in}{2.876424in}}%
\pgfpathlineto{\pgfqpoint{1.309979in}{1.926024in}}%
\pgfpathlineto{\pgfqpoint{1.310297in}{2.674275in}}%
\pgfpathlineto{\pgfqpoint{1.310614in}{2.816662in}}%
\pgfpathlineto{\pgfqpoint{1.310932in}{1.978974in}}%
\pgfpathlineto{\pgfqpoint{1.311566in}{2.955045in}}%
\pgfpathlineto{\pgfqpoint{1.311884in}{2.154248in}}%
\pgfpathlineto{\pgfqpoint{1.312201in}{2.139563in}}%
\pgfpathlineto{\pgfqpoint{1.312519in}{2.942305in}}%
\pgfpathlineto{\pgfqpoint{1.313154in}{1.962358in}}%
\pgfpathlineto{\pgfqpoint{1.313471in}{2.796367in}}%
\pgfpathlineto{\pgfqpoint{1.313789in}{2.678067in}}%
\pgfpathlineto{\pgfqpoint{1.314106in}{1.924023in}}%
\pgfpathlineto{\pgfqpoint{1.314741in}{2.881801in}}%
\pgfpathlineto{\pgfqpoint{1.315058in}{2.031576in}}%
\pgfpathlineto{\pgfqpoint{1.315693in}{2.962857in}}%
\pgfpathlineto{\pgfqpoint{1.316011in}{2.251214in}}%
\pgfpathlineto{\pgfqpoint{1.316328in}{2.050372in}}%
\pgfpathlineto{\pgfqpoint{1.316646in}{2.898611in}}%
\pgfpathlineto{\pgfqpoint{1.317281in}{1.932737in}}%
\pgfpathlineto{\pgfqpoint{1.317598in}{2.714975in}}%
\pgfpathlineto{\pgfqpoint{1.317916in}{2.783137in}}%
\pgfpathlineto{\pgfqpoint{1.318233in}{1.959146in}}%
\pgfpathlineto{\pgfqpoint{1.318868in}{2.935542in}}%
\pgfpathlineto{\pgfqpoint{1.319185in}{2.112921in}}%
\pgfpathlineto{\pgfqpoint{1.319503in}{2.172622in}}%
\pgfpathlineto{\pgfqpoint{1.319820in}{2.948447in}}%
\pgfpathlineto{\pgfqpoint{1.320455in}{1.977830in}}%
\pgfpathlineto{\pgfqpoint{1.320773in}{2.827605in}}%
\pgfpathlineto{\pgfqpoint{1.321408in}{1.920014in}}%
\pgfpathlineto{\pgfqpoint{1.321725in}{2.596903in}}%
\pgfpathlineto{\pgfqpoint{1.322043in}{2.853912in}}%
\pgfpathlineto{\pgfqpoint{1.322360in}{2.001590in}}%
\pgfpathlineto{\pgfqpoint{1.322995in}{2.956685in}}%
\pgfpathlineto{\pgfqpoint{1.323312in}{2.206208in}}%
\pgfpathlineto{\pgfqpoint{1.323630in}{2.080632in}}%
\pgfpathlineto{\pgfqpoint{1.323947in}{2.918746in}}%
\pgfpathlineto{\pgfqpoint{1.324582in}{1.943820in}}%
\pgfpathlineto{\pgfqpoint{1.324900in}{2.753412in}}%
\pgfpathlineto{\pgfqpoint{1.325217in}{2.741695in}}%
\pgfpathlineto{\pgfqpoint{1.325535in}{1.943403in}}%
\pgfpathlineto{\pgfqpoint{1.326169in}{2.913065in}}%
\pgfpathlineto{\pgfqpoint{1.326487in}{2.073529in}}%
\pgfpathlineto{\pgfqpoint{1.326804in}{2.207101in}}%
\pgfpathlineto{\pgfqpoint{1.327122in}{2.952734in}}%
\pgfpathlineto{\pgfqpoint{1.327757in}{2.000863in}}%
\pgfpathlineto{\pgfqpoint{1.328074in}{2.853373in}}%
\pgfpathlineto{\pgfqpoint{1.328709in}{1.916243in}}%
\pgfpathlineto{\pgfqpoint{1.329027in}{2.638302in}}%
\pgfpathlineto{\pgfqpoint{1.329344in}{2.820233in}}%
\pgfpathlineto{\pgfqpoint{1.329662in}{1.976506in}}%
\pgfpathlineto{\pgfqpoint{1.330296in}{2.946794in}}%
\pgfpathlineto{\pgfqpoint{1.330614in}{2.163983in}}%
\pgfpathlineto{\pgfqpoint{1.330931in}{2.117271in}}%
\pgfpathlineto{\pgfqpoint{1.331249in}{2.938230in}}%
\pgfpathlineto{\pgfqpoint{1.331884in}{1.959802in}}%
\pgfpathlineto{\pgfqpoint{1.332201in}{2.782518in}}%
\pgfpathlineto{\pgfqpoint{1.332519in}{2.699006in}}%
\pgfpathlineto{\pgfqpoint{1.332836in}{1.926491in}}%
\pgfpathlineto{\pgfqpoint{1.333471in}{2.886121in}}%
\pgfpathlineto{\pgfqpoint{1.333788in}{2.037915in}}%
\pgfpathlineto{\pgfqpoint{1.334423in}{2.952841in}}%
\pgfpathlineto{\pgfqpoint{1.334741in}{2.264894in}}%
\pgfpathlineto{\pgfqpoint{1.335058in}{2.024736in}}%
\pgfpathlineto{\pgfqpoint{1.335376in}{2.878265in}}%
\pgfpathlineto{\pgfqpoint{1.336011in}{1.919737in}}%
\pgfpathlineto{\pgfqpoint{1.336328in}{2.675937in}}%
\pgfpathlineto{\pgfqpoint{1.336646in}{2.786013in}}%
\pgfpathlineto{\pgfqpoint{1.336963in}{1.952897in}}%
\pgfpathlineto{\pgfqpoint{1.337598in}{2.935267in}}%
\pgfpathlineto{\pgfqpoint{1.337915in}{2.125628in}}%
\pgfpathlineto{\pgfqpoint{1.338233in}{2.157000in}}%
\pgfpathlineto{\pgfqpoint{1.338550in}{2.948943in}}%
\pgfpathlineto{\pgfqpoint{1.339185in}{1.971344in}}%
\pgfpathlineto{\pgfqpoint{1.339503in}{2.809579in}}%
\pgfpathlineto{\pgfqpoint{1.339820in}{2.650281in}}%
\pgfpathlineto{\pgfqpoint{1.340138in}{1.915872in}}%
\pgfpathlineto{\pgfqpoint{1.340773in}{2.858376in}}%
\pgfpathlineto{\pgfqpoint{1.341090in}{2.004796in}}%
\pgfpathlineto{\pgfqpoint{1.341725in}{2.949599in}}%
\pgfpathlineto{\pgfqpoint{1.342042in}{2.218822in}}%
\pgfpathlineto{\pgfqpoint{1.342360in}{2.054824in}}%
\pgfpathlineto{\pgfqpoint{1.342677in}{2.897232in}}%
\pgfpathlineto{\pgfqpoint{1.343312in}{1.923588in}}%
\pgfpathlineto{\pgfqpoint{1.343630in}{2.714464in}}%
\pgfpathlineto{\pgfqpoint{1.343947in}{2.746096in}}%
\pgfpathlineto{\pgfqpoint{1.344265in}{1.935581in}}%
\pgfpathlineto{\pgfqpoint{1.344900in}{2.920671in}}%
\pgfpathlineto{\pgfqpoint{1.345217in}{2.089091in}}%
\pgfpathlineto{\pgfqpoint{1.345534in}{2.194891in}}%
\pgfpathlineto{\pgfqpoint{1.345852in}{2.953940in}}%
\pgfpathlineto{\pgfqpoint{1.346487in}{1.988428in}}%
\pgfpathlineto{\pgfqpoint{1.346804in}{2.832538in}}%
\pgfpathlineto{\pgfqpoint{1.347439in}{1.905825in}}%
\pgfpathlineto{\pgfqpoint{1.347757in}{2.605698in}}%
\pgfpathlineto{\pgfqpoint{1.348074in}{2.825957in}}%
\pgfpathlineto{\pgfqpoint{1.348392in}{1.977515in}}%
\pgfpathlineto{\pgfqpoint{1.349026in}{2.942294in}}%
\pgfpathlineto{\pgfqpoint{1.349344in}{2.175248in}}%
\pgfpathlineto{\pgfqpoint{1.349661in}{2.085484in}}%
\pgfpathlineto{\pgfqpoint{1.349979in}{2.914391in}}%
\pgfpathlineto{\pgfqpoint{1.350614in}{1.934988in}}%
\pgfpathlineto{\pgfqpoint{1.350931in}{2.748422in}}%
\pgfpathlineto{\pgfqpoint{1.351249in}{2.706453in}}%
\pgfpathlineto{\pgfqpoint{1.351566in}{1.919304in}}%
\pgfpathlineto{\pgfqpoint{1.352201in}{2.899590in}}%
\pgfpathlineto{\pgfqpoint{1.352518in}{2.054433in}}%
\pgfpathlineto{\pgfqpoint{1.353153in}{2.952718in}}%
\pgfpathlineto{\pgfqpoint{1.353471in}{2.283662in}}%
\pgfpathlineto{\pgfqpoint{1.353788in}{2.006179in}}%
\pgfpathlineto{\pgfqpoint{1.354106in}{2.856323in}}%
\pgfpathlineto{\pgfqpoint{1.354741in}{1.905911in}}%
\pgfpathlineto{\pgfqpoint{1.355058in}{2.645319in}}%
\pgfpathlineto{\pgfqpoint{1.355376in}{2.793331in}}%
\pgfpathlineto{\pgfqpoint{1.355693in}{1.952669in}}%
\pgfpathlineto{\pgfqpoint{1.356328in}{2.930219in}}%
\pgfpathlineto{\pgfqpoint{1.356645in}{2.132928in}}%
\pgfpathlineto{\pgfqpoint{1.356963in}{2.120513in}}%
\pgfpathlineto{\pgfqpoint{1.357280in}{2.925768in}}%
\pgfpathlineto{\pgfqpoint{1.357915in}{1.947061in}}%
\pgfpathlineto{\pgfqpoint{1.358233in}{2.782728in}}%
\pgfpathlineto{\pgfqpoint{1.358550in}{2.661710in}}%
\pgfpathlineto{\pgfqpoint{1.358868in}{1.911198in}}%
\pgfpathlineto{\pgfqpoint{1.359503in}{2.874840in}}%
\pgfpathlineto{\pgfqpoint{1.359820in}{2.020515in}}%
\pgfpathlineto{\pgfqpoint{1.360455in}{2.947791in}}%
\pgfpathlineto{\pgfqpoint{1.360772in}{2.233733in}}%
\pgfpathlineto{\pgfqpoint{1.361090in}{2.031399in}}%
\pgfpathlineto{\pgfqpoint{1.361407in}{2.876008in}}%
\pgfpathlineto{\pgfqpoint{1.362042in}{1.908043in}}%
\pgfpathlineto{\pgfqpoint{1.362360in}{2.686479in}}%
\pgfpathlineto{\pgfqpoint{1.362677in}{2.755005in}}%
\pgfpathlineto{\pgfqpoint{1.362995in}{1.934235in}}%
\pgfpathlineto{\pgfqpoint{1.363630in}{2.914270in}}%
\pgfpathlineto{\pgfqpoint{1.363947in}{2.092370in}}%
\pgfpathlineto{\pgfqpoint{1.364264in}{2.156285in}}%
\pgfpathlineto{\pgfqpoint{1.364582in}{2.934596in}}%
\pgfpathlineto{\pgfqpoint{1.365217in}{1.966618in}}%
\pgfpathlineto{\pgfqpoint{1.365534in}{2.813108in}}%
\pgfpathlineto{\pgfqpoint{1.366169in}{1.906059in}}%
\pgfpathlineto{\pgfqpoint{1.366487in}{2.595066in}}%
\pgfpathlineto{\pgfqpoint{1.366804in}{2.842651in}}%
\pgfpathlineto{\pgfqpoint{1.367122in}{1.990277in}}%
\pgfpathlineto{\pgfqpoint{1.367756in}{2.939199in}}%
\pgfpathlineto{\pgfqpoint{1.368074in}{2.186693in}}%
\pgfpathlineto{\pgfqpoint{1.368391in}{2.058609in}}%
\pgfpathlineto{\pgfqpoint{1.368709in}{2.895175in}}%
\pgfpathlineto{\pgfqpoint{1.369344in}{1.918470in}}%
\pgfpathlineto{\pgfqpoint{1.369661in}{2.722834in}}%
\pgfpathlineto{\pgfqpoint{1.369979in}{2.716253in}}%
\pgfpathlineto{\pgfqpoint{1.370296in}{1.916317in}}%
\pgfpathlineto{\pgfqpoint{1.370931in}{2.892601in}}%
\pgfpathlineto{\pgfqpoint{1.371248in}{2.054585in}}%
\pgfpathlineto{\pgfqpoint{1.371566in}{2.195562in}}%
\pgfpathlineto{\pgfqpoint{1.371883in}{2.938314in}}%
\pgfpathlineto{\pgfqpoint{1.372518in}{1.987201in}}%
\pgfpathlineto{\pgfqpoint{1.372836in}{2.846008in}}%
\pgfpathlineto{\pgfqpoint{1.373471in}{1.908228in}}%
\pgfpathlineto{\pgfqpoint{1.373788in}{2.634175in}}%
\pgfpathlineto{\pgfqpoint{1.374106in}{2.808490in}}%
\pgfpathlineto{\pgfqpoint{1.374423in}{1.961182in}}%
\pgfpathlineto{\pgfqpoint{1.375058in}{2.926326in}}%
\pgfpathlineto{\pgfqpoint{1.375375in}{2.141345in}}%
\pgfpathlineto{\pgfqpoint{1.375693in}{2.091844in}}%
\pgfpathlineto{\pgfqpoint{1.376010in}{2.909037in}}%
\pgfpathlineto{\pgfqpoint{1.376645in}{1.929726in}}%
\pgfpathlineto{\pgfqpoint{1.376963in}{2.759335in}}%
\pgfpathlineto{\pgfqpoint{1.377280in}{2.671576in}}%
\pgfpathlineto{\pgfqpoint{1.377598in}{1.905222in}}%
\pgfpathlineto{\pgfqpoint{1.378233in}{2.868542in}}%
\pgfpathlineto{\pgfqpoint{1.378550in}{2.018733in}}%
\pgfpathlineto{\pgfqpoint{1.379185in}{2.939624in}}%
\pgfpathlineto{\pgfqpoint{1.379502in}{2.245723in}}%
\pgfpathlineto{\pgfqpoint{1.379820in}{2.019096in}}%
\pgfpathlineto{\pgfqpoint{1.380137in}{2.871843in}}%
\pgfpathlineto{\pgfqpoint{1.380772in}{1.909713in}}%
\pgfpathlineto{\pgfqpoint{1.381090in}{2.672246in}}%
\pgfpathlineto{\pgfqpoint{1.381407in}{2.767580in}}%
\pgfpathlineto{\pgfqpoint{1.381725in}{1.937389in}}%
\pgfpathlineto{\pgfqpoint{1.382360in}{2.910598in}}%
\pgfpathlineto{\pgfqpoint{1.382677in}{2.098188in}}%
\pgfpathlineto{\pgfqpoint{1.382994in}{2.127069in}}%
\pgfpathlineto{\pgfqpoint{1.383312in}{2.920747in}}%
\pgfpathlineto{\pgfqpoint{1.383947in}{1.948509in}}%
\pgfpathlineto{\pgfqpoint{1.384264in}{2.790183in}}%
\pgfpathlineto{\pgfqpoint{1.384582in}{2.627424in}}%
\pgfpathlineto{\pgfqpoint{1.384899in}{1.894912in}}%
\pgfpathlineto{\pgfqpoint{1.385534in}{2.838686in}}%
\pgfpathlineto{\pgfqpoint{1.385852in}{1.987737in}}%
\pgfpathlineto{\pgfqpoint{1.386486in}{2.940049in}}%
\pgfpathlineto{\pgfqpoint{1.386804in}{2.202480in}}%
\pgfpathlineto{\pgfqpoint{1.387121in}{2.051637in}}%
\pgfpathlineto{\pgfqpoint{1.387439in}{2.892666in}}%
\pgfpathlineto{\pgfqpoint{1.388074in}{1.916009in}}%
\pgfpathlineto{\pgfqpoint{1.388391in}{2.704488in}}%
\pgfpathlineto{\pgfqpoint{1.388709in}{2.726217in}}%
\pgfpathlineto{\pgfqpoint{1.389026in}{1.914416in}}%
\pgfpathlineto{\pgfqpoint{1.389661in}{2.890182in}}%
\pgfpathlineto{\pgfqpoint{1.389978in}{2.058515in}}%
\pgfpathlineto{\pgfqpoint{1.390296in}{2.166970in}}%
\pgfpathlineto{\pgfqpoint{1.390613in}{2.926985in}}%
\pgfpathlineto{\pgfqpoint{1.391248in}{1.968119in}}%
\pgfpathlineto{\pgfqpoint{1.391566in}{2.820411in}}%
\pgfpathlineto{\pgfqpoint{1.392201in}{1.892392in}}%
\pgfpathlineto{\pgfqpoint{1.392518in}{2.596703in}}%
\pgfpathlineto{\pgfqpoint{1.392836in}{2.807350in}}%
\pgfpathlineto{\pgfqpoint{1.393153in}{1.958217in}}%
\pgfpathlineto{\pgfqpoint{1.393788in}{2.933516in}}%
\pgfpathlineto{\pgfqpoint{1.394105in}{2.158289in}}%
\pgfpathlineto{\pgfqpoint{1.394423in}{2.085026in}}%
\pgfpathlineto{\pgfqpoint{1.394740in}{2.905286in}}%
\pgfpathlineto{\pgfqpoint{1.395375in}{1.920611in}}%
\pgfpathlineto{\pgfqpoint{1.395693in}{2.737504in}}%
\pgfpathlineto{\pgfqpoint{1.396010in}{2.679597in}}%
\pgfpathlineto{\pgfqpoint{1.396328in}{1.899114in}}%
\pgfpathlineto{\pgfqpoint{1.396963in}{2.867601in}}%
\pgfpathlineto{\pgfqpoint{1.397280in}{2.021073in}}%
\pgfpathlineto{\pgfqpoint{1.397915in}{2.929447in}}%
\pgfpathlineto{\pgfqpoint{1.398232in}{2.254483in}}%
\pgfpathlineto{\pgfqpoint{1.398550in}{1.994279in}}%
\pgfpathlineto{\pgfqpoint{1.398867in}{2.844680in}}%
\pgfpathlineto{\pgfqpoint{1.399502in}{1.890857in}}%
\pgfpathlineto{\pgfqpoint{1.399820in}{2.639694in}}%
\pgfpathlineto{\pgfqpoint{1.400137in}{2.770294in}}%
\pgfpathlineto{\pgfqpoint{1.400455in}{1.935582in}}%
\pgfpathlineto{\pgfqpoint{1.401090in}{2.921090in}}%
\pgfpathlineto{\pgfqpoint{1.401407in}{2.114834in}}%
\pgfpathlineto{\pgfqpoint{1.401724in}{2.118395in}}%
\pgfpathlineto{\pgfqpoint{1.402042in}{2.915052in}}%
\pgfpathlineto{\pgfqpoint{1.402677in}{1.934145in}}%
\pgfpathlineto{\pgfqpoint{1.402994in}{2.767110in}}%
\pgfpathlineto{\pgfqpoint{1.403312in}{2.634226in}}%
\pgfpathlineto{\pgfqpoint{1.403629in}{1.885675in}}%
\pgfpathlineto{\pgfqpoint{1.404264in}{2.839559in}}%
\pgfpathlineto{\pgfqpoint{1.404582in}{1.988742in}}%
\pgfpathlineto{\pgfqpoint{1.405216in}{2.926869in}}%
\pgfpathlineto{\pgfqpoint{1.405534in}{2.206503in}}%
\pgfpathlineto{\pgfqpoint{1.405851in}{2.021485in}}%
\pgfpathlineto{\pgfqpoint{1.406169in}{2.867552in}}%
\pgfpathlineto{\pgfqpoint{1.406804in}{1.897469in}}%
\pgfpathlineto{\pgfqpoint{1.407121in}{2.678445in}}%
\pgfpathlineto{\pgfqpoint{1.407439in}{2.734655in}}%
\pgfpathlineto{\pgfqpoint{1.407756in}{1.915441in}}%
\pgfpathlineto{\pgfqpoint{1.408391in}{2.901663in}}%
\pgfpathlineto{\pgfqpoint{1.408708in}{2.073065in}}%
\pgfpathlineto{\pgfqpoint{1.409026in}{2.154449in}}%
\pgfpathlineto{\pgfqpoint{1.409343in}{2.919280in}}%
\pgfpathlineto{\pgfqpoint{1.409978in}{1.949498in}}%
\pgfpathlineto{\pgfqpoint{1.410296in}{2.798006in}}%
\pgfpathlineto{\pgfqpoint{1.410931in}{1.881720in}}%
\pgfpathlineto{\pgfqpoint{1.411248in}{2.570769in}}%
\pgfpathlineto{\pgfqpoint{1.411566in}{2.809957in}}%
\pgfpathlineto{\pgfqpoint{1.411883in}{1.958269in}}%
\pgfpathlineto{\pgfqpoint{1.412518in}{2.919358in}}%
\pgfpathlineto{\pgfqpoint{1.412835in}{2.159320in}}%
\pgfpathlineto{\pgfqpoint{1.413153in}{2.054078in}}%
\pgfpathlineto{\pgfqpoint{1.413470in}{2.884524in}}%
\pgfpathlineto{\pgfqpoint{1.414105in}{1.905265in}}%
\pgfpathlineto{\pgfqpoint{1.414423in}{2.720464in}}%
\pgfpathlineto{\pgfqpoint{1.414740in}{2.692708in}}%
\pgfpathlineto{\pgfqpoint{1.415058in}{1.902474in}}%
\pgfpathlineto{\pgfqpoint{1.415693in}{2.878418in}}%
\pgfpathlineto{\pgfqpoint{1.416010in}{2.032645in}}%
\pgfpathlineto{\pgfqpoint{1.416327in}{2.192180in}}%
\pgfpathlineto{\pgfqpoint{1.416645in}{2.920663in}}%
\pgfpathlineto{\pgfqpoint{1.417280in}{1.973067in}}%
\pgfpathlineto{\pgfqpoint{1.417597in}{2.824116in}}%
\pgfpathlineto{\pgfqpoint{1.418232in}{1.879390in}}%
\pgfpathlineto{\pgfqpoint{1.418550in}{2.615899in}}%
\pgfpathlineto{\pgfqpoint{1.418867in}{2.773713in}}%
\pgfpathlineto{\pgfqpoint{1.419185in}{1.933755in}}%
\pgfpathlineto{\pgfqpoint{1.419820in}{2.907503in}}%
\pgfpathlineto{\pgfqpoint{1.420137in}{2.113978in}}%
\pgfpathlineto{\pgfqpoint{1.420454in}{2.088319in}}%
\pgfpathlineto{\pgfqpoint{1.420772in}{2.899104in}}%
\pgfpathlineto{\pgfqpoint{1.421407in}{1.920810in}}%
\pgfpathlineto{\pgfqpoint{1.421724in}{2.757850in}}%
\pgfpathlineto{\pgfqpoint{1.422042in}{2.650536in}}%
\pgfpathlineto{\pgfqpoint{1.422359in}{1.890808in}}%
\pgfpathlineto{\pgfqpoint{1.422994in}{2.848499in}}%
\pgfpathlineto{\pgfqpoint{1.423312in}{1.996613in}}%
\pgfpathlineto{\pgfqpoint{1.423946in}{2.917825in}}%
\pgfpathlineto{\pgfqpoint{1.424264in}{2.211914in}}%
\pgfpathlineto{\pgfqpoint{1.424581in}{1.998756in}}%
\pgfpathlineto{\pgfqpoint{1.424899in}{2.849319in}}%
\pgfpathlineto{\pgfqpoint{1.425534in}{1.885225in}}%
\pgfpathlineto{\pgfqpoint{1.425851in}{2.656204in}}%
\pgfpathlineto{\pgfqpoint{1.426169in}{2.736255in}}%
\pgfpathlineto{\pgfqpoint{1.426486in}{1.910094in}}%
\pgfpathlineto{\pgfqpoint{1.427121in}{2.890005in}}%
\pgfpathlineto{\pgfqpoint{1.427438in}{2.071414in}}%
\pgfpathlineto{\pgfqpoint{1.427756in}{2.126910in}}%
\pgfpathlineto{\pgfqpoint{1.428073in}{2.907956in}}%
\pgfpathlineto{\pgfqpoint{1.428708in}{1.940077in}}%
\pgfpathlineto{\pgfqpoint{1.429026in}{2.794169in}}%
\pgfpathlineto{\pgfqpoint{1.429661in}{1.885913in}}%
\pgfpathlineto{\pgfqpoint{1.429978in}{2.560367in}}%
\pgfpathlineto{\pgfqpoint{1.430296in}{2.816674in}}%
\pgfpathlineto{\pgfqpoint{1.430613in}{1.962100in}}%
\pgfpathlineto{\pgfqpoint{1.431248in}{2.910743in}}%
\pgfpathlineto{\pgfqpoint{1.431565in}{2.163192in}}%
\pgfpathlineto{\pgfqpoint{1.431883in}{2.030844in}}%
\pgfpathlineto{\pgfqpoint{1.432200in}{2.868620in}}%
\pgfpathlineto{\pgfqpoint{1.432835in}{1.892183in}}%
\pgfpathlineto{\pgfqpoint{1.433153in}{2.696916in}}%
\pgfpathlineto{\pgfqpoint{1.433470in}{2.692531in}}%
\pgfpathlineto{\pgfqpoint{1.433788in}{1.893525in}}%
\pgfpathlineto{\pgfqpoint{1.434423in}{2.869448in}}%
\pgfpathlineto{\pgfqpoint{1.434740in}{2.030843in}}%
\pgfpathlineto{\pgfqpoint{1.435057in}{2.167699in}}%
\pgfpathlineto{\pgfqpoint{1.435375in}{2.916994in}}%
\pgfpathlineto{\pgfqpoint{1.436010in}{1.968292in}}%
\pgfpathlineto{\pgfqpoint{1.436327in}{2.822362in}}%
\pgfpathlineto{\pgfqpoint{1.436962in}{1.881061in}}%
\pgfpathlineto{\pgfqpoint{1.437280in}{2.602088in}}%
\pgfpathlineto{\pgfqpoint{1.437597in}{2.778516in}}%
\pgfpathlineto{\pgfqpoint{1.437915in}{1.933672in}}%
\pgfpathlineto{\pgfqpoint{1.438550in}{2.899976in}}%
\pgfpathlineto{\pgfqpoint{1.438867in}{2.116429in}}%
\pgfpathlineto{\pgfqpoint{1.439184in}{2.065154in}}%
\pgfpathlineto{\pgfqpoint{1.439502in}{2.885514in}}%
\pgfpathlineto{\pgfqpoint{1.440137in}{1.906960in}}%
\pgfpathlineto{\pgfqpoint{1.440454in}{2.732054in}}%
\pgfpathlineto{\pgfqpoint{1.440772in}{2.648700in}}%
\pgfpathlineto{\pgfqpoint{1.441089in}{1.878182in}}%
\pgfpathlineto{\pgfqpoint{1.441724in}{2.842715in}}%
\pgfpathlineto{\pgfqpoint{1.442042in}{1.994865in}}%
\pgfpathlineto{\pgfqpoint{1.442676in}{2.921620in}}%
\pgfpathlineto{\pgfqpoint{1.442994in}{2.225753in}}%
\pgfpathlineto{\pgfqpoint{1.443311in}{1.996661in}}%
\pgfpathlineto{\pgfqpoint{1.443629in}{2.846448in}}%
\pgfpathlineto{\pgfqpoint{1.444264in}{1.882713in}}%
\pgfpathlineto{\pgfqpoint{1.444581in}{2.639077in}}%
\pgfpathlineto{\pgfqpoint{1.444899in}{2.739360in}}%
\pgfpathlineto{\pgfqpoint{1.445216in}{1.906542in}}%
\pgfpathlineto{\pgfqpoint{1.445851in}{2.883980in}}%
\pgfpathlineto{\pgfqpoint{1.446168in}{2.072691in}}%
\pgfpathlineto{\pgfqpoint{1.446486in}{2.104277in}}%
\pgfpathlineto{\pgfqpoint{1.446803in}{2.896364in}}%
\pgfpathlineto{\pgfqpoint{1.447438in}{1.923049in}}%
\pgfpathlineto{\pgfqpoint{1.447756in}{2.766890in}}%
\pgfpathlineto{\pgfqpoint{1.448391in}{1.870997in}}%
\pgfpathlineto{\pgfqpoint{1.448708in}{2.529772in}}%
\pgfpathlineto{\pgfqpoint{1.449026in}{2.814339in}}%
\pgfpathlineto{\pgfqpoint{1.449343in}{1.961236in}}%
\pgfpathlineto{\pgfqpoint{1.449978in}{2.918220in}}%
\pgfpathlineto{\pgfqpoint{1.450295in}{2.176905in}}%
\pgfpathlineto{\pgfqpoint{1.450613in}{2.027408in}}%
\pgfpathlineto{\pgfqpoint{1.450930in}{2.863505in}}%
\pgfpathlineto{\pgfqpoint{1.451565in}{1.884276in}}%
\pgfpathlineto{\pgfqpoint{1.451883in}{2.677342in}}%
\pgfpathlineto{\pgfqpoint{1.452200in}{2.694537in}}%
\pgfpathlineto{\pgfqpoint{1.452518in}{1.887219in}}%
\pgfpathlineto{\pgfqpoint{1.453153in}{2.864798in}}%
\pgfpathlineto{\pgfqpoint{1.453470in}{2.030897in}}%
\pgfpathlineto{\pgfqpoint{1.453787in}{2.145713in}}%
\pgfpathlineto{\pgfqpoint{1.454105in}{2.903290in}}%
\pgfpathlineto{\pgfqpoint{1.454740in}{1.946256in}}%
\pgfpathlineto{\pgfqpoint{1.455057in}{2.795732in}}%
\pgfpathlineto{\pgfqpoint{1.455692in}{1.865199in}}%
\pgfpathlineto{\pgfqpoint{1.456010in}{2.576177in}}%
\pgfpathlineto{\pgfqpoint{1.456327in}{2.780779in}}%
\pgfpathlineto{\pgfqpoint{1.456645in}{1.934517in}}%
\pgfpathlineto{\pgfqpoint{1.457280in}{2.908681in}}%
\pgfpathlineto{\pgfqpoint{1.457597in}{2.128905in}}%
\pgfpathlineto{\pgfqpoint{1.457914in}{2.059060in}}%
\pgfpathlineto{\pgfqpoint{1.458232in}{2.877888in}}%
\pgfpathlineto{\pgfqpoint{1.458867in}{1.894535in}}%
\pgfpathlineto{\pgfqpoint{1.459184in}{2.711482in}}%
\pgfpathlineto{\pgfqpoint{1.459502in}{2.650061in}}%
\pgfpathlineto{\pgfqpoint{1.459819in}{1.869751in}}%
\pgfpathlineto{\pgfqpoint{1.460454in}{2.839573in}}%
\pgfpathlineto{\pgfqpoint{1.460772in}{1.993913in}}%
\pgfpathlineto{\pgfqpoint{1.461406in}{2.904694in}}%
\pgfpathlineto{\pgfqpoint{1.461724in}{2.224451in}}%
\pgfpathlineto{\pgfqpoint{1.462041in}{1.971178in}}%
\pgfpathlineto{\pgfqpoint{1.462359in}{2.823330in}}%
\pgfpathlineto{\pgfqpoint{1.462994in}{1.867911in}}%
\pgfpathlineto{\pgfqpoint{1.463311in}{2.620108in}}%
\pgfpathlineto{\pgfqpoint{1.463629in}{2.746692in}}%
\pgfpathlineto{\pgfqpoint{1.463946in}{1.909641in}}%
\pgfpathlineto{\pgfqpoint{1.464581in}{2.892035in}}%
\pgfpathlineto{\pgfqpoint{1.464898in}{2.082694in}}%
\pgfpathlineto{\pgfqpoint{1.465216in}{2.094406in}}%
\pgfpathlineto{\pgfqpoint{1.465533in}{2.886698in}}%
\pgfpathlineto{\pgfqpoint{1.466168in}{1.907342in}}%
\pgfpathlineto{\pgfqpoint{1.466486in}{2.746915in}}%
\pgfpathlineto{\pgfqpoint{1.466803in}{2.600775in}}%
\pgfpathlineto{\pgfqpoint{1.467121in}{1.861545in}}%
\pgfpathlineto{\pgfqpoint{1.467756in}{2.811973in}}%
\pgfpathlineto{\pgfqpoint{1.468073in}{1.958693in}}%
\pgfpathlineto{\pgfqpoint{1.468708in}{2.901064in}}%
\pgfpathlineto{\pgfqpoint{1.469025in}{2.173710in}}%
\pgfpathlineto{\pgfqpoint{1.469343in}{2.002147in}}%
\pgfpathlineto{\pgfqpoint{1.469660in}{2.844782in}}%
\pgfpathlineto{\pgfqpoint{1.470295in}{1.872184in}}%
\pgfpathlineto{\pgfqpoint{1.470613in}{2.666643in}}%
\pgfpathlineto{\pgfqpoint{1.470930in}{2.705135in}}%
\pgfpathlineto{\pgfqpoint{1.471248in}{1.891427in}}%
\pgfpathlineto{\pgfqpoint{1.471883in}{2.871471in}}%
\pgfpathlineto{\pgfqpoint{1.472200in}{2.038106in}}%
\pgfpathlineto{\pgfqpoint{1.472517in}{2.132422in}}%
\pgfpathlineto{\pgfqpoint{1.472835in}{2.892781in}}%
\pgfpathlineto{\pgfqpoint{1.473470in}{1.928813in}}%
\pgfpathlineto{\pgfqpoint{1.473787in}{2.777354in}}%
\pgfpathlineto{\pgfqpoint{1.474422in}{1.855229in}}%
\pgfpathlineto{\pgfqpoint{1.474740in}{2.556636in}}%
\pgfpathlineto{\pgfqpoint{1.475057in}{2.777554in}}%
\pgfpathlineto{\pgfqpoint{1.475375in}{1.929584in}}%
\pgfpathlineto{\pgfqpoint{1.476010in}{2.892625in}}%
\pgfpathlineto{\pgfqpoint{1.476327in}{2.124666in}}%
\pgfpathlineto{\pgfqpoint{1.476644in}{2.035371in}}%
\pgfpathlineto{\pgfqpoint{1.476962in}{2.863740in}}%
\pgfpathlineto{\pgfqpoint{1.477597in}{1.884660in}}%
\pgfpathlineto{\pgfqpoint{1.477914in}{2.706995in}}%
\pgfpathlineto{\pgfqpoint{1.478232in}{2.662381in}}%
\pgfpathlineto{\pgfqpoint{1.478549in}{1.874816in}}%
\pgfpathlineto{\pgfqpoint{1.479184in}{2.844025in}}%
\pgfpathlineto{\pgfqpoint{1.479502in}{1.997993in}}%
\pgfpathlineto{\pgfqpoint{1.480136in}{2.894033in}}%
\pgfpathlineto{\pgfqpoint{1.480454in}{2.224975in}}%
\pgfpathlineto{\pgfqpoint{1.480771in}{1.952791in}}%
\pgfpathlineto{\pgfqpoint{1.481089in}{2.806834in}}%
\pgfpathlineto{\pgfqpoint{1.481724in}{1.857413in}}%
\pgfpathlineto{\pgfqpoint{1.482041in}{2.600296in}}%
\pgfpathlineto{\pgfqpoint{1.482359in}{2.741177in}}%
\pgfpathlineto{\pgfqpoint{1.482676in}{1.901693in}}%
\pgfpathlineto{\pgfqpoint{1.483311in}{2.878255in}}%
\pgfpathlineto{\pgfqpoint{1.483628in}{2.078342in}}%
\pgfpathlineto{\pgfqpoint{1.483946in}{2.073345in}}%
\pgfpathlineto{\pgfqpoint{1.484263in}{2.876590in}}%
\pgfpathlineto{\pgfqpoint{1.484898in}{1.900855in}}%
\pgfpathlineto{\pgfqpoint{1.485216in}{2.746574in}}%
\pgfpathlineto{\pgfqpoint{1.485533in}{2.613025in}}%
\pgfpathlineto{\pgfqpoint{1.485851in}{1.865690in}}%
\pgfpathlineto{\pgfqpoint{1.486486in}{2.814284in}}%
\pgfpathlineto{\pgfqpoint{1.486803in}{1.959577in}}%
\pgfpathlineto{\pgfqpoint{1.487438in}{2.890682in}}%
\pgfpathlineto{\pgfqpoint{1.487755in}{2.173086in}}%
\pgfpathlineto{\pgfqpoint{1.488073in}{1.983534in}}%
\pgfpathlineto{\pgfqpoint{1.488390in}{2.830220in}}%
\pgfpathlineto{\pgfqpoint{1.489025in}{1.861259in}}%
\pgfpathlineto{\pgfqpoint{1.489343in}{2.644643in}}%
\pgfpathlineto{\pgfqpoint{1.489660in}{2.698162in}}%
\pgfpathlineto{\pgfqpoint{1.489978in}{1.881104in}}%
\pgfpathlineto{\pgfqpoint{1.490613in}{2.860203in}}%
\pgfpathlineto{\pgfqpoint{1.490930in}{2.033836in}}%
\pgfpathlineto{\pgfqpoint{1.491247in}{2.114027in}}%
\pgfpathlineto{\pgfqpoint{1.491565in}{2.887465in}}%
\pgfpathlineto{\pgfqpoint{1.492200in}{1.925842in}}%
\pgfpathlineto{\pgfqpoint{1.492517in}{2.778681in}}%
\pgfpathlineto{\pgfqpoint{1.493152in}{1.857560in}}%
\pgfpathlineto{\pgfqpoint{1.493470in}{2.548064in}}%
\pgfpathlineto{\pgfqpoint{1.493787in}{2.778111in}}%
\pgfpathlineto{\pgfqpoint{1.494105in}{1.927633in}}%
\pgfpathlineto{\pgfqpoint{1.494740in}{2.883015in}}%
\pgfpathlineto{\pgfqpoint{1.495057in}{2.123038in}}%
\pgfpathlineto{\pgfqpoint{1.495374in}{2.016908in}}%
\pgfpathlineto{\pgfqpoint{1.495692in}{2.850924in}}%
\pgfpathlineto{\pgfqpoint{1.496327in}{1.873227in}}%
\pgfpathlineto{\pgfqpoint{1.496644in}{2.683371in}}%
\pgfpathlineto{\pgfqpoint{1.496962in}{2.654411in}}%
\pgfpathlineto{\pgfqpoint{1.497279in}{1.862003in}}%
\pgfpathlineto{\pgfqpoint{1.497914in}{2.835628in}}%
\pgfpathlineto{\pgfqpoint{1.498232in}{1.994067in}}%
\pgfpathlineto{\pgfqpoint{1.498866in}{2.896019in}}%
\pgfpathlineto{\pgfqpoint{1.499184in}{2.233008in}}%
\pgfpathlineto{\pgfqpoint{1.499501in}{1.953199in}}%
\pgfpathlineto{\pgfqpoint{1.499819in}{2.807512in}}%
\pgfpathlineto{\pgfqpoint{1.500454in}{1.857002in}}%
\pgfpathlineto{\pgfqpoint{1.500771in}{2.589031in}}%
\pgfpathlineto{\pgfqpoint{1.501089in}{2.740243in}}%
\pgfpathlineto{\pgfqpoint{1.501406in}{1.897122in}}%
\pgfpathlineto{\pgfqpoint{1.502041in}{2.869712in}}%
\pgfpathlineto{\pgfqpoint{1.502358in}{2.075804in}}%
\pgfpathlineto{\pgfqpoint{1.502676in}{2.055344in}}%
\pgfpathlineto{\pgfqpoint{1.502993in}{2.865389in}}%
\pgfpathlineto{\pgfqpoint{1.503628in}{1.887107in}}%
\pgfpathlineto{\pgfqpoint{1.503946in}{2.721903in}}%
\pgfpathlineto{\pgfqpoint{1.504263in}{2.604821in}}%
\pgfpathlineto{\pgfqpoint{1.504581in}{1.851347in}}%
\pgfpathlineto{\pgfqpoint{1.505216in}{2.808265in}}%
\pgfpathlineto{\pgfqpoint{1.505533in}{1.955994in}}%
\pgfpathlineto{\pgfqpoint{1.506168in}{2.896870in}}%
\pgfpathlineto{\pgfqpoint{1.506485in}{2.181684in}}%
\pgfpathlineto{\pgfqpoint{1.506803in}{1.984310in}}%
\pgfpathlineto{\pgfqpoint{1.507120in}{2.829180in}}%
\pgfpathlineto{\pgfqpoint{1.507755in}{1.857098in}}%
\pgfpathlineto{\pgfqpoint{1.508073in}{2.630860in}}%
\pgfpathlineto{\pgfqpoint{1.508390in}{2.696008in}}%
\pgfpathlineto{\pgfqpoint{1.508708in}{1.874391in}}%
\pgfpathlineto{\pgfqpoint{1.509343in}{2.852566in}}%
\pgfpathlineto{\pgfqpoint{1.509660in}{2.030302in}}%
\pgfpathlineto{\pgfqpoint{1.509977in}{2.096725in}}%
\pgfpathlineto{\pgfqpoint{1.510295in}{2.875717in}}%
\pgfpathlineto{\pgfqpoint{1.510930in}{1.908564in}}%
\pgfpathlineto{\pgfqpoint{1.511247in}{2.754290in}}%
\pgfpathlineto{\pgfqpoint{1.511882in}{1.842435in}}%
\pgfpathlineto{\pgfqpoint{1.512200in}{2.526169in}}%
\pgfpathlineto{\pgfqpoint{1.512517in}{2.775602in}}%
\pgfpathlineto{\pgfqpoint{1.512835in}{1.925519in}}%
\pgfpathlineto{\pgfqpoint{1.513470in}{2.890911in}}%
\pgfpathlineto{\pgfqpoint{1.513787in}{2.131128in}}%
\pgfpathlineto{\pgfqpoint{1.514104in}{2.016210in}}%
\pgfpathlineto{\pgfqpoint{1.514422in}{2.846798in}}%
\pgfpathlineto{\pgfqpoint{1.515057in}{1.865088in}}%
\pgfpathlineto{\pgfqpoint{1.515374in}{2.668130in}}%
\pgfpathlineto{\pgfqpoint{1.515692in}{2.651252in}}%
\pgfpathlineto{\pgfqpoint{1.516009in}{1.853479in}}%
\pgfpathlineto{\pgfqpoint{1.516644in}{2.828856in}}%
\pgfpathlineto{\pgfqpoint{1.516962in}{1.989635in}}%
\pgfpathlineto{\pgfqpoint{1.517279in}{2.141012in}}%
\pgfpathlineto{\pgfqpoint{1.517596in}{2.880088in}}%
\pgfpathlineto{\pgfqpoint{1.518231in}{1.932349in}}%
\pgfpathlineto{\pgfqpoint{1.518549in}{2.785359in}}%
\pgfpathlineto{\pgfqpoint{1.519184in}{1.842364in}}%
\pgfpathlineto{\pgfqpoint{1.519501in}{2.571235in}}%
\pgfpathlineto{\pgfqpoint{1.519819in}{2.741179in}}%
\pgfpathlineto{\pgfqpoint{1.520136in}{1.896649in}}%
\pgfpathlineto{\pgfqpoint{1.520771in}{2.877359in}}%
\pgfpathlineto{\pgfqpoint{1.521088in}{2.082435in}}%
\pgfpathlineto{\pgfqpoint{1.521406in}{2.052454in}}%
\pgfpathlineto{\pgfqpoint{1.521723in}{2.859126in}}%
\pgfpathlineto{\pgfqpoint{1.522358in}{1.875931in}}%
\pgfpathlineto{\pgfqpoint{1.522676in}{2.705821in}}%
\pgfpathlineto{\pgfqpoint{1.522993in}{2.601060in}}%
\pgfpathlineto{\pgfqpoint{1.523311in}{1.841560in}}%
\pgfpathlineto{\pgfqpoint{1.523946in}{2.802308in}}%
\pgfpathlineto{\pgfqpoint{1.524263in}{1.950774in}}%
\pgfpathlineto{\pgfqpoint{1.524898in}{2.879184in}}%
\pgfpathlineto{\pgfqpoint{1.525215in}{2.174292in}}%
\pgfpathlineto{\pgfqpoint{1.525533in}{1.962541in}}%
\pgfpathlineto{\pgfqpoint{1.525850in}{2.810032in}}%
\pgfpathlineto{\pgfqpoint{1.526485in}{1.844355in}}%
\pgfpathlineto{\pgfqpoint{1.526803in}{2.619489in}}%
\pgfpathlineto{\pgfqpoint{1.527120in}{2.700359in}}%
\pgfpathlineto{\pgfqpoint{1.527438in}{1.875380in}}%
\pgfpathlineto{\pgfqpoint{1.528073in}{2.859130in}}%
\pgfpathlineto{\pgfqpoint{1.528390in}{2.034972in}}%
\pgfpathlineto{\pgfqpoint{1.528707in}{2.091189in}}%
\pgfpathlineto{\pgfqpoint{1.529025in}{2.867558in}}%
\pgfpathlineto{\pgfqpoint{1.529660in}{1.895253in}}%
\pgfpathlineto{\pgfqpoint{1.529977in}{2.738577in}}%
\pgfpathlineto{\pgfqpoint{1.530612in}{1.832072in}}%
\pgfpathlineto{\pgfqpoint{1.530930in}{2.510328in}}%
\pgfpathlineto{\pgfqpoint{1.531247in}{2.768791in}}%
\pgfpathlineto{\pgfqpoint{1.531565in}{1.918477in}}%
\pgfpathlineto{\pgfqpoint{1.532200in}{2.873071in}}%
\pgfpathlineto{\pgfqpoint{1.532517in}{2.122285in}}%
\pgfpathlineto{\pgfqpoint{1.532834in}{1.995570in}}%
\pgfpathlineto{\pgfqpoint{1.533152in}{2.831999in}}%
\pgfpathlineto{\pgfqpoint{1.533787in}{1.854944in}}%
\pgfpathlineto{\pgfqpoint{1.534104in}{2.662975in}}%
\pgfpathlineto{\pgfqpoint{1.534422in}{2.657790in}}%
\pgfpathlineto{\pgfqpoint{1.534739in}{1.856018in}}%
\pgfpathlineto{\pgfqpoint{1.535374in}{2.833355in}}%
\pgfpathlineto{\pgfqpoint{1.535692in}{1.991959in}}%
\pgfpathlineto{\pgfqpoint{1.536009in}{2.132859in}}%
\pgfpathlineto{\pgfqpoint{1.536326in}{2.870705in}}%
\pgfpathlineto{\pgfqpoint{1.536961in}{1.917831in}}%
\pgfpathlineto{\pgfqpoint{1.537279in}{2.770781in}}%
\pgfpathlineto{\pgfqpoint{1.537914in}{1.831964in}}%
\pgfpathlineto{\pgfqpoint{1.538231in}{2.557023in}}%
\pgfpathlineto{\pgfqpoint{1.538549in}{2.733079in}}%
\pgfpathlineto{\pgfqpoint{1.538866in}{1.887817in}}%
\pgfpathlineto{\pgfqpoint{1.539501in}{2.860568in}}%
\pgfpathlineto{\pgfqpoint{1.539819in}{2.073141in}}%
\pgfpathlineto{\pgfqpoint{1.540136in}{2.033493in}}%
\pgfpathlineto{\pgfqpoint{1.540453in}{2.847619in}}%
\pgfpathlineto{\pgfqpoint{1.541088in}{1.867884in}}%
\pgfpathlineto{\pgfqpoint{1.541406in}{2.705676in}}%
\pgfpathlineto{\pgfqpoint{1.541723in}{2.608119in}}%
\pgfpathlineto{\pgfqpoint{1.542041in}{1.844789in}}%
\pgfpathlineto{\pgfqpoint{1.542676in}{2.804595in}}%
\pgfpathlineto{\pgfqpoint{1.542993in}{1.950726in}}%
\pgfpathlineto{\pgfqpoint{1.543628in}{2.869300in}}%
\pgfpathlineto{\pgfqpoint{1.543945in}{2.169188in}}%
\pgfpathlineto{\pgfqpoint{1.544263in}{1.947642in}}%
\pgfpathlineto{\pgfqpoint{1.544580in}{2.796673in}}%
\pgfpathlineto{\pgfqpoint{1.545215in}{1.833907in}}%
\pgfpathlineto{\pgfqpoint{1.545533in}{2.604410in}}%
\pgfpathlineto{\pgfqpoint{1.545850in}{2.690148in}}%
\pgfpathlineto{\pgfqpoint{1.546168in}{1.864756in}}%
\pgfpathlineto{\pgfqpoint{1.546803in}{2.843742in}}%
\pgfpathlineto{\pgfqpoint{1.547120in}{2.025502in}}%
\pgfpathlineto{\pgfqpoint{1.547437in}{2.074731in}}%
\pgfpathlineto{\pgfqpoint{1.547755in}{2.859057in}}%
\pgfpathlineto{\pgfqpoint{1.548390in}{1.889733in}}%
\pgfpathlineto{\pgfqpoint{1.548707in}{2.741832in}}%
\pgfpathlineto{\pgfqpoint{1.549342in}{1.835391in}}%
\pgfpathlineto{\pgfqpoint{1.549660in}{2.511420in}}%
\pgfpathlineto{\pgfqpoint{1.549977in}{2.768759in}}%
\pgfpathlineto{\pgfqpoint{1.550295in}{1.916115in}}%
\pgfpathlineto{\pgfqpoint{1.550930in}{2.862732in}}%
\pgfpathlineto{\pgfqpoint{1.551247in}{2.116010in}}%
\pgfpathlineto{\pgfqpoint{1.551564in}{1.980712in}}%
\pgfpathlineto{\pgfqpoint{1.551882in}{2.819970in}}%
\pgfpathlineto{\pgfqpoint{1.552517in}{1.844626in}}%
\pgfpathlineto{\pgfqpoint{1.552834in}{2.646173in}}%
\pgfpathlineto{\pgfqpoint{1.553152in}{2.645875in}}%
\pgfpathlineto{\pgfqpoint{1.553469in}{1.843361in}}%
\pgfpathlineto{\pgfqpoint{1.554104in}{2.819970in}}%
\pgfpathlineto{\pgfqpoint{1.554422in}{1.982901in}}%
\pgfpathlineto{\pgfqpoint{1.554739in}{2.119075in}}%
\pgfpathlineto{\pgfqpoint{1.555056in}{2.864867in}}%
\pgfpathlineto{\pgfqpoint{1.555691in}{1.915929in}}%
\pgfpathlineto{\pgfqpoint{1.556009in}{2.775452in}}%
\pgfpathlineto{\pgfqpoint{1.556644in}{1.834412in}}%
\pgfpathlineto{\pgfqpoint{1.556961in}{2.556135in}}%
\pgfpathlineto{\pgfqpoint{1.557279in}{2.731170in}}%
\pgfpathlineto{\pgfqpoint{1.557596in}{1.883412in}}%
\pgfpathlineto{\pgfqpoint{1.558231in}{2.850324in}}%
\pgfpathlineto{\pgfqpoint{1.558549in}{2.066106in}}%
\pgfpathlineto{\pgfqpoint{1.558866in}{2.019014in}}%
\pgfpathlineto{\pgfqpoint{1.559183in}{2.836687in}}%
\pgfpathlineto{\pgfqpoint{1.559818in}{1.857760in}}%
\pgfpathlineto{\pgfqpoint{1.560136in}{2.687553in}}%
\pgfpathlineto{\pgfqpoint{1.560453in}{2.595450in}}%
\pgfpathlineto{\pgfqpoint{1.560771in}{1.830679in}}%
\pgfpathlineto{\pgfqpoint{1.561406in}{2.792937in}}%
\pgfpathlineto{\pgfqpoint{1.561723in}{1.941999in}}%
\pgfpathlineto{\pgfqpoint{1.562358in}{2.868694in}}%
\pgfpathlineto{\pgfqpoint{1.562675in}{2.169610in}}%
\pgfpathlineto{\pgfqpoint{1.562993in}{1.949614in}}%
\pgfpathlineto{\pgfqpoint{1.563310in}{2.801615in}}%
\pgfpathlineto{\pgfqpoint{1.563945in}{1.835209in}}%
\pgfpathlineto{\pgfqpoint{1.564263in}{2.601069in}}%
\pgfpathlineto{\pgfqpoint{1.564580in}{2.686459in}}%
\pgfpathlineto{\pgfqpoint{1.564898in}{1.858568in}}%
\pgfpathlineto{\pgfqpoint{1.565533in}{2.833681in}}%
\pgfpathlineto{\pgfqpoint{1.565850in}{2.017609in}}%
\pgfpathlineto{\pgfqpoint{1.566167in}{2.060969in}}%
\pgfpathlineto{\pgfqpoint{1.566485in}{2.849151in}}%
\pgfpathlineto{\pgfqpoint{1.567120in}{1.878766in}}%
\pgfpathlineto{\pgfqpoint{1.567437in}{2.722567in}}%
\pgfpathlineto{\pgfqpoint{1.568072in}{1.820074in}}%
\pgfpathlineto{\pgfqpoint{1.568390in}{2.490683in}}%
\pgfpathlineto{\pgfqpoint{1.568707in}{2.758631in}}%
\pgfpathlineto{\pgfqpoint{1.569025in}{1.907889in}}%
\pgfpathlineto{\pgfqpoint{1.569660in}{2.866737in}}%
\pgfpathlineto{\pgfqpoint{1.569977in}{2.117509in}}%
\pgfpathlineto{\pgfqpoint{1.570294in}{1.985179in}}%
\pgfpathlineto{\pgfqpoint{1.570612in}{2.824276in}}%
\pgfpathlineto{\pgfqpoint{1.571247in}{1.843899in}}%
\pgfpathlineto{\pgfqpoint{1.571564in}{2.640477in}}%
\pgfpathlineto{\pgfqpoint{1.571882in}{2.640470in}}%
\pgfpathlineto{\pgfqpoint{1.572199in}{1.835560in}}%
\pgfpathlineto{\pgfqpoint{1.572834in}{2.810093in}}%
\pgfpathlineto{\pgfqpoint{1.573152in}{1.974378in}}%
\pgfpathlineto{\pgfqpoint{1.573469in}{2.106034in}}%
\pgfpathlineto{\pgfqpoint{1.573786in}{2.855307in}}%
\pgfpathlineto{\pgfqpoint{1.574421in}{1.902861in}}%
\pgfpathlineto{\pgfqpoint{1.574739in}{2.755924in}}%
\pgfpathlineto{\pgfqpoint{1.575374in}{1.818632in}}%
\pgfpathlineto{\pgfqpoint{1.575691in}{2.538016in}}%
\pgfpathlineto{\pgfqpoint{1.576009in}{2.721912in}}%
\pgfpathlineto{\pgfqpoint{1.576326in}{1.875503in}}%
\pgfpathlineto{\pgfqpoint{1.576961in}{2.856514in}}%
\pgfpathlineto{\pgfqpoint{1.577279in}{2.068252in}}%
\pgfpathlineto{\pgfqpoint{1.577596in}{2.024234in}}%
\pgfpathlineto{\pgfqpoint{1.577913in}{2.838969in}}%
\pgfpathlineto{\pgfqpoint{1.578548in}{1.854250in}}%
\pgfpathlineto{\pgfqpoint{1.578866in}{2.679549in}}%
\pgfpathlineto{\pgfqpoint{1.579183in}{2.588477in}}%
\pgfpathlineto{\pgfqpoint{1.579501in}{1.821556in}}%
\pgfpathlineto{\pgfqpoint{1.580136in}{2.783083in}}%
\pgfpathlineto{\pgfqpoint{1.580453in}{1.932787in}}%
\pgfpathlineto{\pgfqpoint{1.581088in}{2.855621in}}%
\pgfpathlineto{\pgfqpoint{1.581405in}{2.159796in}}%
\pgfpathlineto{\pgfqpoint{1.581723in}{1.933468in}}%
\pgfpathlineto{\pgfqpoint{1.582040in}{2.782611in}}%
\pgfpathlineto{\pgfqpoint{1.582675in}{1.819747in}}%
\pgfpathlineto{\pgfqpoint{1.582993in}{2.586396in}}%
\pgfpathlineto{\pgfqpoint{1.583310in}{2.678649in}}%
\pgfpathlineto{\pgfqpoint{1.583628in}{1.851727in}}%
\pgfpathlineto{\pgfqpoint{1.584263in}{2.840551in}}%
\pgfpathlineto{\pgfqpoint{1.584580in}{2.019725in}}%
\pgfpathlineto{\pgfqpoint{1.584897in}{2.066092in}}%
\pgfpathlineto{\pgfqpoint{1.585215in}{2.848971in}}%
\pgfpathlineto{\pgfqpoint{1.585850in}{1.873208in}}%
\pgfpathlineto{\pgfqpoint{1.586167in}{2.712848in}}%
\pgfpathlineto{\pgfqpoint{1.586802in}{1.809689in}}%
\pgfpathlineto{\pgfqpoint{1.587120in}{2.478515in}}%
\pgfpathlineto{\pgfqpoint{1.587437in}{2.748882in}}%
\pgfpathlineto{\pgfqpoint{1.587755in}{1.898328in}}%
\pgfpathlineto{\pgfqpoint{1.588390in}{2.850479in}}%
\pgfpathlineto{\pgfqpoint{1.588707in}{2.105462in}}%
\pgfpathlineto{\pgfqpoint{1.589024in}{1.967592in}}%
\pgfpathlineto{\pgfqpoint{1.589342in}{2.806129in}}%
\pgfpathlineto{\pgfqpoint{1.589977in}{1.829789in}}%
\pgfpathlineto{\pgfqpoint{1.590294in}{2.629167in}}%
\pgfpathlineto{\pgfqpoint{1.590612in}{2.635052in}}%
\pgfpathlineto{\pgfqpoint{1.590929in}{1.830621in}}%
\pgfpathlineto{\pgfqpoint{1.591564in}{2.816328in}}%
\pgfpathlineto{\pgfqpoint{1.591882in}{1.975830in}}%
\pgfpathlineto{\pgfqpoint{1.592199in}{2.110042in}}%
\pgfpathlineto{\pgfqpoint{1.592516in}{2.852571in}}%
\pgfpathlineto{\pgfqpoint{1.593151in}{1.895096in}}%
\pgfpathlineto{\pgfqpoint{1.593469in}{2.744932in}}%
\pgfpathlineto{\pgfqpoint{1.594104in}{1.807734in}}%
\pgfpathlineto{\pgfqpoint{1.594421in}{2.526878in}}%
\pgfpathlineto{\pgfqpoint{1.594739in}{2.712231in}}%
\pgfpathlineto{\pgfqpoint{1.595056in}{1.865792in}}%
\pgfpathlineto{\pgfqpoint{1.595691in}{2.838434in}}%
\pgfpathlineto{\pgfqpoint{1.596009in}{2.054445in}}%
\pgfpathlineto{\pgfqpoint{1.596326in}{2.006574in}}%
\pgfpathlineto{\pgfqpoint{1.596643in}{2.823049in}}%
\pgfpathlineto{\pgfqpoint{1.597278in}{1.842910in}}%
\pgfpathlineto{\pgfqpoint{1.597596in}{2.671904in}}%
\pgfpathlineto{\pgfqpoint{1.597913in}{2.584804in}}%
\pgfpathlineto{\pgfqpoint{1.598231in}{1.818553in}}%
\pgfpathlineto{\pgfqpoint{1.598866in}{2.788266in}}%
\pgfpathlineto{\pgfqpoint{1.599183in}{1.933364in}}%
\pgfpathlineto{\pgfqpoint{1.599818in}{2.850789in}}%
\pgfpathlineto{\pgfqpoint{1.600135in}{2.152255in}}%
\pgfpathlineto{\pgfqpoint{1.600453in}{1.924488in}}%
\pgfpathlineto{\pgfqpoint{1.600770in}{2.771337in}}%
\pgfpathlineto{\pgfqpoint{1.601405in}{1.808874in}}%
\pgfpathlineto{\pgfqpoint{1.601723in}{2.576317in}}%
\pgfpathlineto{\pgfqpoint{1.602040in}{2.668683in}}%
\pgfpathlineto{\pgfqpoint{1.602358in}{1.841856in}}%
\pgfpathlineto{\pgfqpoint{1.602993in}{2.821756in}}%
\pgfpathlineto{\pgfqpoint{1.603310in}{2.004816in}}%
\pgfpathlineto{\pgfqpoint{1.603627in}{2.049402in}}%
\pgfpathlineto{\pgfqpoint{1.603945in}{2.835240in}}%
\pgfpathlineto{\pgfqpoint{1.604580in}{1.864098in}}%
\pgfpathlineto{\pgfqpoint{1.604897in}{2.709301in}}%
\pgfpathlineto{\pgfqpoint{1.605532in}{1.809386in}}%
\pgfpathlineto{\pgfqpoint{1.605850in}{2.488738in}}%
\pgfpathlineto{\pgfqpoint{1.606167in}{2.751931in}}%
\pgfpathlineto{\pgfqpoint{1.606485in}{1.897562in}}%
\pgfpathlineto{\pgfqpoint{1.607120in}{2.843896in}}%
\pgfpathlineto{\pgfqpoint{1.607437in}{2.096361in}}%
\pgfpathlineto{\pgfqpoint{1.607754in}{1.958051in}}%
\pgfpathlineto{\pgfqpoint{1.608072in}{2.794970in}}%
\pgfpathlineto{\pgfqpoint{1.608707in}{1.819306in}}%
\pgfpathlineto{\pgfqpoint{1.609024in}{2.620206in}}%
\pgfpathlineto{\pgfqpoint{1.609342in}{2.623204in}}%
\pgfpathlineto{\pgfqpoint{1.609659in}{1.819838in}}%
\pgfpathlineto{\pgfqpoint{1.610294in}{2.797427in}}%
\pgfpathlineto{\pgfqpoint{1.610612in}{1.960644in}}%
\pgfpathlineto{\pgfqpoint{1.610929in}{2.095091in}}%
\pgfpathlineto{\pgfqpoint{1.611246in}{2.841200in}}%
\pgfpathlineto{\pgfqpoint{1.611881in}{1.888877in}}%
\pgfpathlineto{\pgfqpoint{1.612199in}{2.744707in}}%
\pgfpathlineto{\pgfqpoint{1.612834in}{1.808980in}}%
\pgfpathlineto{\pgfqpoint{1.613151in}{2.537549in}}%
\pgfpathlineto{\pgfqpoint{1.613469in}{2.713296in}}%
\pgfpathlineto{\pgfqpoint{1.613786in}{1.864058in}}%
\pgfpathlineto{\pgfqpoint{1.614421in}{2.830009in}}%
\pgfpathlineto{\pgfqpoint{1.614739in}{2.044179in}}%
\pgfpathlineto{\pgfqpoint{1.615056in}{1.996759in}}%
\pgfpathlineto{\pgfqpoint{1.615373in}{2.812121in}}%
\pgfpathlineto{\pgfqpoint{1.616008in}{1.833136in}}%
\pgfpathlineto{\pgfqpoint{1.616326in}{2.663709in}}%
\pgfpathlineto{\pgfqpoint{1.616643in}{2.571382in}}%
\pgfpathlineto{\pgfqpoint{1.616961in}{1.807099in}}%
\pgfpathlineto{\pgfqpoint{1.617596in}{2.769327in}}%
\pgfpathlineto{\pgfqpoint{1.617913in}{1.918023in}}%
\pgfpathlineto{\pgfqpoint{1.618548in}{2.841040in}}%
\pgfpathlineto{\pgfqpoint{1.618865in}{2.142247in}}%
\pgfpathlineto{\pgfqpoint{1.619183in}{1.920268in}}%
\pgfpathlineto{\pgfqpoint{1.619500in}{2.773831in}}%
\pgfpathlineto{\pgfqpoint{1.620135in}{1.811782in}}%
\pgfpathlineto{\pgfqpoint{1.620453in}{2.586514in}}%
\pgfpathlineto{\pgfqpoint{1.620770in}{2.667367in}}%
\pgfpathlineto{\pgfqpoint{1.621088in}{1.839042in}}%
\pgfpathlineto{\pgfqpoint{1.621723in}{2.811836in}}%
\pgfpathlineto{\pgfqpoint{1.622040in}{1.993580in}}%
\pgfpathlineto{\pgfqpoint{1.622357in}{2.039857in}}%
\pgfpathlineto{\pgfqpoint{1.622675in}{2.824498in}}%
\pgfpathlineto{\pgfqpoint{1.623310in}{1.854981in}}%
\pgfpathlineto{\pgfqpoint{1.623627in}{2.700457in}}%
\pgfpathlineto{\pgfqpoint{1.624262in}{1.796742in}}%
\pgfpathlineto{\pgfqpoint{1.624580in}{2.470748in}}%
\pgfpathlineto{\pgfqpoint{1.624897in}{2.733559in}}%
\pgfpathlineto{\pgfqpoint{1.625215in}{1.882821in}}%
\pgfpathlineto{\pgfqpoint{1.625850in}{2.835303in}}%
\pgfpathlineto{\pgfqpoint{1.626167in}{2.086575in}}%
\pgfpathlineto{\pgfqpoint{1.626484in}{1.956382in}}%
\pgfpathlineto{\pgfqpoint{1.626802in}{2.799306in}}%
\pgfpathlineto{\pgfqpoint{1.627437in}{1.823350in}}%
\pgfpathlineto{\pgfqpoint{1.627754in}{2.629104in}}%
\pgfpathlineto{\pgfqpoint{1.628072in}{2.619772in}}%
\pgfpathlineto{\pgfqpoint{1.628389in}{1.816225in}}%
\pgfpathlineto{\pgfqpoint{1.629024in}{2.786369in}}%
\pgfpathlineto{\pgfqpoint{1.629342in}{1.948723in}}%
\pgfpathlineto{\pgfqpoint{1.629659in}{2.085902in}}%
\pgfpathlineto{\pgfqpoint{1.629976in}{2.830687in}}%
\pgfpathlineto{\pgfqpoint{1.630611in}{1.880753in}}%
\pgfpathlineto{\pgfqpoint{1.630929in}{2.735131in}}%
\pgfpathlineto{\pgfqpoint{1.631564in}{1.795808in}}%
\pgfpathlineto{\pgfqpoint{1.631881in}{2.519717in}}%
\pgfpathlineto{\pgfqpoint{1.632199in}{2.695030in}}%
\pgfpathlineto{\pgfqpoint{1.632516in}{1.849561in}}%
\pgfpathlineto{\pgfqpoint{1.633151in}{2.822602in}}%
\pgfpathlineto{\pgfqpoint{1.633469in}{2.034863in}}%
\pgfpathlineto{\pgfqpoint{1.633786in}{1.999216in}}%
\pgfpathlineto{\pgfqpoint{1.634103in}{2.818307in}}%
\pgfpathlineto{\pgfqpoint{1.634738in}{1.838367in}}%
\pgfpathlineto{\pgfqpoint{1.635056in}{2.670389in}}%
\pgfpathlineto{\pgfqpoint{1.635373in}{2.565775in}}%
\pgfpathlineto{\pgfqpoint{1.635691in}{1.802619in}}%
\pgfpathlineto{\pgfqpoint{1.636326in}{2.757289in}}%
\pgfpathlineto{\pgfqpoint{1.636643in}{1.905578in}}%
\pgfpathlineto{\pgfqpoint{1.637278in}{2.830504in}}%
\pgfpathlineto{\pgfqpoint{1.637595in}{2.129266in}}%
\pgfpathlineto{\pgfqpoint{1.637913in}{1.913083in}}%
\pgfpathlineto{\pgfqpoint{1.638230in}{2.762809in}}%
\pgfpathlineto{\pgfqpoint{1.638865in}{1.797966in}}%
\pgfpathlineto{\pgfqpoint{1.639183in}{2.569551in}}%
\pgfpathlineto{\pgfqpoint{1.639500in}{2.649332in}}%
\pgfpathlineto{\pgfqpoint{1.639818in}{1.825140in}}%
\pgfpathlineto{\pgfqpoint{1.640453in}{2.806825in}}%
\pgfpathlineto{\pgfqpoint{1.640770in}{1.985314in}}%
\pgfpathlineto{\pgfqpoint{1.641087in}{2.046387in}}%
\pgfpathlineto{\pgfqpoint{1.641405in}{2.832250in}}%
\pgfpathlineto{\pgfqpoint{1.642040in}{1.860495in}}%
\pgfpathlineto{\pgfqpoint{1.642357in}{2.704484in}}%
\pgfpathlineto{\pgfqpoint{1.642992in}{1.791386in}}%
\pgfpathlineto{\pgfqpoint{1.643310in}{2.466410in}}%
\pgfpathlineto{\pgfqpoint{1.643627in}{2.720617in}}%
\pgfpathlineto{\pgfqpoint{1.643945in}{1.870187in}}%
\pgfpathlineto{\pgfqpoint{1.644580in}{2.824590in}}%
\pgfpathlineto{\pgfqpoint{1.644897in}{2.073142in}}%
\pgfpathlineto{\pgfqpoint{1.645214in}{1.949785in}}%
\pgfpathlineto{\pgfqpoint{1.645532in}{2.786612in}}%
\pgfpathlineto{\pgfqpoint{1.646167in}{1.809161in}}%
\pgfpathlineto{\pgfqpoint{1.646484in}{2.613515in}}%
\pgfpathlineto{\pgfqpoint{1.646802in}{2.601735in}}%
\pgfpathlineto{\pgfqpoint{1.647119in}{1.803130in}}%
\pgfpathlineto{\pgfqpoint{1.647754in}{2.783030in}}%
\pgfpathlineto{\pgfqpoint{1.648072in}{1.942123in}}%
\pgfpathlineto{\pgfqpoint{1.648389in}{2.095910in}}%
\pgfpathlineto{\pgfqpoint{1.648706in}{2.838253in}}%
\pgfpathlineto{\pgfqpoint{1.649341in}{1.886122in}}%
\pgfpathlineto{\pgfqpoint{1.649659in}{2.736568in}}%
\pgfpathlineto{\pgfqpoint{1.650294in}{1.789500in}}%
\pgfpathlineto{\pgfqpoint{1.650611in}{2.514792in}}%
\pgfpathlineto{\pgfqpoint{1.650929in}{2.681029in}}%
\pgfpathlineto{\pgfqpoint{1.651246in}{1.836778in}}%
\pgfpathlineto{\pgfqpoint{1.651881in}{2.811143in}}%
\pgfpathlineto{\pgfqpoint{1.652199in}{2.021068in}}%
\pgfpathlineto{\pgfqpoint{1.652516in}{1.991057in}}%
\pgfpathlineto{\pgfqpoint{1.652833in}{2.803480in}}%
\pgfpathlineto{\pgfqpoint{1.653468in}{1.824116in}}%
\pgfpathlineto{\pgfqpoint{1.653786in}{2.656776in}}%
\pgfpathlineto{\pgfqpoint{1.654103in}{2.548054in}}%
\pgfpathlineto{\pgfqpoint{1.654421in}{1.790721in}}%
\pgfpathlineto{\pgfqpoint{1.655056in}{2.755476in}}%
\pgfpathlineto{\pgfqpoint{1.655373in}{1.900746in}}%
\pgfpathlineto{\pgfqpoint{1.656008in}{2.836085in}}%
\pgfpathlineto{\pgfqpoint{1.656325in}{2.122960in}}%
\pgfpathlineto{\pgfqpoint{1.656643in}{1.917545in}}%
\pgfpathlineto{\pgfqpoint{1.656960in}{2.761316in}}%
\pgfpathlineto{\pgfqpoint{1.657595in}{1.790564in}}%
\pgfpathlineto{\pgfqpoint{1.657913in}{2.564206in}}%
\pgfpathlineto{\pgfqpoint{1.658230in}{2.634408in}}%
\pgfpathlineto{\pgfqpoint{1.658548in}{1.812499in}}%
\pgfpathlineto{\pgfqpoint{1.659183in}{2.792839in}}%
\pgfpathlineto{\pgfqpoint{1.659500in}{1.970350in}}%
\pgfpathlineto{\pgfqpoint{1.659817in}{2.036886in}}%
\pgfpathlineto{\pgfqpoint{1.660135in}{2.814785in}}%
\pgfpathlineto{\pgfqpoint{1.660770in}{1.847082in}}%
\pgfpathlineto{\pgfqpoint{1.661087in}{2.693395in}}%
\pgfpathlineto{\pgfqpoint{1.661722in}{1.781270in}}%
\pgfpathlineto{\pgfqpoint{1.662040in}{2.468276in}}%
\pgfpathlineto{\pgfqpoint{1.662357in}{2.718787in}}%
\pgfpathlineto{\pgfqpoint{1.662675in}{1.866418in}}%
\pgfpathlineto{\pgfqpoint{1.663310in}{2.828243in}}%
\pgfpathlineto{\pgfqpoint{1.663627in}{2.065770in}}%
\pgfpathlineto{\pgfqpoint{1.663944in}{1.953593in}}%
\pgfpathlineto{\pgfqpoint{1.664262in}{2.782622in}}%
\pgfpathlineto{\pgfqpoint{1.664897in}{1.801636in}}%
\pgfpathlineto{\pgfqpoint{1.665214in}{2.607623in}}%
\pgfpathlineto{\pgfqpoint{1.665532in}{2.585775in}}%
\pgfpathlineto{\pgfqpoint{1.665849in}{1.790843in}}%
\pgfpathlineto{\pgfqpoint{1.666484in}{2.766514in}}%
\pgfpathlineto{\pgfqpoint{1.666802in}{1.925974in}}%
\pgfpathlineto{\pgfqpoint{1.667119in}{2.085228in}}%
\pgfpathlineto{\pgfqpoint{1.667436in}{2.819634in}}%
\pgfpathlineto{\pgfqpoint{1.668071in}{1.874429in}}%
\pgfpathlineto{\pgfqpoint{1.668389in}{2.727489in}}%
\pgfpathlineto{\pgfqpoint{1.669024in}{1.781520in}}%
\pgfpathlineto{\pgfqpoint{1.669341in}{2.520484in}}%
\pgfpathlineto{\pgfqpoint{1.669659in}{2.679603in}}%
\pgfpathlineto{\pgfqpoint{1.669976in}{1.834381in}}%
\pgfpathlineto{\pgfqpoint{1.670611in}{2.812032in}}%
\pgfpathlineto{\pgfqpoint{1.670929in}{2.012634in}}%
\pgfpathlineto{\pgfqpoint{1.671246in}{1.993877in}}%
\pgfpathlineto{\pgfqpoint{1.671563in}{2.797201in}}%
\pgfpathlineto{\pgfqpoint{1.672198in}{1.816526in}}%
\pgfpathlineto{\pgfqpoint{1.672516in}{2.650467in}}%
\pgfpathlineto{\pgfqpoint{1.672833in}{2.531462in}}%
\pgfpathlineto{\pgfqpoint{1.673151in}{1.779212in}}%
\pgfpathlineto{\pgfqpoint{1.673786in}{2.736286in}}%
\pgfpathlineto{\pgfqpoint{1.674103in}{1.883406in}}%
\pgfpathlineto{\pgfqpoint{1.674738in}{2.817515in}}%
\pgfpathlineto{\pgfqpoint{1.675055in}{2.102674in}}%
\pgfpathlineto{\pgfqpoint{1.675373in}{1.908198in}}%
\pgfpathlineto{\pgfqpoint{1.675690in}{2.754572in}}%
\pgfpathlineto{\pgfqpoint{1.676325in}{1.785392in}}%
\pgfpathlineto{\pgfqpoint{1.676643in}{2.572926in}}%
\pgfpathlineto{\pgfqpoint{1.676960in}{2.632158in}}%
\pgfpathlineto{\pgfqpoint{1.677278in}{1.810952in}}%
\pgfpathlineto{\pgfqpoint{1.677913in}{2.791417in}}%
\pgfpathlineto{\pgfqpoint{1.678230in}{1.961137in}}%
\pgfpathlineto{\pgfqpoint{1.678547in}{2.039518in}}%
\pgfpathlineto{\pgfqpoint{1.678865in}{2.806885in}}%
\pgfpathlineto{\pgfqpoint{1.679500in}{1.839971in}}%
\pgfpathlineto{\pgfqpoint{1.679817in}{2.686994in}}%
\pgfpathlineto{\pgfqpoint{1.680452in}{1.770724in}}%
\pgfpathlineto{\pgfqpoint{1.680770in}{2.464234in}}%
\pgfpathlineto{\pgfqpoint{1.681087in}{2.698044in}}%
\pgfpathlineto{\pgfqpoint{1.681405in}{1.849024in}}%
\pgfpathlineto{\pgfqpoint{1.682040in}{2.809594in}}%
\pgfpathlineto{\pgfqpoint{1.682357in}{2.045567in}}%
\pgfpathlineto{\pgfqpoint{1.682674in}{1.946790in}}%
\pgfpathlineto{\pgfqpoint{1.682992in}{2.777244in}}%
\pgfpathlineto{\pgfqpoint{1.683627in}{1.798424in}}%
\pgfpathlineto{\pgfqpoint{1.683944in}{2.618942in}}%
\pgfpathlineto{\pgfqpoint{1.684262in}{2.582610in}}%
\pgfpathlineto{\pgfqpoint{1.684579in}{1.790817in}}%
\pgfpathlineto{\pgfqpoint{1.685214in}{2.762035in}}%
\pgfpathlineto{\pgfqpoint{1.685532in}{1.916032in}}%
\pgfpathlineto{\pgfqpoint{1.686166in}{2.810118in}}%
\pgfpathlineto{\pgfqpoint{1.686484in}{2.142498in}}%
\pgfpathlineto{\pgfqpoint{1.686801in}{1.868163in}}%
\pgfpathlineto{\pgfqpoint{1.687119in}{2.720999in}}%
\pgfpathlineto{\pgfqpoint{1.687754in}{1.772229in}}%
\pgfpathlineto{\pgfqpoint{1.688071in}{2.514940in}}%
\pgfpathlineto{\pgfqpoint{1.688389in}{2.656677in}}%
\pgfpathlineto{\pgfqpoint{1.688706in}{1.816876in}}%
\pgfpathlineto{\pgfqpoint{1.689341in}{2.793680in}}%
\pgfpathlineto{\pgfqpoint{1.689659in}{1.993206in}}%
\pgfpathlineto{\pgfqpoint{1.689976in}{1.989778in}}%
\pgfpathlineto{\pgfqpoint{1.690293in}{2.792976in}}%
\pgfpathlineto{\pgfqpoint{1.690928in}{1.816214in}}%
\pgfpathlineto{\pgfqpoint{1.691246in}{2.663078in}}%
\pgfpathlineto{\pgfqpoint{1.691563in}{2.526542in}}%
\pgfpathlineto{\pgfqpoint{1.691881in}{1.780226in}}%
\pgfpathlineto{\pgfqpoint{1.692516in}{2.729041in}}%
\pgfpathlineto{\pgfqpoint{1.692833in}{1.872915in}}%
\pgfpathlineto{\pgfqpoint{1.693468in}{2.806439in}}%
\pgfpathlineto{\pgfqpoint{1.693785in}{2.084094in}}%
\pgfpathlineto{\pgfqpoint{1.694103in}{1.902880in}}%
\pgfpathlineto{\pgfqpoint{1.694420in}{2.747934in}}%
\pgfpathlineto{\pgfqpoint{1.695055in}{1.777625in}}%
\pgfpathlineto{\pgfqpoint{1.695373in}{2.565904in}}%
\pgfpathlineto{\pgfqpoint{1.695690in}{2.607913in}}%
\pgfpathlineto{\pgfqpoint{1.696008in}{1.794057in}}%
\pgfpathlineto{\pgfqpoint{1.696643in}{2.772795in}}%
\pgfpathlineto{\pgfqpoint{1.696960in}{1.942208in}}%
\pgfpathlineto{\pgfqpoint{1.697277in}{2.037807in}}%
\pgfpathlineto{\pgfqpoint{1.697595in}{2.802546in}}%
\pgfpathlineto{\pgfqpoint{1.698230in}{1.842245in}}%
\pgfpathlineto{\pgfqpoint{1.698547in}{2.699911in}}%
\pgfpathlineto{\pgfqpoint{1.699182in}{1.773080in}}%
\pgfpathlineto{\pgfqpoint{1.699500in}{2.477839in}}%
\pgfpathlineto{\pgfqpoint{1.699817in}{2.688301in}}%
\pgfpathlineto{\pgfqpoint{1.700135in}{1.838396in}}%
\pgfpathlineto{\pgfqpoint{1.700770in}{2.797003in}}%
\pgfpathlineto{\pgfqpoint{1.701087in}{2.026692in}}%
\pgfpathlineto{\pgfqpoint{1.701404in}{1.942730in}}%
\pgfpathlineto{\pgfqpoint{1.701722in}{2.770134in}}%
\pgfpathlineto{\pgfqpoint{1.702357in}{1.792006in}}%
\pgfpathlineto{\pgfqpoint{1.702674in}{2.610413in}}%
\pgfpathlineto{\pgfqpoint{1.702992in}{2.556847in}}%
\pgfpathlineto{\pgfqpoint{1.703309in}{1.774145in}}%
\pgfpathlineto{\pgfqpoint{1.703944in}{2.743567in}}%
\pgfpathlineto{\pgfqpoint{1.704262in}{1.898241in}}%
\pgfpathlineto{\pgfqpoint{1.704896in}{2.805752in}}%
\pgfpathlineto{\pgfqpoint{1.705214in}{2.127527in}}%
\pgfpathlineto{\pgfqpoint{1.705531in}{1.874199in}}%
\pgfpathlineto{\pgfqpoint{1.705849in}{2.733194in}}%
\pgfpathlineto{\pgfqpoint{1.706484in}{1.775554in}}%
\pgfpathlineto{\pgfqpoint{1.706801in}{2.527128in}}%
\pgfpathlineto{\pgfqpoint{1.707119in}{2.644602in}}%
\pgfpathlineto{\pgfqpoint{1.707436in}{1.806342in}}%
\pgfpathlineto{\pgfqpoint{1.708071in}{2.779685in}}%
\pgfpathlineto{\pgfqpoint{1.708389in}{1.974392in}}%
\pgfpathlineto{\pgfqpoint{1.708706in}{1.986890in}}%
\pgfpathlineto{\pgfqpoint{1.709023in}{2.785408in}}%
\pgfpathlineto{\pgfqpoint{1.709658in}{1.811071in}}%
\pgfpathlineto{\pgfqpoint{1.709976in}{2.653390in}}%
\pgfpathlineto{\pgfqpoint{1.710293in}{2.499978in}}%
\pgfpathlineto{\pgfqpoint{1.710611in}{1.764216in}}%
\pgfpathlineto{\pgfqpoint{1.711246in}{2.710310in}}%
\pgfpathlineto{\pgfqpoint{1.711563in}{1.856224in}}%
\pgfpathlineto{\pgfqpoint{1.712198in}{2.803140in}}%
\pgfpathlineto{\pgfqpoint{1.712515in}{2.070273in}}%
\pgfpathlineto{\pgfqpoint{1.712833in}{1.912665in}}%
\pgfpathlineto{\pgfqpoint{1.713150in}{2.759123in}}%
\pgfpathlineto{\pgfqpoint{1.713785in}{1.782157in}}%
\pgfpathlineto{\pgfqpoint{1.714103in}{2.576325in}}%
\pgfpathlineto{\pgfqpoint{1.714420in}{2.593913in}}%
\pgfpathlineto{\pgfqpoint{1.714738in}{1.784113in}}%
\pgfpathlineto{\pgfqpoint{1.715373in}{2.757396in}}%
\pgfpathlineto{\pgfqpoint{1.715690in}{1.923455in}}%
\pgfpathlineto{\pgfqpoint{1.716008in}{2.036285in}}%
\pgfpathlineto{\pgfqpoint{1.716325in}{2.794155in}}%
\pgfpathlineto{\pgfqpoint{1.716960in}{1.837745in}}%
\pgfpathlineto{\pgfqpoint{1.717277in}{2.689266in}}%
\pgfpathlineto{\pgfqpoint{1.717912in}{1.757979in}}%
\pgfpathlineto{\pgfqpoint{1.718230in}{2.470920in}}%
\pgfpathlineto{\pgfqpoint{1.718547in}{2.668939in}}%
\pgfpathlineto{\pgfqpoint{1.718865in}{1.822986in}}%
\pgfpathlineto{\pgfqpoint{1.719500in}{2.795092in}}%
\pgfpathlineto{\pgfqpoint{1.719817in}{2.013679in}}%
\pgfpathlineto{\pgfqpoint{1.720134in}{1.955800in}}%
\pgfpathlineto{\pgfqpoint{1.720452in}{2.780570in}}%
\pgfpathlineto{\pgfqpoint{1.721087in}{1.797307in}}%
\pgfpathlineto{\pgfqpoint{1.721404in}{2.618650in}}%
\pgfpathlineto{\pgfqpoint{1.721722in}{2.540653in}}%
\pgfpathlineto{\pgfqpoint{1.722039in}{1.764753in}}%
\pgfpathlineto{\pgfqpoint{1.722674in}{2.726751in}}%
\pgfpathlineto{\pgfqpoint{1.722992in}{1.879958in}}%
\pgfpathlineto{\pgfqpoint{1.723626in}{2.796218in}}%
\pgfpathlineto{\pgfqpoint{1.723944in}{2.106249in}}%
\pgfpathlineto{\pgfqpoint{1.724261in}{1.869781in}}%
\pgfpathlineto{\pgfqpoint{1.724579in}{2.721614in}}%
\pgfpathlineto{\pgfqpoint{1.725214in}{1.761593in}}%
\pgfpathlineto{\pgfqpoint{1.725531in}{2.521321in}}%
\pgfpathlineto{\pgfqpoint{1.725849in}{2.624399in}}%
\pgfpathlineto{\pgfqpoint{1.726166in}{1.792380in}}%
\pgfpathlineto{\pgfqpoint{1.726801in}{2.778525in}}%
\pgfpathlineto{\pgfqpoint{1.727119in}{1.962611in}}%
\pgfpathlineto{\pgfqpoint{1.727436in}{2.002756in}}%
\pgfpathlineto{\pgfqpoint{1.727753in}{2.794603in}}%
\pgfpathlineto{\pgfqpoint{1.728388in}{1.816873in}}%
\pgfpathlineto{\pgfqpoint{1.728706in}{2.658940in}}%
\pgfpathlineto{\pgfqpoint{1.729341in}{1.755594in}}%
\pgfpathlineto{\pgfqpoint{1.729658in}{2.416825in}}%
\pgfpathlineto{\pgfqpoint{1.729976in}{2.691920in}}%
\pgfpathlineto{\pgfqpoint{1.730293in}{1.838472in}}%
\pgfpathlineto{\pgfqpoint{1.730928in}{2.790311in}}%
\pgfpathlineto{\pgfqpoint{1.731245in}{2.047920in}}%
\pgfpathlineto{\pgfqpoint{1.731563in}{1.907597in}}%
\pgfpathlineto{\pgfqpoint{1.731880in}{2.746666in}}%
\pgfpathlineto{\pgfqpoint{1.732515in}{1.769748in}}%
\pgfpathlineto{\pgfqpoint{1.732833in}{2.571616in}}%
\pgfpathlineto{\pgfqpoint{1.733150in}{2.572753in}}%
\pgfpathlineto{\pgfqpoint{1.733468in}{1.771627in}}%
\pgfpathlineto{\pgfqpoint{1.734103in}{2.755626in}}%
\pgfpathlineto{\pgfqpoint{1.734420in}{1.913044in}}%
\pgfpathlineto{\pgfqpoint{1.734738in}{2.054580in}}%
\pgfpathlineto{\pgfqpoint{1.735055in}{2.800789in}}%
\pgfpathlineto{\pgfqpoint{1.735690in}{1.844363in}}%
\pgfpathlineto{\pgfqpoint{1.736007in}{2.692695in}}%
\pgfpathlineto{\pgfqpoint{1.736642in}{1.749884in}}%
\pgfpathlineto{\pgfqpoint{1.736960in}{2.472519in}}%
\pgfpathlineto{\pgfqpoint{1.737277in}{2.649073in}}%
\pgfpathlineto{\pgfqpoint{1.737595in}{1.806115in}}%
\pgfpathlineto{\pgfqpoint{1.738230in}{2.778512in}}%
\pgfpathlineto{\pgfqpoint{1.738547in}{1.990536in}}%
\pgfpathlineto{\pgfqpoint{1.738864in}{1.950967in}}%
\pgfpathlineto{\pgfqpoint{1.739182in}{2.766162in}}%
\pgfpathlineto{\pgfqpoint{1.739817in}{1.786852in}}%
\pgfpathlineto{\pgfqpoint{1.740134in}{2.615370in}}%
\pgfpathlineto{\pgfqpoint{1.740452in}{2.518747in}}%
\pgfpathlineto{\pgfqpoint{1.740769in}{1.754308in}}%
\pgfpathlineto{\pgfqpoint{1.741404in}{2.723929in}}%
\pgfpathlineto{\pgfqpoint{1.741722in}{1.871128in}}%
\pgfpathlineto{\pgfqpoint{1.742356in}{2.799885in}}%
\pgfpathlineto{\pgfqpoint{1.742674in}{2.090357in}}%
\pgfpathlineto{\pgfqpoint{1.742991in}{1.876812in}}%
\pgfpathlineto{\pgfqpoint{1.743309in}{2.722043in}}%
\pgfpathlineto{\pgfqpoint{1.743944in}{1.754214in}}%
\pgfpathlineto{\pgfqpoint{1.744261in}{2.522684in}}%
\pgfpathlineto{\pgfqpoint{1.744579in}{2.602924in}}%
\pgfpathlineto{\pgfqpoint{1.744896in}{1.776575in}}%
\pgfpathlineto{\pgfqpoint{1.745531in}{2.758000in}}%
\pgfpathlineto{\pgfqpoint{1.745849in}{1.939105in}}%
\pgfpathlineto{\pgfqpoint{1.746166in}{1.997940in}}%
\pgfpathlineto{\pgfqpoint{1.746483in}{2.778564in}}%
\pgfpathlineto{\pgfqpoint{1.747118in}{1.809249in}}%
\pgfpathlineto{\pgfqpoint{1.747436in}{2.656919in}}%
\pgfpathlineto{\pgfqpoint{1.748071in}{1.747234in}}%
\pgfpathlineto{\pgfqpoint{1.748388in}{2.430753in}}%
\pgfpathlineto{\pgfqpoint{1.748706in}{2.687671in}}%
\pgfpathlineto{\pgfqpoint{1.749023in}{1.831087in}}%
\pgfpathlineto{\pgfqpoint{1.749658in}{2.791006in}}%
\pgfpathlineto{\pgfqpoint{1.749975in}{2.031669in}}%
\pgfpathlineto{\pgfqpoint{1.750293in}{1.915097in}}%
\pgfpathlineto{\pgfqpoint{1.750610in}{2.744628in}}%
\pgfpathlineto{\pgfqpoint{1.751245in}{1.763753in}}%
\pgfpathlineto{\pgfqpoint{1.751563in}{2.572106in}}%
\pgfpathlineto{\pgfqpoint{1.751880in}{2.549982in}}%
\pgfpathlineto{\pgfqpoint{1.752198in}{1.757298in}}%
\pgfpathlineto{\pgfqpoint{1.752833in}{2.732507in}}%
\pgfpathlineto{\pgfqpoint{1.753150in}{1.889338in}}%
\pgfpathlineto{\pgfqpoint{1.753468in}{2.050479in}}%
\pgfpathlineto{\pgfqpoint{1.753785in}{2.783694in}}%
\pgfpathlineto{\pgfqpoint{1.754420in}{1.839067in}}%
\pgfpathlineto{\pgfqpoint{1.754737in}{2.691260in}}%
\pgfpathlineto{\pgfqpoint{1.755372in}{1.744477in}}%
\pgfpathlineto{\pgfqpoint{1.755690in}{2.488790in}}%
\pgfpathlineto{\pgfqpoint{1.756007in}{2.643081in}}%
\pgfpathlineto{\pgfqpoint{1.756325in}{1.800161in}}%
\pgfpathlineto{\pgfqpoint{1.756960in}{2.776055in}}%
\pgfpathlineto{\pgfqpoint{1.757277in}{1.973832in}}%
\pgfpathlineto{\pgfqpoint{1.757594in}{1.958954in}}%
\pgfpathlineto{\pgfqpoint{1.757912in}{2.761451in}}%
\pgfpathlineto{\pgfqpoint{1.758547in}{1.782163in}}%
\pgfpathlineto{\pgfqpoint{1.758864in}{2.615088in}}%
\pgfpathlineto{\pgfqpoint{1.759182in}{2.494636in}}%
\pgfpathlineto{\pgfqpoint{1.759499in}{1.741778in}}%
\pgfpathlineto{\pgfqpoint{1.760134in}{2.698240in}}%
\pgfpathlineto{\pgfqpoint{1.760452in}{1.847718in}}%
\pgfpathlineto{\pgfqpoint{1.761086in}{2.782202in}}%
\pgfpathlineto{\pgfqpoint{1.761404in}{2.063052in}}%
\pgfpathlineto{\pgfqpoint{1.761721in}{1.874714in}}%
\pgfpathlineto{\pgfqpoint{1.762039in}{2.721210in}}%
\pgfpathlineto{\pgfqpoint{1.762674in}{1.751719in}}%
\pgfpathlineto{\pgfqpoint{1.762991in}{2.540499in}}%
\pgfpathlineto{\pgfqpoint{1.763309in}{2.595138in}}%
\pgfpathlineto{\pgfqpoint{1.763626in}{1.772390in}}%
\pgfpathlineto{\pgfqpoint{1.764261in}{2.752313in}}%
\pgfpathlineto{\pgfqpoint{1.764579in}{1.922581in}}%
\pgfpathlineto{\pgfqpoint{1.764896in}{2.006501in}}%
\pgfpathlineto{\pgfqpoint{1.765213in}{2.771579in}}%
\pgfpathlineto{\pgfqpoint{1.765848in}{1.806249in}}%
\pgfpathlineto{\pgfqpoint{1.766166in}{2.655806in}}%
\pgfpathlineto{\pgfqpoint{1.766801in}{1.736614in}}%
\pgfpathlineto{\pgfqpoint{1.767118in}{2.432627in}}%
\pgfpathlineto{\pgfqpoint{1.767436in}{2.659803in}}%
\pgfpathlineto{\pgfqpoint{1.767753in}{1.808545in}}%
\pgfpathlineto{\pgfqpoint{1.768388in}{2.772180in}}%
\pgfpathlineto{\pgfqpoint{1.768705in}{2.005080in}}%
\pgfpathlineto{\pgfqpoint{1.769023in}{1.915718in}}%
\pgfpathlineto{\pgfqpoint{1.769340in}{2.743772in}}%
\pgfpathlineto{\pgfqpoint{1.769975in}{1.764264in}}%
\pgfpathlineto{\pgfqpoint{1.770293in}{2.590854in}}%
\pgfpathlineto{\pgfqpoint{1.770610in}{2.539874in}}%
\pgfpathlineto{\pgfqpoint{1.770928in}{1.755089in}}%
\pgfpathlineto{\pgfqpoint{1.771563in}{2.723357in}}%
\pgfpathlineto{\pgfqpoint{1.771880in}{1.872975in}}%
\pgfpathlineto{\pgfqpoint{1.772515in}{2.774353in}}%
\pgfpathlineto{\pgfqpoint{1.772832in}{2.099349in}}%
\pgfpathlineto{\pgfqpoint{1.773150in}{1.837731in}}%
\pgfpathlineto{\pgfqpoint{1.773467in}{2.689254in}}%
\pgfpathlineto{\pgfqpoint{1.774102in}{1.736145in}}%
\pgfpathlineto{\pgfqpoint{1.774420in}{2.488863in}}%
\pgfpathlineto{\pgfqpoint{1.774737in}{2.613199in}}%
\pgfpathlineto{\pgfqpoint{1.775055in}{1.778991in}}%
\pgfpathlineto{\pgfqpoint{1.775690in}{2.756214in}}%
\pgfpathlineto{\pgfqpoint{1.776007in}{1.948162in}}%
\pgfpathlineto{\pgfqpoint{1.776324in}{1.962757in}}%
\pgfpathlineto{\pgfqpoint{1.776642in}{2.759916in}}%
\pgfpathlineto{\pgfqpoint{1.777277in}{1.785606in}}%
\pgfpathlineto{\pgfqpoint{1.777594in}{2.634060in}}%
\pgfpathlineto{\pgfqpoint{1.777912in}{2.482356in}}%
\pgfpathlineto{\pgfqpoint{1.778229in}{1.741810in}}%
\pgfpathlineto{\pgfqpoint{1.778864in}{2.685870in}}%
\pgfpathlineto{\pgfqpoint{1.779182in}{1.832083in}}%
\pgfpathlineto{\pgfqpoint{1.779816in}{2.770447in}}%
\pgfpathlineto{\pgfqpoint{1.780134in}{2.036735in}}%
\pgfpathlineto{\pgfqpoint{1.780451in}{1.875353in}}%
\pgfpathlineto{\pgfqpoint{1.780769in}{2.717970in}}%
\pgfpathlineto{\pgfqpoint{1.781404in}{1.745570in}}%
\pgfpathlineto{\pgfqpoint{1.781721in}{2.539042in}}%
\pgfpathlineto{\pgfqpoint{1.782039in}{2.563202in}}%
\pgfpathlineto{\pgfqpoint{1.782356in}{1.752750in}}%
\pgfpathlineto{\pgfqpoint{1.782991in}{2.731149in}}%
\pgfpathlineto{\pgfqpoint{1.783309in}{1.898173in}}%
\pgfpathlineto{\pgfqpoint{1.783626in}{2.012779in}}%
\pgfpathlineto{\pgfqpoint{1.783943in}{2.768887in}}%
\pgfpathlineto{\pgfqpoint{1.784578in}{1.813216in}}%
\pgfpathlineto{\pgfqpoint{1.784896in}{2.673368in}}%
\pgfpathlineto{\pgfqpoint{1.785531in}{1.738820in}}%
\pgfpathlineto{\pgfqpoint{1.785848in}{2.453776in}}%
\pgfpathlineto{\pgfqpoint{1.786166in}{2.644477in}}%
\pgfpathlineto{\pgfqpoint{1.786483in}{1.793893in}}%
\pgfpathlineto{\pgfqpoint{1.787118in}{2.757957in}}%
\pgfpathlineto{\pgfqpoint{1.787435in}{1.979275in}}%
\pgfpathlineto{\pgfqpoint{1.787753in}{1.918108in}}%
\pgfpathlineto{\pgfqpoint{1.788070in}{2.739102in}}%
\pgfpathlineto{\pgfqpoint{1.788705in}{1.760413in}}%
\pgfpathlineto{\pgfqpoint{1.789023in}{2.587763in}}%
\pgfpathlineto{\pgfqpoint{1.789340in}{2.506601in}}%
\pgfpathlineto{\pgfqpoint{1.789658in}{1.736977in}}%
\pgfpathlineto{\pgfqpoint{1.790293in}{2.701002in}}%
\pgfpathlineto{\pgfqpoint{1.790610in}{1.850105in}}%
\pgfpathlineto{\pgfqpoint{1.791245in}{2.769767in}}%
\pgfpathlineto{\pgfqpoint{1.791562in}{2.076276in}}%
\pgfpathlineto{\pgfqpoint{1.791880in}{1.848222in}}%
\pgfpathlineto{\pgfqpoint{1.792197in}{2.705475in}}%
\pgfpathlineto{\pgfqpoint{1.792832in}{1.740898in}}%
\pgfpathlineto{\pgfqpoint{1.793150in}{2.508757in}}%
\pgfpathlineto{\pgfqpoint{1.793467in}{2.595275in}}%
\pgfpathlineto{\pgfqpoint{1.793785in}{1.765818in}}%
\pgfpathlineto{\pgfqpoint{1.794420in}{2.739551in}}%
\pgfpathlineto{\pgfqpoint{1.794737in}{1.922796in}}%
\pgfpathlineto{\pgfqpoint{1.795054in}{1.967072in}}%
\pgfpathlineto{\pgfqpoint{1.795372in}{2.753513in}}%
\pgfpathlineto{\pgfqpoint{1.796007in}{1.783847in}}%
\pgfpathlineto{\pgfqpoint{1.796324in}{2.629451in}}%
\pgfpathlineto{\pgfqpoint{1.796959in}{1.725561in}}%
\pgfpathlineto{\pgfqpoint{1.797277in}{2.399478in}}%
\pgfpathlineto{\pgfqpoint{1.797594in}{2.662020in}}%
\pgfpathlineto{\pgfqpoint{1.797912in}{1.811013in}}%
\pgfpathlineto{\pgfqpoint{1.798546in}{2.764984in}}%
\pgfpathlineto{\pgfqpoint{1.798864in}{2.014744in}}%
\pgfpathlineto{\pgfqpoint{1.799181in}{1.890238in}}%
\pgfpathlineto{\pgfqpoint{1.799499in}{2.731597in}}%
\pgfpathlineto{\pgfqpoint{1.800134in}{1.752581in}}%
\pgfpathlineto{\pgfqpoint{1.800451in}{2.557383in}}%
\pgfpathlineto{\pgfqpoint{1.800769in}{2.542903in}}%
\pgfpathlineto{\pgfqpoint{1.801086in}{1.741518in}}%
\pgfpathlineto{\pgfqpoint{1.801721in}{2.712151in}}%
\pgfpathlineto{\pgfqpoint{1.802039in}{1.873808in}}%
\pgfpathlineto{\pgfqpoint{1.802356in}{2.018657in}}%
\pgfpathlineto{\pgfqpoint{1.802673in}{2.760605in}}%
\pgfpathlineto{\pgfqpoint{1.803308in}{1.813369in}}%
\pgfpathlineto{\pgfqpoint{1.803626in}{2.667356in}}%
\pgfpathlineto{\pgfqpoint{1.804261in}{1.724578in}}%
\pgfpathlineto{\pgfqpoint{1.804578in}{2.454396in}}%
\pgfpathlineto{\pgfqpoint{1.804896in}{2.618773in}}%
\pgfpathlineto{\pgfqpoint{1.805213in}{1.774807in}}%
\pgfpathlineto{\pgfqpoint{1.805848in}{2.751080in}}%
\pgfpathlineto{\pgfqpoint{1.806165in}{1.958098in}}%
\pgfpathlineto{\pgfqpoint{1.806483in}{1.936032in}}%
\pgfpathlineto{\pgfqpoint{1.806800in}{2.751877in}}%
\pgfpathlineto{\pgfqpoint{1.807435in}{1.770053in}}%
\pgfpathlineto{\pgfqpoint{1.807753in}{2.603572in}}%
\pgfpathlineto{\pgfqpoint{1.808070in}{2.484106in}}%
\pgfpathlineto{\pgfqpoint{1.808388in}{1.727747in}}%
\pgfpathlineto{\pgfqpoint{1.809023in}{2.679525in}}%
\pgfpathlineto{\pgfqpoint{1.809340in}{1.826889in}}%
\pgfpathlineto{\pgfqpoint{1.809975in}{2.759200in}}%
\pgfpathlineto{\pgfqpoint{1.810292in}{2.047460in}}%
\pgfpathlineto{\pgfqpoint{1.810610in}{1.849647in}}%
\pgfpathlineto{\pgfqpoint{1.810927in}{2.697808in}}%
\pgfpathlineto{\pgfqpoint{1.811562in}{1.728989in}}%
\pgfpathlineto{\pgfqpoint{1.811880in}{2.509390in}}%
\pgfpathlineto{\pgfqpoint{1.812197in}{2.567769in}}%
\pgfpathlineto{\pgfqpoint{1.812515in}{1.748912in}}%
\pgfpathlineto{\pgfqpoint{1.813150in}{2.732285in}}%
\pgfpathlineto{\pgfqpoint{1.813467in}{1.903171in}}%
\pgfpathlineto{\pgfqpoint{1.813784in}{1.988600in}}%
\pgfpathlineto{\pgfqpoint{1.814102in}{2.764259in}}%
\pgfpathlineto{\pgfqpoint{1.814737in}{1.795441in}}%
\pgfpathlineto{\pgfqpoint{1.815054in}{2.642461in}}%
\pgfpathlineto{\pgfqpoint{1.815689in}{1.718589in}}%
\pgfpathlineto{\pgfqpoint{1.816007in}{2.410115in}}%
\pgfpathlineto{\pgfqpoint{1.816324in}{2.638261in}}%
\pgfpathlineto{\pgfqpoint{1.816642in}{1.789422in}}%
\pgfpathlineto{\pgfqpoint{1.817276in}{2.750987in}}%
\pgfpathlineto{\pgfqpoint{1.817594in}{1.985379in}}%
\pgfpathlineto{\pgfqpoint{1.817911in}{1.892453in}}%
\pgfpathlineto{\pgfqpoint{1.818229in}{2.722255in}}%
\pgfpathlineto{\pgfqpoint{1.818864in}{1.742998in}}%
\pgfpathlineto{\pgfqpoint{1.819181in}{2.557894in}}%
\pgfpathlineto{\pgfqpoint{1.819499in}{2.513342in}}%
\pgfpathlineto{\pgfqpoint{1.819816in}{1.726983in}}%
\pgfpathlineto{\pgfqpoint{1.820451in}{2.702253in}}%
\pgfpathlineto{\pgfqpoint{1.820769in}{1.855798in}}%
\pgfpathlineto{\pgfqpoint{1.821403in}{2.769026in}}%
\pgfpathlineto{\pgfqpoint{1.821721in}{2.088816in}}%
\pgfpathlineto{\pgfqpoint{1.822038in}{1.827710in}}%
\pgfpathlineto{\pgfqpoint{1.822356in}{2.677869in}}%
\pgfpathlineto{\pgfqpoint{1.822991in}{1.719507in}}%
\pgfpathlineto{\pgfqpoint{1.823308in}{2.464374in}}%
\pgfpathlineto{\pgfqpoint{1.823626in}{2.592529in}}%
\pgfpathlineto{\pgfqpoint{1.823943in}{1.755057in}}%
\pgfpathlineto{\pgfqpoint{1.824578in}{2.733405in}}%
\pgfpathlineto{\pgfqpoint{1.824895in}{1.929383in}}%
\pgfpathlineto{\pgfqpoint{1.825213in}{1.939487in}}%
\pgfpathlineto{\pgfqpoint{1.825530in}{2.739122in}}%
\pgfpathlineto{\pgfqpoint{1.826165in}{1.763293in}}%
\pgfpathlineto{\pgfqpoint{1.826483in}{2.604055in}}%
\pgfpathlineto{\pgfqpoint{1.826800in}{2.453075in}}%
\pgfpathlineto{\pgfqpoint{1.827118in}{1.715858in}}%
\pgfpathlineto{\pgfqpoint{1.827753in}{2.668153in}}%
\pgfpathlineto{\pgfqpoint{1.828070in}{1.811575in}}%
\pgfpathlineto{\pgfqpoint{1.828705in}{2.764033in}}%
\pgfpathlineto{\pgfqpoint{1.829022in}{2.025771in}}%
\pgfpathlineto{\pgfqpoint{1.829340in}{1.865817in}}%
\pgfpathlineto{\pgfqpoint{1.829657in}{2.705073in}}%
\pgfpathlineto{\pgfqpoint{1.830292in}{1.725994in}}%
\pgfpathlineto{\pgfqpoint{1.830610in}{2.518403in}}%
\pgfpathlineto{\pgfqpoint{1.830927in}{2.539397in}}%
\pgfpathlineto{\pgfqpoint{1.831245in}{1.731297in}}%
\pgfpathlineto{\pgfqpoint{1.831880in}{2.709774in}}%
\pgfpathlineto{\pgfqpoint{1.832197in}{1.874872in}}%
\pgfpathlineto{\pgfqpoint{1.832514in}{1.992919in}}%
\pgfpathlineto{\pgfqpoint{1.832832in}{2.748058in}}%
\pgfpathlineto{\pgfqpoint{1.833467in}{1.791590in}}%
\pgfpathlineto{\pgfqpoint{1.833784in}{2.642869in}}%
\pgfpathlineto{\pgfqpoint{1.834419in}{1.709842in}}%
\pgfpathlineto{\pgfqpoint{1.834737in}{2.426219in}}%
\pgfpathlineto{\pgfqpoint{1.835054in}{2.623573in}}%
\pgfpathlineto{\pgfqpoint{1.835372in}{1.776117in}}%
\pgfpathlineto{\pgfqpoint{1.836006in}{2.752809in}}%
\pgfpathlineto{\pgfqpoint{1.836324in}{1.964244in}}%
\pgfpathlineto{\pgfqpoint{1.836641in}{1.911132in}}%
\pgfpathlineto{\pgfqpoint{1.836959in}{2.726443in}}%
\pgfpathlineto{\pgfqpoint{1.837594in}{1.742602in}}%
\pgfpathlineto{\pgfqpoint{1.837911in}{2.565231in}}%
\pgfpathlineto{\pgfqpoint{1.838229in}{2.482613in}}%
\pgfpathlineto{\pgfqpoint{1.838546in}{1.711836in}}%
\pgfpathlineto{\pgfqpoint{1.839181in}{2.676797in}}%
\pgfpathlineto{\pgfqpoint{1.839499in}{1.828881in}}%
\pgfpathlineto{\pgfqpoint{1.840133in}{2.749755in}}%
\pgfpathlineto{\pgfqpoint{1.840451in}{2.051908in}}%
\pgfpathlineto{\pgfqpoint{1.840768in}{1.826787in}}%
\pgfpathlineto{\pgfqpoint{1.841086in}{2.676863in}}%
\pgfpathlineto{\pgfqpoint{1.841721in}{1.714255in}}%
\pgfpathlineto{\pgfqpoint{1.842038in}{2.481110in}}%
\pgfpathlineto{\pgfqpoint{1.842356in}{2.576069in}}%
\pgfpathlineto{\pgfqpoint{1.842673in}{1.744827in}}%
\pgfpathlineto{\pgfqpoint{1.843308in}{2.731218in}}%
\pgfpathlineto{\pgfqpoint{1.843625in}{1.909146in}}%
\pgfpathlineto{\pgfqpoint{1.843943in}{1.959474in}}%
\pgfpathlineto{\pgfqpoint{1.844260in}{2.739287in}}%
\pgfpathlineto{\pgfqpoint{1.844895in}{1.765447in}}%
\pgfpathlineto{\pgfqpoint{1.845213in}{2.609219in}}%
\pgfpathlineto{\pgfqpoint{1.845848in}{1.703330in}}%
\pgfpathlineto{\pgfqpoint{1.846165in}{2.378331in}}%
\pgfpathlineto{\pgfqpoint{1.846483in}{2.638672in}}%
\pgfpathlineto{\pgfqpoint{1.846800in}{1.785573in}}%
\pgfpathlineto{\pgfqpoint{1.847435in}{2.742168in}}%
\pgfpathlineto{\pgfqpoint{1.847752in}{1.990241in}}%
\pgfpathlineto{\pgfqpoint{1.848070in}{1.867817in}}%
\pgfpathlineto{\pgfqpoint{1.848387in}{2.703140in}}%
\pgfpathlineto{\pgfqpoint{1.849022in}{1.724878in}}%
\pgfpathlineto{\pgfqpoint{1.849340in}{2.534734in}}%
\pgfpathlineto{\pgfqpoint{1.849657in}{2.520254in}}%
\pgfpathlineto{\pgfqpoint{1.849975in}{1.723475in}}%
\pgfpathlineto{\pgfqpoint{1.850610in}{2.704377in}}%
\pgfpathlineto{\pgfqpoint{1.850927in}{1.856212in}}%
\pgfpathlineto{\pgfqpoint{1.851244in}{2.015228in}}%
\pgfpathlineto{\pgfqpoint{1.851562in}{2.745037in}}%
\pgfpathlineto{\pgfqpoint{1.852197in}{1.796349in}}%
\pgfpathlineto{\pgfqpoint{1.852514in}{2.645995in}}%
\pgfpathlineto{\pgfqpoint{1.853149in}{1.700383in}}%
\pgfpathlineto{\pgfqpoint{1.853467in}{2.437780in}}%
\pgfpathlineto{\pgfqpoint{1.853784in}{2.592033in}}%
\pgfpathlineto{\pgfqpoint{1.854102in}{1.752529in}}%
\pgfpathlineto{\pgfqpoint{1.854736in}{2.727921in}}%
\pgfpathlineto{\pgfqpoint{1.855054in}{1.929460in}}%
\pgfpathlineto{\pgfqpoint{1.855371in}{1.915875in}}%
\pgfpathlineto{\pgfqpoint{1.855689in}{2.722311in}}%
\pgfpathlineto{\pgfqpoint{1.856324in}{1.744670in}}%
\pgfpathlineto{\pgfqpoint{1.856641in}{2.581610in}}%
\pgfpathlineto{\pgfqpoint{1.856959in}{2.461346in}}%
\pgfpathlineto{\pgfqpoint{1.857276in}{1.707886in}}%
\pgfpathlineto{\pgfqpoint{1.857911in}{2.667262in}}%
\pgfpathlineto{\pgfqpoint{1.858229in}{1.811820in}}%
\pgfpathlineto{\pgfqpoint{1.858863in}{2.742800in}}%
\pgfpathlineto{\pgfqpoint{1.859181in}{2.020077in}}%
\pgfpathlineto{\pgfqpoint{1.859498in}{1.833898in}}%
\pgfpathlineto{\pgfqpoint{1.859816in}{2.677359in}}%
\pgfpathlineto{\pgfqpoint{1.860451in}{1.707795in}}%
\pgfpathlineto{\pgfqpoint{1.860768in}{2.491380in}}%
\pgfpathlineto{\pgfqpoint{1.861086in}{2.541120in}}%
\pgfpathlineto{\pgfqpoint{1.861403in}{1.723451in}}%
\pgfpathlineto{\pgfqpoint{1.862038in}{2.703830in}}%
\pgfpathlineto{\pgfqpoint{1.862355in}{1.876019in}}%
\pgfpathlineto{\pgfqpoint{1.862673in}{1.967215in}}%
\pgfpathlineto{\pgfqpoint{1.862990in}{2.733818in}}%
\pgfpathlineto{\pgfqpoint{1.863625in}{1.771508in}}%
\pgfpathlineto{\pgfqpoint{1.863943in}{2.624287in}}%
\pgfpathlineto{\pgfqpoint{1.864578in}{1.702810in}}%
\pgfpathlineto{\pgfqpoint{1.864895in}{2.410481in}}%
\pgfpathlineto{\pgfqpoint{1.865213in}{2.625519in}}%
\pgfpathlineto{\pgfqpoint{1.865530in}{1.770639in}}%
\pgfpathlineto{\pgfqpoint{1.866165in}{2.731859in}}%
\pgfpathlineto{\pgfqpoint{1.866482in}{1.959301in}}%
\pgfpathlineto{\pgfqpoint{1.866800in}{1.877417in}}%
\pgfpathlineto{\pgfqpoint{1.867117in}{2.701090in}}%
\pgfpathlineto{\pgfqpoint{1.867752in}{1.721810in}}%
\pgfpathlineto{\pgfqpoint{1.868070in}{2.543355in}}%
\pgfpathlineto{\pgfqpoint{1.868387in}{2.483196in}}%
\pgfpathlineto{\pgfqpoint{1.868705in}{1.705219in}}%
\pgfpathlineto{\pgfqpoint{1.869340in}{2.673790in}}%
\pgfpathlineto{\pgfqpoint{1.869657in}{1.824635in}}%
\pgfpathlineto{\pgfqpoint{1.870292in}{2.736427in}}%
\pgfpathlineto{\pgfqpoint{1.870609in}{2.053189in}}%
\pgfpathlineto{\pgfqpoint{1.870927in}{1.805612in}}%
\pgfpathlineto{\pgfqpoint{1.871244in}{2.659524in}}%
\pgfpathlineto{\pgfqpoint{1.871879in}{1.703846in}}%
\pgfpathlineto{\pgfqpoint{1.872197in}{2.469589in}}%
\pgfpathlineto{\pgfqpoint{1.872514in}{2.575184in}}%
\pgfpathlineto{\pgfqpoint{1.872832in}{1.740151in}}%
\pgfpathlineto{\pgfqpoint{1.873466in}{2.713482in}}%
\pgfpathlineto{\pgfqpoint{1.873784in}{1.899298in}}%
\pgfpathlineto{\pgfqpoint{1.874101in}{1.927791in}}%
\pgfpathlineto{\pgfqpoint{1.874419in}{2.717217in}}%
\pgfpathlineto{\pgfqpoint{1.875054in}{1.744831in}}%
\pgfpathlineto{\pgfqpoint{1.875371in}{2.588374in}}%
\pgfpathlineto{\pgfqpoint{1.876006in}{1.692344in}}%
\pgfpathlineto{\pgfqpoint{1.876324in}{2.358050in}}%
\pgfpathlineto{\pgfqpoint{1.876641in}{2.634273in}}%
\pgfpathlineto{\pgfqpoint{1.876959in}{1.782742in}}%
\pgfpathlineto{\pgfqpoint{1.877593in}{2.731753in}}%
\pgfpathlineto{\pgfqpoint{1.877911in}{1.987440in}}%
\pgfpathlineto{\pgfqpoint{1.878228in}{1.847192in}}%
\pgfpathlineto{\pgfqpoint{1.878546in}{2.688072in}}%
\pgfpathlineto{\pgfqpoint{1.879181in}{1.714728in}}%
\pgfpathlineto{\pgfqpoint{1.879498in}{2.522385in}}%
\pgfpathlineto{\pgfqpoint{1.879816in}{2.520993in}}%
\pgfpathlineto{\pgfqpoint{1.880133in}{1.714223in}}%
\pgfpathlineto{\pgfqpoint{1.880768in}{2.685689in}}%
\pgfpathlineto{\pgfqpoint{1.881085in}{1.847637in}}%
\pgfpathlineto{\pgfqpoint{1.881403in}{1.981298in}}%
\pgfpathlineto{\pgfqpoint{1.881720in}{2.725613in}}%
\pgfpathlineto{\pgfqpoint{1.882355in}{1.775143in}}%
\pgfpathlineto{\pgfqpoint{1.882673in}{2.628833in}}%
\pgfpathlineto{\pgfqpoint{1.883308in}{1.690297in}}%
\pgfpathlineto{\pgfqpoint{1.883625in}{2.416186in}}%
\pgfpathlineto{\pgfqpoint{1.883943in}{2.589630in}}%
\pgfpathlineto{\pgfqpoint{1.884260in}{1.744279in}}%
\pgfpathlineto{\pgfqpoint{1.884895in}{2.717133in}}%
\pgfpathlineto{\pgfqpoint{1.885212in}{1.927724in}}%
\pgfpathlineto{\pgfqpoint{1.885530in}{1.893615in}}%
\pgfpathlineto{\pgfqpoint{1.885847in}{2.709357in}}%
\pgfpathlineto{\pgfqpoint{1.886482in}{1.733013in}}%
\pgfpathlineto{\pgfqpoint{1.886800in}{2.572272in}}%
\pgfpathlineto{\pgfqpoint{1.887117in}{2.460161in}}%
\pgfpathlineto{\pgfqpoint{1.887435in}{1.699420in}}%
\pgfpathlineto{\pgfqpoint{1.888070in}{2.651908in}}%
\pgfpathlineto{\pgfqpoint{1.888387in}{1.798111in}}%
\pgfpathlineto{\pgfqpoint{1.889022in}{2.724739in}}%
\pgfpathlineto{\pgfqpoint{1.889339in}{2.016493in}}%
\pgfpathlineto{\pgfqpoint{1.889657in}{1.812372in}}%
\pgfpathlineto{\pgfqpoint{1.889974in}{2.661625in}}%
\pgfpathlineto{\pgfqpoint{1.890609in}{1.694891in}}%
\pgfpathlineto{\pgfqpoint{1.890927in}{2.474109in}}%
\pgfpathlineto{\pgfqpoint{1.891244in}{2.536882in}}%
\pgfpathlineto{\pgfqpoint{1.891562in}{1.716651in}}%
\pgfpathlineto{\pgfqpoint{1.892197in}{2.695851in}}%
\pgfpathlineto{\pgfqpoint{1.892514in}{1.869440in}}%
\pgfpathlineto{\pgfqpoint{1.892831in}{1.947308in}}%
\pgfpathlineto{\pgfqpoint{1.893149in}{2.722252in}}%
\pgfpathlineto{\pgfqpoint{1.893784in}{1.759870in}}%
\pgfpathlineto{\pgfqpoint{1.894101in}{2.614649in}}%
\pgfpathlineto{\pgfqpoint{1.894736in}{1.690496in}}%
\pgfpathlineto{\pgfqpoint{1.895054in}{2.383451in}}%
\pgfpathlineto{\pgfqpoint{1.895371in}{2.609161in}}%
\pgfpathlineto{\pgfqpoint{1.895689in}{1.758787in}}%
\pgfpathlineto{\pgfqpoint{1.896323in}{2.716493in}}%
\pgfpathlineto{\pgfqpoint{1.896641in}{1.951488in}}%
\pgfpathlineto{\pgfqpoint{1.896958in}{1.857173in}}%
\pgfpathlineto{\pgfqpoint{1.897276in}{2.687379in}}%
\pgfpathlineto{\pgfqpoint{1.897911in}{1.709243in}}%
\pgfpathlineto{\pgfqpoint{1.898228in}{2.525293in}}%
\pgfpathlineto{\pgfqpoint{1.898546in}{2.479874in}}%
\pgfpathlineto{\pgfqpoint{1.898863in}{1.693885in}}%
\pgfpathlineto{\pgfqpoint{1.899498in}{2.664505in}}%
\pgfpathlineto{\pgfqpoint{1.899815in}{1.819741in}}%
\pgfpathlineto{\pgfqpoint{1.900450in}{2.728740in}}%
\pgfpathlineto{\pgfqpoint{1.900768in}{2.052298in}}%
\pgfpathlineto{\pgfqpoint{1.901085in}{1.794989in}}%
\pgfpathlineto{\pgfqpoint{1.901403in}{2.651330in}}%
\pgfpathlineto{\pgfqpoint{1.902038in}{1.692265in}}%
\pgfpathlineto{\pgfqpoint{1.902355in}{2.440756in}}%
\pgfpathlineto{\pgfqpoint{1.902673in}{2.561197in}}%
\pgfpathlineto{\pgfqpoint{1.902990in}{1.723304in}}%
\pgfpathlineto{\pgfqpoint{1.903625in}{2.698136in}}%
\pgfpathlineto{\pgfqpoint{1.903942in}{1.893115in}}%
\pgfpathlineto{\pgfqpoint{1.904260in}{1.906381in}}%
\pgfpathlineto{\pgfqpoint{1.904577in}{2.705331in}}%
\pgfpathlineto{\pgfqpoint{1.905212in}{1.731179in}}%
\pgfpathlineto{\pgfqpoint{1.905530in}{2.573419in}}%
\pgfpathlineto{\pgfqpoint{1.905847in}{2.417022in}}%
\pgfpathlineto{\pgfqpoint{1.906165in}{1.682354in}}%
\pgfpathlineto{\pgfqpoint{1.906800in}{2.627245in}}%
\pgfpathlineto{\pgfqpoint{1.907117in}{1.772886in}}%
\pgfpathlineto{\pgfqpoint{1.907752in}{2.724958in}}%
\pgfpathlineto{\pgfqpoint{1.908069in}{1.987527in}}%
\pgfpathlineto{\pgfqpoint{1.908387in}{1.836176in}}%
\pgfpathlineto{\pgfqpoint{1.908704in}{2.680804in}}%
\pgfpathlineto{\pgfqpoint{1.909339in}{1.700934in}}%
\pgfpathlineto{\pgfqpoint{1.909657in}{2.496969in}}%
\pgfpathlineto{\pgfqpoint{1.909974in}{2.505761in}}%
\pgfpathlineto{\pgfqpoint{1.910292in}{1.699113in}}%
\pgfpathlineto{\pgfqpoint{1.910927in}{2.673008in}}%
\pgfpathlineto{\pgfqpoint{1.911244in}{1.836477in}}%
\pgfpathlineto{\pgfqpoint{1.911561in}{1.962832in}}%
\pgfpathlineto{\pgfqpoint{1.911879in}{2.714242in}}%
\pgfpathlineto{\pgfqpoint{1.912514in}{1.761142in}}%
\pgfpathlineto{\pgfqpoint{1.912831in}{2.613902in}}%
\pgfpathlineto{\pgfqpoint{1.913466in}{1.677119in}}%
\pgfpathlineto{\pgfqpoint{1.913784in}{2.398945in}}%
\pgfpathlineto{\pgfqpoint{1.914101in}{2.580822in}}%
\pgfpathlineto{\pgfqpoint{1.914419in}{1.736462in}}%
\pgfpathlineto{\pgfqpoint{1.915053in}{2.713548in}}%
\pgfpathlineto{\pgfqpoint{1.915371in}{1.923312in}}%
\pgfpathlineto{\pgfqpoint{1.915688in}{1.884896in}}%
\pgfpathlineto{\pgfqpoint{1.916006in}{2.703470in}}%
\pgfpathlineto{\pgfqpoint{1.916641in}{1.718863in}}%
\pgfpathlineto{\pgfqpoint{1.916958in}{2.545727in}}%
\pgfpathlineto{\pgfqpoint{1.917276in}{2.445764in}}%
\pgfpathlineto{\pgfqpoint{1.917593in}{1.680105in}}%
\pgfpathlineto{\pgfqpoint{1.918228in}{2.637934in}}%
\pgfpathlineto{\pgfqpoint{1.918545in}{1.789070in}}%
\pgfpathlineto{\pgfqpoint{1.919180in}{2.715335in}}%
\pgfpathlineto{\pgfqpoint{1.919498in}{2.009685in}}%
\pgfpathlineto{\pgfqpoint{1.919815in}{1.799262in}}%
\pgfpathlineto{\pgfqpoint{1.920133in}{2.648542in}}%
\pgfpathlineto{\pgfqpoint{1.920768in}{1.682565in}}%
\pgfpathlineto{\pgfqpoint{1.921085in}{2.455466in}}%
\pgfpathlineto{\pgfqpoint{1.921403in}{2.529318in}}%
\pgfpathlineto{\pgfqpoint{1.921720in}{1.704433in}}%
\pgfpathlineto{\pgfqpoint{1.922355in}{2.691693in}}%
\pgfpathlineto{\pgfqpoint{1.922672in}{1.866588in}}%
\pgfpathlineto{\pgfqpoint{1.922990in}{1.937433in}}%
\pgfpathlineto{\pgfqpoint{1.923307in}{2.717855in}}%
\pgfpathlineto{\pgfqpoint{1.923942in}{1.744787in}}%
\pgfpathlineto{\pgfqpoint{1.924260in}{2.590760in}}%
\pgfpathlineto{\pgfqpoint{1.924895in}{1.672323in}}%
\pgfpathlineto{\pgfqpoint{1.925212in}{2.355932in}}%
\pgfpathlineto{\pgfqpoint{1.925530in}{2.596867in}}%
\pgfpathlineto{\pgfqpoint{1.925847in}{1.744929in}}%
\pgfpathlineto{\pgfqpoint{1.926482in}{2.706127in}}%
\pgfpathlineto{\pgfqpoint{1.926799in}{1.945947in}}%
\pgfpathlineto{\pgfqpoint{1.927117in}{1.843071in}}%
\pgfpathlineto{\pgfqpoint{1.927434in}{2.675090in}}%
\pgfpathlineto{\pgfqpoint{1.928069in}{1.695522in}}%
\pgfpathlineto{\pgfqpoint{1.928387in}{2.510218in}}%
\pgfpathlineto{\pgfqpoint{1.928704in}{2.470582in}}%
\pgfpathlineto{\pgfqpoint{1.929022in}{1.683726in}}%
\pgfpathlineto{\pgfqpoint{1.929657in}{2.662568in}}%
\pgfpathlineto{\pgfqpoint{1.929974in}{1.812332in}}%
\pgfpathlineto{\pgfqpoint{1.930609in}{2.722222in}}%
\pgfpathlineto{\pgfqpoint{1.930926in}{2.045975in}}%
\pgfpathlineto{\pgfqpoint{1.931244in}{1.778639in}}%
\pgfpathlineto{\pgfqpoint{1.931561in}{2.628083in}}%
\pgfpathlineto{\pgfqpoint{1.932196in}{1.671004in}}%
\pgfpathlineto{\pgfqpoint{1.932514in}{2.417618in}}%
\pgfpathlineto{\pgfqpoint{1.932831in}{2.547066in}}%
\pgfpathlineto{\pgfqpoint{1.933149in}{1.711662in}}%
\pgfpathlineto{\pgfqpoint{1.933783in}{2.689602in}}%
\pgfpathlineto{\pgfqpoint{1.934101in}{1.883264in}}%
\pgfpathlineto{\pgfqpoint{1.934418in}{1.894823in}}%
\pgfpathlineto{\pgfqpoint{1.934736in}{2.693339in}}%
\pgfpathlineto{\pgfqpoint{1.935371in}{1.717712in}}%
\pgfpathlineto{\pgfqpoint{1.935688in}{2.557720in}}%
\pgfpathlineto{\pgfqpoint{1.936006in}{2.407698in}}%
\pgfpathlineto{\pgfqpoint{1.936323in}{1.668975in}}%
\pgfpathlineto{\pgfqpoint{1.936958in}{2.623173in}}%
\pgfpathlineto{\pgfqpoint{1.937275in}{1.768131in}}%
\pgfpathlineto{\pgfqpoint{1.937910in}{2.718662in}}%
\pgfpathlineto{\pgfqpoint{1.938228in}{1.976960in}}%
\pgfpathlineto{\pgfqpoint{1.938545in}{1.820674in}}%
\pgfpathlineto{\pgfqpoint{1.938863in}{2.658709in}}%
\pgfpathlineto{\pgfqpoint{1.939498in}{1.680387in}}%
\pgfpathlineto{\pgfqpoint{1.939815in}{2.472672in}}%
\pgfpathlineto{\pgfqpoint{1.940133in}{2.492104in}}%
\pgfpathlineto{\pgfqpoint{1.940450in}{1.683318in}}%
\pgfpathlineto{\pgfqpoint{1.941085in}{2.662507in}}%
\pgfpathlineto{\pgfqpoint{1.941402in}{1.828814in}}%
\pgfpathlineto{\pgfqpoint{1.941720in}{1.949575in}}%
\pgfpathlineto{\pgfqpoint{1.942037in}{2.703471in}}%
\pgfpathlineto{\pgfqpoint{1.942672in}{1.748232in}}%
\pgfpathlineto{\pgfqpoint{1.942990in}{2.600475in}}%
\pgfpathlineto{\pgfqpoint{1.943625in}{1.665465in}}%
\pgfpathlineto{\pgfqpoint{1.943942in}{2.384788in}}%
\pgfpathlineto{\pgfqpoint{1.944260in}{2.577801in}}%
\pgfpathlineto{\pgfqpoint{1.944577in}{1.727492in}}%
\pgfpathlineto{\pgfqpoint{1.945212in}{2.704500in}}%
\pgfpathlineto{\pgfqpoint{1.945529in}{1.914619in}}%
\pgfpathlineto{\pgfqpoint{1.945847in}{1.867896in}}%
\pgfpathlineto{\pgfqpoint{1.946164in}{2.681266in}}%
\pgfpathlineto{\pgfqpoint{1.946799in}{1.697929in}}%
\pgfpathlineto{\pgfqpoint{1.947117in}{2.524879in}}%
\pgfpathlineto{\pgfqpoint{1.947434in}{2.430590in}}%
\pgfpathlineto{\pgfqpoint{1.947752in}{1.666522in}}%
\pgfpathlineto{\pgfqpoint{1.948387in}{2.628834in}}%
\pgfpathlineto{\pgfqpoint{1.948704in}{1.777053in}}%
\pgfpathlineto{\pgfqpoint{1.949339in}{2.703554in}}%
\pgfpathlineto{\pgfqpoint{1.949656in}{1.999914in}}%
\pgfpathlineto{\pgfqpoint{1.949974in}{1.785841in}}%
\pgfpathlineto{\pgfqpoint{1.950291in}{2.635200in}}%
\pgfpathlineto{\pgfqpoint{1.950926in}{1.669354in}}%
\pgfpathlineto{\pgfqpoint{1.951244in}{2.445614in}}%
\pgfpathlineto{\pgfqpoint{1.951561in}{2.524025in}}%
\pgfpathlineto{\pgfqpoint{1.951879in}{1.698013in}}%
\pgfpathlineto{\pgfqpoint{1.952513in}{2.683118in}}%
\pgfpathlineto{\pgfqpoint{1.952831in}{1.853837in}}%
\pgfpathlineto{\pgfqpoint{1.953148in}{1.922768in}}%
\pgfpathlineto{\pgfqpoint{1.953466in}{2.694539in}}%
\pgfpathlineto{\pgfqpoint{1.954101in}{1.724224in}}%
\pgfpathlineto{\pgfqpoint{1.954418in}{2.569606in}}%
\pgfpathlineto{\pgfqpoint{1.955053in}{1.656251in}}%
\pgfpathlineto{\pgfqpoint{1.955371in}{2.345163in}}%
\pgfpathlineto{\pgfqpoint{1.955688in}{2.585432in}}%
\pgfpathlineto{\pgfqpoint{1.956005in}{1.735686in}}%
\pgfpathlineto{\pgfqpoint{1.956640in}{2.695811in}}%
\pgfpathlineto{\pgfqpoint{1.956958in}{1.932345in}}%
\pgfpathlineto{\pgfqpoint{1.957275in}{1.831974in}}%
\pgfpathlineto{\pgfqpoint{1.957593in}{2.662424in}}%
\pgfpathlineto{\pgfqpoint{1.958228in}{1.683428in}}%
\pgfpathlineto{\pgfqpoint{1.958545in}{2.499640in}}%
\pgfpathlineto{\pgfqpoint{1.958863in}{2.465339in}}%
\pgfpathlineto{\pgfqpoint{1.959180in}{1.673940in}}%
\pgfpathlineto{\pgfqpoint{1.959815in}{2.651099in}}%
\pgfpathlineto{\pgfqpoint{1.960132in}{1.802140in}}%
\pgfpathlineto{\pgfqpoint{1.960767in}{2.700093in}}%
\pgfpathlineto{\pgfqpoint{1.961085in}{2.025493in}}%
\pgfpathlineto{\pgfqpoint{1.961402in}{1.759182in}}%
\pgfpathlineto{\pgfqpoint{1.961720in}{2.608953in}}%
\pgfpathlineto{\pgfqpoint{1.962355in}{1.657116in}}%
\pgfpathlineto{\pgfqpoint{1.962672in}{2.404865in}}%
\pgfpathlineto{\pgfqpoint{1.962990in}{2.536273in}}%
\pgfpathlineto{\pgfqpoint{1.963307in}{1.698807in}}%
\pgfpathlineto{\pgfqpoint{1.963942in}{2.677217in}}%
\pgfpathlineto{\pgfqpoint{1.964259in}{1.871936in}}%
\pgfpathlineto{\pgfqpoint{1.964577in}{1.882552in}}%
\pgfpathlineto{\pgfqpoint{1.964894in}{2.681320in}}%
\pgfpathlineto{\pgfqpoint{1.965529in}{1.705945in}}%
\pgfpathlineto{\pgfqpoint{1.965847in}{2.549579in}}%
\pgfpathlineto{\pgfqpoint{1.966164in}{2.400097in}}%
\pgfpathlineto{\pgfqpoint{1.966482in}{1.661910in}}%
\pgfpathlineto{\pgfqpoint{1.967117in}{2.612425in}}%
\pgfpathlineto{\pgfqpoint{1.967434in}{1.753476in}}%
\pgfpathlineto{\pgfqpoint{1.968069in}{2.695170in}}%
\pgfpathlineto{\pgfqpoint{1.968386in}{1.958531in}}%
\pgfpathlineto{\pgfqpoint{1.968704in}{1.800427in}}%
\pgfpathlineto{\pgfqpoint{1.969021in}{2.639998in}}%
\pgfpathlineto{\pgfqpoint{1.969656in}{1.665921in}}%
\pgfpathlineto{\pgfqpoint{1.969974in}{2.463349in}}%
\pgfpathlineto{\pgfqpoint{1.970291in}{2.479137in}}%
\pgfpathlineto{\pgfqpoint{1.970609in}{1.673371in}}%
\pgfpathlineto{\pgfqpoint{1.971243in}{2.651337in}}%
\pgfpathlineto{\pgfqpoint{1.971561in}{1.813490in}}%
\pgfpathlineto{\pgfqpoint{1.971878in}{1.941066in}}%
\pgfpathlineto{\pgfqpoint{1.972196in}{2.690553in}}%
\pgfpathlineto{\pgfqpoint{1.972831in}{1.736762in}}%
\pgfpathlineto{\pgfqpoint{1.973148in}{2.591731in}}%
\pgfpathlineto{\pgfqpoint{1.973783in}{1.656928in}}%
\pgfpathlineto{\pgfqpoint{1.974101in}{2.386394in}}%
\pgfpathlineto{\pgfqpoint{1.974418in}{2.564243in}}%
\pgfpathlineto{\pgfqpoint{1.974735in}{1.715783in}}%
\pgfpathlineto{\pgfqpoint{1.975370in}{2.682572in}}%
\pgfpathlineto{\pgfqpoint{1.975688in}{1.892433in}}%
\pgfpathlineto{\pgfqpoint{1.976005in}{1.850537in}}%
\pgfpathlineto{\pgfqpoint{1.976323in}{2.662975in}}%
\pgfpathlineto{\pgfqpoint{1.976958in}{1.684557in}}%
\pgfpathlineto{\pgfqpoint{1.977275in}{2.514757in}}%
\pgfpathlineto{\pgfqpoint{1.977593in}{2.417074in}}%
\pgfpathlineto{\pgfqpoint{1.977910in}{1.654078in}}%
\pgfpathlineto{\pgfqpoint{1.978545in}{2.614924in}}%
\pgfpathlineto{\pgfqpoint{1.978862in}{1.764721in}}%
\pgfpathlineto{\pgfqpoint{1.979497in}{2.691539in}}%
\pgfpathlineto{\pgfqpoint{1.979815in}{1.982475in}}%
\pgfpathlineto{\pgfqpoint{1.980132in}{1.776529in}}%
\pgfpathlineto{\pgfqpoint{1.980450in}{2.626831in}}%
\pgfpathlineto{\pgfqpoint{1.981085in}{1.662730in}}%
\pgfpathlineto{\pgfqpoint{1.981402in}{2.445359in}}%
\pgfpathlineto{\pgfqpoint{1.981720in}{2.510628in}}%
\pgfpathlineto{\pgfqpoint{1.982037in}{1.683408in}}%
\pgfpathlineto{\pgfqpoint{1.982672in}{2.658741in}}%
\pgfpathlineto{\pgfqpoint{1.982989in}{1.834329in}}%
\pgfpathlineto{\pgfqpoint{1.983307in}{1.904308in}}%
\pgfpathlineto{\pgfqpoint{1.983624in}{2.677418in}}%
\pgfpathlineto{\pgfqpoint{1.984259in}{1.712111in}}%
\pgfpathlineto{\pgfqpoint{1.984577in}{2.561781in}}%
\pgfpathlineto{\pgfqpoint{1.985212in}{1.646407in}}%
\pgfpathlineto{\pgfqpoint{1.985529in}{2.337885in}}%
\pgfpathlineto{\pgfqpoint{1.985847in}{2.571945in}}%
\pgfpathlineto{\pgfqpoint{1.986164in}{1.719783in}}%
\pgfpathlineto{\pgfqpoint{1.986799in}{2.681586in}}%
\pgfpathlineto{\pgfqpoint{1.987116in}{1.916998in}}%
\pgfpathlineto{\pgfqpoint{1.987434in}{1.822111in}}%
\pgfpathlineto{\pgfqpoint{1.987751in}{2.653791in}}%
\pgfpathlineto{\pgfqpoint{1.988386in}{1.676972in}}%
\pgfpathlineto{\pgfqpoint{1.988704in}{2.501581in}}%
\pgfpathlineto{\pgfqpoint{1.989021in}{2.449604in}}%
\pgfpathlineto{\pgfqpoint{1.989339in}{1.662823in}}%
\pgfpathlineto{\pgfqpoint{1.989973in}{2.627711in}}%
\pgfpathlineto{\pgfqpoint{1.990291in}{1.778814in}}%
\pgfpathlineto{\pgfqpoint{1.990926in}{2.681732in}}%
\pgfpathlineto{\pgfqpoint{1.991243in}{2.006798in}}%
\pgfpathlineto{\pgfqpoint{1.991561in}{1.747345in}}%
\pgfpathlineto{\pgfqpoint{1.991878in}{2.600617in}}%
\pgfpathlineto{\pgfqpoint{1.992513in}{1.646529in}}%
\pgfpathlineto{\pgfqpoint{1.992831in}{2.401190in}}%
\pgfpathlineto{\pgfqpoint{1.993148in}{2.519933in}}%
\pgfpathlineto{\pgfqpoint{1.993465in}{1.686161in}}%
\pgfpathlineto{\pgfqpoint{1.994100in}{2.663766in}}%
\pgfpathlineto{\pgfqpoint{1.994418in}{1.852773in}}%
\pgfpathlineto{\pgfqpoint{1.994735in}{1.876363in}}%
\pgfpathlineto{\pgfqpoint{1.995053in}{2.671464in}}%
\pgfpathlineto{\pgfqpoint{1.995688in}{1.700587in}}%
\pgfpathlineto{\pgfqpoint{1.996005in}{2.550349in}}%
\pgfpathlineto{\pgfqpoint{1.996640in}{1.649207in}}%
\pgfpathlineto{\pgfqpoint{1.996958in}{2.311191in}}%
\pgfpathlineto{\pgfqpoint{1.997275in}{2.586736in}}%
\pgfpathlineto{\pgfqpoint{1.997592in}{1.733639in}}%
\pgfpathlineto{\pgfqpoint{1.998227in}{2.677610in}}%
\pgfpathlineto{\pgfqpoint{1.998545in}{1.936350in}}%
\pgfpathlineto{\pgfqpoint{1.998862in}{1.791740in}}%
\pgfpathlineto{\pgfqpoint{1.999180in}{2.631936in}}%
\pgfpathlineto{\pgfqpoint{1.999815in}{1.657381in}}%
\pgfpathlineto{\pgfqpoint{2.000132in}{2.458053in}}%
\pgfpathlineto{\pgfqpoint{2.000450in}{2.462241in}}%
\pgfpathlineto{\pgfqpoint{2.000767in}{1.658285in}}%
\pgfpathlineto{\pgfqpoint{2.001402in}{2.634871in}}%
\pgfpathlineto{\pgfqpoint{2.001719in}{1.797395in}}%
\pgfpathlineto{\pgfqpoint{2.002037in}{1.933557in}}%
\pgfpathlineto{\pgfqpoint{2.002354in}{2.681327in}}%
\pgfpathlineto{\pgfqpoint{2.002989in}{1.733818in}}%
\pgfpathlineto{\pgfqpoint{2.003307in}{2.592713in}}%
\pgfpathlineto{\pgfqpoint{2.003942in}{1.647123in}}%
\pgfpathlineto{\pgfqpoint{2.004259in}{2.374127in}}%
\pgfpathlineto{\pgfqpoint{2.004577in}{2.539210in}}%
\pgfpathlineto{\pgfqpoint{2.004894in}{1.693006in}}%
\pgfpathlineto{\pgfqpoint{2.005529in}{2.662381in}}%
\pgfpathlineto{\pgfqpoint{2.005846in}{1.872844in}}%
\pgfpathlineto{\pgfqpoint{2.006164in}{1.841172in}}%
\pgfpathlineto{\pgfqpoint{2.006481in}{2.654637in}}%
\pgfpathlineto{\pgfqpoint{2.007116in}{1.677177in}}%
\pgfpathlineto{\pgfqpoint{2.007434in}{2.511445in}}%
\pgfpathlineto{\pgfqpoint{2.007751in}{2.398063in}}%
\pgfpathlineto{\pgfqpoint{2.008069in}{1.642276in}}%
\pgfpathlineto{\pgfqpoint{2.008703in}{2.598722in}}%
\pgfpathlineto{\pgfqpoint{2.009021in}{1.745198in}}%
\pgfpathlineto{\pgfqpoint{2.009656in}{2.680540in}}%
\pgfpathlineto{\pgfqpoint{2.009973in}{1.966530in}}%
\pgfpathlineto{\pgfqpoint{2.010291in}{1.774376in}}%
\pgfpathlineto{\pgfqpoint{2.010608in}{2.626966in}}%
\pgfpathlineto{\pgfqpoint{2.011243in}{1.653434in}}%
\pgfpathlineto{\pgfqpoint{2.011561in}{2.435905in}}%
\pgfpathlineto{\pgfqpoint{2.011878in}{2.483482in}}%
\pgfpathlineto{\pgfqpoint{2.012195in}{1.664236in}}%
\pgfpathlineto{\pgfqpoint{2.012830in}{2.639194in}}%
\pgfpathlineto{\pgfqpoint{2.013148in}{1.810996in}}%
\pgfpathlineto{\pgfqpoint{2.013465in}{1.899223in}}%
\pgfpathlineto{\pgfqpoint{2.013783in}{2.667535in}}%
\pgfpathlineto{\pgfqpoint{2.014418in}{1.705810in}}%
\pgfpathlineto{\pgfqpoint{2.014735in}{2.557079in}}%
\pgfpathlineto{\pgfqpoint{2.015370in}{1.633831in}}%
\pgfpathlineto{\pgfqpoint{2.015688in}{2.335659in}}%
\pgfpathlineto{\pgfqpoint{2.016005in}{2.552584in}}%
\pgfpathlineto{\pgfqpoint{2.016322in}{1.703932in}}%
\pgfpathlineto{\pgfqpoint{2.016957in}{2.671376in}}%
\pgfpathlineto{\pgfqpoint{2.017275in}{1.897248in}}%
\pgfpathlineto{\pgfqpoint{2.017592in}{1.823877in}}%
\pgfpathlineto{\pgfqpoint{2.017910in}{2.653310in}}%
\pgfpathlineto{\pgfqpoint{2.018545in}{1.669884in}}%
\pgfpathlineto{\pgfqpoint{2.018862in}{2.490328in}}%
\pgfpathlineto{\pgfqpoint{2.019180in}{2.422013in}}%
\pgfpathlineto{\pgfqpoint{2.019497in}{1.641926in}}%
\pgfpathlineto{\pgfqpoint{2.020132in}{2.605125in}}%
\pgfpathlineto{\pgfqpoint{2.020449in}{1.758815in}}%
\pgfpathlineto{\pgfqpoint{2.021084in}{2.671758in}}%
\pgfpathlineto{\pgfqpoint{2.021402in}{1.984368in}}%
\pgfpathlineto{\pgfqpoint{2.021719in}{1.743976in}}%
\pgfpathlineto{\pgfqpoint{2.022037in}{2.596020in}}%
\pgfpathlineto{\pgfqpoint{2.022672in}{1.636719in}}%
\pgfpathlineto{\pgfqpoint{2.022989in}{2.396938in}}%
\pgfpathlineto{\pgfqpoint{2.023307in}{2.499992in}}%
\pgfpathlineto{\pgfqpoint{2.023624in}{1.667910in}}%
\pgfpathlineto{\pgfqpoint{2.024259in}{2.650092in}}%
\pgfpathlineto{\pgfqpoint{2.024576in}{1.835515in}}%
\pgfpathlineto{\pgfqpoint{2.024894in}{1.877407in}}%
\pgfpathlineto{\pgfqpoint{2.025211in}{2.672043in}}%
\pgfpathlineto{\pgfqpoint{2.025846in}{1.695690in}}%
\pgfpathlineto{\pgfqpoint{2.026164in}{2.539908in}}%
\pgfpathlineto{\pgfqpoint{2.026799in}{1.631615in}}%
\pgfpathlineto{\pgfqpoint{2.027116in}{2.299397in}}%
\pgfpathlineto{\pgfqpoint{2.027433in}{2.563787in}}%
\pgfpathlineto{\pgfqpoint{2.027751in}{1.710541in}}%
\pgfpathlineto{\pgfqpoint{2.028386in}{2.664548in}}%
\pgfpathlineto{\pgfqpoint{2.028703in}{1.916196in}}%
\pgfpathlineto{\pgfqpoint{2.029021in}{1.788527in}}%
\pgfpathlineto{\pgfqpoint{2.029338in}{2.626466in}}%
\pgfpathlineto{\pgfqpoint{2.029973in}{1.648642in}}%
\pgfpathlineto{\pgfqpoint{2.030291in}{2.455984in}}%
\pgfpathlineto{\pgfqpoint{2.030608in}{2.439641in}}%
\pgfpathlineto{\pgfqpoint{2.030925in}{1.643838in}}%
\pgfpathlineto{\pgfqpoint{2.031560in}{2.621470in}}%
\pgfpathlineto{\pgfqpoint{2.031878in}{1.776507in}}%
\pgfpathlineto{\pgfqpoint{2.032195in}{1.939291in}}%
\pgfpathlineto{\pgfqpoint{2.032513in}{2.679272in}}%
\pgfpathlineto{\pgfqpoint{2.033148in}{1.729869in}}%
\pgfpathlineto{\pgfqpoint{2.033465in}{2.581799in}}%
\pgfpathlineto{\pgfqpoint{2.034100in}{1.629228in}}%
\pgfpathlineto{\pgfqpoint{2.034418in}{2.365397in}}%
\pgfpathlineto{\pgfqpoint{2.034735in}{2.513062in}}%
\pgfpathlineto{\pgfqpoint{2.035052in}{1.673730in}}%
\pgfpathlineto{\pgfqpoint{2.035687in}{2.648954in}}%
\pgfpathlineto{\pgfqpoint{2.036005in}{1.849106in}}%
\pgfpathlineto{\pgfqpoint{2.036322in}{1.842336in}}%
\pgfpathlineto{\pgfqpoint{2.036640in}{2.647486in}}%
\pgfpathlineto{\pgfqpoint{2.037275in}{1.670371in}}%
\pgfpathlineto{\pgfqpoint{2.037592in}{2.507675in}}%
\pgfpathlineto{\pgfqpoint{2.037910in}{2.373882in}}%
\pgfpathlineto{\pgfqpoint{2.038227in}{1.627023in}}%
\pgfpathlineto{\pgfqpoint{2.038862in}{2.581249in}}%
\pgfpathlineto{\pgfqpoint{2.039179in}{1.727858in}}%
\pgfpathlineto{\pgfqpoint{2.039814in}{2.677887in}}%
\pgfpathlineto{\pgfqpoint{2.040132in}{1.943826in}}%
\pgfpathlineto{\pgfqpoint{2.040449in}{1.773964in}}%
\pgfpathlineto{\pgfqpoint{2.040767in}{2.616228in}}%
\pgfpathlineto{\pgfqpoint{2.041402in}{1.637784in}}%
\pgfpathlineto{\pgfqpoint{2.041719in}{2.424952in}}%
\pgfpathlineto{\pgfqpoint{2.042037in}{2.455918in}}%
\pgfpathlineto{\pgfqpoint{2.042354in}{1.642980in}}%
\pgfpathlineto{\pgfqpoint{2.042989in}{2.621775in}}%
\pgfpathlineto{\pgfqpoint{2.043306in}{1.790774in}}%
\pgfpathlineto{\pgfqpoint{2.043624in}{1.899493in}}%
\pgfpathlineto{\pgfqpoint{2.043941in}{2.659625in}}%
\pgfpathlineto{\pgfqpoint{2.044576in}{1.701992in}}%
\pgfpathlineto{\pgfqpoint{2.044894in}{2.553657in}}%
\pgfpathlineto{\pgfqpoint{2.045529in}{1.622032in}}%
\pgfpathlineto{\pgfqpoint{2.045846in}{2.335120in}}%
\pgfpathlineto{\pgfqpoint{2.046163in}{2.534009in}}%
\pgfpathlineto{\pgfqpoint{2.046481in}{1.684303in}}%
\pgfpathlineto{\pgfqpoint{2.047116in}{2.664273in}}%
\pgfpathlineto{\pgfqpoint{2.047433in}{1.877299in}}%
\pgfpathlineto{\pgfqpoint{2.047751in}{1.823541in}}%
\pgfpathlineto{\pgfqpoint{2.048068in}{2.641798in}}%
\pgfpathlineto{\pgfqpoint{2.048703in}{1.656113in}}%
\pgfpathlineto{\pgfqpoint{2.049021in}{2.480913in}}%
\pgfpathlineto{\pgfqpoint{2.049338in}{2.391839in}}%
\pgfpathlineto{\pgfqpoint{2.049655in}{1.624283in}}%
\pgfpathlineto{\pgfqpoint{2.050290in}{2.586795in}}%
\pgfpathlineto{\pgfqpoint{2.050608in}{1.735711in}}%
\pgfpathlineto{\pgfqpoint{2.051243in}{2.660500in}}%
\pgfpathlineto{\pgfqpoint{2.051560in}{1.959839in}}%
\pgfpathlineto{\pgfqpoint{2.051878in}{1.741107in}}%
\pgfpathlineto{\pgfqpoint{2.052195in}{2.591232in}}%
\pgfpathlineto{\pgfqpoint{2.052830in}{1.626017in}}%
\pgfpathlineto{\pgfqpoint{2.053148in}{2.398914in}}%
\pgfpathlineto{\pgfqpoint{2.053465in}{2.477456in}}%
\pgfpathlineto{\pgfqpoint{2.053782in}{1.652406in}}%
\pgfpathlineto{\pgfqpoint{2.054417in}{2.642465in}}%
\pgfpathlineto{\pgfqpoint{2.054735in}{1.812792in}}%
\pgfpathlineto{\pgfqpoint{2.055052in}{1.882282in}}%
\pgfpathlineto{\pgfqpoint{2.055370in}{2.657226in}}%
\pgfpathlineto{\pgfqpoint{2.056005in}{1.684021in}}%
\pgfpathlineto{\pgfqpoint{2.056322in}{2.528940in}}%
\pgfpathlineto{\pgfqpoint{2.056957in}{1.613608in}}%
\pgfpathlineto{\pgfqpoint{2.057274in}{2.301434in}}%
\pgfpathlineto{\pgfqpoint{2.057592in}{2.541386in}}%
\pgfpathlineto{\pgfqpoint{2.057909in}{1.691641in}}%
\pgfpathlineto{\pgfqpoint{2.058544in}{2.652551in}}%
\pgfpathlineto{\pgfqpoint{2.058862in}{1.888466in}}%
\pgfpathlineto{\pgfqpoint{2.059179in}{1.790101in}}%
\pgfpathlineto{\pgfqpoint{2.059497in}{2.620039in}}%
\pgfpathlineto{\pgfqpoint{2.060132in}{1.640643in}}%
\pgfpathlineto{\pgfqpoint{2.060449in}{2.455882in}}%
\pgfpathlineto{\pgfqpoint{2.060767in}{2.415174in}}%
\pgfpathlineto{\pgfqpoint{2.061084in}{1.627351in}}%
\pgfpathlineto{\pgfqpoint{2.061719in}{2.608700in}}%
\pgfpathlineto{\pgfqpoint{2.062036in}{1.758341in}}%
\pgfpathlineto{\pgfqpoint{2.062671in}{2.663225in}}%
\pgfpathlineto{\pgfqpoint{2.062989in}{1.982277in}}%
\pgfpathlineto{\pgfqpoint{2.063306in}{1.721995in}}%
\pgfpathlineto{\pgfqpoint{2.063624in}{2.570280in}}%
\pgfpathlineto{\pgfqpoint{2.064259in}{1.614617in}}%
\pgfpathlineto{\pgfqpoint{2.064576in}{2.364812in}}%
\pgfpathlineto{\pgfqpoint{2.064893in}{2.489011in}}%
\pgfpathlineto{\pgfqpoint{2.065211in}{1.653232in}}%
\pgfpathlineto{\pgfqpoint{2.065846in}{2.632703in}}%
\pgfpathlineto{\pgfqpoint{2.066163in}{1.825003in}}%
\pgfpathlineto{\pgfqpoint{2.066481in}{1.843768in}}%
\pgfpathlineto{\pgfqpoint{2.066798in}{2.639877in}}%
\pgfpathlineto{\pgfqpoint{2.067433in}{1.665390in}}%
\pgfpathlineto{\pgfqpoint{2.067751in}{2.508148in}}%
\pgfpathlineto{\pgfqpoint{2.068068in}{2.346818in}}%
\pgfpathlineto{\pgfqpoint{2.068386in}{1.614417in}}%
\pgfpathlineto{\pgfqpoint{2.069020in}{2.567396in}}%
\pgfpathlineto{\pgfqpoint{2.069338in}{1.707953in}}%
\pgfpathlineto{\pgfqpoint{2.069973in}{2.657701in}}%
\pgfpathlineto{\pgfqpoint{2.070290in}{1.912384in}}%
\pgfpathlineto{\pgfqpoint{2.070608in}{1.766883in}}%
\pgfpathlineto{\pgfqpoint{2.070925in}{2.603113in}}%
\pgfpathlineto{\pgfqpoint{2.071560in}{1.625392in}}%
\pgfpathlineto{\pgfqpoint{2.071878in}{2.425709in}}%
\pgfpathlineto{\pgfqpoint{2.072195in}{2.428560in}}%
\pgfpathlineto{\pgfqpoint{2.072512in}{1.626968in}}%
\pgfpathlineto{\pgfqpoint{2.073147in}{2.604432in}}%
\pgfpathlineto{\pgfqpoint{2.073465in}{1.764081in}}%
\pgfpathlineto{\pgfqpoint{2.073782in}{1.906274in}}%
\pgfpathlineto{\pgfqpoint{2.074100in}{2.648682in}}%
\pgfpathlineto{\pgfqpoint{2.074735in}{1.698685in}}%
\pgfpathlineto{\pgfqpoint{2.075052in}{2.552235in}}%
\pgfpathlineto{\pgfqpoint{2.075687in}{1.610483in}}%
\pgfpathlineto{\pgfqpoint{2.076004in}{2.347777in}}%
\pgfpathlineto{\pgfqpoint{2.076322in}{2.515831in}}%
\pgfpathlineto{\pgfqpoint{2.076639in}{1.668991in}}%
\pgfpathlineto{\pgfqpoint{2.077274in}{2.643412in}}%
\pgfpathlineto{\pgfqpoint{2.077592in}{1.843492in}}%
\pgfpathlineto{\pgfqpoint{2.077909in}{1.821170in}}%
\pgfpathlineto{\pgfqpoint{2.078227in}{2.625950in}}%
\pgfpathlineto{\pgfqpoint{2.078862in}{1.646238in}}%
\pgfpathlineto{\pgfqpoint{2.079179in}{2.479328in}}%
\pgfpathlineto{\pgfqpoint{2.079497in}{2.362125in}}%
\pgfpathlineto{\pgfqpoint{2.079814in}{1.608467in}}%
\pgfpathlineto{\pgfqpoint{2.080449in}{2.565016in}}%
\pgfpathlineto{\pgfqpoint{2.080766in}{1.713714in}}%
\pgfpathlineto{\pgfqpoint{2.081401in}{2.648364in}}%
\pgfpathlineto{\pgfqpoint{2.081719in}{1.927135in}}%
\pgfpathlineto{\pgfqpoint{2.082036in}{1.742395in}}%
\pgfpathlineto{\pgfqpoint{2.082354in}{2.588244in}}%
\pgfpathlineto{\pgfqpoint{2.082989in}{1.617942in}}%
\pgfpathlineto{\pgfqpoint{2.083306in}{2.409667in}}%
\pgfpathlineto{\pgfqpoint{2.083623in}{2.457683in}}%
\pgfpathlineto{\pgfqpoint{2.083941in}{1.636334in}}%
\pgfpathlineto{\pgfqpoint{2.084576in}{2.617232in}}%
\pgfpathlineto{\pgfqpoint{2.084893in}{1.783696in}}%
\pgfpathlineto{\pgfqpoint{2.085211in}{1.879448in}}%
\pgfpathlineto{\pgfqpoint{2.085528in}{2.639776in}}%
\pgfpathlineto{\pgfqpoint{2.086163in}{1.677764in}}%
\pgfpathlineto{\pgfqpoint{2.086481in}{2.527129in}}%
\pgfpathlineto{\pgfqpoint{2.087116in}{1.601919in}}%
\pgfpathlineto{\pgfqpoint{2.087433in}{2.308828in}}%
\pgfpathlineto{\pgfqpoint{2.087750in}{2.517872in}}%
\pgfpathlineto{\pgfqpoint{2.088068in}{1.668349in}}%
\pgfpathlineto{\pgfqpoint{2.088703in}{2.635909in}}%
\pgfpathlineto{\pgfqpoint{2.089020in}{1.859215in}}%
\pgfpathlineto{\pgfqpoint{2.089338in}{1.791942in}}%
\pgfpathlineto{\pgfqpoint{2.089655in}{2.615256in}}%
\pgfpathlineto{\pgfqpoint{2.090290in}{1.635463in}}%
\pgfpathlineto{\pgfqpoint{2.090608in}{2.466761in}}%
\pgfpathlineto{\pgfqpoint{2.090925in}{2.391817in}}%
\pgfpathlineto{\pgfqpoint{2.091242in}{1.616505in}}%
\pgfpathlineto{\pgfqpoint{2.091877in}{2.582490in}}%
\pgfpathlineto{\pgfqpoint{2.092195in}{1.727251in}}%
\pgfpathlineto{\pgfqpoint{2.092830in}{2.642103in}}%
\pgfpathlineto{\pgfqpoint{2.093147in}{1.946596in}}%
\pgfpathlineto{\pgfqpoint{2.093465in}{1.716855in}}%
\pgfpathlineto{\pgfqpoint{2.093782in}{2.566394in}}%
\pgfpathlineto{\pgfqpoint{2.094417in}{1.605074in}}%
\pgfpathlineto{\pgfqpoint{2.094734in}{2.374269in}}%
\pgfpathlineto{\pgfqpoint{2.095052in}{2.461558in}}%
\pgfpathlineto{\pgfqpoint{2.095369in}{1.635048in}}%
\pgfpathlineto{\pgfqpoint{2.096004in}{2.614533in}}%
\pgfpathlineto{\pgfqpoint{2.096322in}{1.793139in}}%
\pgfpathlineto{\pgfqpoint{2.096639in}{1.851145in}}%
\pgfpathlineto{\pgfqpoint{2.096957in}{2.631554in}}%
\pgfpathlineto{\pgfqpoint{2.097592in}{1.662660in}}%
\pgfpathlineto{\pgfqpoint{2.097909in}{2.516544in}}%
\pgfpathlineto{\pgfqpoint{2.098544in}{1.605314in}}%
\pgfpathlineto{\pgfqpoint{2.098861in}{2.294599in}}%
\pgfpathlineto{\pgfqpoint{2.099179in}{2.536791in}}%
\pgfpathlineto{\pgfqpoint{2.099496in}{1.682020in}}%
\pgfpathlineto{\pgfqpoint{2.100131in}{2.635261in}}%
\pgfpathlineto{\pgfqpoint{2.100449in}{1.873653in}}%
\pgfpathlineto{\pgfqpoint{2.100766in}{1.766579in}}%
\pgfpathlineto{\pgfqpoint{2.101084in}{2.596570in}}%
\pgfpathlineto{\pgfqpoint{2.101719in}{1.619030in}}%
\pgfpathlineto{\pgfqpoint{2.102036in}{2.432733in}}%
\pgfpathlineto{\pgfqpoint{2.102353in}{2.398424in}}%
\pgfpathlineto{\pgfqpoint{2.102671in}{1.609075in}}%
\pgfpathlineto{\pgfqpoint{2.103306in}{2.581300in}}%
\pgfpathlineto{\pgfqpoint{2.103623in}{1.736972in}}%
\pgfpathlineto{\pgfqpoint{2.104258in}{2.638496in}}%
\pgfpathlineto{\pgfqpoint{2.104576in}{1.965426in}}%
\pgfpathlineto{\pgfqpoint{2.104893in}{1.700772in}}%
\pgfpathlineto{\pgfqpoint{2.105211in}{2.557884in}}%
\pgfpathlineto{\pgfqpoint{2.105846in}{1.605766in}}%
\pgfpathlineto{\pgfqpoint{2.106163in}{2.360417in}}%
\pgfpathlineto{\pgfqpoint{2.106480in}{2.484013in}}%
\pgfpathlineto{\pgfqpoint{2.106798in}{1.643090in}}%
\pgfpathlineto{\pgfqpoint{2.107433in}{2.616020in}}%
\pgfpathlineto{\pgfqpoint{2.107750in}{1.808839in}}%
\pgfpathlineto{\pgfqpoint{2.108068in}{1.821153in}}%
\pgfpathlineto{\pgfqpoint{2.108385in}{2.617432in}}%
\pgfpathlineto{\pgfqpoint{2.109020in}{1.643920in}}%
\pgfpathlineto{\pgfqpoint{2.109338in}{2.486430in}}%
\pgfpathlineto{\pgfqpoint{2.109655in}{2.329388in}}%
\pgfpathlineto{\pgfqpoint{2.109972in}{1.595557in}}%
\pgfpathlineto{\pgfqpoint{2.110607in}{2.539720in}}%
\pgfpathlineto{\pgfqpoint{2.110925in}{1.685237in}}%
\pgfpathlineto{\pgfqpoint{2.111560in}{2.633208in}}%
\pgfpathlineto{\pgfqpoint{2.111877in}{1.893688in}}%
\pgfpathlineto{\pgfqpoint{2.112195in}{1.745811in}}%
\pgfpathlineto{\pgfqpoint{2.112512in}{2.590980in}}%
\pgfpathlineto{\pgfqpoint{2.113147in}{1.616533in}}%
\pgfpathlineto{\pgfqpoint{2.113464in}{2.423204in}}%
\pgfpathlineto{\pgfqpoint{2.113782in}{2.422641in}}%
\pgfpathlineto{\pgfqpoint{2.114099in}{1.616510in}}%
\pgfpathlineto{\pgfqpoint{2.114734in}{2.587617in}}%
\pgfpathlineto{\pgfqpoint{2.115052in}{1.746883in}}%
\pgfpathlineto{\pgfqpoint{2.115369in}{1.885206in}}%
\pgfpathlineto{\pgfqpoint{2.115687in}{2.626880in}}%
\pgfpathlineto{\pgfqpoint{2.116322in}{1.677496in}}%
\pgfpathlineto{\pgfqpoint{2.116639in}{2.531709in}}%
\pgfpathlineto{\pgfqpoint{2.117274in}{1.591446in}}%
\pgfpathlineto{\pgfqpoint{2.117591in}{2.322475in}}%
\pgfpathlineto{\pgfqpoint{2.117909in}{2.488012in}}%
\pgfpathlineto{\pgfqpoint{2.118226in}{1.645384in}}%
\pgfpathlineto{\pgfqpoint{2.118861in}{2.618598in}}%
\pgfpathlineto{\pgfqpoint{2.119179in}{1.823075in}}%
\pgfpathlineto{\pgfqpoint{2.119496in}{1.801196in}}%
\pgfpathlineto{\pgfqpoint{2.119814in}{2.612989in}}%
\pgfpathlineto{\pgfqpoint{2.120449in}{1.637339in}}%
\pgfpathlineto{\pgfqpoint{2.120766in}{2.478770in}}%
\pgfpathlineto{\pgfqpoint{2.121083in}{2.355119in}}%
\pgfpathlineto{\pgfqpoint{2.121401in}{1.598563in}}%
\pgfpathlineto{\pgfqpoint{2.122036in}{2.548257in}}%
\pgfpathlineto{\pgfqpoint{2.122353in}{1.695689in}}%
\pgfpathlineto{\pgfqpoint{2.122988in}{2.626723in}}%
\pgfpathlineto{\pgfqpoint{2.123306in}{1.906768in}}%
\pgfpathlineto{\pgfqpoint{2.123623in}{1.722112in}}%
\pgfpathlineto{\pgfqpoint{2.123941in}{2.568442in}}%
\pgfpathlineto{\pgfqpoint{2.124576in}{1.598996in}}%
\pgfpathlineto{\pgfqpoint{2.124893in}{2.385436in}}%
\pgfpathlineto{\pgfqpoint{2.125210in}{2.428852in}}%
\pgfpathlineto{\pgfqpoint{2.125528in}{1.612550in}}%
\pgfpathlineto{\pgfqpoint{2.126163in}{2.591629in}}%
\pgfpathlineto{\pgfqpoint{2.126480in}{1.761679in}}%
\pgfpathlineto{\pgfqpoint{2.126798in}{1.860492in}}%
\pgfpathlineto{\pgfqpoint{2.127115in}{2.626781in}}%
\pgfpathlineto{\pgfqpoint{2.127750in}{1.669668in}}%
\pgfpathlineto{\pgfqpoint{2.128068in}{2.526866in}}%
\pgfpathlineto{\pgfqpoint{2.128702in}{1.592856in}}%
\pgfpathlineto{\pgfqpoint{2.129020in}{2.301011in}}%
\pgfpathlineto{\pgfqpoint{2.129337in}{2.500421in}}%
\pgfpathlineto{\pgfqpoint{2.129655in}{1.649962in}}%
\pgfpathlineto{\pgfqpoint{2.130290in}{2.614086in}}%
\pgfpathlineto{\pgfqpoint{2.130607in}{1.837608in}}%
\pgfpathlineto{\pgfqpoint{2.130925in}{1.772655in}}%
\pgfpathlineto{\pgfqpoint{2.131242in}{2.595836in}}%
\pgfpathlineto{\pgfqpoint{2.131877in}{1.617444in}}%
\pgfpathlineto{\pgfqpoint{2.132194in}{2.444223in}}%
\pgfpathlineto{\pgfqpoint{2.132512in}{2.362524in}}%
\pgfpathlineto{\pgfqpoint{2.132829in}{1.592301in}}%
\pgfpathlineto{\pgfqpoint{2.133464in}{2.555742in}}%
\pgfpathlineto{\pgfqpoint{2.133782in}{1.704094in}}%
\pgfpathlineto{\pgfqpoint{2.134417in}{2.628239in}}%
\pgfpathlineto{\pgfqpoint{2.134734in}{1.930868in}}%
\pgfpathlineto{\pgfqpoint{2.135052in}{1.710020in}}%
\pgfpathlineto{\pgfqpoint{2.135369in}{2.566261in}}%
\pgfpathlineto{\pgfqpoint{2.136004in}{1.597516in}}%
\pgfpathlineto{\pgfqpoint{2.136321in}{2.368660in}}%
\pgfpathlineto{\pgfqpoint{2.136639in}{2.443626in}}%
\pgfpathlineto{\pgfqpoint{2.136956in}{1.616800in}}%
\pgfpathlineto{\pgfqpoint{2.137591in}{2.591997in}}%
\pgfpathlineto{\pgfqpoint{2.137909in}{1.770384in}}%
\pgfpathlineto{\pgfqpoint{2.138226in}{1.833321in}}%
\pgfpathlineto{\pgfqpoint{2.138544in}{2.612049in}}%
\pgfpathlineto{\pgfqpoint{2.139179in}{1.645568in}}%
\pgfpathlineto{\pgfqpoint{2.139496in}{2.495092in}}%
\pgfpathlineto{\pgfqpoint{2.140131in}{1.581406in}}%
\pgfpathlineto{\pgfqpoint{2.140448in}{2.269788in}}%
\pgfpathlineto{\pgfqpoint{2.140766in}{2.508872in}}%
\pgfpathlineto{\pgfqpoint{2.141083in}{1.658073in}}%
\pgfpathlineto{\pgfqpoint{2.141718in}{2.620493in}}%
\pgfpathlineto{\pgfqpoint{2.142036in}{1.855945in}}%
\pgfpathlineto{\pgfqpoint{2.142353in}{1.761342in}}%
\pgfpathlineto{\pgfqpoint{2.142671in}{2.595877in}}%
\pgfpathlineto{\pgfqpoint{2.143306in}{1.613070in}}%
\pgfpathlineto{\pgfqpoint{2.143623in}{2.428964in}}%
\pgfpathlineto{\pgfqpoint{2.143940in}{2.379448in}}%
\pgfpathlineto{\pgfqpoint{2.144258in}{1.591768in}}%
\pgfpathlineto{\pgfqpoint{2.144893in}{2.557885in}}%
\pgfpathlineto{\pgfqpoint{2.145210in}{1.713386in}}%
\pgfpathlineto{\pgfqpoint{2.145845in}{2.618556in}}%
\pgfpathlineto{\pgfqpoint{2.146163in}{1.941257in}}%
\pgfpathlineto{\pgfqpoint{2.146480in}{1.685056in}}%
\pgfpathlineto{\pgfqpoint{2.146798in}{2.537945in}}%
\pgfpathlineto{\pgfqpoint{2.147432in}{1.582570in}}%
\pgfpathlineto{\pgfqpoint{2.147750in}{2.336333in}}%
\pgfpathlineto{\pgfqpoint{2.148067in}{2.453884in}}%
\pgfpathlineto{\pgfqpoint{2.148385in}{1.618657in}}%
\pgfpathlineto{\pgfqpoint{2.149020in}{2.598752in}}%
\pgfpathlineto{\pgfqpoint{2.149337in}{1.788836in}}%
\pgfpathlineto{\pgfqpoint{2.149655in}{1.816941in}}%
\pgfpathlineto{\pgfqpoint{2.149972in}{2.617371in}}%
\pgfpathlineto{\pgfqpoint{2.150607in}{1.640004in}}%
\pgfpathlineto{\pgfqpoint{2.150924in}{2.483286in}}%
\pgfpathlineto{\pgfqpoint{2.151559in}{1.579537in}}%
\pgfpathlineto{\pgfqpoint{2.151877in}{2.241291in}}%
\pgfpathlineto{\pgfqpoint{2.152194in}{2.514898in}}%
\pgfpathlineto{\pgfqpoint{2.152512in}{1.661389in}}%
\pgfpathlineto{\pgfqpoint{2.153147in}{2.612203in}}%
\pgfpathlineto{\pgfqpoint{2.153464in}{1.868279in}}%
\pgfpathlineto{\pgfqpoint{2.153782in}{1.731598in}}%
\pgfpathlineto{\pgfqpoint{2.154099in}{2.571709in}}%
\pgfpathlineto{\pgfqpoint{2.154734in}{1.594820in}}%
\pgfpathlineto{\pgfqpoint{2.155051in}{2.399704in}}%
\pgfpathlineto{\pgfqpoint{2.155369in}{2.390634in}}%
\pgfpathlineto{\pgfqpoint{2.155686in}{1.591910in}}%
\pgfpathlineto{\pgfqpoint{2.156321in}{2.568406in}}%
\pgfpathlineto{\pgfqpoint{2.156639in}{1.725428in}}%
\pgfpathlineto{\pgfqpoint{2.156956in}{1.882221in}}%
\pgfpathlineto{\pgfqpoint{2.157274in}{2.626047in}}%
\pgfpathlineto{\pgfqpoint{2.157909in}{1.675820in}}%
\pgfpathlineto{\pgfqpoint{2.158226in}{2.529168in}}%
\pgfpathlineto{\pgfqpoint{2.158861in}{1.577430in}}%
\pgfpathlineto{\pgfqpoint{2.159178in}{2.312101in}}%
\pgfpathlineto{\pgfqpoint{2.159496in}{2.461860in}}%
\pgfpathlineto{\pgfqpoint{2.159813in}{1.621615in}}%
\pgfpathlineto{\pgfqpoint{2.160448in}{2.596049in}}%
\pgfpathlineto{\pgfqpoint{2.160766in}{1.796691in}}%
\pgfpathlineto{\pgfqpoint{2.161083in}{1.788910in}}%
\pgfpathlineto{\pgfqpoint{2.161401in}{2.594461in}}%
\pgfpathlineto{\pgfqpoint{2.162036in}{1.617481in}}%
\pgfpathlineto{\pgfqpoint{2.162353in}{2.455538in}}%
\pgfpathlineto{\pgfqpoint{2.162670in}{2.320506in}}%
\pgfpathlineto{\pgfqpoint{2.162988in}{1.574344in}}%
\pgfpathlineto{\pgfqpoint{2.163623in}{2.525875in}}%
\pgfpathlineto{\pgfqpoint{2.163940in}{1.673165in}}%
\pgfpathlineto{\pgfqpoint{2.164575in}{2.624507in}}%
\pgfpathlineto{\pgfqpoint{2.164893in}{1.889431in}}%
\pgfpathlineto{\pgfqpoint{2.165210in}{1.723273in}}%
\pgfpathlineto{\pgfqpoint{2.165528in}{2.566417in}}%
\pgfpathlineto{\pgfqpoint{2.166162in}{1.586790in}}%
\pgfpathlineto{\pgfqpoint{2.166480in}{2.376346in}}%
\pgfpathlineto{\pgfqpoint{2.166797in}{2.400873in}}%
\pgfpathlineto{\pgfqpoint{2.167115in}{1.589681in}}%
\pgfpathlineto{\pgfqpoint{2.167750in}{2.567202in}}%
\pgfpathlineto{\pgfqpoint{2.168067in}{1.734678in}}%
\pgfpathlineto{\pgfqpoint{2.168385in}{1.850054in}}%
\pgfpathlineto{\pgfqpoint{2.168702in}{2.607329in}}%
\pgfpathlineto{\pgfqpoint{2.169337in}{1.652005in}}%
\pgfpathlineto{\pgfqpoint{2.169654in}{2.504118in}}%
\pgfpathlineto{\pgfqpoint{2.170289in}{1.569254in}}%
\pgfpathlineto{\pgfqpoint{2.170607in}{2.287850in}}%
\pgfpathlineto{\pgfqpoint{2.170924in}{2.474948in}}%
\pgfpathlineto{\pgfqpoint{2.171242in}{1.627513in}}%
\pgfpathlineto{\pgfqpoint{2.171877in}{2.609436in}}%
\pgfpathlineto{\pgfqpoint{2.172194in}{1.818877in}}%
\pgfpathlineto{\pgfqpoint{2.172512in}{1.776641in}}%
\pgfpathlineto{\pgfqpoint{2.172829in}{2.593967in}}%
\pgfpathlineto{\pgfqpoint{2.173464in}{1.607913in}}%
\pgfpathlineto{\pgfqpoint{2.173781in}{2.435773in}}%
\pgfpathlineto{\pgfqpoint{2.174099in}{2.332938in}}%
\pgfpathlineto{\pgfqpoint{2.174416in}{1.570575in}}%
\pgfpathlineto{\pgfqpoint{2.175051in}{2.528940in}}%
\pgfpathlineto{\pgfqpoint{2.175369in}{1.677041in}}%
\pgfpathlineto{\pgfqpoint{2.176004in}{2.607284in}}%
\pgfpathlineto{\pgfqpoint{2.176321in}{1.898107in}}%
\pgfpathlineto{\pgfqpoint{2.176639in}{1.694453in}}%
\pgfpathlineto{\pgfqpoint{2.176956in}{2.543702in}}%
\pgfpathlineto{\pgfqpoint{2.177591in}{1.575342in}}%
\pgfpathlineto{\pgfqpoint{2.177908in}{2.355083in}}%
\pgfpathlineto{\pgfqpoint{2.178226in}{2.414688in}}%
\pgfpathlineto{\pgfqpoint{2.178543in}{1.594628in}}%
\pgfpathlineto{\pgfqpoint{2.179178in}{2.584539in}}%
\pgfpathlineto{\pgfqpoint{2.179496in}{1.750754in}}%
\pgfpathlineto{\pgfqpoint{2.179813in}{1.840112in}}%
\pgfpathlineto{\pgfqpoint{2.180131in}{2.609611in}}%
\pgfpathlineto{\pgfqpoint{2.180766in}{1.639291in}}%
\pgfpathlineto{\pgfqpoint{2.181083in}{2.486965in}}%
\pgfpathlineto{\pgfqpoint{2.181718in}{1.561610in}}%
\pgfpathlineto{\pgfqpoint{2.182035in}{2.261055in}}%
\pgfpathlineto{\pgfqpoint{2.182353in}{2.479707in}}%
\pgfpathlineto{\pgfqpoint{2.182670in}{1.631311in}}%
\pgfpathlineto{\pgfqpoint{2.183305in}{2.597119in}}%
\pgfpathlineto{\pgfqpoint{2.183623in}{1.822948in}}%
\pgfpathlineto{\pgfqpoint{2.183940in}{1.748289in}}%
\pgfpathlineto{\pgfqpoint{2.184258in}{2.572648in}}%
\pgfpathlineto{\pgfqpoint{2.184892in}{1.592531in}}%
\pgfpathlineto{\pgfqpoint{2.185210in}{2.415237in}}%
\pgfpathlineto{\pgfqpoint{2.185527in}{2.347209in}}%
\pgfpathlineto{\pgfqpoint{2.185845in}{1.570764in}}%
\pgfpathlineto{\pgfqpoint{2.186480in}{2.546791in}}%
\pgfpathlineto{\pgfqpoint{2.186797in}{1.693586in}}%
\pgfpathlineto{\pgfqpoint{2.187432in}{2.614773in}}%
\pgfpathlineto{\pgfqpoint{2.187750in}{1.915329in}}%
\pgfpathlineto{\pgfqpoint{2.188067in}{1.682599in}}%
\pgfpathlineto{\pgfqpoint{2.188384in}{2.529484in}}%
\pgfpathlineto{\pgfqpoint{2.189019in}{1.564847in}}%
\pgfpathlineto{\pgfqpoint{2.189337in}{2.328481in}}%
\pgfpathlineto{\pgfqpoint{2.189654in}{2.421928in}}%
\pgfpathlineto{\pgfqpoint{2.189972in}{1.593021in}}%
\pgfpathlineto{\pgfqpoint{2.190607in}{2.573731in}}%
\pgfpathlineto{\pgfqpoint{2.190924in}{1.756701in}}%
\pgfpathlineto{\pgfqpoint{2.191242in}{1.806960in}}%
\pgfpathlineto{\pgfqpoint{2.191559in}{2.591664in}}%
\pgfpathlineto{\pgfqpoint{2.192194in}{1.621948in}}%
\pgfpathlineto{\pgfqpoint{2.192511in}{2.468869in}}%
\pgfpathlineto{\pgfqpoint{2.193146in}{1.559668in}}%
\pgfpathlineto{\pgfqpoint{2.193464in}{2.242490in}}%
\pgfpathlineto{\pgfqpoint{2.193781in}{2.499511in}}%
\pgfpathlineto{\pgfqpoint{2.194099in}{1.642216in}}%
\pgfpathlineto{\pgfqpoint{2.194734in}{2.606398in}}%
\pgfpathlineto{\pgfqpoint{2.195051in}{1.842253in}}%
\pgfpathlineto{\pgfqpoint{2.195369in}{1.733045in}}%
\pgfpathlineto{\pgfqpoint{2.195686in}{2.562928in}}%
\pgfpathlineto{\pgfqpoint{2.196321in}{1.580348in}}%
\pgfpathlineto{\pgfqpoint{2.196638in}{2.391596in}}%
\pgfpathlineto{\pgfqpoint{2.196956in}{2.356208in}}%
\pgfpathlineto{\pgfqpoint{2.197273in}{1.567709in}}%
\pgfpathlineto{\pgfqpoint{2.197908in}{2.540505in}}%
\pgfpathlineto{\pgfqpoint{2.198226in}{1.694275in}}%
\pgfpathlineto{\pgfqpoint{2.198861in}{2.597809in}}%
\pgfpathlineto{\pgfqpoint{2.199178in}{1.925669in}}%
\pgfpathlineto{\pgfqpoint{2.199496in}{1.660133in}}%
\pgfpathlineto{\pgfqpoint{2.199813in}{2.513720in}}%
\pgfpathlineto{\pgfqpoint{2.200448in}{1.559761in}}%
\pgfpathlineto{\pgfqpoint{2.200765in}{2.312595in}}%
\pgfpathlineto{\pgfqpoint{2.201083in}{2.441934in}}%
\pgfpathlineto{\pgfqpoint{2.201400in}{1.603434in}}%
\pgfpathlineto{\pgfqpoint{2.202035in}{2.587674in}}%
\pgfpathlineto{\pgfqpoint{2.202353in}{1.770976in}}%
\pgfpathlineto{\pgfqpoint{2.202670in}{1.794257in}}%
\pgfpathlineto{\pgfqpoint{2.202988in}{2.584416in}}%
\pgfpathlineto{\pgfqpoint{2.203622in}{1.606263in}}%
\pgfpathlineto{\pgfqpoint{2.203940in}{2.446864in}}%
\pgfpathlineto{\pgfqpoint{2.204575in}{1.552499in}}%
\pgfpathlineto{\pgfqpoint{2.204892in}{2.215793in}}%
\pgfpathlineto{\pgfqpoint{2.205210in}{2.495409in}}%
\pgfpathlineto{\pgfqpoint{2.205527in}{1.643386in}}%
\pgfpathlineto{\pgfqpoint{2.206162in}{2.593556in}}%
\pgfpathlineto{\pgfqpoint{2.206480in}{1.847494in}}%
\pgfpathlineto{\pgfqpoint{2.206797in}{1.710350in}}%
\pgfpathlineto{\pgfqpoint{2.207114in}{2.548335in}}%
\pgfpathlineto{\pgfqpoint{2.207749in}{1.571543in}}%
\pgfpathlineto{\pgfqpoint{2.208067in}{2.376215in}}%
\pgfpathlineto{\pgfqpoint{2.208384in}{2.376244in}}%
\pgfpathlineto{\pgfqpoint{2.208702in}{1.572774in}}%
\pgfpathlineto{\pgfqpoint{2.209337in}{2.555976in}}%
\pgfpathlineto{\pgfqpoint{2.209654in}{1.710206in}}%
\pgfpathlineto{\pgfqpoint{2.209972in}{1.859239in}}%
\pgfpathlineto{\pgfqpoint{2.210289in}{2.595643in}}%
\pgfpathlineto{\pgfqpoint{2.210924in}{1.644969in}}%
\pgfpathlineto{\pgfqpoint{2.211241in}{2.494405in}}%
\pgfpathlineto{\pgfqpoint{2.211876in}{1.549843in}}%
\pgfpathlineto{\pgfqpoint{2.212194in}{2.285518in}}%
\pgfpathlineto{\pgfqpoint{2.212511in}{2.441294in}}%
\pgfpathlineto{\pgfqpoint{2.212829in}{1.599628in}}%
\pgfpathlineto{\pgfqpoint{2.213464in}{2.575760in}}%
\pgfpathlineto{\pgfqpoint{2.213781in}{1.777594in}}%
\pgfpathlineto{\pgfqpoint{2.214099in}{1.766346in}}%
\pgfpathlineto{\pgfqpoint{2.214416in}{2.573010in}}%
\pgfpathlineto{\pgfqpoint{2.215051in}{1.595838in}}%
\pgfpathlineto{\pgfqpoint{2.215368in}{2.434041in}}%
\pgfpathlineto{\pgfqpoint{2.215686in}{2.303808in}}%
\pgfpathlineto{\pgfqpoint{2.216003in}{1.555484in}}%
\pgfpathlineto{\pgfqpoint{2.216638in}{2.514260in}}%
\pgfpathlineto{\pgfqpoint{2.216956in}{1.654442in}}%
\pgfpathlineto{\pgfqpoint{2.217591in}{2.593235in}}%
\pgfpathlineto{\pgfqpoint{2.217908in}{1.861872in}}%
\pgfpathlineto{\pgfqpoint{2.218226in}{1.691340in}}%
\pgfpathlineto{\pgfqpoint{2.218543in}{2.532532in}}%
\pgfpathlineto{\pgfqpoint{2.219178in}{1.559335in}}%
\pgfpathlineto{\pgfqpoint{2.219495in}{2.352138in}}%
\pgfpathlineto{\pgfqpoint{2.219813in}{2.378168in}}%
\pgfpathlineto{\pgfqpoint{2.220130in}{1.568757in}}%
\pgfpathlineto{\pgfqpoint{2.220765in}{2.547597in}}%
\pgfpathlineto{\pgfqpoint{2.221083in}{1.710952in}}%
\pgfpathlineto{\pgfqpoint{2.221400in}{1.832741in}}%
\pgfpathlineto{\pgfqpoint{2.221718in}{2.584903in}}%
\pgfpathlineto{\pgfqpoint{2.222352in}{1.629655in}}%
\pgfpathlineto{\pgfqpoint{2.222670in}{2.483274in}}%
\pgfpathlineto{\pgfqpoint{2.223305in}{1.549247in}}%
\pgfpathlineto{\pgfqpoint{2.223622in}{2.277792in}}%
\pgfpathlineto{\pgfqpoint{2.223940in}{2.461261in}}%
\pgfpathlineto{\pgfqpoint{2.224257in}{1.610894in}}%
\pgfpathlineto{\pgfqpoint{2.224892in}{2.580833in}}%
\pgfpathlineto{\pgfqpoint{2.225210in}{1.787569in}}%
\pgfpathlineto{\pgfqpoint{2.225527in}{1.749554in}}%
\pgfpathlineto{\pgfqpoint{2.225844in}{2.559389in}}%
\pgfpathlineto{\pgfqpoint{2.226479in}{1.579944in}}%
\pgfpathlineto{\pgfqpoint{2.226797in}{2.411338in}}%
\pgfpathlineto{\pgfqpoint{2.227114in}{2.307113in}}%
\pgfpathlineto{\pgfqpoint{2.227432in}{1.547739in}}%
\pgfpathlineto{\pgfqpoint{2.228067in}{2.506970in}}%
\pgfpathlineto{\pgfqpoint{2.228384in}{1.655540in}}%
\pgfpathlineto{\pgfqpoint{2.229019in}{2.586185in}}%
\pgfpathlineto{\pgfqpoint{2.229337in}{1.870393in}}%
\pgfpathlineto{\pgfqpoint{2.229654in}{1.676047in}}%
\pgfpathlineto{\pgfqpoint{2.229971in}{2.523106in}}%
\pgfpathlineto{\pgfqpoint{2.230606in}{1.555570in}}%
\pgfpathlineto{\pgfqpoint{2.230924in}{2.345271in}}%
\pgfpathlineto{\pgfqpoint{2.231241in}{2.399352in}}%
\pgfpathlineto{\pgfqpoint{2.231559in}{1.575556in}}%
\pgfpathlineto{\pgfqpoint{2.232194in}{2.554539in}}%
\pgfpathlineto{\pgfqpoint{2.232511in}{1.722764in}}%
\pgfpathlineto{\pgfqpoint{2.232829in}{1.812130in}}%
\pgfpathlineto{\pgfqpoint{2.233146in}{2.575661in}}%
\pgfpathlineto{\pgfqpoint{2.233781in}{1.613591in}}%
\pgfpathlineto{\pgfqpoint{2.234098in}{2.463105in}}%
\pgfpathlineto{\pgfqpoint{2.234733in}{1.539590in}}%
\pgfpathlineto{\pgfqpoint{2.235051in}{2.245430in}}%
\pgfpathlineto{\pgfqpoint{2.235368in}{2.456582in}}%
\pgfpathlineto{\pgfqpoint{2.235686in}{1.606763in}}%
\pgfpathlineto{\pgfqpoint{2.236321in}{2.573670in}}%
\pgfpathlineto{\pgfqpoint{2.236638in}{1.797156in}}%
\pgfpathlineto{\pgfqpoint{2.236956in}{1.729086in}}%
\pgfpathlineto{\pgfqpoint{2.237273in}{2.553015in}}%
\pgfpathlineto{\pgfqpoint{2.237908in}{1.574010in}}%
\pgfpathlineto{\pgfqpoint{2.238225in}{2.406207in}}%
\pgfpathlineto{\pgfqpoint{2.238543in}{2.329486in}}%
\pgfpathlineto{\pgfqpoint{2.238860in}{1.553766in}}%
\pgfpathlineto{\pgfqpoint{2.239495in}{2.518020in}}%
\pgfpathlineto{\pgfqpoint{2.239813in}{1.662859in}}%
\pgfpathlineto{\pgfqpoint{2.240448in}{2.578923in}}%
\pgfpathlineto{\pgfqpoint{2.240765in}{1.881326in}}%
\pgfpathlineto{\pgfqpoint{2.241082in}{1.655782in}}%
\pgfpathlineto{\pgfqpoint{2.241400in}{2.505742in}}%
\pgfpathlineto{\pgfqpoint{2.242035in}{1.543627in}}%
\pgfpathlineto{\pgfqpoint{2.242352in}{2.315349in}}%
\pgfpathlineto{\pgfqpoint{2.242670in}{2.396325in}}%
\pgfpathlineto{\pgfqpoint{2.242987in}{1.570873in}}%
\pgfpathlineto{\pgfqpoint{2.243622in}{2.550385in}}%
\pgfpathlineto{\pgfqpoint{2.243940in}{1.726637in}}%
\pgfpathlineto{\pgfqpoint{2.244257in}{1.793272in}}%
\pgfpathlineto{\pgfqpoint{2.244574in}{2.570205in}}%
\pgfpathlineto{\pgfqpoint{2.245209in}{1.603228in}}%
\pgfpathlineto{\pgfqpoint{2.245527in}{2.459690in}}%
\pgfpathlineto{\pgfqpoint{2.246162in}{1.543032in}}%
\pgfpathlineto{\pgfqpoint{2.246479in}{2.238152in}}%
\pgfpathlineto{\pgfqpoint{2.246797in}{2.469259in}}%
\pgfpathlineto{\pgfqpoint{2.247114in}{1.614846in}}%
\pgfpathlineto{\pgfqpoint{2.247749in}{2.571088in}}%
\pgfpathlineto{\pgfqpoint{2.248067in}{1.803524in}}%
\pgfpathlineto{\pgfqpoint{2.248384in}{1.710397in}}%
\pgfpathlineto{\pgfqpoint{2.248701in}{2.537276in}}%
\pgfpathlineto{\pgfqpoint{2.249336in}{1.559282in}}%
\pgfpathlineto{\pgfqpoint{2.249654in}{2.378241in}}%
\pgfpathlineto{\pgfqpoint{2.249971in}{2.327520in}}%
\pgfpathlineto{\pgfqpoint{2.250289in}{1.544721in}}%
\pgfpathlineto{\pgfqpoint{2.250924in}{2.514032in}}%
\pgfpathlineto{\pgfqpoint{2.251241in}{1.666979in}}%
\pgfpathlineto{\pgfqpoint{2.251876in}{2.576600in}}%
\pgfpathlineto{\pgfqpoint{2.252193in}{1.891809in}}%
\pgfpathlineto{\pgfqpoint{2.252511in}{1.645535in}}%
\pgfpathlineto{\pgfqpoint{2.252828in}{2.502368in}}%
\pgfpathlineto{\pgfqpoint{2.253463in}{1.544666in}}%
\pgfpathlineto{\pgfqpoint{2.253781in}{2.309230in}}%
\pgfpathlineto{\pgfqpoint{2.254098in}{2.411594in}}%
\pgfpathlineto{\pgfqpoint{2.254416in}{1.575335in}}%
\pgfpathlineto{\pgfqpoint{2.255051in}{2.549540in}}%
\pgfpathlineto{\pgfqpoint{2.255368in}{1.734960in}}%
\pgfpathlineto{\pgfqpoint{2.255686in}{1.770523in}}%
\pgfpathlineto{\pgfqpoint{2.256003in}{2.558343in}}%
\pgfpathlineto{\pgfqpoint{2.256638in}{1.588301in}}%
\pgfpathlineto{\pgfqpoint{2.256955in}{2.434205in}}%
\pgfpathlineto{\pgfqpoint{2.257590in}{1.531732in}}%
\pgfpathlineto{\pgfqpoint{2.257908in}{2.206436in}}%
\pgfpathlineto{\pgfqpoint{2.258225in}{2.467413in}}%
\pgfpathlineto{\pgfqpoint{2.258543in}{1.613628in}}%
\pgfpathlineto{\pgfqpoint{2.259178in}{2.568960in}}%
\pgfpathlineto{\pgfqpoint{2.259495in}{1.815648in}}%
\pgfpathlineto{\pgfqpoint{2.259812in}{1.695670in}}%
\pgfpathlineto{\pgfqpoint{2.260130in}{2.536048in}}%
\pgfpathlineto{\pgfqpoint{2.260765in}{1.559170in}}%
\pgfpathlineto{\pgfqpoint{2.261082in}{2.375263in}}%
\pgfpathlineto{\pgfqpoint{2.261400in}{2.344962in}}%
\pgfpathlineto{\pgfqpoint{2.261717in}{1.548951in}}%
\pgfpathlineto{\pgfqpoint{2.262352in}{2.516797in}}%
\pgfpathlineto{\pgfqpoint{2.262670in}{1.670684in}}%
\pgfpathlineto{\pgfqpoint{2.263305in}{2.565978in}}%
\pgfpathlineto{\pgfqpoint{2.263622in}{1.900969in}}%
\pgfpathlineto{\pgfqpoint{2.263939in}{1.626597in}}%
\pgfpathlineto{\pgfqpoint{2.264257in}{2.481199in}}%
\pgfpathlineto{\pgfqpoint{2.264892in}{1.530928in}}%
\pgfpathlineto{\pgfqpoint{2.265209in}{2.279395in}}%
\pgfpathlineto{\pgfqpoint{2.265527in}{2.410175in}}%
\pgfpathlineto{\pgfqpoint{2.265844in}{1.573126in}}%
\pgfpathlineto{\pgfqpoint{2.266479in}{2.550227in}}%
\pgfpathlineto{\pgfqpoint{2.266797in}{1.741647in}}%
\pgfpathlineto{\pgfqpoint{2.267114in}{1.757579in}}%
\pgfpathlineto{\pgfqpoint{2.267431in}{2.556274in}}%
\pgfpathlineto{\pgfqpoint{2.268066in}{1.584126in}}%
\pgfpathlineto{\pgfqpoint{2.268384in}{2.433479in}}%
\pgfpathlineto{\pgfqpoint{2.268701in}{2.270631in}}%
\pgfpathlineto{\pgfqpoint{2.269019in}{1.533963in}}%
\pgfpathlineto{\pgfqpoint{2.269654in}{2.472394in}}%
\pgfpathlineto{\pgfqpoint{2.269971in}{1.618371in}}%
\pgfpathlineto{\pgfqpoint{2.270606in}{2.562455in}}%
\pgfpathlineto{\pgfqpoint{2.270923in}{1.820212in}}%
\pgfpathlineto{\pgfqpoint{2.271241in}{1.677836in}}%
\pgfpathlineto{\pgfqpoint{2.271558in}{2.517171in}}%
\pgfpathlineto{\pgfqpoint{2.272193in}{1.542279in}}%
\pgfpathlineto{\pgfqpoint{2.272511in}{2.345822in}}%
\pgfpathlineto{\pgfqpoint{2.272828in}{2.343784in}}%
\pgfpathlineto{\pgfqpoint{2.273146in}{1.542267in}}%
\pgfpathlineto{\pgfqpoint{2.273781in}{2.517911in}}%
\pgfpathlineto{\pgfqpoint{2.274098in}{1.678092in}}%
\pgfpathlineto{\pgfqpoint{2.274416in}{1.823627in}}%
\pgfpathlineto{\pgfqpoint{2.274733in}{2.567021in}}%
\pgfpathlineto{\pgfqpoint{2.275368in}{1.622837in}}%
\pgfpathlineto{\pgfqpoint{2.275685in}{2.481691in}}%
\pgfpathlineto{\pgfqpoint{2.276320in}{1.531807in}}%
\pgfpathlineto{\pgfqpoint{2.276638in}{2.267947in}}%
\pgfpathlineto{\pgfqpoint{2.276955in}{2.417587in}}%
\pgfpathlineto{\pgfqpoint{2.277273in}{1.573908in}}%
\pgfpathlineto{\pgfqpoint{2.277908in}{2.544828in}}%
\pgfpathlineto{\pgfqpoint{2.278225in}{1.748114in}}%
\pgfpathlineto{\pgfqpoint{2.278542in}{1.735291in}}%
\pgfpathlineto{\pgfqpoint{2.278860in}{2.542683in}}%
\pgfpathlineto{\pgfqpoint{2.279495in}{1.567214in}}%
\pgfpathlineto{\pgfqpoint{2.279812in}{2.405709in}}%
\pgfpathlineto{\pgfqpoint{2.280130in}{2.270618in}}%
\pgfpathlineto{\pgfqpoint{2.280447in}{1.524779in}}%
\pgfpathlineto{\pgfqpoint{2.281082in}{2.474832in}}%
\pgfpathlineto{\pgfqpoint{2.281400in}{1.620461in}}%
\pgfpathlineto{\pgfqpoint{2.282035in}{2.563397in}}%
\pgfpathlineto{\pgfqpoint{2.282352in}{1.834854in}}%
\pgfpathlineto{\pgfqpoint{2.282669in}{1.669800in}}%
\pgfpathlineto{\pgfqpoint{2.282987in}{2.520021in}}%
\pgfpathlineto{\pgfqpoint{2.283622in}{1.542790in}}%
\pgfpathlineto{\pgfqpoint{2.283939in}{2.338187in}}%
\pgfpathlineto{\pgfqpoint{2.284257in}{2.353960in}}%
\pgfpathlineto{\pgfqpoint{2.284574in}{1.543081in}}%
\pgfpathlineto{\pgfqpoint{2.285209in}{2.516025in}}%
\pgfpathlineto{\pgfqpoint{2.285527in}{1.679742in}}%
\pgfpathlineto{\pgfqpoint{2.285844in}{1.803770in}}%
\pgfpathlineto{\pgfqpoint{2.286161in}{2.554645in}}%
\pgfpathlineto{\pgfqpoint{2.286796in}{1.602057in}}%
\pgfpathlineto{\pgfqpoint{2.287114in}{2.456828in}}%
\pgfpathlineto{\pgfqpoint{2.287749in}{1.519620in}}%
\pgfpathlineto{\pgfqpoint{2.288066in}{2.244014in}}%
\pgfpathlineto{\pgfqpoint{2.288384in}{2.420442in}}%
\pgfpathlineto{\pgfqpoint{2.288701in}{1.575582in}}%
\pgfpathlineto{\pgfqpoint{2.289336in}{2.548371in}}%
\pgfpathlineto{\pgfqpoint{2.289653in}{1.756710in}}%
\pgfpathlineto{\pgfqpoint{2.289971in}{1.728622in}}%
\pgfpathlineto{\pgfqpoint{2.290288in}{2.546331in}}%
\pgfpathlineto{\pgfqpoint{2.290923in}{1.565038in}}%
\pgfpathlineto{\pgfqpoint{2.291241in}{2.400355in}}%
\pgfpathlineto{\pgfqpoint{2.291558in}{2.281458in}}%
\pgfpathlineto{\pgfqpoint{2.291876in}{1.523379in}}%
\pgfpathlineto{\pgfqpoint{2.292511in}{2.474427in}}%
\pgfpathlineto{\pgfqpoint{2.292828in}{1.622976in}}%
\pgfpathlineto{\pgfqpoint{2.293463in}{2.555338in}}%
\pgfpathlineto{\pgfqpoint{2.293780in}{1.836774in}}%
\pgfpathlineto{\pgfqpoint{2.294098in}{1.650255in}}%
\pgfpathlineto{\pgfqpoint{2.294415in}{2.497113in}}%
\pgfpathlineto{\pgfqpoint{2.295050in}{1.526997in}}%
\pgfpathlineto{\pgfqpoint{2.295368in}{2.313450in}}%
\pgfpathlineto{\pgfqpoint{2.295685in}{2.356326in}}%
\pgfpathlineto{\pgfqpoint{2.296003in}{1.540232in}}%
\pgfpathlineto{\pgfqpoint{2.296638in}{2.519380in}}%
\pgfpathlineto{\pgfqpoint{2.296955in}{1.688757in}}%
\pgfpathlineto{\pgfqpoint{2.297272in}{1.792073in}}%
\pgfpathlineto{\pgfqpoint{2.297590in}{2.562515in}}%
\pgfpathlineto{\pgfqpoint{2.298225in}{1.601022in}}%
\pgfpathlineto{\pgfqpoint{2.298542in}{2.453447in}}%
\pgfpathlineto{\pgfqpoint{2.299177in}{1.517189in}}%
\pgfpathlineto{\pgfqpoint{2.299495in}{2.226929in}}%
\pgfpathlineto{\pgfqpoint{2.299812in}{2.422321in}}%
\pgfpathlineto{\pgfqpoint{2.300130in}{1.574012in}}%
\pgfpathlineto{\pgfqpoint{2.300765in}{2.541444in}}%
\pgfpathlineto{\pgfqpoint{2.301082in}{1.761504in}}%
\pgfpathlineto{\pgfqpoint{2.301399in}{1.705422in}}%
\pgfpathlineto{\pgfqpoint{2.301717in}{2.527010in}}%
\pgfpathlineto{\pgfqpoint{2.302352in}{1.548248in}}%
\pgfpathlineto{\pgfqpoint{2.302669in}{2.376867in}}%
\pgfpathlineto{\pgfqpoint{2.302987in}{2.284649in}}%
\pgfpathlineto{\pgfqpoint{2.303304in}{1.518448in}}%
\pgfpathlineto{\pgfqpoint{2.303939in}{2.479684in}}%
\pgfpathlineto{\pgfqpoint{2.304257in}{1.627111in}}%
\pgfpathlineto{\pgfqpoint{2.304891in}{2.563150in}}%
\pgfpathlineto{\pgfqpoint{2.305209in}{1.854045in}}%
\pgfpathlineto{\pgfqpoint{2.305526in}{1.645876in}}%
\pgfpathlineto{\pgfqpoint{2.305844in}{2.497530in}}%
\pgfpathlineto{\pgfqpoint{2.306479in}{1.524243in}}%
\pgfpathlineto{\pgfqpoint{2.306796in}{2.299733in}}%
\pgfpathlineto{\pgfqpoint{2.307114in}{2.360450in}}%
\pgfpathlineto{\pgfqpoint{2.307431in}{1.538474in}}%
\pgfpathlineto{\pgfqpoint{2.308066in}{2.515964in}}%
\pgfpathlineto{\pgfqpoint{2.308383in}{1.689599in}}%
\pgfpathlineto{\pgfqpoint{2.308701in}{1.772206in}}%
\pgfpathlineto{\pgfqpoint{2.309018in}{2.543245in}}%
\pgfpathlineto{\pgfqpoint{2.309653in}{1.579973in}}%
\pgfpathlineto{\pgfqpoint{2.309971in}{2.431750in}}%
\pgfpathlineto{\pgfqpoint{2.310606in}{1.509114in}}%
\pgfpathlineto{\pgfqpoint{2.310923in}{2.210231in}}%
\pgfpathlineto{\pgfqpoint{2.311241in}{2.428005in}}%
\pgfpathlineto{\pgfqpoint{2.311558in}{1.578300in}}%
\pgfpathlineto{\pgfqpoint{2.312193in}{2.551782in}}%
\pgfpathlineto{\pgfqpoint{2.312510in}{1.773302in}}%
\pgfpathlineto{\pgfqpoint{2.312828in}{1.703307in}}%
\pgfpathlineto{\pgfqpoint{2.313145in}{2.529560in}}%
\pgfpathlineto{\pgfqpoint{2.313780in}{1.542985in}}%
\pgfpathlineto{\pgfqpoint{2.314098in}{2.365035in}}%
\pgfpathlineto{\pgfqpoint{2.314415in}{2.289373in}}%
\pgfpathlineto{\pgfqpoint{2.314733in}{1.514189in}}%
\pgfpathlineto{\pgfqpoint{2.315368in}{2.477087in}}%
\pgfpathlineto{\pgfqpoint{2.315685in}{1.629016in}}%
\pgfpathlineto{\pgfqpoint{2.316320in}{2.547918in}}%
\pgfpathlineto{\pgfqpoint{2.316637in}{1.851423in}}%
\pgfpathlineto{\pgfqpoint{2.316955in}{1.625589in}}%
\pgfpathlineto{\pgfqpoint{2.317272in}{2.476112in}}%
\pgfpathlineto{\pgfqpoint{2.317907in}{1.512707in}}%
\pgfpathlineto{\pgfqpoint{2.318225in}{2.282184in}}%
\pgfpathlineto{\pgfqpoint{2.318542in}{2.366021in}}%
\pgfpathlineto{\pgfqpoint{2.318860in}{1.538830in}}%
\pgfpathlineto{\pgfqpoint{2.319495in}{2.525135in}}%
\pgfpathlineto{\pgfqpoint{2.319812in}{1.701927in}}%
\pgfpathlineto{\pgfqpoint{2.320129in}{1.766146in}}%
\pgfpathlineto{\pgfqpoint{2.320447in}{2.550597in}}%
\pgfpathlineto{\pgfqpoint{2.321082in}{1.576376in}}%
\pgfpathlineto{\pgfqpoint{2.321399in}{2.422445in}}%
\pgfpathlineto{\pgfqpoint{2.322034in}{1.503503in}}%
\pgfpathlineto{\pgfqpoint{2.322352in}{2.190232in}}%
\pgfpathlineto{\pgfqpoint{2.322669in}{2.427143in}}%
\pgfpathlineto{\pgfqpoint{2.322987in}{1.575876in}}%
\pgfpathlineto{\pgfqpoint{2.323621in}{2.537668in}}%
\pgfpathlineto{\pgfqpoint{2.323939in}{1.773675in}}%
\pgfpathlineto{\pgfqpoint{2.324256in}{1.678820in}}%
\pgfpathlineto{\pgfqpoint{2.324574in}{2.510068in}}%
\pgfpathlineto{\pgfqpoint{2.325209in}{1.530517in}}%
\pgfpathlineto{\pgfqpoint{2.325526in}{2.348613in}}%
\pgfpathlineto{\pgfqpoint{2.325844in}{2.295722in}}%
\pgfpathlineto{\pgfqpoint{2.326161in}{1.513032in}}%
\pgfpathlineto{\pgfqpoint{2.326796in}{2.487209in}}%
\pgfpathlineto{\pgfqpoint{2.327113in}{1.635707in}}%
\pgfpathlineto{\pgfqpoint{2.327748in}{2.556309in}}%
\pgfpathlineto{\pgfqpoint{2.328066in}{1.866327in}}%
\pgfpathlineto{\pgfqpoint{2.328383in}{1.619419in}}%
\pgfpathlineto{\pgfqpoint{2.328701in}{2.470475in}}%
\pgfpathlineto{\pgfqpoint{2.329336in}{1.506280in}}%
\pgfpathlineto{\pgfqpoint{2.329653in}{2.265040in}}%
\pgfpathlineto{\pgfqpoint{2.329971in}{2.366753in}}%
\pgfpathlineto{\pgfqpoint{2.330288in}{1.536084in}}%
\pgfpathlineto{\pgfqpoint{2.330923in}{2.515477in}}%
\pgfpathlineto{\pgfqpoint{2.331240in}{1.698878in}}%
\pgfpathlineto{\pgfqpoint{2.331558in}{1.744166in}}%
\pgfpathlineto{\pgfqpoint{2.331875in}{2.530289in}}%
\pgfpathlineto{\pgfqpoint{2.332510in}{1.559364in}}%
\pgfpathlineto{\pgfqpoint{2.332828in}{2.406770in}}%
\pgfpathlineto{\pgfqpoint{2.333463in}{1.499771in}}%
\pgfpathlineto{\pgfqpoint{2.333780in}{2.179029in}}%
\pgfpathlineto{\pgfqpoint{2.334098in}{2.436396in}}%
\pgfpathlineto{\pgfqpoint{2.334415in}{1.582114in}}%
\pgfpathlineto{\pgfqpoint{2.335050in}{2.549839in}}%
\pgfpathlineto{\pgfqpoint{2.335367in}{1.784411in}}%
\pgfpathlineto{\pgfqpoint{2.335685in}{1.675950in}}%
\pgfpathlineto{\pgfqpoint{2.336002in}{2.506725in}}%
\pgfpathlineto{\pgfqpoint{2.336637in}{1.521751in}}%
\pgfpathlineto{\pgfqpoint{2.336955in}{2.332933in}}%
\pgfpathlineto{\pgfqpoint{2.337272in}{2.296648in}}%
\pgfpathlineto{\pgfqpoint{2.337590in}{1.507480in}}%
\pgfpathlineto{\pgfqpoint{2.338225in}{2.479232in}}%
\pgfpathlineto{\pgfqpoint{2.338542in}{1.635048in}}%
\pgfpathlineto{\pgfqpoint{2.339177in}{2.538771in}}%
\pgfpathlineto{\pgfqpoint{2.339494in}{1.863267in}}%
\pgfpathlineto{\pgfqpoint{2.339812in}{1.602587in}}%
\pgfpathlineto{\pgfqpoint{2.340129in}{2.454669in}}%
\pgfpathlineto{\pgfqpoint{2.340764in}{1.499774in}}%
\pgfpathlineto{\pgfqpoint{2.341082in}{2.253158in}}%
\pgfpathlineto{\pgfqpoint{2.341399in}{2.374638in}}%
\pgfpathlineto{\pgfqpoint{2.341717in}{1.538194in}}%
\pgfpathlineto{\pgfqpoint{2.342351in}{2.527738in}}%
\pgfpathlineto{\pgfqpoint{2.342669in}{1.711373in}}%
\pgfpathlineto{\pgfqpoint{2.342986in}{1.738559in}}%
\pgfpathlineto{\pgfqpoint{2.343304in}{2.531801in}}%
\pgfpathlineto{\pgfqpoint{2.343939in}{1.552283in}}%
\pgfpathlineto{\pgfqpoint{2.344256in}{2.392845in}}%
\pgfpathlineto{\pgfqpoint{2.344891in}{1.492589in}}%
\pgfpathlineto{\pgfqpoint{2.345209in}{2.159314in}}%
\pgfpathlineto{\pgfqpoint{2.345526in}{2.431521in}}%
\pgfpathlineto{\pgfqpoint{2.345843in}{1.578435in}}%
\pgfpathlineto{\pgfqpoint{2.346478in}{2.532123in}}%
\pgfpathlineto{\pgfqpoint{2.346796in}{1.783471in}}%
\pgfpathlineto{\pgfqpoint{2.347113in}{1.654020in}}%
\pgfpathlineto{\pgfqpoint{2.347431in}{2.492197in}}%
\pgfpathlineto{\pgfqpoint{2.348066in}{1.514303in}}%
\pgfpathlineto{\pgfqpoint{2.348383in}{2.322160in}}%
\pgfpathlineto{\pgfqpoint{2.348701in}{2.304468in}}%
\pgfpathlineto{\pgfqpoint{2.349018in}{1.508343in}}%
\pgfpathlineto{\pgfqpoint{2.349653in}{2.493497in}}%
\pgfpathlineto{\pgfqpoint{2.349970in}{1.642893in}}%
\pgfpathlineto{\pgfqpoint{2.350605in}{2.542154in}}%
\pgfpathlineto{\pgfqpoint{2.350923in}{1.873477in}}%
\pgfpathlineto{\pgfqpoint{2.351240in}{1.593164in}}%
\pgfpathlineto{\pgfqpoint{2.351558in}{2.443849in}}%
\pgfpathlineto{\pgfqpoint{2.352193in}{1.491705in}}%
\pgfpathlineto{\pgfqpoint{2.352510in}{2.235722in}}%
\pgfpathlineto{\pgfqpoint{2.352828in}{2.372545in}}%
\pgfpathlineto{\pgfqpoint{2.353145in}{1.534996in}}%
\pgfpathlineto{\pgfqpoint{2.353780in}{2.513080in}}%
\pgfpathlineto{\pgfqpoint{2.354097in}{1.706233in}}%
\pgfpathlineto{\pgfqpoint{2.354415in}{1.718126in}}%
\pgfpathlineto{\pgfqpoint{2.354732in}{2.516021in}}%
\pgfpathlineto{\pgfqpoint{2.355367in}{1.540380in}}%
\pgfpathlineto{\pgfqpoint{2.355685in}{2.383138in}}%
\pgfpathlineto{\pgfqpoint{2.356002in}{2.226064in}}%
\pgfpathlineto{\pgfqpoint{2.356320in}{1.491432in}}%
\pgfpathlineto{\pgfqpoint{2.356955in}{2.445179in}}%
\pgfpathlineto{\pgfqpoint{2.357272in}{1.587076in}}%
\pgfpathlineto{\pgfqpoint{2.357907in}{2.539846in}}%
\pgfpathlineto{\pgfqpoint{2.358224in}{1.790209in}}%
\pgfpathlineto{\pgfqpoint{2.358542in}{1.648135in}}%
\pgfpathlineto{\pgfqpoint{2.358859in}{2.483150in}}%
\pgfpathlineto{\pgfqpoint{2.359494in}{1.503721in}}%
\pgfpathlineto{\pgfqpoint{2.359812in}{2.305130in}}%
\pgfpathlineto{\pgfqpoint{2.360129in}{2.303432in}}%
\pgfpathlineto{\pgfqpoint{2.360447in}{1.502788in}}%
\pgfpathlineto{\pgfqpoint{2.361081in}{2.479634in}}%
\pgfpathlineto{\pgfqpoint{2.361399in}{1.639765in}}%
\pgfpathlineto{\pgfqpoint{2.361716in}{1.787020in}}%
\pgfpathlineto{\pgfqpoint{2.362034in}{2.527979in}}%
\pgfpathlineto{\pgfqpoint{2.362669in}{1.581304in}}%
\pgfpathlineto{\pgfqpoint{2.362986in}{2.434056in}}%
\pgfpathlineto{\pgfqpoint{2.363621in}{1.488265in}}%
\pgfpathlineto{\pgfqpoint{2.363939in}{2.227754in}}%
\pgfpathlineto{\pgfqpoint{2.364256in}{2.385561in}}%
\pgfpathlineto{\pgfqpoint{2.364573in}{1.540136in}}%
\pgfpathlineto{\pgfqpoint{2.365208in}{2.521878in}}%
\pgfpathlineto{\pgfqpoint{2.365526in}{1.715623in}}%
\pgfpathlineto{\pgfqpoint{2.365843in}{1.709999in}}%
\pgfpathlineto{\pgfqpoint{2.366161in}{2.511447in}}%
\pgfpathlineto{\pgfqpoint{2.366796in}{1.531218in}}%
\pgfpathlineto{\pgfqpoint{2.367113in}{2.367296in}}%
\pgfpathlineto{\pgfqpoint{2.367431in}{2.227352in}}%
\pgfpathlineto{\pgfqpoint{2.367748in}{1.484419in}}%
\pgfpathlineto{\pgfqpoint{2.368383in}{2.434024in}}%
\pgfpathlineto{\pgfqpoint{2.368700in}{1.580207in}}%
\pgfpathlineto{\pgfqpoint{2.369335in}{2.524572in}}%
\pgfpathlineto{\pgfqpoint{2.369653in}{1.790609in}}%
\pgfpathlineto{\pgfqpoint{2.369970in}{1.631036in}}%
\pgfpathlineto{\pgfqpoint{2.370288in}{2.474656in}}%
\pgfpathlineto{\pgfqpoint{2.370923in}{1.499869in}}%
\pgfpathlineto{\pgfqpoint{2.371240in}{2.298086in}}%
\pgfpathlineto{\pgfqpoint{2.371558in}{2.315903in}}%
\pgfpathlineto{\pgfqpoint{2.371875in}{1.506639in}}%
\pgfpathlineto{\pgfqpoint{2.372510in}{2.491491in}}%
\pgfpathlineto{\pgfqpoint{2.372827in}{1.645818in}}%
\pgfpathlineto{\pgfqpoint{2.373145in}{1.783151in}}%
\pgfpathlineto{\pgfqpoint{2.373462in}{2.524914in}}%
\pgfpathlineto{\pgfqpoint{2.374097in}{1.569398in}}%
\pgfpathlineto{\pgfqpoint{2.374415in}{2.420528in}}%
\pgfpathlineto{\pgfqpoint{2.375050in}{1.480111in}}%
\pgfpathlineto{\pgfqpoint{2.375367in}{2.210328in}}%
\pgfpathlineto{\pgfqpoint{2.375685in}{2.376796in}}%
\pgfpathlineto{\pgfqpoint{2.376002in}{1.533947in}}%
\pgfpathlineto{\pgfqpoint{2.376637in}{2.508533in}}%
\pgfpathlineto{\pgfqpoint{2.376954in}{1.711246in}}%
\pgfpathlineto{\pgfqpoint{2.377272in}{1.694019in}}%
\pgfpathlineto{\pgfqpoint{2.377589in}{2.501520in}}%
\pgfpathlineto{\pgfqpoint{2.378224in}{1.523439in}}%
\pgfpathlineto{\pgfqpoint{2.378542in}{2.361312in}}%
\pgfpathlineto{\pgfqpoint{2.378859in}{2.237278in}}%
\pgfpathlineto{\pgfqpoint{2.379177in}{1.485690in}}%
\pgfpathlineto{\pgfqpoint{2.379811in}{2.446859in}}%
\pgfpathlineto{\pgfqpoint{2.380129in}{1.588047in}}%
\pgfpathlineto{\pgfqpoint{2.380764in}{2.525692in}}%
\pgfpathlineto{\pgfqpoint{2.381081in}{1.793015in}}%
\pgfpathlineto{\pgfqpoint{2.381399in}{1.622441in}}%
\pgfpathlineto{\pgfqpoint{2.381716in}{2.462178in}}%
\pgfpathlineto{\pgfqpoint{2.382351in}{1.488958in}}%
\pgfpathlineto{\pgfqpoint{2.382669in}{2.281622in}}%
\pgfpathlineto{\pgfqpoint{2.382986in}{2.308670in}}%
\pgfpathlineto{\pgfqpoint{2.383303in}{1.498693in}}%
\pgfpathlineto{\pgfqpoint{2.383938in}{2.477567in}}%
\pgfpathlineto{\pgfqpoint{2.384256in}{1.642426in}}%
\pgfpathlineto{\pgfqpoint{2.384573in}{1.762373in}}%
\pgfpathlineto{\pgfqpoint{2.384891in}{2.516423in}}%
\pgfpathlineto{\pgfqpoint{2.385526in}{1.562243in}}%
\pgfpathlineto{\pgfqpoint{2.385843in}{2.414731in}}%
\pgfpathlineto{\pgfqpoint{2.386478in}{1.478827in}}%
\pgfpathlineto{\pgfqpoint{2.386796in}{2.209025in}}%
\pgfpathlineto{\pgfqpoint{2.387113in}{2.390146in}}%
\pgfpathlineto{\pgfqpoint{2.387430in}{1.538973in}}%
\pgfpathlineto{\pgfqpoint{2.388065in}{2.511320in}}%
\pgfpathlineto{\pgfqpoint{2.388383in}{1.716764in}}%
\pgfpathlineto{\pgfqpoint{2.388700in}{1.682943in}}%
\pgfpathlineto{\pgfqpoint{2.389018in}{2.492710in}}%
\pgfpathlineto{\pgfqpoint{2.389653in}{1.513526in}}%
\pgfpathlineto{\pgfqpoint{2.389970in}{2.345759in}}%
\pgfpathlineto{\pgfqpoint{2.390288in}{2.232942in}}%
\pgfpathlineto{\pgfqpoint{2.390605in}{1.477383in}}%
\pgfpathlineto{\pgfqpoint{2.391240in}{2.434138in}}%
\pgfpathlineto{\pgfqpoint{2.391557in}{1.580482in}}%
\pgfpathlineto{\pgfqpoint{2.392192in}{2.515755in}}%
\pgfpathlineto{\pgfqpoint{2.392510in}{1.795324in}}%
\pgfpathlineto{\pgfqpoint{2.392827in}{1.610371in}}%
\pgfpathlineto{\pgfqpoint{2.393145in}{2.458069in}}%
\pgfpathlineto{\pgfqpoint{2.393780in}{1.487344in}}%
\pgfpathlineto{\pgfqpoint{2.394097in}{2.281608in}}%
\pgfpathlineto{\pgfqpoint{2.394415in}{2.322825in}}%
\pgfpathlineto{\pgfqpoint{2.394732in}{1.503429in}}%
\pgfpathlineto{\pgfqpoint{2.395367in}{2.483756in}}%
\pgfpathlineto{\pgfqpoint{2.395684in}{1.644910in}}%
\pgfpathlineto{\pgfqpoint{2.396002in}{1.755297in}}%
\pgfpathlineto{\pgfqpoint{2.396319in}{2.508317in}}%
\pgfpathlineto{\pgfqpoint{2.396954in}{1.549178in}}%
\pgfpathlineto{\pgfqpoint{2.397272in}{2.401008in}}%
\pgfpathlineto{\pgfqpoint{2.397907in}{1.470291in}}%
\pgfpathlineto{\pgfqpoint{2.398224in}{2.188013in}}%
\pgfpathlineto{\pgfqpoint{2.398541in}{2.378397in}}%
\pgfpathlineto{\pgfqpoint{2.398859in}{1.531690in}}%
\pgfpathlineto{\pgfqpoint{2.399494in}{2.502192in}}%
\pgfpathlineto{\pgfqpoint{2.399811in}{1.713982in}}%
\pgfpathlineto{\pgfqpoint{2.400129in}{1.672271in}}%
\pgfpathlineto{\pgfqpoint{2.400446in}{2.487541in}}%
\pgfpathlineto{\pgfqpoint{2.401081in}{1.508201in}}%
\pgfpathlineto{\pgfqpoint{2.401399in}{2.346081in}}%
\pgfpathlineto{\pgfqpoint{2.401716in}{2.245124in}}%
\pgfpathlineto{\pgfqpoint{2.402033in}{1.480782in}}%
\pgfpathlineto{\pgfqpoint{2.402668in}{2.441975in}}%
\pgfpathlineto{\pgfqpoint{2.402986in}{1.585098in}}%
\pgfpathlineto{\pgfqpoint{2.403621in}{2.511492in}}%
\pgfpathlineto{\pgfqpoint{2.403938in}{1.794286in}}%
\pgfpathlineto{\pgfqpoint{2.404256in}{1.600161in}}%
\pgfpathlineto{\pgfqpoint{2.404573in}{2.444508in}}%
\pgfpathlineto{\pgfqpoint{2.405208in}{1.476545in}}%
\pgfpathlineto{\pgfqpoint{2.405526in}{2.261257in}}%
\pgfpathlineto{\pgfqpoint{2.405843in}{2.311256in}}%
\pgfpathlineto{\pgfqpoint{2.406160in}{1.493989in}}%
\pgfpathlineto{\pgfqpoint{2.406795in}{2.473377in}}%
\pgfpathlineto{\pgfqpoint{2.407113in}{1.643004in}}%
\pgfpathlineto{\pgfqpoint{2.407430in}{1.740069in}}%
\pgfpathlineto{\pgfqpoint{2.407748in}{2.504893in}}%
\pgfpathlineto{\pgfqpoint{2.408383in}{1.545279in}}%
\pgfpathlineto{\pgfqpoint{2.408700in}{2.400434in}}%
\pgfpathlineto{\pgfqpoint{2.409335in}{1.472334in}}%
\pgfpathlineto{\pgfqpoint{2.409652in}{2.189678in}}%
\pgfpathlineto{\pgfqpoint{2.409970in}{2.387360in}}%
\pgfpathlineto{\pgfqpoint{2.410287in}{1.534052in}}%
\pgfpathlineto{\pgfqpoint{2.410922in}{2.499063in}}%
\pgfpathlineto{\pgfqpoint{2.411240in}{1.715737in}}%
\pgfpathlineto{\pgfqpoint{2.411557in}{1.658922in}}%
\pgfpathlineto{\pgfqpoint{2.411875in}{2.476778in}}%
\pgfpathlineto{\pgfqpoint{2.412510in}{1.498752in}}%
\pgfpathlineto{\pgfqpoint{2.412827in}{2.327385in}}%
\pgfpathlineto{\pgfqpoint{2.413145in}{2.236059in}}%
\pgfpathlineto{\pgfqpoint{2.413462in}{1.470418in}}%
\pgfpathlineto{\pgfqpoint{2.414097in}{2.431648in}}%
\pgfpathlineto{\pgfqpoint{2.414414in}{1.578871in}}%
\pgfpathlineto{\pgfqpoint{2.415049in}{2.506438in}}%
\pgfpathlineto{\pgfqpoint{2.415367in}{1.797786in}}%
\pgfpathlineto{\pgfqpoint{2.415684in}{1.592145in}}%
\pgfpathlineto{\pgfqpoint{2.416002in}{2.444405in}}%
\pgfpathlineto{\pgfqpoint{2.416637in}{1.478670in}}%
\pgfpathlineto{\pgfqpoint{2.416954in}{2.264798in}}%
\pgfpathlineto{\pgfqpoint{2.417271in}{2.322237in}}%
\pgfpathlineto{\pgfqpoint{2.417589in}{1.497132in}}%
\pgfpathlineto{\pgfqpoint{2.418224in}{2.473377in}}%
\pgfpathlineto{\pgfqpoint{2.418541in}{1.641779in}}%
\pgfpathlineto{\pgfqpoint{2.418859in}{1.730537in}}%
\pgfpathlineto{\pgfqpoint{2.419176in}{2.494206in}}%
\pgfpathlineto{\pgfqpoint{2.419811in}{1.532384in}}%
\pgfpathlineto{\pgfqpoint{2.420129in}{2.384553in}}%
\pgfpathlineto{\pgfqpoint{2.420763in}{1.461181in}}%
\pgfpathlineto{\pgfqpoint{2.421081in}{2.167879in}}%
\pgfpathlineto{\pgfqpoint{2.421398in}{2.377194in}}%
\pgfpathlineto{\pgfqpoint{2.421716in}{1.527915in}}%
\pgfpathlineto{\pgfqpoint{2.422351in}{2.494801in}}%
\pgfpathlineto{\pgfqpoint{2.422668in}{1.714634in}}%
\pgfpathlineto{\pgfqpoint{2.422986in}{1.653181in}}%
\pgfpathlineto{\pgfqpoint{2.423303in}{2.474675in}}%
\pgfpathlineto{\pgfqpoint{2.423938in}{1.497589in}}%
\pgfpathlineto{\pgfqpoint{2.424256in}{2.332471in}}%
\pgfpathlineto{\pgfqpoint{2.424573in}{2.246271in}}%
\pgfpathlineto{\pgfqpoint{2.424890in}{1.473240in}}%
\pgfpathlineto{\pgfqpoint{2.425525in}{2.433035in}}%
\pgfpathlineto{\pgfqpoint{2.425843in}{1.579721in}}%
\pgfpathlineto{\pgfqpoint{2.426478in}{2.498847in}}%
\pgfpathlineto{\pgfqpoint{2.426795in}{1.794218in}}%
\pgfpathlineto{\pgfqpoint{2.427113in}{1.581622in}}%
\pgfpathlineto{\pgfqpoint{2.427430in}{2.429895in}}%
\pgfpathlineto{\pgfqpoint{2.428065in}{1.465598in}}%
\pgfpathlineto{\pgfqpoint{2.428382in}{2.243128in}}%
\pgfpathlineto{\pgfqpoint{2.428700in}{2.310934in}}%
\pgfpathlineto{\pgfqpoint{2.429017in}{1.488243in}}%
\pgfpathlineto{\pgfqpoint{2.429652in}{2.467607in}}%
\pgfpathlineto{\pgfqpoint{2.429970in}{1.641698in}}%
\pgfpathlineto{\pgfqpoint{2.430287in}{1.720581in}}%
\pgfpathlineto{\pgfqpoint{2.430605in}{2.493561in}}%
\pgfpathlineto{\pgfqpoint{2.431240in}{1.532190in}}%
\pgfpathlineto{\pgfqpoint{2.431557in}{2.389371in}}%
\pgfpathlineto{\pgfqpoint{2.432192in}{1.463550in}}%
\pgfpathlineto{\pgfqpoint{2.432509in}{2.167685in}}%
\pgfpathlineto{\pgfqpoint{2.432827in}{2.380091in}}%
\pgfpathlineto{\pgfqpoint{2.433144in}{1.526958in}}%
\pgfpathlineto{\pgfqpoint{2.433779in}{2.487672in}}%
\pgfpathlineto{\pgfqpoint{2.434097in}{1.713646in}}%
\pgfpathlineto{\pgfqpoint{2.434414in}{1.639243in}}%
\pgfpathlineto{\pgfqpoint{2.434732in}{2.463853in}}%
\pgfpathlineto{\pgfqpoint{2.435367in}{1.486100in}}%
\pgfpathlineto{\pgfqpoint{2.435684in}{2.311094in}}%
\pgfpathlineto{\pgfqpoint{2.436001in}{2.236157in}}%
\pgfpathlineto{\pgfqpoint{2.436319in}{1.462838in}}%
\pgfpathlineto{\pgfqpoint{2.436954in}{2.427143in}}%
\pgfpathlineto{\pgfqpoint{2.437271in}{1.575624in}}%
\pgfpathlineto{\pgfqpoint{2.437906in}{2.496981in}}%
\pgfpathlineto{\pgfqpoint{2.438224in}{1.798055in}}%
\pgfpathlineto{\pgfqpoint{2.438541in}{1.577082in}}%
\pgfpathlineto{\pgfqpoint{2.438859in}{2.435691in}}%
\pgfpathlineto{\pgfqpoint{2.439494in}{1.469220in}}%
\pgfpathlineto{\pgfqpoint{2.439811in}{2.245532in}}%
\pgfpathlineto{\pgfqpoint{2.440128in}{2.316370in}}%
\pgfpathlineto{\pgfqpoint{2.440446in}{1.488296in}}%
\pgfpathlineto{\pgfqpoint{2.441081in}{2.462828in}}%
\pgfpathlineto{\pgfqpoint{2.441398in}{1.637624in}}%
\pgfpathlineto{\pgfqpoint{2.441716in}{1.710020in}}%
\pgfpathlineto{\pgfqpoint{2.442033in}{2.482593in}}%
\pgfpathlineto{\pgfqpoint{2.442668in}{1.518432in}}%
\pgfpathlineto{\pgfqpoint{2.442986in}{2.370331in}}%
\pgfpathlineto{\pgfqpoint{2.443620in}{1.452121in}}%
\pgfpathlineto{\pgfqpoint{2.443938in}{2.149640in}}%
\pgfpathlineto{\pgfqpoint{2.444255in}{2.373596in}}%
\pgfpathlineto{\pgfqpoint{2.444573in}{1.522747in}}%
\pgfpathlineto{\pgfqpoint{2.445208in}{2.486887in}}%
\pgfpathlineto{\pgfqpoint{2.445525in}{1.713114in}}%
\pgfpathlineto{\pgfqpoint{2.445843in}{1.637000in}}%
\pgfpathlineto{\pgfqpoint{2.446160in}{2.467203in}}%
\pgfpathlineto{\pgfqpoint{2.446795in}{1.487689in}}%
\pgfpathlineto{\pgfqpoint{2.447112in}{2.315984in}}%
\pgfpathlineto{\pgfqpoint{2.447430in}{2.241450in}}%
\pgfpathlineto{\pgfqpoint{2.447747in}{1.463020in}}%
\pgfpathlineto{\pgfqpoint{2.448382in}{2.423446in}}%
\pgfpathlineto{\pgfqpoint{2.448700in}{1.573504in}}%
\pgfpathlineto{\pgfqpoint{2.449335in}{2.488323in}}%
\pgfpathlineto{\pgfqpoint{2.449652in}{1.792820in}}%
\pgfpathlineto{\pgfqpoint{2.449970in}{1.566599in}}%
\pgfpathlineto{\pgfqpoint{2.450287in}{2.417313in}}%
\pgfpathlineto{\pgfqpoint{2.450922in}{1.455188in}}%
\pgfpathlineto{\pgfqpoint{2.451239in}{2.226600in}}%
\pgfpathlineto{\pgfqpoint{2.451557in}{2.307859in}}%
\pgfpathlineto{\pgfqpoint{2.451874in}{1.481407in}}%
\pgfpathlineto{\pgfqpoint{2.452509in}{2.460873in}}%
\pgfpathlineto{\pgfqpoint{2.452827in}{1.638564in}}%
\pgfpathlineto{\pgfqpoint{2.453144in}{1.703682in}}%
\pgfpathlineto{\pgfqpoint{2.453462in}{2.486626in}}%
\pgfpathlineto{\pgfqpoint{2.454097in}{1.522026in}}%
\pgfpathlineto{\pgfqpoint{2.454414in}{2.375839in}}%
\pgfpathlineto{\pgfqpoint{2.455049in}{1.452354in}}%
\pgfpathlineto{\pgfqpoint{2.455366in}{2.145858in}}%
\pgfpathlineto{\pgfqpoint{2.455684in}{2.371225in}}%
\pgfpathlineto{\pgfqpoint{2.456001in}{1.518800in}}%
\pgfpathlineto{\pgfqpoint{2.456636in}{2.477869in}}%
\pgfpathlineto{\pgfqpoint{2.456954in}{1.710546in}}%
\pgfpathlineto{\pgfqpoint{2.457271in}{1.623482in}}%
\pgfpathlineto{\pgfqpoint{2.457589in}{2.452761in}}%
\pgfpathlineto{\pgfqpoint{2.458224in}{1.474794in}}%
\pgfpathlineto{\pgfqpoint{2.458541in}{2.296409in}}%
\pgfpathlineto{\pgfqpoint{2.458858in}{2.233303in}}%
\pgfpathlineto{\pgfqpoint{2.459176in}{1.454529in}}%
\pgfpathlineto{\pgfqpoint{2.459811in}{2.421194in}}%
\pgfpathlineto{\pgfqpoint{2.460128in}{1.570873in}}%
\pgfpathlineto{\pgfqpoint{2.460763in}{2.490477in}}%
\pgfpathlineto{\pgfqpoint{2.461081in}{1.799617in}}%
\pgfpathlineto{\pgfqpoint{2.461398in}{1.567185in}}%
\pgfpathlineto{\pgfqpoint{2.461716in}{2.424746in}}%
\pgfpathlineto{\pgfqpoint{2.462350in}{1.457544in}}%
\pgfpathlineto{\pgfqpoint{2.462668in}{2.225406in}}%
\pgfpathlineto{\pgfqpoint{2.462985in}{2.307747in}}%
\pgfpathlineto{\pgfqpoint{2.463303in}{1.478243in}}%
\pgfpathlineto{\pgfqpoint{2.463938in}{2.453546in}}%
\pgfpathlineto{\pgfqpoint{2.464255in}{1.632730in}}%
\pgfpathlineto{\pgfqpoint{2.464573in}{1.693728in}}%
\pgfpathlineto{\pgfqpoint{2.464890in}{2.472653in}}%
\pgfpathlineto{\pgfqpoint{2.465525in}{1.506373in}}%
\pgfpathlineto{\pgfqpoint{2.465842in}{2.357339in}}%
\pgfpathlineto{\pgfqpoint{2.466477in}{1.442705in}}%
\pgfpathlineto{\pgfqpoint{2.466795in}{2.133447in}}%
\pgfpathlineto{\pgfqpoint{2.467112in}{2.368104in}}%
\pgfpathlineto{\pgfqpoint{2.467430in}{1.516362in}}%
\pgfpathlineto{\pgfqpoint{2.468065in}{2.479848in}}%
\pgfpathlineto{\pgfqpoint{2.468382in}{1.711740in}}%
\pgfpathlineto{\pgfqpoint{2.468700in}{1.626310in}}%
\pgfpathlineto{\pgfqpoint{2.469017in}{2.459065in}}%
\pgfpathlineto{\pgfqpoint{2.469652in}{1.475996in}}%
\pgfpathlineto{\pgfqpoint{2.469969in}{2.298063in}}%
\pgfpathlineto{\pgfqpoint{2.470287in}{2.233226in}}%
\pgfpathlineto{\pgfqpoint{2.470604in}{1.451746in}}%
\pgfpathlineto{\pgfqpoint{2.471239in}{2.414460in}}%
\pgfpathlineto{\pgfqpoint{2.471557in}{1.566618in}}%
\pgfpathlineto{\pgfqpoint{2.472192in}{2.479317in}}%
\pgfpathlineto{\pgfqpoint{2.472509in}{1.789509in}}%
\pgfpathlineto{\pgfqpoint{2.472827in}{1.554065in}}%
\pgfpathlineto{\pgfqpoint{2.473144in}{2.405811in}}%
\pgfpathlineto{\pgfqpoint{2.473779in}{1.444959in}}%
\pgfpathlineto{\pgfqpoint{2.474096in}{2.211858in}}%
\pgfpathlineto{\pgfqpoint{2.474414in}{2.302455in}}%
\pgfpathlineto{\pgfqpoint{2.474731in}{1.473642in}}%
\pgfpathlineto{\pgfqpoint{2.475366in}{2.453327in}}%
\pgfpathlineto{\pgfqpoint{2.475684in}{1.634132in}}%
\pgfpathlineto{\pgfqpoint{2.476001in}{1.692334in}}%
\pgfpathlineto{\pgfqpoint{2.476319in}{2.481868in}}%
\pgfpathlineto{\pgfqpoint{2.476954in}{1.510812in}}%
\pgfpathlineto{\pgfqpoint{2.477271in}{2.359689in}}%
\pgfpathlineto{\pgfqpoint{2.477906in}{1.439831in}}%
\pgfpathlineto{\pgfqpoint{2.478223in}{2.126391in}}%
\pgfpathlineto{\pgfqpoint{2.478541in}{2.361939in}}%
\pgfpathlineto{\pgfqpoint{2.478858in}{1.510177in}}%
\pgfpathlineto{\pgfqpoint{2.479493in}{2.469405in}}%
\pgfpathlineto{\pgfqpoint{2.479811in}{1.705859in}}%
\pgfpathlineto{\pgfqpoint{2.480128in}{1.610769in}}%
\pgfpathlineto{\pgfqpoint{2.480446in}{2.442594in}}%
\pgfpathlineto{\pgfqpoint{2.481080in}{1.464084in}}%
\pgfpathlineto{\pgfqpoint{2.481398in}{2.283061in}}%
\pgfpathlineto{\pgfqpoint{2.481715in}{2.227751in}}%
\pgfpathlineto{\pgfqpoint{2.482033in}{1.445576in}}%
\pgfpathlineto{\pgfqpoint{2.482668in}{2.413907in}}%
\pgfpathlineto{\pgfqpoint{2.482985in}{1.564377in}}%
\pgfpathlineto{\pgfqpoint{2.483620in}{2.486844in}}%
\pgfpathlineto{\pgfqpoint{2.483938in}{1.796973in}}%
\pgfpathlineto{\pgfqpoint{2.484255in}{1.556421in}}%
\pgfpathlineto{\pgfqpoint{2.484572in}{2.411296in}}%
\pgfpathlineto{\pgfqpoint{2.485207in}{1.444718in}}%
\pgfpathlineto{\pgfqpoint{2.485525in}{2.207301in}}%
\pgfpathlineto{\pgfqpoint{2.485842in}{2.298298in}}%
\pgfpathlineto{\pgfqpoint{2.486160in}{1.468147in}}%
\pgfpathlineto{\pgfqpoint{2.486795in}{2.445191in}}%
\pgfpathlineto{\pgfqpoint{2.487112in}{1.626474in}}%
\pgfpathlineto{\pgfqpoint{2.487430in}{1.681044in}}%
\pgfpathlineto{\pgfqpoint{2.487747in}{2.463559in}}%
\pgfpathlineto{\pgfqpoint{2.488382in}{1.495454in}}%
\pgfpathlineto{\pgfqpoint{2.488699in}{2.345393in}}%
\pgfpathlineto{\pgfqpoint{2.489334in}{1.432994in}}%
\pgfpathlineto{\pgfqpoint{2.489652in}{2.119765in}}%
\pgfpathlineto{\pgfqpoint{2.489969in}{2.360954in}}%
\pgfpathlineto{\pgfqpoint{2.490287in}{1.508720in}}%
\pgfpathlineto{\pgfqpoint{2.490922in}{2.476858in}}%
\pgfpathlineto{\pgfqpoint{2.491239in}{1.708553in}}%
\pgfpathlineto{\pgfqpoint{2.491557in}{1.616651in}}%
\pgfpathlineto{\pgfqpoint{2.491874in}{2.448795in}}%
\pgfpathlineto{\pgfqpoint{2.492509in}{1.462665in}}%
\pgfpathlineto{\pgfqpoint{2.492826in}{2.280541in}}%
\pgfpathlineto{\pgfqpoint{2.493144in}{2.222914in}}%
\pgfpathlineto{\pgfqpoint{2.493461in}{1.440264in}}%
\pgfpathlineto{\pgfqpoint{2.494096in}{2.405826in}}%
\pgfpathlineto{\pgfqpoint{2.494414in}{1.558941in}}%
\pgfpathlineto{\pgfqpoint{2.495049in}{2.470952in}}%
\pgfpathlineto{\pgfqpoint{2.495366in}{1.783610in}}%
\pgfpathlineto{\pgfqpoint{2.495684in}{1.543260in}}%
\pgfpathlineto{\pgfqpoint{2.496001in}{2.395083in}}%
\pgfpathlineto{\pgfqpoint{2.496636in}{1.434745in}}%
\pgfpathlineto{\pgfqpoint{2.496953in}{2.199271in}}%
\pgfpathlineto{\pgfqpoint{2.497271in}{2.295029in}}%
\pgfpathlineto{\pgfqpoint{2.497588in}{1.465053in}}%
\pgfpathlineto{\pgfqpoint{2.498223in}{2.449367in}}%
\pgfpathlineto{\pgfqpoint{2.498541in}{1.629686in}}%
\pgfpathlineto{\pgfqpoint{2.498858in}{1.683782in}}%
\pgfpathlineto{\pgfqpoint{2.499176in}{2.473442in}}%
\pgfpathlineto{\pgfqpoint{2.499810in}{1.498045in}}%
\pgfpathlineto{\pgfqpoint{2.500128in}{2.343996in}}%
\pgfpathlineto{\pgfqpoint{2.500763in}{1.427528in}}%
\pgfpathlineto{\pgfqpoint{2.501080in}{2.111238in}}%
\pgfpathlineto{\pgfqpoint{2.501398in}{2.352733in}}%
\pgfpathlineto{\pgfqpoint{2.501715in}{1.501275in}}%
\pgfpathlineto{\pgfqpoint{2.502350in}{2.461175in}}%
\pgfpathlineto{\pgfqpoint{2.502668in}{1.698795in}}%
\pgfpathlineto{\pgfqpoint{2.502985in}{1.600359in}}%
\pgfpathlineto{\pgfqpoint{2.503302in}{2.432923in}}%
\pgfpathlineto{\pgfqpoint{2.503937in}{1.453815in}}%
\pgfpathlineto{\pgfqpoint{2.504255in}{2.271471in}}%
\pgfpathlineto{\pgfqpoint{2.504572in}{2.219830in}}%
\pgfpathlineto{\pgfqpoint{2.504890in}{1.436164in}}%
\pgfpathlineto{\pgfqpoint{2.505525in}{2.407443in}}%
\pgfpathlineto{\pgfqpoint{2.505842in}{1.557697in}}%
\pgfpathlineto{\pgfqpoint{2.506477in}{2.480114in}}%
\pgfpathlineto{\pgfqpoint{2.506795in}{1.788984in}}%
\pgfpathlineto{\pgfqpoint{2.507112in}{1.544758in}}%
\pgfpathlineto{\pgfqpoint{2.507429in}{2.397369in}}%
\pgfpathlineto{\pgfqpoint{2.508064in}{1.431557in}}%
\pgfpathlineto{\pgfqpoint{2.508382in}{2.192292in}}%
\pgfpathlineto{\pgfqpoint{2.508699in}{2.288105in}}%
\pgfpathlineto{\pgfqpoint{2.509017in}{1.457984in}}%
\pgfpathlineto{\pgfqpoint{2.509652in}{2.436878in}}%
\pgfpathlineto{\pgfqpoint{2.509969in}{1.618497in}}%
\pgfpathlineto{\pgfqpoint{2.510287in}{1.671212in}}%
\pgfpathlineto{\pgfqpoint{2.510604in}{2.454624in}}%
\pgfpathlineto{\pgfqpoint{2.511239in}{1.485404in}}%
\pgfpathlineto{\pgfqpoint{2.511556in}{2.334781in}}%
\pgfpathlineto{\pgfqpoint{2.512191in}{1.423183in}}%
\pgfpathlineto{\pgfqpoint{2.512509in}{2.108935in}}%
\pgfpathlineto{\pgfqpoint{2.512826in}{2.352595in}}%
\pgfpathlineto{\pgfqpoint{2.513144in}{1.499895in}}%
\pgfpathlineto{\pgfqpoint{2.513779in}{2.471570in}}%
\pgfpathlineto{\pgfqpoint{2.514096in}{1.700784in}}%
\pgfpathlineto{\pgfqpoint{2.514414in}{1.606542in}}%
\pgfpathlineto{\pgfqpoint{2.514731in}{2.436042in}}%
\pgfpathlineto{\pgfqpoint{2.515366in}{1.449450in}}%
\pgfpathlineto{\pgfqpoint{2.515683in}{2.266425in}}%
\pgfpathlineto{\pgfqpoint{2.516001in}{2.211766in}}%
\pgfpathlineto{\pgfqpoint{2.516318in}{1.429354in}}%
\pgfpathlineto{\pgfqpoint{2.516953in}{2.396938in}}%
\pgfpathlineto{\pgfqpoint{2.517271in}{1.549975in}}%
\pgfpathlineto{\pgfqpoint{2.517906in}{2.462238in}}%
\pgfpathlineto{\pgfqpoint{2.518223in}{1.774697in}}%
\pgfpathlineto{\pgfqpoint{2.518540in}{1.533779in}}%
\pgfpathlineto{\pgfqpoint{2.518858in}{2.385240in}}%
\pgfpathlineto{\pgfqpoint{2.519493in}{1.424788in}}%
\pgfpathlineto{\pgfqpoint{2.519810in}{2.189277in}}%
\pgfpathlineto{\pgfqpoint{2.520128in}{2.285530in}}%
\pgfpathlineto{\pgfqpoint{2.520445in}{1.455510in}}%
\pgfpathlineto{\pgfqpoint{2.521080in}{2.444609in}}%
\pgfpathlineto{\pgfqpoint{2.521398in}{1.621939in}}%
\pgfpathlineto{\pgfqpoint{2.521715in}{1.675554in}}%
\pgfpathlineto{\pgfqpoint{2.522032in}{2.462077in}}%
\pgfpathlineto{\pgfqpoint{2.522667in}{1.485793in}}%
\pgfpathlineto{\pgfqpoint{2.522985in}{2.330466in}}%
\pgfpathlineto{\pgfqpoint{2.523620in}{1.415720in}}%
\pgfpathlineto{\pgfqpoint{2.523937in}{2.100339in}}%
\pgfpathlineto{\pgfqpoint{2.524255in}{2.342947in}}%
\pgfpathlineto{\pgfqpoint{2.524572in}{1.491617in}}%
\pgfpathlineto{\pgfqpoint{2.525207in}{2.452538in}}%
\pgfpathlineto{\pgfqpoint{2.525525in}{1.689152in}}%
\pgfpathlineto{\pgfqpoint{2.525842in}{1.591727in}}%
\pgfpathlineto{\pgfqpoint{2.526159in}{2.423645in}}%
\pgfpathlineto{\pgfqpoint{2.526794in}{1.444156in}}%
\pgfpathlineto{\pgfqpoint{2.527112in}{2.262067in}}%
\pgfpathlineto{\pgfqpoint{2.527429in}{2.209620in}}%
\pgfpathlineto{\pgfqpoint{2.527747in}{1.426344in}}%
\pgfpathlineto{\pgfqpoint{2.528382in}{2.403028in}}%
\pgfpathlineto{\pgfqpoint{2.528699in}{1.550030in}}%
\pgfpathlineto{\pgfqpoint{2.529334in}{2.469871in}}%
\pgfpathlineto{\pgfqpoint{2.529651in}{1.776591in}}%
\pgfpathlineto{\pgfqpoint{2.529969in}{1.533319in}}%
\pgfpathlineto{\pgfqpoint{2.530286in}{2.383906in}}%
\pgfpathlineto{\pgfqpoint{2.530921in}{1.419569in}}%
\pgfpathlineto{\pgfqpoint{2.531239in}{2.181736in}}%
\pgfpathlineto{\pgfqpoint{2.531556in}{2.277191in}}%
\pgfpathlineto{\pgfqpoint{2.531874in}{1.447677in}}%
\pgfpathlineto{\pgfqpoint{2.532509in}{2.427747in}}%
\pgfpathlineto{\pgfqpoint{2.532826in}{1.608281in}}%
\pgfpathlineto{\pgfqpoint{2.533144in}{1.663436in}}%
\pgfpathlineto{\pgfqpoint{2.533461in}{2.445559in}}%
\pgfpathlineto{\pgfqpoint{2.534096in}{1.476322in}}%
\pgfpathlineto{\pgfqpoint{2.534413in}{2.325912in}}%
\pgfpathlineto{\pgfqpoint{2.535048in}{1.413448in}}%
\pgfpathlineto{\pgfqpoint{2.535366in}{2.100831in}}%
\pgfpathlineto{\pgfqpoint{2.535683in}{2.346195in}}%
\pgfpathlineto{\pgfqpoint{2.536001in}{1.491187in}}%
\pgfpathlineto{\pgfqpoint{2.536636in}{2.462377in}}%
\pgfpathlineto{\pgfqpoint{2.536953in}{1.688857in}}%
\pgfpathlineto{\pgfqpoint{2.537270in}{1.596955in}}%
\pgfpathlineto{\pgfqpoint{2.537588in}{2.423524in}}%
\pgfpathlineto{\pgfqpoint{2.538223in}{1.438032in}}%
\pgfpathlineto{\pgfqpoint{2.538540in}{2.255806in}}%
\pgfpathlineto{\pgfqpoint{2.538858in}{2.199323in}}%
\pgfpathlineto{\pgfqpoint{2.539175in}{1.418710in}}%
\pgfpathlineto{\pgfqpoint{2.539810in}{2.387132in}}%
\pgfpathlineto{\pgfqpoint{2.540128in}{1.539310in}}%
\pgfpathlineto{\pgfqpoint{2.540762in}{2.453117in}}%
\pgfpathlineto{\pgfqpoint{2.541080in}{1.762853in}}%
\pgfpathlineto{\pgfqpoint{2.541397in}{1.525621in}}%
\pgfpathlineto{\pgfqpoint{2.541715in}{2.376588in}}%
\pgfpathlineto{\pgfqpoint{2.542350in}{1.415315in}}%
\pgfpathlineto{\pgfqpoint{2.542667in}{2.181911in}}%
\pgfpathlineto{\pgfqpoint{2.542985in}{2.276100in}}%
\pgfpathlineto{\pgfqpoint{2.543302in}{1.445885in}}%
\pgfpathlineto{\pgfqpoint{2.543937in}{2.436627in}}%
\pgfpathlineto{\pgfqpoint{2.544255in}{1.610403in}}%
\pgfpathlineto{\pgfqpoint{2.544572in}{1.667770in}}%
\pgfpathlineto{\pgfqpoint{2.544889in}{2.449142in}}%
\pgfpathlineto{\pgfqpoint{2.545524in}{1.474676in}}%
\pgfpathlineto{\pgfqpoint{2.545842in}{2.319305in}}%
\pgfpathlineto{\pgfqpoint{2.546477in}{1.404929in}}%
\pgfpathlineto{\pgfqpoint{2.546794in}{2.093645in}}%
\pgfpathlineto{\pgfqpoint{2.547112in}{2.332011in}}%
\pgfpathlineto{\pgfqpoint{2.547429in}{1.480820in}}%
\pgfpathlineto{\pgfqpoint{2.548064in}{2.443040in}}%
\pgfpathlineto{\pgfqpoint{2.548381in}{1.676761in}}%
\pgfpathlineto{\pgfqpoint{2.548699in}{1.584685in}}%
\pgfpathlineto{\pgfqpoint{2.549016in}{2.415008in}}%
\pgfpathlineto{\pgfqpoint{2.549651in}{1.435377in}}%
\pgfpathlineto{\pgfqpoint{2.549969in}{2.255016in}}%
\pgfpathlineto{\pgfqpoint{2.550286in}{2.197202in}}%
\pgfpathlineto{\pgfqpoint{2.550604in}{1.415998in}}%
\pgfpathlineto{\pgfqpoint{2.551239in}{2.394974in}}%
\pgfpathlineto{\pgfqpoint{2.551556in}{1.539103in}}%
\pgfpathlineto{\pgfqpoint{2.552191in}{2.457690in}}%
\pgfpathlineto{\pgfqpoint{2.552508in}{1.761209in}}%
\pgfpathlineto{\pgfqpoint{2.552826in}{1.523551in}}%
\pgfpathlineto{\pgfqpoint{2.553143in}{2.372888in}}%
\pgfpathlineto{\pgfqpoint{2.553778in}{1.409265in}}%
\pgfpathlineto{\pgfqpoint{2.554096in}{2.174674in}}%
\pgfpathlineto{\pgfqpoint{2.554413in}{2.264912in}}%
\pgfpathlineto{\pgfqpoint{2.554731in}{1.436857in}}%
\pgfpathlineto{\pgfqpoint{2.555366in}{2.417502in}}%
\pgfpathlineto{\pgfqpoint{2.555683in}{1.595705in}}%
\pgfpathlineto{\pgfqpoint{2.556000in}{1.657656in}}%
\pgfpathlineto{\pgfqpoint{2.556318in}{2.436629in}}%
\pgfpathlineto{\pgfqpoint{2.556953in}{1.468452in}}%
\pgfpathlineto{\pgfqpoint{2.557270in}{2.319009in}}%
\pgfpathlineto{\pgfqpoint{2.557905in}{1.403734in}}%
\pgfpathlineto{\pgfqpoint{2.558223in}{2.096846in}}%
\pgfpathlineto{\pgfqpoint{2.558540in}{2.338131in}}%
\pgfpathlineto{\pgfqpoint{2.558858in}{1.481567in}}%
\pgfpathlineto{\pgfqpoint{2.559492in}{2.449894in}}%
\pgfpathlineto{\pgfqpoint{2.559810in}{1.673204in}}%
\pgfpathlineto{\pgfqpoint{2.560127in}{1.588201in}}%
\pgfpathlineto{\pgfqpoint{2.560445in}{2.411298in}}%
\pgfpathlineto{\pgfqpoint{2.561080in}{1.427961in}}%
\pgfpathlineto{\pgfqpoint{2.561397in}{2.248366in}}%
\pgfpathlineto{\pgfqpoint{2.561715in}{2.185520in}}%
\pgfpathlineto{\pgfqpoint{2.562032in}{1.408319in}}%
\pgfpathlineto{\pgfqpoint{2.562667in}{2.375941in}}%
\pgfpathlineto{\pgfqpoint{2.562985in}{1.526715in}}%
\pgfpathlineto{\pgfqpoint{2.563619in}{2.443574in}}%
\pgfpathlineto{\pgfqpoint{2.563937in}{1.748186in}}%
\pgfpathlineto{\pgfqpoint{2.564254in}{1.518976in}}%
\pgfpathlineto{\pgfqpoint{2.564572in}{2.369421in}}%
\pgfpathlineto{\pgfqpoint{2.565207in}{1.406427in}}%
\pgfpathlineto{\pgfqpoint{2.565524in}{2.177129in}}%
\pgfpathlineto{\pgfqpoint{2.565842in}{2.266638in}}%
\pgfpathlineto{\pgfqpoint{2.566159in}{1.436724in}}%
\pgfpathlineto{\pgfqpoint{2.566794in}{2.424462in}}%
\pgfpathlineto{\pgfqpoint{2.567111in}{1.595311in}}%
\pgfpathlineto{\pgfqpoint{2.567429in}{1.661057in}}%
\pgfpathlineto{\pgfqpoint{2.567746in}{2.436642in}}%
\pgfpathlineto{\pgfqpoint{2.568381in}{1.465700in}}%
\pgfpathlineto{\pgfqpoint{2.568699in}{2.311446in}}%
\pgfpathlineto{\pgfqpoint{2.569334in}{1.395010in}}%
\pgfpathlineto{\pgfqpoint{2.569651in}{2.090019in}}%
\pgfpathlineto{\pgfqpoint{2.569969in}{2.319415in}}%
\pgfpathlineto{\pgfqpoint{2.570286in}{1.468659in}}%
\pgfpathlineto{\pgfqpoint{2.570921in}{2.432647in}}%
\pgfpathlineto{\pgfqpoint{2.571238in}{1.661699in}}%
\pgfpathlineto{\pgfqpoint{2.571556in}{1.579454in}}%
\pgfpathlineto{\pgfqpoint{2.571873in}{2.407288in}}%
\pgfpathlineto{\pgfqpoint{2.572508in}{1.427681in}}%
\pgfpathlineto{\pgfqpoint{2.572826in}{2.250165in}}%
\pgfpathlineto{\pgfqpoint{2.573143in}{2.186068in}}%
\pgfpathlineto{\pgfqpoint{2.573461in}{1.407086in}}%
\pgfpathlineto{\pgfqpoint{2.574096in}{2.382759in}}%
\pgfpathlineto{\pgfqpoint{2.574413in}{1.525403in}}%
\pgfpathlineto{\pgfqpoint{2.575048in}{2.443741in}}%
\pgfpathlineto{\pgfqpoint{2.575365in}{1.742772in}}%
\pgfpathlineto{\pgfqpoint{2.575683in}{1.515253in}}%
\pgfpathlineto{\pgfqpoint{2.576000in}{2.363809in}}%
\pgfpathlineto{\pgfqpoint{2.576635in}{1.400141in}}%
\pgfpathlineto{\pgfqpoint{2.576953in}{2.170914in}}%
\pgfpathlineto{\pgfqpoint{2.577270in}{2.250972in}}%
\pgfpathlineto{\pgfqpoint{2.577588in}{1.425367in}}%
\pgfpathlineto{\pgfqpoint{2.578222in}{2.405909in}}%
\pgfpathlineto{\pgfqpoint{2.578540in}{1.580753in}}%
\pgfpathlineto{\pgfqpoint{2.578857in}{1.653924in}}%
\pgfpathlineto{\pgfqpoint{2.579175in}{2.428015in}}%
\pgfpathlineto{\pgfqpoint{2.579810in}{1.462100in}}%
\pgfpathlineto{\pgfqpoint{2.580127in}{2.314038in}}%
\pgfpathlineto{\pgfqpoint{2.580762in}{1.394607in}}%
\pgfpathlineto{\pgfqpoint{2.581080in}{2.097908in}}%
\pgfpathlineto{\pgfqpoint{2.581397in}{2.325846in}}%
\pgfpathlineto{\pgfqpoint{2.581715in}{1.469139in}}%
\pgfpathlineto{\pgfqpoint{2.582349in}{2.435198in}}%
\pgfpathlineto{\pgfqpoint{2.582667in}{1.654863in}}%
\pgfpathlineto{\pgfqpoint{2.582984in}{1.581491in}}%
\pgfpathlineto{\pgfqpoint{2.583302in}{2.401129in}}%
\pgfpathlineto{\pgfqpoint{2.583937in}{1.419828in}}%
\pgfpathlineto{\pgfqpoint{2.584254in}{2.244294in}}%
\pgfpathlineto{\pgfqpoint{2.584572in}{2.169765in}}%
\pgfpathlineto{\pgfqpoint{2.584889in}{1.397783in}}%
\pgfpathlineto{\pgfqpoint{2.585524in}{2.362981in}}%
\pgfpathlineto{\pgfqpoint{2.585841in}{1.512104in}}%
\pgfpathlineto{\pgfqpoint{2.586476in}{2.433722in}}%
\pgfpathlineto{\pgfqpoint{2.586794in}{1.730861in}}%
\pgfpathlineto{\pgfqpoint{2.587111in}{1.514170in}}%
\pgfpathlineto{\pgfqpoint{2.587429in}{2.363777in}}%
\pgfpathlineto{\pgfqpoint{2.588064in}{1.398114in}}%
\pgfpathlineto{\pgfqpoint{2.588381in}{2.178526in}}%
\pgfpathlineto{\pgfqpoint{2.588699in}{2.254219in}}%
\pgfpathlineto{\pgfqpoint{2.589016in}{1.425957in}}%
\pgfpathlineto{\pgfqpoint{2.589651in}{2.408988in}}%
\pgfpathlineto{\pgfqpoint{2.589968in}{1.577323in}}%
\pgfpathlineto{\pgfqpoint{2.590286in}{1.655647in}}%
\pgfpathlineto{\pgfqpoint{2.590603in}{2.424397in}}%
\pgfpathlineto{\pgfqpoint{2.591238in}{1.458592in}}%
\pgfpathlineto{\pgfqpoint{2.591556in}{2.306399in}}%
\pgfpathlineto{\pgfqpoint{2.592191in}{1.385623in}}%
\pgfpathlineto{\pgfqpoint{2.592508in}{2.089311in}}%
\pgfpathlineto{\pgfqpoint{2.592826in}{2.304852in}}%
\pgfpathlineto{\pgfqpoint{2.593143in}{1.455040in}}%
\pgfpathlineto{\pgfqpoint{2.593778in}{2.421360in}}%
\pgfpathlineto{\pgfqpoint{2.594095in}{1.644123in}}%
\pgfpathlineto{\pgfqpoint{2.594413in}{1.576333in}}%
\pgfpathlineto{\pgfqpoint{2.594730in}{2.400583in}}%
\pgfpathlineto{\pgfqpoint{2.595365in}{1.421118in}}%
\pgfpathlineto{\pgfqpoint{2.595683in}{2.249878in}}%
\pgfpathlineto{\pgfqpoint{2.596000in}{2.172712in}}%
\pgfpathlineto{\pgfqpoint{2.596318in}{1.397848in}}%
\pgfpathlineto{\pgfqpoint{2.596952in}{2.366568in}}%
\pgfpathlineto{\pgfqpoint{2.597270in}{1.508704in}}%
\pgfpathlineto{\pgfqpoint{2.597905in}{2.429789in}}%
\pgfpathlineto{\pgfqpoint{2.598222in}{1.722311in}}%
\pgfpathlineto{\pgfqpoint{2.598540in}{1.509499in}}%
\pgfpathlineto{\pgfqpoint{2.598857in}{2.357450in}}%
\pgfpathlineto{\pgfqpoint{2.599492in}{1.392217in}}%
\pgfpathlineto{\pgfqpoint{2.599810in}{2.170077in}}%
\pgfpathlineto{\pgfqpoint{2.600127in}{2.234642in}}%
\pgfpathlineto{\pgfqpoint{2.600445in}{1.412825in}}%
\pgfpathlineto{\pgfqpoint{2.601079in}{2.392833in}}%
\pgfpathlineto{\pgfqpoint{2.601397in}{1.563539in}}%
\pgfpathlineto{\pgfqpoint{2.601714in}{1.652463in}}%
\pgfpathlineto{\pgfqpoint{2.602032in}{2.419832in}}%
\pgfpathlineto{\pgfqpoint{2.602667in}{1.457413in}}%
\pgfpathlineto{\pgfqpoint{2.602984in}{2.311683in}}%
\pgfpathlineto{\pgfqpoint{2.603619in}{1.387489in}}%
\pgfpathlineto{\pgfqpoint{2.603937in}{2.101059in}}%
\pgfpathlineto{\pgfqpoint{2.604254in}{2.309026in}}%
\pgfpathlineto{\pgfqpoint{2.604571in}{1.453739in}}%
\pgfpathlineto{\pgfqpoint{2.605206in}{2.419341in}}%
\pgfpathlineto{\pgfqpoint{2.605524in}{1.634243in}}%
\pgfpathlineto{\pgfqpoint{2.605841in}{1.577009in}}%
\pgfpathlineto{\pgfqpoint{2.606159in}{2.392705in}}%
\pgfpathlineto{\pgfqpoint{2.606794in}{1.413407in}}%
\pgfpathlineto{\pgfqpoint{2.607111in}{2.243079in}}%
\pgfpathlineto{\pgfqpoint{2.607429in}{2.151576in}}%
\pgfpathlineto{\pgfqpoint{2.607746in}{1.386933in}}%
\pgfpathlineto{\pgfqpoint{2.608381in}{2.348100in}}%
\pgfpathlineto{\pgfqpoint{2.608698in}{1.495562in}}%
\pgfpathlineto{\pgfqpoint{2.609333in}{2.423663in}}%
\pgfpathlineto{\pgfqpoint{2.609651in}{1.711035in}}%
\pgfpathlineto{\pgfqpoint{2.609968in}{1.511453in}}%
\pgfpathlineto{\pgfqpoint{2.610286in}{2.359370in}}%
\pgfpathlineto{\pgfqpoint{2.610921in}{1.392540in}}%
\pgfpathlineto{\pgfqpoint{2.611238in}{2.182570in}}%
\pgfpathlineto{\pgfqpoint{2.611556in}{2.236542in}}%
\pgfpathlineto{\pgfqpoint{2.611873in}{1.412765in}}%
\pgfpathlineto{\pgfqpoint{2.612508in}{2.391574in}}%
\pgfpathlineto{\pgfqpoint{2.612825in}{1.557046in}}%
\pgfpathlineto{\pgfqpoint{2.613143in}{1.652610in}}%
\pgfpathlineto{\pgfqpoint{2.613460in}{2.413592in}}%
\pgfpathlineto{\pgfqpoint{2.614095in}{1.453838in}}%
\pgfpathlineto{\pgfqpoint{2.614413in}{2.303994in}}%
\pgfpathlineto{\pgfqpoint{2.615048in}{1.376577in}}%
\pgfpathlineto{\pgfqpoint{2.615365in}{2.091094in}}%
\pgfpathlineto{\pgfqpoint{2.615683in}{2.287848in}}%
\pgfpathlineto{\pgfqpoint{2.616000in}{1.439882in}}%
\pgfpathlineto{\pgfqpoint{2.616635in}{2.409235in}}%
\pgfpathlineto{\pgfqpoint{2.616952in}{1.624239in}}%
\pgfpathlineto{\pgfqpoint{2.617270in}{1.575620in}}%
\pgfpathlineto{\pgfqpoint{2.617587in}{2.394688in}}%
\pgfpathlineto{\pgfqpoint{2.618222in}{1.416557in}}%
\pgfpathlineto{\pgfqpoint{2.618540in}{2.253433in}}%
\pgfpathlineto{\pgfqpoint{2.618857in}{2.154331in}}%
\pgfpathlineto{\pgfqpoint{2.619175in}{1.387411in}}%
\pgfpathlineto{\pgfqpoint{2.619809in}{2.347200in}}%
\pgfpathlineto{\pgfqpoint{2.620127in}{1.489578in}}%
\pgfpathlineto{\pgfqpoint{2.620762in}{2.416618in}}%
\pgfpathlineto{\pgfqpoint{2.621079in}{1.699977in}}%
\pgfpathlineto{\pgfqpoint{2.621397in}{1.506465in}}%
\pgfpathlineto{\pgfqpoint{2.621714in}{2.353319in}}%
\pgfpathlineto{\pgfqpoint{2.622349in}{1.385316in}}%
\pgfpathlineto{\pgfqpoint{2.622667in}{2.171657in}}%
\pgfpathlineto{\pgfqpoint{2.622984in}{2.215651in}}%
\pgfpathlineto{\pgfqpoint{2.623301in}{1.399306in}}%
\pgfpathlineto{\pgfqpoint{2.623936in}{2.378292in}}%
\pgfpathlineto{\pgfqpoint{2.624254in}{1.544291in}}%
\pgfpathlineto{\pgfqpoint{2.624571in}{1.653584in}}%
\pgfpathlineto{\pgfqpoint{2.624889in}{2.411978in}}%
\pgfpathlineto{\pgfqpoint{2.625524in}{1.454404in}}%
\pgfpathlineto{\pgfqpoint{2.625841in}{2.313969in}}%
\pgfpathlineto{\pgfqpoint{2.626476in}{1.380225in}}%
\pgfpathlineto{\pgfqpoint{2.626794in}{2.104163in}}%
\pgfpathlineto{\pgfqpoint{2.627111in}{2.288073in}}%
\pgfpathlineto{\pgfqpoint{2.627428in}{1.436177in}}%
\pgfpathlineto{\pgfqpoint{2.628063in}{2.403457in}}%
\pgfpathlineto{\pgfqpoint{2.628381in}{1.611789in}}%
\pgfpathlineto{\pgfqpoint{2.628698in}{1.575467in}}%
\pgfpathlineto{\pgfqpoint{2.629016in}{2.386189in}}%
\pgfpathlineto{\pgfqpoint{2.629651in}{1.408739in}}%
\pgfpathlineto{\pgfqpoint{2.629968in}{2.244149in}}%
\pgfpathlineto{\pgfqpoint{2.630286in}{2.130599in}}%
\pgfpathlineto{\pgfqpoint{2.630603in}{1.375753in}}%
\pgfpathlineto{\pgfqpoint{2.631238in}{2.331195in}}%
\pgfpathlineto{\pgfqpoint{2.631555in}{1.477321in}}%
\pgfpathlineto{\pgfqpoint{2.632190in}{2.413355in}}%
\pgfpathlineto{\pgfqpoint{2.632508in}{1.688760in}}%
\pgfpathlineto{\pgfqpoint{2.632825in}{1.510856in}}%
\pgfpathlineto{\pgfqpoint{2.633143in}{2.358615in}}%
\pgfpathlineto{\pgfqpoint{2.633778in}{1.388087in}}%
\pgfpathlineto{\pgfqpoint{2.634095in}{2.186377in}}%
\pgfpathlineto{\pgfqpoint{2.634413in}{2.214258in}}%
\pgfpathlineto{\pgfqpoint{2.634730in}{1.397988in}}%
\pgfpathlineto{\pgfqpoint{2.635365in}{2.373027in}}%
\pgfpathlineto{\pgfqpoint{2.635682in}{1.535099in}}%
\pgfpathlineto{\pgfqpoint{2.636000in}{1.652949in}}%
\pgfpathlineto{\pgfqpoint{2.636317in}{2.404351in}}%
\pgfpathlineto{\pgfqpoint{2.636952in}{1.451423in}}%
\pgfpathlineto{\pgfqpoint{2.637270in}{2.303517in}}%
\pgfpathlineto{\pgfqpoint{2.637905in}{1.367800in}}%
\pgfpathlineto{\pgfqpoint{2.638222in}{2.095058in}}%
\pgfpathlineto{\pgfqpoint{2.638539in}{2.268398in}}%
\pgfpathlineto{\pgfqpoint{2.638857in}{1.423494in}}%
\pgfpathlineto{\pgfqpoint{2.639492in}{2.396259in}}%
\pgfpathlineto{\pgfqpoint{2.639809in}{1.602211in}}%
\pgfpathlineto{\pgfqpoint{2.640127in}{1.577455in}}%
\pgfpathlineto{\pgfqpoint{2.640444in}{2.389947in}}%
\pgfpathlineto{\pgfqpoint{2.641079in}{1.414883in}}%
\pgfpathlineto{\pgfqpoint{2.641397in}{2.257235in}}%
\pgfpathlineto{\pgfqpoint{2.641714in}{2.131122in}}%
\pgfpathlineto{\pgfqpoint{2.642031in}{1.375720in}}%
\pgfpathlineto{\pgfqpoint{2.642666in}{2.325664in}}%
\pgfpathlineto{\pgfqpoint{2.642984in}{1.468970in}}%
\pgfpathlineto{\pgfqpoint{2.643619in}{2.404226in}}%
\pgfpathlineto{\pgfqpoint{2.643936in}{1.675627in}}%
\pgfpathlineto{\pgfqpoint{2.644254in}{1.506224in}}%
\pgfpathlineto{\pgfqpoint{2.644571in}{2.350949in}}%
\pgfpathlineto{\pgfqpoint{2.645206in}{1.379487in}}%
\pgfpathlineto{\pgfqpoint{2.645524in}{2.175355in}}%
\pgfpathlineto{\pgfqpoint{2.645841in}{2.193945in}}%
\pgfpathlineto{\pgfqpoint{2.646158in}{1.385083in}}%
\pgfpathlineto{\pgfqpoint{2.646793in}{2.362312in}}%
\pgfpathlineto{\pgfqpoint{2.647111in}{1.523284in}}%
\pgfpathlineto{\pgfqpoint{2.647428in}{1.657509in}}%
\pgfpathlineto{\pgfqpoint{2.647746in}{2.404092in}}%
\pgfpathlineto{\pgfqpoint{2.648381in}{1.454907in}}%
\pgfpathlineto{\pgfqpoint{2.648698in}{2.316906in}}%
\pgfpathlineto{\pgfqpoint{2.649333in}{1.372099in}}%
\pgfpathlineto{\pgfqpoint{2.649650in}{2.107657in}}%
\pgfpathlineto{\pgfqpoint{2.649968in}{2.264509in}}%
\pgfpathlineto{\pgfqpoint{2.650285in}{1.417606in}}%
\pgfpathlineto{\pgfqpoint{2.650920in}{2.387743in}}%
\pgfpathlineto{\pgfqpoint{2.651238in}{1.587871in}}%
\pgfpathlineto{\pgfqpoint{2.651555in}{1.577393in}}%
\pgfpathlineto{\pgfqpoint{2.651873in}{2.381046in}}%
\pgfpathlineto{\pgfqpoint{2.652508in}{1.405665in}}%
\pgfpathlineto{\pgfqpoint{2.652825in}{2.246786in}}%
\pgfpathlineto{\pgfqpoint{2.653143in}{2.106615in}}%
\pgfpathlineto{\pgfqpoint{2.653460in}{1.364387in}}%
\pgfpathlineto{\pgfqpoint{2.654095in}{2.312339in}}%
\pgfpathlineto{\pgfqpoint{2.654412in}{1.457729in}}%
\pgfpathlineto{\pgfqpoint{2.655047in}{2.402521in}}%
\pgfpathlineto{\pgfqpoint{2.655365in}{1.664040in}}%
\pgfpathlineto{\pgfqpoint{2.655682in}{1.512931in}}%
\pgfpathlineto{\pgfqpoint{2.656000in}{2.359583in}}%
\pgfpathlineto{\pgfqpoint{2.656635in}{1.383633in}}%
\pgfpathlineto{\pgfqpoint{2.656952in}{2.190194in}}%
\pgfpathlineto{\pgfqpoint{2.657269in}{2.188626in}}%
\pgfpathlineto{\pgfqpoint{2.657587in}{1.382359in}}%
\pgfpathlineto{\pgfqpoint{2.658222in}{2.353570in}}%
\pgfpathlineto{\pgfqpoint{2.658539in}{1.511863in}}%
\pgfpathlineto{\pgfqpoint{2.658857in}{1.656756in}}%
\pgfpathlineto{\pgfqpoint{2.659174in}{2.395920in}}%
\pgfpathlineto{\pgfqpoint{2.659809in}{1.451332in}}%
\pgfpathlineto{\pgfqpoint{2.660127in}{2.304658in}}%
\pgfpathlineto{\pgfqpoint{2.660761in}{1.359593in}}%
\pgfpathlineto{\pgfqpoint{2.661079in}{2.101232in}}%
\pgfpathlineto{\pgfqpoint{2.661396in}{2.246525in}}%
\pgfpathlineto{\pgfqpoint{2.661714in}{1.406208in}}%
\pgfpathlineto{\pgfqpoint{2.662349in}{2.382272in}}%
\pgfpathlineto{\pgfqpoint{2.662666in}{1.578128in}}%
\pgfpathlineto{\pgfqpoint{2.662984in}{1.581810in}}%
\pgfpathlineto{\pgfqpoint{2.663301in}{2.387619in}}%
\pgfpathlineto{\pgfqpoint{2.663936in}{1.414293in}}%
\pgfpathlineto{\pgfqpoint{2.664254in}{2.260717in}}%
\pgfpathlineto{\pgfqpoint{2.664571in}{2.104056in}}%
\pgfpathlineto{\pgfqpoint{2.664888in}{1.363344in}}%
\pgfpathlineto{\pgfqpoint{2.665523in}{2.302965in}}%
\pgfpathlineto{\pgfqpoint{2.665841in}{1.447660in}}%
\pgfpathlineto{\pgfqpoint{2.666476in}{2.392207in}}%
\pgfpathlineto{\pgfqpoint{2.666793in}{1.649401in}}%
\pgfpathlineto{\pgfqpoint{2.667111in}{1.508899in}}%
\pgfpathlineto{\pgfqpoint{2.667428in}{2.349505in}}%
\pgfpathlineto{\pgfqpoint{2.668063in}{1.374657in}}%
\pgfpathlineto{\pgfqpoint{2.668380in}{2.180711in}}%
\pgfpathlineto{\pgfqpoint{2.668698in}{2.169445in}}%
\pgfpathlineto{\pgfqpoint{2.669015in}{1.370468in}}%
\pgfpathlineto{\pgfqpoint{2.669650in}{2.344764in}}%
\pgfpathlineto{\pgfqpoint{2.669968in}{1.500690in}}%
\pgfpathlineto{\pgfqpoint{2.670603in}{2.397326in}}%
\pgfpathlineto{\pgfqpoint{2.670920in}{1.735724in}}%
\pgfpathlineto{\pgfqpoint{2.671238in}{1.458144in}}%
\pgfpathlineto{\pgfqpoint{2.671555in}{2.319395in}}%
\pgfpathlineto{\pgfqpoint{2.672190in}{1.363715in}}%
\pgfpathlineto{\pgfqpoint{2.672507in}{2.112424in}}%
\pgfpathlineto{\pgfqpoint{2.672825in}{2.238729in}}%
\pgfpathlineto{\pgfqpoint{2.673142in}{1.398331in}}%
\pgfpathlineto{\pgfqpoint{2.673777in}{2.371370in}}%
\pgfpathlineto{\pgfqpoint{2.674095in}{1.562357in}}%
\pgfpathlineto{\pgfqpoint{2.674412in}{1.582449in}}%
\pgfpathlineto{\pgfqpoint{2.674730in}{2.376574in}}%
\pgfpathlineto{\pgfqpoint{2.675365in}{1.404503in}}%
\pgfpathlineto{\pgfqpoint{2.675682in}{2.251091in}}%
\pgfpathlineto{\pgfqpoint{2.676317in}{1.353258in}}%
\pgfpathlineto{\pgfqpoint{2.676634in}{2.021761in}}%
\pgfpathlineto{\pgfqpoint{2.676952in}{2.291395in}}%
\pgfpathlineto{\pgfqpoint{2.677269in}{1.437023in}}%
\pgfpathlineto{\pgfqpoint{2.677904in}{2.391076in}}%
\pgfpathlineto{\pgfqpoint{2.678222in}{1.638778in}}%
\pgfpathlineto{\pgfqpoint{2.678539in}{1.519489in}}%
\pgfpathlineto{\pgfqpoint{2.678857in}{2.360066in}}%
\pgfpathlineto{\pgfqpoint{2.679491in}{1.379487in}}%
\pgfpathlineto{\pgfqpoint{2.679809in}{2.194659in}}%
\pgfpathlineto{\pgfqpoint{2.680126in}{2.160134in}}%
\pgfpathlineto{\pgfqpoint{2.680444in}{1.366350in}}%
\pgfpathlineto{\pgfqpoint{2.681079in}{2.333319in}}%
\pgfpathlineto{\pgfqpoint{2.681396in}{1.487622in}}%
\pgfpathlineto{\pgfqpoint{2.682031in}{2.387631in}}%
\pgfpathlineto{\pgfqpoint{2.682349in}{1.713743in}}%
\pgfpathlineto{\pgfqpoint{2.682666in}{1.453555in}}%
\pgfpathlineto{\pgfqpoint{2.682984in}{2.306691in}}%
\pgfpathlineto{\pgfqpoint{2.683618in}{1.352091in}}%
\pgfpathlineto{\pgfqpoint{2.683936in}{2.109435in}}%
\pgfpathlineto{\pgfqpoint{2.684253in}{2.222150in}}%
\pgfpathlineto{\pgfqpoint{2.684571in}{1.388355in}}%
\pgfpathlineto{\pgfqpoint{2.685206in}{2.366883in}}%
\pgfpathlineto{\pgfqpoint{2.685523in}{1.552168in}}%
\pgfpathlineto{\pgfqpoint{2.685841in}{1.590075in}}%
\pgfpathlineto{\pgfqpoint{2.686158in}{2.386307in}}%
\pgfpathlineto{\pgfqpoint{2.686793in}{1.414841in}}%
\pgfpathlineto{\pgfqpoint{2.687110in}{2.264052in}}%
\pgfpathlineto{\pgfqpoint{2.687745in}{1.350765in}}%
\pgfpathlineto{\pgfqpoint{2.688063in}{2.030091in}}%
\pgfpathlineto{\pgfqpoint{2.688380in}{2.278465in}}%
\pgfpathlineto{\pgfqpoint{2.688698in}{1.425368in}}%
\pgfpathlineto{\pgfqpoint{2.689333in}{2.379742in}}%
\pgfpathlineto{\pgfqpoint{2.689650in}{1.621277in}}%
\pgfpathlineto{\pgfqpoint{2.689968in}{1.514473in}}%
\pgfpathlineto{\pgfqpoint{2.690285in}{2.348815in}}%
\pgfpathlineto{\pgfqpoint{2.690920in}{1.371375in}}%
\pgfpathlineto{\pgfqpoint{2.691237in}{2.187880in}}%
\pgfpathlineto{\pgfqpoint{2.691555in}{2.142110in}}%
\pgfpathlineto{\pgfqpoint{2.691872in}{1.355805in}}%
\pgfpathlineto{\pgfqpoint{2.692507in}{2.325369in}}%
\pgfpathlineto{\pgfqpoint{2.692825in}{1.476702in}}%
\pgfpathlineto{\pgfqpoint{2.693460in}{2.392994in}}%
\pgfpathlineto{\pgfqpoint{2.693777in}{1.708231in}}%
\pgfpathlineto{\pgfqpoint{2.694095in}{1.463411in}}%
\pgfpathlineto{\pgfqpoint{2.694412in}{2.320823in}}%
\pgfpathlineto{\pgfqpoint{2.695047in}{1.355590in}}%
\pgfpathlineto{\pgfqpoint{2.695364in}{2.118751in}}%
\pgfpathlineto{\pgfqpoint{2.695682in}{2.210674in}}%
\pgfpathlineto{\pgfqpoint{2.695999in}{1.378719in}}%
\pgfpathlineto{\pgfqpoint{2.696634in}{2.354342in}}%
\pgfpathlineto{\pgfqpoint{2.696952in}{1.535463in}}%
\pgfpathlineto{\pgfqpoint{2.697269in}{1.590744in}}%
\pgfpathlineto{\pgfqpoint{2.697587in}{2.372201in}}%
\pgfpathlineto{\pgfqpoint{2.698221in}{1.405276in}}%
\pgfpathlineto{\pgfqpoint{2.698539in}{2.256476in}}%
\pgfpathlineto{\pgfqpoint{2.699174in}{1.342637in}}%
\pgfpathlineto{\pgfqpoint{2.699491in}{2.032570in}}%
\pgfpathlineto{\pgfqpoint{2.699809in}{2.268172in}}%
\pgfpathlineto{\pgfqpoint{2.700126in}{1.415483in}}%
\pgfpathlineto{\pgfqpoint{2.700761in}{2.381006in}}%
\pgfpathlineto{\pgfqpoint{2.701079in}{1.611705in}}%
\pgfpathlineto{\pgfqpoint{2.701396in}{1.528333in}}%
\pgfpathlineto{\pgfqpoint{2.701714in}{2.360434in}}%
\pgfpathlineto{\pgfqpoint{2.702348in}{1.375912in}}%
\pgfpathlineto{\pgfqpoint{2.702666in}{2.199646in}}%
\pgfpathlineto{\pgfqpoint{2.702983in}{2.128562in}}%
\pgfpathlineto{\pgfqpoint{2.703301in}{1.350262in}}%
\pgfpathlineto{\pgfqpoint{2.703936in}{2.311714in}}%
\pgfpathlineto{\pgfqpoint{2.704253in}{1.462268in}}%
\pgfpathlineto{\pgfqpoint{2.704888in}{2.379135in}}%
\pgfpathlineto{\pgfqpoint{2.705206in}{1.683053in}}%
\pgfpathlineto{\pgfqpoint{2.705523in}{1.458348in}}%
\pgfpathlineto{\pgfqpoint{2.705840in}{2.309478in}}%
\pgfpathlineto{\pgfqpoint{2.706475in}{1.345742in}}%
\pgfpathlineto{\pgfqpoint{2.706793in}{2.119613in}}%
\pgfpathlineto{\pgfqpoint{2.707110in}{2.195095in}}%
\pgfpathlineto{\pgfqpoint{2.707428in}{1.370243in}}%
\pgfpathlineto{\pgfqpoint{2.708063in}{2.350361in}}%
\pgfpathlineto{\pgfqpoint{2.708380in}{1.525813in}}%
\pgfpathlineto{\pgfqpoint{2.708698in}{1.602369in}}%
\pgfpathlineto{\pgfqpoint{2.709015in}{2.384344in}}%
\pgfpathlineto{\pgfqpoint{2.709650in}{1.416240in}}%
\pgfpathlineto{\pgfqpoint{2.709967in}{2.267021in}}%
\pgfpathlineto{\pgfqpoint{2.710602in}{1.338770in}}%
\pgfpathlineto{\pgfqpoint{2.710920in}{2.039671in}}%
\pgfpathlineto{\pgfqpoint{2.711237in}{2.252351in}}%
\pgfpathlineto{\pgfqpoint{2.711555in}{1.402784in}}%
\pgfpathlineto{\pgfqpoint{2.712190in}{2.366627in}}%
\pgfpathlineto{\pgfqpoint{2.712507in}{1.591297in}}%
\pgfpathlineto{\pgfqpoint{2.712825in}{1.522957in}}%
\pgfpathlineto{\pgfqpoint{2.713142in}{2.348105in}}%
\pgfpathlineto{\pgfqpoint{2.713777in}{1.369681in}}%
\pgfpathlineto{\pgfqpoint{2.714094in}{2.196416in}}%
\pgfpathlineto{\pgfqpoint{2.714412in}{2.111823in}}%
\pgfpathlineto{\pgfqpoint{2.714729in}{1.341482in}}%
\pgfpathlineto{\pgfqpoint{2.715364in}{2.303760in}}%
\pgfpathlineto{\pgfqpoint{2.715682in}{1.451503in}}%
\pgfpathlineto{\pgfqpoint{2.716317in}{2.387128in}}%
\pgfpathlineto{\pgfqpoint{2.716634in}{1.676814in}}%
\pgfpathlineto{\pgfqpoint{2.716951in}{1.469704in}}%
\pgfpathlineto{\pgfqpoint{2.717269in}{2.322239in}}%
\pgfpathlineto{\pgfqpoint{2.717904in}{1.348156in}}%
\pgfpathlineto{\pgfqpoint{2.718221in}{2.126780in}}%
\pgfpathlineto{\pgfqpoint{2.718539in}{2.180285in}}%
\pgfpathlineto{\pgfqpoint{2.718856in}{1.359234in}}%
\pgfpathlineto{\pgfqpoint{2.719491in}{2.336164in}}%
\pgfpathlineto{\pgfqpoint{2.719809in}{1.507209in}}%
\pgfpathlineto{\pgfqpoint{2.720126in}{1.602379in}}%
\pgfpathlineto{\pgfqpoint{2.720444in}{2.367582in}}%
\pgfpathlineto{\pgfqpoint{2.721078in}{1.408227in}}%
\pgfpathlineto{\pgfqpoint{2.721396in}{2.262674in}}%
\pgfpathlineto{\pgfqpoint{2.722031in}{1.332914in}}%
\pgfpathlineto{\pgfqpoint{2.722348in}{2.045696in}}%
\pgfpathlineto{\pgfqpoint{2.722666in}{2.242397in}}%
\pgfpathlineto{\pgfqpoint{2.722983in}{1.393407in}}%
\pgfpathlineto{\pgfqpoint{2.723618in}{2.370458in}}%
\pgfpathlineto{\pgfqpoint{2.723936in}{1.582172in}}%
\pgfpathlineto{\pgfqpoint{2.724253in}{1.539272in}}%
\pgfpathlineto{\pgfqpoint{2.724570in}{2.359445in}}%
\pgfpathlineto{\pgfqpoint{2.725205in}{1.373301in}}%
\pgfpathlineto{\pgfqpoint{2.725523in}{2.205745in}}%
\pgfpathlineto{\pgfqpoint{2.725840in}{2.094626in}}%
\pgfpathlineto{\pgfqpoint{2.726158in}{1.335056in}}%
\pgfpathlineto{\pgfqpoint{2.726793in}{2.288609in}}%
\pgfpathlineto{\pgfqpoint{2.727110in}{1.436260in}}%
\pgfpathlineto{\pgfqpoint{2.727745in}{2.369701in}}%
\pgfpathlineto{\pgfqpoint{2.728063in}{1.650035in}}%
\pgfpathlineto{\pgfqpoint{2.728380in}{1.465681in}}%
\pgfpathlineto{\pgfqpoint{2.728697in}{2.312402in}}%
\pgfpathlineto{\pgfqpoint{2.729332in}{1.340831in}}%
\pgfpathlineto{\pgfqpoint{2.729650in}{2.131591in}}%
\pgfpathlineto{\pgfqpoint{2.729967in}{2.165199in}}%
\pgfpathlineto{\pgfqpoint{2.730285in}{1.352265in}}%
\pgfpathlineto{\pgfqpoint{2.730920in}{2.333930in}}%
\pgfpathlineto{\pgfqpoint{2.731237in}{1.498195in}}%
\pgfpathlineto{\pgfqpoint{2.731555in}{1.617289in}}%
\pgfpathlineto{\pgfqpoint{2.731872in}{2.379819in}}%
\pgfpathlineto{\pgfqpoint{2.732507in}{1.419049in}}%
\pgfpathlineto{\pgfqpoint{2.732824in}{2.270848in}}%
\pgfpathlineto{\pgfqpoint{2.733459in}{1.327839in}}%
\pgfpathlineto{\pgfqpoint{2.733777in}{2.051925in}}%
\pgfpathlineto{\pgfqpoint{2.734094in}{2.224173in}}%
\pgfpathlineto{\pgfqpoint{2.734412in}{1.380196in}}%
\pgfpathlineto{\pgfqpoint{2.735047in}{2.352188in}}%
\pgfpathlineto{\pgfqpoint{2.735364in}{1.559509in}}%
\pgfpathlineto{\pgfqpoint{2.735681in}{1.534519in}}%
\pgfpathlineto{\pgfqpoint{2.735999in}{2.347061in}}%
\pgfpathlineto{\pgfqpoint{2.736634in}{1.369918in}}%
\pgfpathlineto{\pgfqpoint{2.736951in}{2.206142in}}%
\pgfpathlineto{\pgfqpoint{2.737269in}{2.078475in}}%
\pgfpathlineto{\pgfqpoint{2.737586in}{1.327917in}}%
\pgfpathlineto{\pgfqpoint{2.738221in}{2.279982in}}%
\pgfpathlineto{\pgfqpoint{2.738539in}{1.426029in}}%
\pgfpathlineto{\pgfqpoint{2.739174in}{2.378555in}}%
\pgfpathlineto{\pgfqpoint{2.739491in}{1.642173in}}%
\pgfpathlineto{\pgfqpoint{2.739808in}{1.477886in}}%
\pgfpathlineto{\pgfqpoint{2.740126in}{2.323206in}}%
\pgfpathlineto{\pgfqpoint{2.740761in}{1.342298in}}%
\pgfpathlineto{\pgfqpoint{2.741078in}{2.137234in}}%
\pgfpathlineto{\pgfqpoint{2.741396in}{2.147793in}}%
\pgfpathlineto{\pgfqpoint{2.741713in}{1.340556in}}%
\pgfpathlineto{\pgfqpoint{2.742348in}{2.316135in}}%
\pgfpathlineto{\pgfqpoint{2.742666in}{1.477760in}}%
\pgfpathlineto{\pgfqpoint{2.742983in}{1.617278in}}%
\pgfpathlineto{\pgfqpoint{2.743300in}{2.361901in}}%
\pgfpathlineto{\pgfqpoint{2.743935in}{1.413549in}}%
\pgfpathlineto{\pgfqpoint{2.744253in}{2.269361in}}%
\pgfpathlineto{\pgfqpoint{2.744888in}{1.324532in}}%
\pgfpathlineto{\pgfqpoint{2.745205in}{2.061029in}}%
\pgfpathlineto{\pgfqpoint{2.745523in}{2.213746in}}%
\pgfpathlineto{\pgfqpoint{2.745840in}{1.371124in}}%
\pgfpathlineto{\pgfqpoint{2.746475in}{2.357159in}}%
\pgfpathlineto{\pgfqpoint{2.746793in}{1.550071in}}%
\pgfpathlineto{\pgfqpoint{2.747110in}{1.552346in}}%
\pgfpathlineto{\pgfqpoint{2.747427in}{2.356705in}}%
\pgfpathlineto{\pgfqpoint{2.748062in}{1.372840in}}%
\pgfpathlineto{\pgfqpoint{2.748380in}{2.213556in}}%
\pgfpathlineto{\pgfqpoint{2.748697in}{2.058013in}}%
\pgfpathlineto{\pgfqpoint{2.749015in}{1.321044in}}%
\pgfpathlineto{\pgfqpoint{2.749650in}{2.263331in}}%
\pgfpathlineto{\pgfqpoint{2.749967in}{1.409858in}}%
\pgfpathlineto{\pgfqpoint{2.750602in}{2.358708in}}%
\pgfpathlineto{\pgfqpoint{2.750919in}{1.614825in}}%
\pgfpathlineto{\pgfqpoint{2.751237in}{1.475831in}}%
\pgfpathlineto{\pgfqpoint{2.751554in}{2.315181in}}%
\pgfpathlineto{\pgfqpoint{2.752189in}{1.337808in}}%
\pgfpathlineto{\pgfqpoint{2.752507in}{2.145202in}}%
\pgfpathlineto{\pgfqpoint{2.752824in}{2.132226in}}%
\pgfpathlineto{\pgfqpoint{2.753142in}{1.334808in}}%
\pgfpathlineto{\pgfqpoint{2.753777in}{2.315115in}}%
\pgfpathlineto{\pgfqpoint{2.754094in}{1.468912in}}%
\pgfpathlineto{\pgfqpoint{2.754729in}{2.372815in}}%
\pgfpathlineto{\pgfqpoint{2.755046in}{1.699845in}}%
\pgfpathlineto{\pgfqpoint{2.755364in}{1.424410in}}%
\pgfpathlineto{\pgfqpoint{2.755681in}{2.275475in}}%
\pgfpathlineto{\pgfqpoint{2.756316in}{1.318842in}}%
\pgfpathlineto{\pgfqpoint{2.756634in}{2.067178in}}%
\pgfpathlineto{\pgfqpoint{2.756951in}{2.193625in}}%
\pgfpathlineto{\pgfqpoint{2.757269in}{1.357977in}}%
\pgfpathlineto{\pgfqpoint{2.757904in}{2.335568in}}%
\pgfpathlineto{\pgfqpoint{2.758221in}{1.526019in}}%
\pgfpathlineto{\pgfqpoint{2.758538in}{1.549144in}}%
\pgfpathlineto{\pgfqpoint{2.758856in}{2.345091in}}%
\pgfpathlineto{\pgfqpoint{2.759491in}{1.372497in}}%
\pgfpathlineto{\pgfqpoint{2.759808in}{2.216832in}}%
\pgfpathlineto{\pgfqpoint{2.760443in}{1.315573in}}%
\pgfpathlineto{\pgfqpoint{2.760761in}{1.988980in}}%
\pgfpathlineto{\pgfqpoint{2.761078in}{2.255346in}}%
\pgfpathlineto{\pgfqpoint{2.761396in}{1.400666in}}%
\pgfpathlineto{\pgfqpoint{2.762030in}{2.366925in}}%
\pgfpathlineto{\pgfqpoint{2.762348in}{1.604974in}}%
\pgfpathlineto{\pgfqpoint{2.762665in}{1.488642in}}%
\pgfpathlineto{\pgfqpoint{2.762983in}{2.323506in}}%
\pgfpathlineto{\pgfqpoint{2.763618in}{1.338898in}}%
\pgfpathlineto{\pgfqpoint{2.763935in}{2.149805in}}%
\pgfpathlineto{\pgfqpoint{2.764253in}{2.112575in}}%
\pgfpathlineto{\pgfqpoint{2.764570in}{1.322903in}}%
\pgfpathlineto{\pgfqpoint{2.765205in}{2.293663in}}%
\pgfpathlineto{\pgfqpoint{2.765523in}{1.447464in}}%
\pgfpathlineto{\pgfqpoint{2.766157in}{2.354678in}}%
\pgfpathlineto{\pgfqpoint{2.766475in}{1.674557in}}%
\pgfpathlineto{\pgfqpoint{2.766792in}{1.421642in}}%
\pgfpathlineto{\pgfqpoint{2.767110in}{2.276257in}}%
\pgfpathlineto{\pgfqpoint{2.767745in}{1.317980in}}%
\pgfpathlineto{\pgfqpoint{2.768062in}{2.078323in}}%
\pgfpathlineto{\pgfqpoint{2.768380in}{2.182041in}}%
\pgfpathlineto{\pgfqpoint{2.768697in}{1.349071in}}%
\pgfpathlineto{\pgfqpoint{2.769332in}{2.340005in}}%
\pgfpathlineto{\pgfqpoint{2.769649in}{1.515906in}}%
\pgfpathlineto{\pgfqpoint{2.769967in}{1.568087in}}%
\pgfpathlineto{\pgfqpoint{2.770284in}{2.352882in}}%
\pgfpathlineto{\pgfqpoint{2.770919in}{1.375633in}}%
\pgfpathlineto{\pgfqpoint{2.771237in}{2.222937in}}%
\pgfpathlineto{\pgfqpoint{2.771872in}{1.308878in}}%
\pgfpathlineto{\pgfqpoint{2.772189in}{1.994623in}}%
\pgfpathlineto{\pgfqpoint{2.772507in}{2.235241in}}%
\pgfpathlineto{\pgfqpoint{2.772824in}{1.383288in}}%
\pgfpathlineto{\pgfqpoint{2.773459in}{2.345405in}}%
\pgfpathlineto{\pgfqpoint{2.773776in}{1.577605in}}%
\pgfpathlineto{\pgfqpoint{2.774094in}{1.488999in}}%
\pgfpathlineto{\pgfqpoint{2.774411in}{2.317397in}}%
\pgfpathlineto{\pgfqpoint{2.775046in}{1.337165in}}%
\pgfpathlineto{\pgfqpoint{2.775364in}{2.160149in}}%
\pgfpathlineto{\pgfqpoint{2.775681in}{2.095929in}}%
\pgfpathlineto{\pgfqpoint{2.775999in}{1.318334in}}%
\pgfpathlineto{\pgfqpoint{2.776634in}{2.293547in}}%
\pgfpathlineto{\pgfqpoint{2.776951in}{1.438655in}}%
\pgfpathlineto{\pgfqpoint{2.777586in}{2.363242in}}%
\pgfpathlineto{\pgfqpoint{2.777903in}{1.658746in}}%
\pgfpathlineto{\pgfqpoint{2.778221in}{1.432703in}}%
\pgfpathlineto{\pgfqpoint{2.778538in}{2.280227in}}%
\pgfpathlineto{\pgfqpoint{2.779173in}{1.312305in}}%
\pgfpathlineto{\pgfqpoint{2.779491in}{2.084722in}}%
\pgfpathlineto{\pgfqpoint{2.779808in}{2.160101in}}%
\pgfpathlineto{\pgfqpoint{2.780126in}{1.336431in}}%
\pgfpathlineto{\pgfqpoint{2.780760in}{2.316348in}}%
\pgfpathlineto{\pgfqpoint{2.781078in}{1.491259in}}%
\pgfpathlineto{\pgfqpoint{2.781395in}{1.566983in}}%
\pgfpathlineto{\pgfqpoint{2.781713in}{2.341773in}}%
\pgfpathlineto{\pgfqpoint{2.782348in}{1.377877in}}%
\pgfpathlineto{\pgfqpoint{2.782665in}{2.228151in}}%
\pgfpathlineto{\pgfqpoint{2.783300in}{1.304924in}}%
\pgfpathlineto{\pgfqpoint{2.783618in}{2.009797in}}%
\pgfpathlineto{\pgfqpoint{2.783935in}{2.227408in}}%
\pgfpathlineto{\pgfqpoint{2.784253in}{1.374922in}}%
\pgfpathlineto{\pgfqpoint{2.784887in}{2.352120in}}%
\pgfpathlineto{\pgfqpoint{2.785205in}{1.566032in}}%
\pgfpathlineto{\pgfqpoint{2.785522in}{1.502820in}}%
\pgfpathlineto{\pgfqpoint{2.785840in}{2.323768in}}%
\pgfpathlineto{\pgfqpoint{2.786475in}{1.338821in}}%
\pgfpathlineto{\pgfqpoint{2.786792in}{2.164027in}}%
\pgfpathlineto{\pgfqpoint{2.787110in}{2.074361in}}%
\pgfpathlineto{\pgfqpoint{2.787427in}{1.306743in}}%
\pgfpathlineto{\pgfqpoint{2.788062in}{2.268101in}}%
\pgfpathlineto{\pgfqpoint{2.788379in}{1.416599in}}%
\pgfpathlineto{\pgfqpoint{2.789014in}{2.345364in}}%
\pgfpathlineto{\pgfqpoint{2.789332in}{1.633574in}}%
\pgfpathlineto{\pgfqpoint{2.789649in}{1.432888in}}%
\pgfpathlineto{\pgfqpoint{2.789967in}{2.282954in}}%
\pgfpathlineto{\pgfqpoint{2.790602in}{1.313741in}}%
\pgfpathlineto{\pgfqpoint{2.790919in}{2.097241in}}%
\pgfpathlineto{\pgfqpoint{2.791237in}{2.148127in}}%
\pgfpathlineto{\pgfqpoint{2.791554in}{1.328483in}}%
\pgfpathlineto{\pgfqpoint{2.792189in}{2.319692in}}%
\pgfpathlineto{\pgfqpoint{2.792506in}{1.480455in}}%
\pgfpathlineto{\pgfqpoint{2.792824in}{1.586915in}}%
\pgfpathlineto{\pgfqpoint{2.793141in}{2.346896in}}%
\pgfpathlineto{\pgfqpoint{2.793776in}{1.381427in}}%
\pgfpathlineto{\pgfqpoint{2.794094in}{2.232727in}}%
\pgfpathlineto{\pgfqpoint{2.794729in}{1.298886in}}%
\pgfpathlineto{\pgfqpoint{2.795046in}{2.016370in}}%
\pgfpathlineto{\pgfqpoint{2.795364in}{2.203923in}}%
\pgfpathlineto{\pgfqpoint{2.795681in}{1.357027in}}%
\pgfpathlineto{\pgfqpoint{2.796316in}{2.329516in}}%
\pgfpathlineto{\pgfqpoint{2.796633in}{1.538831in}}%
\pgfpathlineto{\pgfqpoint{2.796951in}{1.505481in}}%
\pgfpathlineto{\pgfqpoint{2.797268in}{2.318627in}}%
\pgfpathlineto{\pgfqpoint{2.797903in}{1.339394in}}%
\pgfpathlineto{\pgfqpoint{2.798221in}{2.176028in}}%
\pgfpathlineto{\pgfqpoint{2.798538in}{2.056119in}}%
\pgfpathlineto{\pgfqpoint{2.798856in}{1.303298in}}%
\pgfpathlineto{\pgfqpoint{2.799490in}{2.267552in}}%
\pgfpathlineto{\pgfqpoint{2.799808in}{1.407532in}}%
\pgfpathlineto{\pgfqpoint{2.800443in}{2.351437in}}%
\pgfpathlineto{\pgfqpoint{2.800760in}{1.616078in}}%
\pgfpathlineto{\pgfqpoint{2.801078in}{1.444810in}}%
\pgfpathlineto{\pgfqpoint{2.801395in}{2.285386in}}%
\pgfpathlineto{\pgfqpoint{2.802030in}{1.308852in}}%
\pgfpathlineto{\pgfqpoint{2.802348in}{2.103773in}}%
\pgfpathlineto{\pgfqpoint{2.802665in}{2.123219in}}%
\pgfpathlineto{\pgfqpoint{2.802983in}{1.316028in}}%
\pgfpathlineto{\pgfqpoint{2.803617in}{2.294010in}}%
\pgfpathlineto{\pgfqpoint{2.803935in}{1.455619in}}%
\pgfpathlineto{\pgfqpoint{2.804252in}{1.588185in}}%
\pgfpathlineto{\pgfqpoint{2.804570in}{2.336642in}}%
\pgfpathlineto{\pgfqpoint{2.805205in}{1.386479in}}%
\pgfpathlineto{\pgfqpoint{2.805522in}{2.239642in}}%
\pgfpathlineto{\pgfqpoint{2.806157in}{1.296491in}}%
\pgfpathlineto{\pgfqpoint{2.806475in}{2.032565in}}%
\pgfpathlineto{\pgfqpoint{2.806792in}{2.195978in}}%
\pgfpathlineto{\pgfqpoint{2.807109in}{1.350194in}}%
\pgfpathlineto{\pgfqpoint{2.807744in}{2.333840in}}%
\pgfpathlineto{\pgfqpoint{2.808062in}{1.525397in}}%
\pgfpathlineto{\pgfqpoint{2.808379in}{1.520165in}}%
\pgfpathlineto{\pgfqpoint{2.808697in}{2.322293in}}%
\pgfpathlineto{\pgfqpoint{2.809332in}{1.341814in}}%
\pgfpathlineto{\pgfqpoint{2.809649in}{2.179102in}}%
\pgfpathlineto{\pgfqpoint{2.809967in}{2.032872in}}%
\pgfpathlineto{\pgfqpoint{2.810284in}{1.292565in}}%
\pgfpathlineto{\pgfqpoint{2.810919in}{2.239097in}}%
\pgfpathlineto{\pgfqpoint{2.811236in}{1.385698in}}%
\pgfpathlineto{\pgfqpoint{2.811871in}{2.333577in}}%
\pgfpathlineto{\pgfqpoint{2.812189in}{1.590727in}}%
\pgfpathlineto{\pgfqpoint{2.812506in}{1.447643in}}%
\pgfpathlineto{\pgfqpoint{2.812824in}{2.288985in}}%
\pgfpathlineto{\pgfqpoint{2.813459in}{1.312329in}}%
\pgfpathlineto{\pgfqpoint{2.813776in}{2.117351in}}%
\pgfpathlineto{\pgfqpoint{2.814094in}{2.110817in}}%
\pgfpathlineto{\pgfqpoint{2.814411in}{1.309440in}}%
\pgfpathlineto{\pgfqpoint{2.815046in}{2.295141in}}%
\pgfpathlineto{\pgfqpoint{2.815363in}{1.444208in}}%
\pgfpathlineto{\pgfqpoint{2.815998in}{2.339430in}}%
\pgfpathlineto{\pgfqpoint{2.816316in}{1.669290in}}%
\pgfpathlineto{\pgfqpoint{2.816633in}{1.391263in}}%
\pgfpathlineto{\pgfqpoint{2.816951in}{2.243126in}}%
\pgfpathlineto{\pgfqpoint{2.817586in}{1.291633in}}%
\pgfpathlineto{\pgfqpoint{2.817903in}{2.039648in}}%
\pgfpathlineto{\pgfqpoint{2.818220in}{2.168976in}}%
\pgfpathlineto{\pgfqpoint{2.818538in}{1.331614in}}%
\pgfpathlineto{\pgfqpoint{2.819173in}{2.310527in}}%
\pgfpathlineto{\pgfqpoint{2.819490in}{1.498904in}}%
\pgfpathlineto{\pgfqpoint{2.819808in}{1.525530in}}%
\pgfpathlineto{\pgfqpoint{2.820125in}{2.318353in}}%
\pgfpathlineto{\pgfqpoint{2.820760in}{1.344965in}}%
\pgfpathlineto{\pgfqpoint{2.821078in}{2.192366in}}%
\pgfpathlineto{\pgfqpoint{2.821713in}{1.290474in}}%
\pgfpathlineto{\pgfqpoint{2.822030in}{1.967539in}}%
\pgfpathlineto{\pgfqpoint{2.822347in}{2.237365in}}%
\pgfpathlineto{\pgfqpoint{2.822665in}{1.376822in}}%
\pgfpathlineto{\pgfqpoint{2.823300in}{2.336243in}}%
\pgfpathlineto{\pgfqpoint{2.823617in}{1.571564in}}%
\pgfpathlineto{\pgfqpoint{2.823935in}{1.460367in}}%
\pgfpathlineto{\pgfqpoint{2.824252in}{2.289488in}}%
\pgfpathlineto{\pgfqpoint{2.824887in}{1.308589in}}%
\pgfpathlineto{\pgfqpoint{2.825205in}{2.124037in}}%
\pgfpathlineto{\pgfqpoint{2.825522in}{2.082724in}}%
\pgfpathlineto{\pgfqpoint{2.825839in}{1.297310in}}%
\pgfpathlineto{\pgfqpoint{2.826474in}{2.268046in}}%
\pgfpathlineto{\pgfqpoint{2.826792in}{1.419592in}}%
\pgfpathlineto{\pgfqpoint{2.827427in}{2.329198in}}%
\pgfpathlineto{\pgfqpoint{2.827744in}{1.643378in}}%
\pgfpathlineto{\pgfqpoint{2.828062in}{1.398755in}}%
\pgfpathlineto{\pgfqpoint{2.828379in}{2.250790in}}%
\pgfpathlineto{\pgfqpoint{2.829014in}{1.290798in}}%
\pgfpathlineto{\pgfqpoint{2.829332in}{2.057572in}}%
\pgfpathlineto{\pgfqpoint{2.829649in}{2.159720in}}%
\pgfpathlineto{\pgfqpoint{2.829966in}{1.326140in}}%
\pgfpathlineto{\pgfqpoint{2.830601in}{2.312029in}}%
\pgfpathlineto{\pgfqpoint{2.830919in}{1.484077in}}%
\pgfpathlineto{\pgfqpoint{2.831236in}{1.541606in}}%
\pgfpathlineto{\pgfqpoint{2.831554in}{2.319817in}}%
\pgfpathlineto{\pgfqpoint{2.832189in}{1.348797in}}%
\pgfpathlineto{\pgfqpoint{2.832506in}{2.194843in}}%
\pgfpathlineto{\pgfqpoint{2.833141in}{1.280787in}}%
\pgfpathlineto{\pgfqpoint{2.833458in}{1.975259in}}%
\pgfpathlineto{\pgfqpoint{2.833776in}{2.206250in}}%
\pgfpathlineto{\pgfqpoint{2.834093in}{1.355343in}}%
\pgfpathlineto{\pgfqpoint{2.834728in}{2.318776in}}%
\pgfpathlineto{\pgfqpoint{2.835046in}{1.546406in}}%
\pgfpathlineto{\pgfqpoint{2.835363in}{1.466243in}}%
\pgfpathlineto{\pgfqpoint{2.835681in}{2.293765in}}%
\pgfpathlineto{\pgfqpoint{2.836316in}{1.314271in}}%
\pgfpathlineto{\pgfqpoint{2.836633in}{2.138151in}}%
\pgfpathlineto{\pgfqpoint{2.836950in}{2.070151in}}%
\pgfpathlineto{\pgfqpoint{2.837268in}{1.292404in}}%
\pgfpathlineto{\pgfqpoint{2.837903in}{2.266306in}}%
\pgfpathlineto{\pgfqpoint{2.838220in}{1.407978in}}%
\pgfpathlineto{\pgfqpoint{2.838855in}{2.329022in}}%
\pgfpathlineto{\pgfqpoint{2.839173in}{1.621298in}}%
\pgfpathlineto{\pgfqpoint{2.839490in}{1.404923in}}%
\pgfpathlineto{\pgfqpoint{2.839808in}{2.252942in}}%
\pgfpathlineto{\pgfqpoint{2.840443in}{1.287502in}}%
\pgfpathlineto{\pgfqpoint{2.840760in}{2.064398in}}%
\pgfpathlineto{\pgfqpoint{2.841077in}{2.130010in}}%
\pgfpathlineto{\pgfqpoint{2.841395in}{1.307576in}}%
\pgfpathlineto{\pgfqpoint{2.842030in}{2.287875in}}%
\pgfpathlineto{\pgfqpoint{2.842347in}{1.458327in}}%
\pgfpathlineto{\pgfqpoint{2.842665in}{1.549392in}}%
\pgfpathlineto{\pgfqpoint{2.842982in}{2.316017in}}%
\pgfpathlineto{\pgfqpoint{2.843617in}{1.354377in}}%
\pgfpathlineto{\pgfqpoint{2.843935in}{2.208546in}}%
\pgfpathlineto{\pgfqpoint{2.844569in}{1.280639in}}%
\pgfpathlineto{\pgfqpoint{2.844887in}{1.996702in}}%
\pgfpathlineto{\pgfqpoint{2.845204in}{2.202375in}}%
\pgfpathlineto{\pgfqpoint{2.845522in}{1.346812in}}%
\pgfpathlineto{\pgfqpoint{2.846157in}{2.317970in}}%
\pgfpathlineto{\pgfqpoint{2.846474in}{1.526181in}}%
\pgfpathlineto{\pgfqpoint{2.846792in}{1.480254in}}%
\pgfpathlineto{\pgfqpoint{2.847109in}{2.292633in}}%
\pgfpathlineto{\pgfqpoint{2.847744in}{1.312161in}}%
\pgfpathlineto{\pgfqpoint{2.848062in}{2.144924in}}%
\pgfpathlineto{\pgfqpoint{2.848379in}{2.038164in}}%
\pgfpathlineto{\pgfqpoint{2.848696in}{1.280716in}}%
\pgfpathlineto{\pgfqpoint{2.849331in}{2.238109in}}%
\pgfpathlineto{\pgfqpoint{2.849649in}{1.383813in}}%
\pgfpathlineto{\pgfqpoint{2.850284in}{2.318835in}}%
\pgfpathlineto{\pgfqpoint{2.850601in}{1.595494in}}%
\pgfpathlineto{\pgfqpoint{2.850919in}{1.415086in}}%
\pgfpathlineto{\pgfqpoint{2.851236in}{2.260961in}}%
\pgfpathlineto{\pgfqpoint{2.851871in}{1.288429in}}%
\pgfpathlineto{\pgfqpoint{2.852188in}{2.084635in}}%
\pgfpathlineto{\pgfqpoint{2.852506in}{2.119342in}}%
\pgfpathlineto{\pgfqpoint{2.852823in}{1.303557in}}%
\pgfpathlineto{\pgfqpoint{2.853458in}{2.286082in}}%
\pgfpathlineto{\pgfqpoint{2.853776in}{1.442414in}}%
\pgfpathlineto{\pgfqpoint{2.854093in}{1.566797in}}%
\pgfpathlineto{\pgfqpoint{2.854411in}{2.314776in}}%
\pgfpathlineto{\pgfqpoint{2.855046in}{1.359791in}}%
\pgfpathlineto{\pgfqpoint{2.855363in}{2.210311in}}%
\pgfpathlineto{\pgfqpoint{2.855998in}{1.272040in}}%
\pgfpathlineto{\pgfqpoint{2.856315in}{2.004097in}}%
\pgfpathlineto{\pgfqpoint{2.856633in}{2.169099in}}%
\pgfpathlineto{\pgfqpoint{2.856950in}{1.326092in}}%
\pgfpathlineto{\pgfqpoint{2.857585in}{2.300391in}}%
\pgfpathlineto{\pgfqpoint{2.857903in}{1.501128in}}%
\pgfpathlineto{\pgfqpoint{2.858220in}{1.489022in}}%
\pgfpathlineto{\pgfqpoint{2.858538in}{2.296632in}}%
\pgfpathlineto{\pgfqpoint{2.859173in}{1.320070in}}%
\pgfpathlineto{\pgfqpoint{2.859490in}{2.159758in}}%
\pgfpathlineto{\pgfqpoint{2.859807in}{2.024851in}}%
\pgfpathlineto{\pgfqpoint{2.860125in}{1.277550in}}%
\pgfpathlineto{\pgfqpoint{2.860760in}{2.233093in}}%
\pgfpathlineto{\pgfqpoint{2.861077in}{1.372473in}}%
\pgfpathlineto{\pgfqpoint{2.861712in}{2.315621in}}%
\pgfpathlineto{\pgfqpoint{2.862030in}{1.571969in}}%
\pgfpathlineto{\pgfqpoint{2.862347in}{1.423037in}}%
\pgfpathlineto{\pgfqpoint{2.862665in}{2.261741in}}%
\pgfpathlineto{\pgfqpoint{2.863299in}{1.287037in}}%
\pgfpathlineto{\pgfqpoint{2.863617in}{2.089955in}}%
\pgfpathlineto{\pgfqpoint{2.863934in}{2.086618in}}%
\pgfpathlineto{\pgfqpoint{2.864252in}{1.285513in}}%
\pgfpathlineto{\pgfqpoint{2.864887in}{2.261123in}}%
\pgfpathlineto{\pgfqpoint{2.865204in}{1.417690in}}%
\pgfpathlineto{\pgfqpoint{2.865522in}{1.577227in}}%
\pgfpathlineto{\pgfqpoint{2.865839in}{2.310920in}}%
\pgfpathlineto{\pgfqpoint{2.866474in}{1.368088in}}%
\pgfpathlineto{\pgfqpoint{2.866792in}{2.223880in}}%
\pgfpathlineto{\pgfqpoint{2.867426in}{1.274506in}}%
\pgfpathlineto{\pgfqpoint{2.867744in}{2.027593in}}%
\pgfpathlineto{\pgfqpoint{2.868061in}{2.162685in}}%
\pgfpathlineto{\pgfqpoint{2.868379in}{1.317946in}}%
\pgfpathlineto{\pgfqpoint{2.869014in}{2.295823in}}%
\pgfpathlineto{\pgfqpoint{2.869331in}{1.480217in}}%
\pgfpathlineto{\pgfqpoint{2.869649in}{1.504359in}}%
\pgfpathlineto{\pgfqpoint{2.869966in}{2.293638in}}%
\pgfpathlineto{\pgfqpoint{2.870601in}{1.319935in}}%
\pgfpathlineto{\pgfqpoint{2.870918in}{2.165738in}}%
\pgfpathlineto{\pgfqpoint{2.871553in}{1.266986in}}%
\pgfpathlineto{\pgfqpoint{2.871871in}{1.940333in}}%
\pgfpathlineto{\pgfqpoint{2.872188in}{2.203640in}}%
\pgfpathlineto{\pgfqpoint{2.872506in}{1.348895in}}%
\pgfpathlineto{\pgfqpoint{2.873141in}{2.304951in}}%
\pgfpathlineto{\pgfqpoint{2.873458in}{1.546327in}}%
\pgfpathlineto{\pgfqpoint{2.873776in}{1.435842in}}%
\pgfpathlineto{\pgfqpoint{2.874093in}{2.269378in}}%
\pgfpathlineto{\pgfqpoint{2.874728in}{1.290289in}}%
\pgfpathlineto{\pgfqpoint{2.875045in}{2.112150in}}%
\pgfpathlineto{\pgfqpoint{2.875363in}{2.073907in}}%
\pgfpathlineto{\pgfqpoint{2.875680in}{1.283001in}}%
\pgfpathlineto{\pgfqpoint{2.876315in}{2.256087in}}%
\pgfpathlineto{\pgfqpoint{2.876633in}{1.401293in}}%
\pgfpathlineto{\pgfqpoint{2.877268in}{2.306732in}}%
\pgfpathlineto{\pgfqpoint{2.877585in}{1.619168in}}%
\pgfpathlineto{\pgfqpoint{2.877903in}{1.375308in}}%
\pgfpathlineto{\pgfqpoint{2.878220in}{2.224882in}}%
\pgfpathlineto{\pgfqpoint{2.878855in}{1.266930in}}%
\pgfpathlineto{\pgfqpoint{2.879172in}{2.034021in}}%
\pgfpathlineto{\pgfqpoint{2.879490in}{2.127289in}}%
\pgfpathlineto{\pgfqpoint{2.879807in}{1.298659in}}%
\pgfpathlineto{\pgfqpoint{2.880442in}{2.277832in}}%
\pgfpathlineto{\pgfqpoint{2.880760in}{1.455456in}}%
\pgfpathlineto{\pgfqpoint{2.881077in}{1.516177in}}%
\pgfpathlineto{\pgfqpoint{2.881395in}{2.296852in}}%
\pgfpathlineto{\pgfqpoint{2.882029in}{1.330237in}}%
\pgfpathlineto{\pgfqpoint{2.882347in}{2.182202in}}%
\pgfpathlineto{\pgfqpoint{2.882982in}{1.265920in}}%
\pgfpathlineto{\pgfqpoint{2.883299in}{1.968641in}}%
\pgfpathlineto{\pgfqpoint{2.883617in}{2.194981in}}%
\pgfpathlineto{\pgfqpoint{2.883934in}{1.338009in}}%
\pgfpathlineto{\pgfqpoint{2.884569in}{2.298690in}}%
\pgfpathlineto{\pgfqpoint{2.884887in}{1.521790in}}%
\pgfpathlineto{\pgfqpoint{2.885204in}{1.445899in}}%
\pgfpathlineto{\pgfqpoint{2.885522in}{2.268637in}}%
\pgfpathlineto{\pgfqpoint{2.886156in}{1.290782in}}%
\pgfpathlineto{\pgfqpoint{2.886474in}{2.115644in}}%
\pgfpathlineto{\pgfqpoint{2.886791in}{2.038635in}}%
\pgfpathlineto{\pgfqpoint{2.887109in}{1.266187in}}%
\pgfpathlineto{\pgfqpoint{2.887744in}{2.229725in}}%
\pgfpathlineto{\pgfqpoint{2.888061in}{1.377684in}}%
\pgfpathlineto{\pgfqpoint{2.888696in}{2.302362in}}%
\pgfpathlineto{\pgfqpoint{2.889014in}{1.594709in}}%
\pgfpathlineto{\pgfqpoint{2.889331in}{1.386456in}}%
\pgfpathlineto{\pgfqpoint{2.889648in}{2.237949in}}%
\pgfpathlineto{\pgfqpoint{2.890283in}{1.272355in}}%
\pgfpathlineto{\pgfqpoint{2.890601in}{2.058862in}}%
\pgfpathlineto{\pgfqpoint{2.890918in}{2.117967in}}%
\pgfpathlineto{\pgfqpoint{2.891236in}{1.291212in}}%
\pgfpathlineto{\pgfqpoint{2.891871in}{2.269766in}}%
\pgfpathlineto{\pgfqpoint{2.892188in}{1.434412in}}%
\pgfpathlineto{\pgfqpoint{2.892506in}{1.532799in}}%
\pgfpathlineto{\pgfqpoint{2.892823in}{2.291699in}}%
\pgfpathlineto{\pgfqpoint{2.893458in}{1.332420in}}%
\pgfpathlineto{\pgfqpoint{2.893775in}{2.185814in}}%
\pgfpathlineto{\pgfqpoint{2.894410in}{1.256843in}}%
\pgfpathlineto{\pgfqpoint{2.894728in}{1.973881in}}%
\pgfpathlineto{\pgfqpoint{2.895045in}{2.164289in}}%
\pgfpathlineto{\pgfqpoint{2.895363in}{1.315639in}}%
\pgfpathlineto{\pgfqpoint{2.895998in}{2.286846in}}%
\pgfpathlineto{\pgfqpoint{2.896315in}{1.496435in}}%
\pgfpathlineto{\pgfqpoint{2.896633in}{1.461312in}}%
\pgfpathlineto{\pgfqpoint{2.896950in}{2.275285in}}%
\pgfpathlineto{\pgfqpoint{2.897585in}{1.297440in}}%
\pgfpathlineto{\pgfqpoint{2.897902in}{2.139842in}}%
\pgfpathlineto{\pgfqpoint{2.898220in}{2.023654in}}%
\pgfpathlineto{\pgfqpoint{2.898537in}{1.265456in}}%
\pgfpathlineto{\pgfqpoint{2.899172in}{2.221060in}}%
\pgfpathlineto{\pgfqpoint{2.899490in}{1.361098in}}%
\pgfpathlineto{\pgfqpoint{2.900125in}{2.295147in}}%
\pgfpathlineto{\pgfqpoint{2.900442in}{1.565548in}}%
\pgfpathlineto{\pgfqpoint{2.900759in}{1.395780in}}%
\pgfpathlineto{\pgfqpoint{2.901077in}{2.237664in}}%
\pgfpathlineto{\pgfqpoint{2.901712in}{1.266094in}}%
\pgfpathlineto{\pgfqpoint{2.902029in}{2.064365in}}%
\pgfpathlineto{\pgfqpoint{2.902347in}{2.080591in}}%
\pgfpathlineto{\pgfqpoint{2.902664in}{1.273850in}}%
\pgfpathlineto{\pgfqpoint{2.903299in}{2.250529in}}%
\pgfpathlineto{\pgfqpoint{2.903617in}{1.410109in}}%
\pgfpathlineto{\pgfqpoint{2.903934in}{1.547877in}}%
\pgfpathlineto{\pgfqpoint{2.904252in}{2.293590in}}%
\pgfpathlineto{\pgfqpoint{2.904886in}{1.345256in}}%
\pgfpathlineto{\pgfqpoint{2.905204in}{2.202778in}}%
\pgfpathlineto{\pgfqpoint{2.905839in}{1.258409in}}%
\pgfpathlineto{\pgfqpoint{2.906156in}{2.003178in}}%
\pgfpathlineto{\pgfqpoint{2.906474in}{2.151859in}}%
\pgfpathlineto{\pgfqpoint{2.906791in}{1.305639in}}%
\pgfpathlineto{\pgfqpoint{2.907426in}{2.277639in}}%
\pgfpathlineto{\pgfqpoint{2.907744in}{1.471159in}}%
\pgfpathlineto{\pgfqpoint{2.908061in}{1.473403in}}%
\pgfpathlineto{\pgfqpoint{2.908378in}{2.272626in}}%
\pgfpathlineto{\pgfqpoint{2.909013in}{1.299391in}}%
\pgfpathlineto{\pgfqpoint{2.909331in}{2.140962in}}%
\pgfpathlineto{\pgfqpoint{2.909648in}{1.986085in}}%
\pgfpathlineto{\pgfqpoint{2.909966in}{1.250483in}}%
\pgfpathlineto{\pgfqpoint{2.910601in}{2.193251in}}%
\pgfpathlineto{\pgfqpoint{2.910918in}{1.339107in}}%
\pgfpathlineto{\pgfqpoint{2.911553in}{2.289578in}}%
\pgfpathlineto{\pgfqpoint{2.911870in}{1.540475in}}%
\pgfpathlineto{\pgfqpoint{2.912188in}{1.409888in}}%
\pgfpathlineto{\pgfqpoint{2.912505in}{2.250588in}}%
\pgfpathlineto{\pgfqpoint{2.913140in}{1.275218in}}%
\pgfpathlineto{\pgfqpoint{2.913458in}{2.090318in}}%
\pgfpathlineto{\pgfqpoint{2.913775in}{2.067926in}}%
\pgfpathlineto{\pgfqpoint{2.914093in}{1.267384in}}%
\pgfpathlineto{\pgfqpoint{2.914728in}{2.238497in}}%
\pgfpathlineto{\pgfqpoint{2.915045in}{1.389292in}}%
\pgfpathlineto{\pgfqpoint{2.915680in}{2.286153in}}%
\pgfpathlineto{\pgfqpoint{2.915997in}{1.612512in}}%
\pgfpathlineto{\pgfqpoint{2.916315in}{1.350096in}}%
\pgfpathlineto{\pgfqpoint{2.916632in}{2.204210in}}%
\pgfpathlineto{\pgfqpoint{2.917267in}{1.251005in}}%
\pgfpathlineto{\pgfqpoint{2.917585in}{2.008336in}}%
\pgfpathlineto{\pgfqpoint{2.917902in}{2.119738in}}%
\pgfpathlineto{\pgfqpoint{2.918220in}{1.284898in}}%
\pgfpathlineto{\pgfqpoint{2.918855in}{2.263864in}}%
\pgfpathlineto{\pgfqpoint{2.919172in}{1.446508in}}%
\pgfpathlineto{\pgfqpoint{2.919489in}{1.491683in}}%
\pgfpathlineto{\pgfqpoint{2.919807in}{2.277803in}}%
\pgfpathlineto{\pgfqpoint{2.920442in}{1.309628in}}%
\pgfpathlineto{\pgfqpoint{2.920759in}{2.165598in}}%
\pgfpathlineto{\pgfqpoint{2.921394in}{1.252104in}}%
\pgfpathlineto{\pgfqpoint{2.921712in}{1.941690in}}%
\pgfpathlineto{\pgfqpoint{2.922029in}{2.180836in}}%
\pgfpathlineto{\pgfqpoint{2.922347in}{1.322846in}}%
\pgfpathlineto{\pgfqpoint{2.922982in}{2.279252in}}%
\pgfpathlineto{\pgfqpoint{2.923299in}{1.510970in}}%
\pgfpathlineto{\pgfqpoint{2.923616in}{1.421313in}}%
\pgfpathlineto{\pgfqpoint{2.923934in}{2.247834in}}%
\pgfpathlineto{\pgfqpoint{2.924569in}{1.270364in}}%
\pgfpathlineto{\pgfqpoint{2.924886in}{2.094649in}}%
\pgfpathlineto{\pgfqpoint{2.925204in}{2.028941in}}%
\pgfpathlineto{\pgfqpoint{2.925521in}{1.252570in}}%
\pgfpathlineto{\pgfqpoint{2.926156in}{2.217917in}}%
\pgfpathlineto{\pgfqpoint{2.926474in}{1.365897in}}%
\pgfpathlineto{\pgfqpoint{2.927108in}{2.286058in}}%
\pgfpathlineto{\pgfqpoint{2.927426in}{1.585500in}}%
\pgfpathlineto{\pgfqpoint{2.927743in}{1.366157in}}%
\pgfpathlineto{\pgfqpoint{2.928061in}{2.221772in}}%
\pgfpathlineto{\pgfqpoint{2.928696in}{1.255825in}}%
\pgfpathlineto{\pgfqpoint{2.929013in}{2.038011in}}%
\pgfpathlineto{\pgfqpoint{2.929331in}{2.103000in}}%
\pgfpathlineto{\pgfqpoint{2.929648in}{1.276002in}}%
\pgfpathlineto{\pgfqpoint{2.930283in}{2.251221in}}%
\pgfpathlineto{\pgfqpoint{2.930600in}{1.420885in}}%
\pgfpathlineto{\pgfqpoint{2.930918in}{1.505969in}}%
\pgfpathlineto{\pgfqpoint{2.931235in}{2.272956in}}%
\pgfpathlineto{\pgfqpoint{2.931870in}{1.313378in}}%
\pgfpathlineto{\pgfqpoint{2.932188in}{2.164905in}}%
\pgfpathlineto{\pgfqpoint{2.932823in}{1.239188in}}%
\pgfpathlineto{\pgfqpoint{2.933140in}{1.952045in}}%
\pgfpathlineto{\pgfqpoint{2.933458in}{2.151297in}}%
\pgfpathlineto{\pgfqpoint{2.933775in}{1.302839in}}%
\pgfpathlineto{\pgfqpoint{2.934410in}{2.271794in}}%
\pgfpathlineto{\pgfqpoint{2.934727in}{1.485750in}}%
\pgfpathlineto{\pgfqpoint{2.935045in}{1.438609in}}%
\pgfpathlineto{\pgfqpoint{2.935362in}{2.259998in}}%
\pgfpathlineto{\pgfqpoint{2.935997in}{1.283070in}}%
\pgfpathlineto{\pgfqpoint{2.936315in}{2.120546in}}%
\pgfpathlineto{\pgfqpoint{2.936632in}{2.012836in}}%
\pgfpathlineto{\pgfqpoint{2.936950in}{1.247762in}}%
\pgfpathlineto{\pgfqpoint{2.937585in}{2.201802in}}%
\pgfpathlineto{\pgfqpoint{2.937902in}{1.345857in}}%
\pgfpathlineto{\pgfqpoint{2.938537in}{2.276147in}}%
\pgfpathlineto{\pgfqpoint{2.938854in}{1.553911in}}%
\pgfpathlineto{\pgfqpoint{2.939172in}{1.373353in}}%
\pgfpathlineto{\pgfqpoint{2.939489in}{2.220188in}}%
\pgfpathlineto{\pgfqpoint{2.940124in}{1.250351in}}%
\pgfpathlineto{\pgfqpoint{2.940442in}{2.043117in}}%
\pgfpathlineto{\pgfqpoint{2.940759in}{2.069765in}}%
\pgfpathlineto{\pgfqpoint{2.941077in}{1.257598in}}%
\pgfpathlineto{\pgfqpoint{2.941712in}{2.235350in}}%
\pgfpathlineto{\pgfqpoint{2.942029in}{1.397366in}}%
\pgfpathlineto{\pgfqpoint{2.942346in}{1.527045in}}%
\pgfpathlineto{\pgfqpoint{2.942664in}{2.276624in}}%
\pgfpathlineto{\pgfqpoint{2.943299in}{1.327829in}}%
\pgfpathlineto{\pgfqpoint{2.943616in}{2.189379in}}%
\pgfpathlineto{\pgfqpoint{2.944251in}{1.243603in}}%
\pgfpathlineto{\pgfqpoint{2.944569in}{1.979655in}}%
\pgfpathlineto{\pgfqpoint{2.944886in}{2.134573in}}%
\pgfpathlineto{\pgfqpoint{2.945204in}{1.287231in}}%
\pgfpathlineto{\pgfqpoint{2.945838in}{2.257940in}}%
\pgfpathlineto{\pgfqpoint{2.946156in}{1.456398in}}%
\pgfpathlineto{\pgfqpoint{2.946473in}{1.452320in}}%
\pgfpathlineto{\pgfqpoint{2.946791in}{2.254406in}}%
\pgfpathlineto{\pgfqpoint{2.947426in}{1.280289in}}%
\pgfpathlineto{\pgfqpoint{2.947743in}{2.123837in}}%
\pgfpathlineto{\pgfqpoint{2.948061in}{1.972236in}}%
\pgfpathlineto{\pgfqpoint{2.948378in}{1.235692in}}%
\pgfpathlineto{\pgfqpoint{2.949013in}{2.179483in}}%
\pgfpathlineto{\pgfqpoint{2.949330in}{1.323732in}}%
\pgfpathlineto{\pgfqpoint{2.949965in}{2.273365in}}%
\pgfpathlineto{\pgfqpoint{2.950283in}{1.527496in}}%
\pgfpathlineto{\pgfqpoint{2.950600in}{1.393244in}}%
\pgfpathlineto{\pgfqpoint{2.950918in}{2.237118in}}%
\pgfpathlineto{\pgfqpoint{2.951553in}{1.258512in}}%
\pgfpathlineto{\pgfqpoint{2.951870in}{2.072306in}}%
\pgfpathlineto{\pgfqpoint{2.952188in}{2.048967in}}%
\pgfpathlineto{\pgfqpoint{2.952505in}{1.250468in}}%
\pgfpathlineto{\pgfqpoint{2.953140in}{2.219169in}}%
\pgfpathlineto{\pgfqpoint{2.953457in}{1.371996in}}%
\pgfpathlineto{\pgfqpoint{2.954092in}{2.268663in}}%
\pgfpathlineto{\pgfqpoint{2.954410in}{1.597722in}}%
\pgfpathlineto{\pgfqpoint{2.954727in}{1.333299in}}%
\pgfpathlineto{\pgfqpoint{2.955045in}{2.186675in}}%
\pgfpathlineto{\pgfqpoint{2.955680in}{1.233210in}}%
\pgfpathlineto{\pgfqpoint{2.955997in}{1.991127in}}%
\pgfpathlineto{\pgfqpoint{2.956315in}{2.103524in}}%
\pgfpathlineto{\pgfqpoint{2.956632in}{1.269833in}}%
\pgfpathlineto{\pgfqpoint{2.957267in}{2.248240in}}%
\pgfpathlineto{\pgfqpoint{2.957584in}{1.431367in}}%
\pgfpathlineto{\pgfqpoint{2.957902in}{1.473007in}}%
\pgfpathlineto{\pgfqpoint{2.958219in}{2.265406in}}%
\pgfpathlineto{\pgfqpoint{2.958854in}{1.296820in}}%
\pgfpathlineto{\pgfqpoint{2.959172in}{2.148940in}}%
\pgfpathlineto{\pgfqpoint{2.959807in}{1.233103in}}%
\pgfpathlineto{\pgfqpoint{2.960124in}{1.920880in}}%
\pgfpathlineto{\pgfqpoint{2.960442in}{2.158859in}}%
\pgfpathlineto{\pgfqpoint{2.960759in}{1.304889in}}%
\pgfpathlineto{\pgfqpoint{2.961394in}{2.260656in}}%
\pgfpathlineto{\pgfqpoint{2.961711in}{1.494805in}}%
\pgfpathlineto{\pgfqpoint{2.962029in}{1.402544in}}%
\pgfpathlineto{\pgfqpoint{2.962346in}{2.232633in}}%
\pgfpathlineto{\pgfqpoint{2.962981in}{1.255551in}}%
\pgfpathlineto{\pgfqpoint{2.963299in}{2.077276in}}%
\pgfpathlineto{\pgfqpoint{2.963616in}{2.014223in}}%
\pgfpathlineto{\pgfqpoint{2.963934in}{1.234721in}}%
\pgfpathlineto{\pgfqpoint{2.964568in}{2.200680in}}%
\pgfpathlineto{\pgfqpoint{2.964886in}{1.349931in}}%
\pgfpathlineto{\pgfqpoint{2.965521in}{2.271071in}}%
\pgfpathlineto{\pgfqpoint{2.965838in}{1.573828in}}%
\pgfpathlineto{\pgfqpoint{2.966156in}{1.352277in}}%
\pgfpathlineto{\pgfqpoint{2.966473in}{2.209702in}}%
\pgfpathlineto{\pgfqpoint{2.967108in}{1.240633in}}%
\pgfpathlineto{\pgfqpoint{2.967426in}{2.017833in}}%
\pgfpathlineto{\pgfqpoint{2.967743in}{2.082662in}}%
\pgfpathlineto{\pgfqpoint{2.968060in}{1.255570in}}%
\pgfpathlineto{\pgfqpoint{2.968695in}{2.230677in}}%
\pgfpathlineto{\pgfqpoint{2.969013in}{1.402784in}}%
\pgfpathlineto{\pgfqpoint{2.969330in}{1.488943in}}%
\pgfpathlineto{\pgfqpoint{2.969648in}{2.256278in}}%
\pgfpathlineto{\pgfqpoint{2.970283in}{1.296534in}}%
\pgfpathlineto{\pgfqpoint{2.970600in}{2.151051in}}%
\pgfpathlineto{\pgfqpoint{2.971235in}{1.224155in}}%
\pgfpathlineto{\pgfqpoint{2.971553in}{1.935385in}}%
\pgfpathlineto{\pgfqpoint{2.971870in}{2.134771in}}%
\pgfpathlineto{\pgfqpoint{2.972187in}{1.284627in}}%
\pgfpathlineto{\pgfqpoint{2.972822in}{2.254721in}}%
\pgfpathlineto{\pgfqpoint{2.973140in}{1.469282in}}%
\pgfpathlineto{\pgfqpoint{2.973457in}{1.426206in}}%
\pgfpathlineto{\pgfqpoint{2.973775in}{2.248186in}}%
\pgfpathlineto{\pgfqpoint{2.974410in}{1.267333in}}%
\pgfpathlineto{\pgfqpoint{2.974727in}{2.105291in}}%
\pgfpathlineto{\pgfqpoint{2.975045in}{1.989647in}}%
\pgfpathlineto{\pgfqpoint{2.975362in}{1.229923in}}%
\pgfpathlineto{\pgfqpoint{2.975997in}{2.180612in}}%
\pgfpathlineto{\pgfqpoint{2.976314in}{1.325354in}}%
\pgfpathlineto{\pgfqpoint{2.976949in}{2.258764in}}%
\pgfpathlineto{\pgfqpoint{2.977267in}{1.534928in}}%
\pgfpathlineto{\pgfqpoint{2.977584in}{1.359544in}}%
\pgfpathlineto{\pgfqpoint{2.977902in}{2.205127in}}%
\pgfpathlineto{\pgfqpoint{2.978537in}{1.233359in}}%
\pgfpathlineto{\pgfqpoint{2.978854in}{2.030022in}}%
\pgfpathlineto{\pgfqpoint{2.979172in}{2.049721in}}%
\pgfpathlineto{\pgfqpoint{2.979489in}{1.241132in}}%
\pgfpathlineto{\pgfqpoint{2.980124in}{2.218183in}}%
\pgfpathlineto{\pgfqpoint{2.980441in}{1.378319in}}%
\pgfpathlineto{\pgfqpoint{2.980759in}{1.513028in}}%
\pgfpathlineto{\pgfqpoint{2.981076in}{2.266083in}}%
\pgfpathlineto{\pgfqpoint{2.981711in}{1.317008in}}%
\pgfpathlineto{\pgfqpoint{2.982029in}{2.174345in}}%
\pgfpathlineto{\pgfqpoint{2.982664in}{1.224249in}}%
\pgfpathlineto{\pgfqpoint{2.982981in}{1.963037in}}%
\pgfpathlineto{\pgfqpoint{2.983298in}{2.109800in}}%
\pgfpathlineto{\pgfqpoint{2.983616in}{1.267643in}}%
\pgfpathlineto{\pgfqpoint{2.984251in}{2.238914in}}%
\pgfpathlineto{\pgfqpoint{2.984568in}{1.436149in}}%
\pgfpathlineto{\pgfqpoint{2.984886in}{1.437821in}}%
\pgfpathlineto{\pgfqpoint{2.985203in}{2.240306in}}%
\pgfpathlineto{\pgfqpoint{2.985838in}{1.267322in}}%
\pgfpathlineto{\pgfqpoint{2.986156in}{2.109834in}}%
\pgfpathlineto{\pgfqpoint{2.986473in}{1.953118in}}%
\pgfpathlineto{\pgfqpoint{2.986791in}{1.217278in}}%
\pgfpathlineto{\pgfqpoint{2.987425in}{2.159317in}}%
\pgfpathlineto{\pgfqpoint{2.987743in}{1.305264in}}%
\pgfpathlineto{\pgfqpoint{2.988378in}{2.259308in}}%
\pgfpathlineto{\pgfqpoint{2.988695in}{1.510924in}}%
\pgfpathlineto{\pgfqpoint{2.989013in}{1.382766in}}%
\pgfpathlineto{\pgfqpoint{2.989330in}{2.225886in}}%
\pgfpathlineto{\pgfqpoint{2.989965in}{1.244139in}}%
\pgfpathlineto{\pgfqpoint{2.990283in}{2.055549in}}%
\pgfpathlineto{\pgfqpoint{2.990600in}{2.025129in}}%
\pgfpathlineto{\pgfqpoint{2.990917in}{1.228907in}}%
\pgfpathlineto{\pgfqpoint{2.991552in}{2.196625in}}%
\pgfpathlineto{\pgfqpoint{2.991870in}{1.351088in}}%
\pgfpathlineto{\pgfqpoint{2.992505in}{2.252476in}}%
\pgfpathlineto{\pgfqpoint{2.992822in}{1.576788in}}%
\pgfpathlineto{\pgfqpoint{2.993140in}{1.319600in}}%
\pgfpathlineto{\pgfqpoint{2.993457in}{2.175161in}}%
\pgfpathlineto{\pgfqpoint{2.994092in}{1.218885in}}%
\pgfpathlineto{\pgfqpoint{2.994409in}{1.978545in}}%
\pgfpathlineto{\pgfqpoint{2.994727in}{2.083462in}}%
\pgfpathlineto{\pgfqpoint{2.995044in}{1.249703in}}%
\pgfpathlineto{\pgfqpoint{2.995679in}{2.229472in}}%
\pgfpathlineto{\pgfqpoint{2.995997in}{1.411541in}}%
\pgfpathlineto{\pgfqpoint{2.996314in}{1.464809in}}%
\pgfpathlineto{\pgfqpoint{2.996632in}{2.254779in}}%
\pgfpathlineto{\pgfqpoint{2.997267in}{1.282984in}}%
\pgfpathlineto{\pgfqpoint{2.997584in}{2.135685in}}%
\pgfpathlineto{\pgfqpoint{2.998219in}{1.215219in}}%
\pgfpathlineto{\pgfqpoint{2.998536in}{1.905310in}}%
\pgfpathlineto{\pgfqpoint{2.998854in}{2.135288in}}%
\pgfpathlineto{\pgfqpoint{2.999171in}{1.282170in}}%
\pgfpathlineto{\pgfqpoint{2.999806in}{2.242305in}}%
\pgfpathlineto{\pgfqpoint{3.000124in}{1.472053in}}%
\pgfpathlineto{\pgfqpoint{3.000441in}{1.392308in}}%
\pgfpathlineto{\pgfqpoint{3.000759in}{2.218953in}}%
\pgfpathlineto{\pgfqpoint{3.001394in}{1.240434in}}%
\pgfpathlineto{\pgfqpoint{3.001711in}{2.067680in}}%
\pgfpathlineto{\pgfqpoint{3.002028in}{1.989808in}}%
\pgfpathlineto{\pgfqpoint{3.002346in}{1.217816in}}%
\pgfpathlineto{\pgfqpoint{3.002981in}{2.180986in}}%
\pgfpathlineto{\pgfqpoint{3.003298in}{1.327733in}}%
\pgfpathlineto{\pgfqpoint{3.003933in}{2.259886in}}%
\pgfpathlineto{\pgfqpoint{3.004251in}{1.552103in}}%
\pgfpathlineto{\pgfqpoint{3.004568in}{1.343804in}}%
\pgfpathlineto{\pgfqpoint{3.004886in}{2.196122in}}%
\pgfpathlineto{\pgfqpoint{3.005521in}{1.222399in}}%
\pgfpathlineto{\pgfqpoint{3.005838in}{2.005544in}}%
\pgfpathlineto{\pgfqpoint{3.006155in}{2.054714in}}%
\pgfpathlineto{\pgfqpoint{3.006473in}{1.235211in}}%
\pgfpathlineto{\pgfqpoint{3.007108in}{2.210054in}}%
\pgfpathlineto{\pgfqpoint{3.007425in}{1.378997in}}%
\pgfpathlineto{\pgfqpoint{3.007743in}{1.479226in}}%
\pgfpathlineto{\pgfqpoint{3.008060in}{2.242219in}}%
\pgfpathlineto{\pgfqpoint{3.008695in}{1.286292in}}%
\pgfpathlineto{\pgfqpoint{3.009013in}{2.139651in}}%
\pgfpathlineto{\pgfqpoint{3.009647in}{1.206301in}}%
\pgfpathlineto{\pgfqpoint{3.009965in}{1.927887in}}%
\pgfpathlineto{\pgfqpoint{3.010282in}{2.110856in}}%
\pgfpathlineto{\pgfqpoint{3.010600in}{1.264539in}}%
\pgfpathlineto{\pgfqpoint{3.011235in}{2.239959in}}%
\pgfpathlineto{\pgfqpoint{3.011552in}{1.447903in}}%
\pgfpathlineto{\pgfqpoint{3.011870in}{1.419538in}}%
\pgfpathlineto{\pgfqpoint{3.012187in}{2.237776in}}%
\pgfpathlineto{\pgfqpoint{3.012822in}{1.254721in}}%
\pgfpathlineto{\pgfqpoint{3.013139in}{2.091188in}}%
\pgfpathlineto{\pgfqpoint{3.013457in}{1.961403in}}%
\pgfpathlineto{\pgfqpoint{3.013774in}{1.208160in}}%
\pgfpathlineto{\pgfqpoint{3.014409in}{2.155275in}}%
\pgfpathlineto{\pgfqpoint{3.014727in}{1.302494in}}%
\pgfpathlineto{\pgfqpoint{3.015362in}{2.241931in}}%
\pgfpathlineto{\pgfqpoint{3.015679in}{1.509507in}}%
\pgfpathlineto{\pgfqpoint{3.015997in}{1.349761in}}%
\pgfpathlineto{\pgfqpoint{3.016314in}{2.194824in}}%
\pgfpathlineto{\pgfqpoint{3.016949in}{1.220769in}}%
\pgfpathlineto{\pgfqpoint{3.017266in}{2.021013in}}%
\pgfpathlineto{\pgfqpoint{3.017584in}{2.025381in}}%
\pgfpathlineto{\pgfqpoint{3.017901in}{1.220143in}}%
\pgfpathlineto{\pgfqpoint{3.018536in}{2.196598in}}%
\pgfpathlineto{\pgfqpoint{3.018854in}{1.355530in}}%
\pgfpathlineto{\pgfqpoint{3.019171in}{1.509053in}}%
\pgfpathlineto{\pgfqpoint{3.019489in}{2.254584in}}%
\pgfpathlineto{\pgfqpoint{3.020124in}{1.306066in}}%
\pgfpathlineto{\pgfqpoint{3.020441in}{2.162986in}}%
\pgfpathlineto{\pgfqpoint{3.021076in}{1.207753in}}%
\pgfpathlineto{\pgfqpoint{3.021393in}{1.952067in}}%
\pgfpathlineto{\pgfqpoint{3.021711in}{2.083182in}}%
\pgfpathlineto{\pgfqpoint{3.022028in}{1.243532in}}%
\pgfpathlineto{\pgfqpoint{3.022663in}{2.218419in}}%
\pgfpathlineto{\pgfqpoint{3.022981in}{1.410226in}}%
\pgfpathlineto{\pgfqpoint{3.023298in}{1.431795in}}%
\pgfpathlineto{\pgfqpoint{3.023616in}{2.227109in}}%
\pgfpathlineto{\pgfqpoint{3.024251in}{1.255197in}}%
\pgfpathlineto{\pgfqpoint{3.024568in}{2.102922in}}%
\pgfpathlineto{\pgfqpoint{3.025203in}{1.201004in}}%
\pgfpathlineto{\pgfqpoint{3.025520in}{1.874660in}}%
\pgfpathlineto{\pgfqpoint{3.025838in}{2.136147in}}%
\pgfpathlineto{\pgfqpoint{3.026155in}{1.280868in}}%
\pgfpathlineto{\pgfqpoint{3.026790in}{2.245994in}}%
\pgfpathlineto{\pgfqpoint{3.027108in}{1.484858in}}%
\pgfpathlineto{\pgfqpoint{3.027425in}{1.377721in}}%
\pgfpathlineto{\pgfqpoint{3.027743in}{2.213786in}}%
\pgfpathlineto{\pgfqpoint{3.028377in}{1.227849in}}%
\pgfpathlineto{\pgfqpoint{3.028695in}{2.046203in}}%
\pgfpathlineto{\pgfqpoint{3.029012in}{1.992642in}}%
\pgfpathlineto{\pgfqpoint{3.029330in}{1.208540in}}%
\pgfpathlineto{\pgfqpoint{3.029965in}{2.173370in}}%
\pgfpathlineto{\pgfqpoint{3.030282in}{1.324598in}}%
\pgfpathlineto{\pgfqpoint{3.030917in}{2.237312in}}%
\pgfpathlineto{\pgfqpoint{3.031235in}{1.547330in}}%
\pgfpathlineto{\pgfqpoint{3.031552in}{1.312894in}}%
\pgfpathlineto{\pgfqpoint{3.031869in}{2.165400in}}%
\pgfpathlineto{\pgfqpoint{3.032504in}{1.202790in}}%
\pgfpathlineto{\pgfqpoint{3.032822in}{1.974902in}}%
\pgfpathlineto{\pgfqpoint{3.033139in}{2.055050in}}%
\pgfpathlineto{\pgfqpoint{3.033457in}{1.229020in}}%
\pgfpathlineto{\pgfqpoint{3.034092in}{2.212404in}}%
\pgfpathlineto{\pgfqpoint{3.034409in}{1.386258in}}%
\pgfpathlineto{\pgfqpoint{3.034727in}{1.462831in}}%
\pgfpathlineto{\pgfqpoint{3.035044in}{2.243157in}}%
\pgfpathlineto{\pgfqpoint{3.035679in}{1.273308in}}%
\pgfpathlineto{\pgfqpoint{3.035996in}{2.124177in}}%
\pgfpathlineto{\pgfqpoint{3.036631in}{1.194825in}}%
\pgfpathlineto{\pgfqpoint{3.036949in}{1.900471in}}%
\pgfpathlineto{\pgfqpoint{3.037266in}{2.106382in}}%
\pgfpathlineto{\pgfqpoint{3.037584in}{1.258142in}}%
\pgfpathlineto{\pgfqpoint{3.038219in}{2.223679in}}%
\pgfpathlineto{\pgfqpoint{3.038536in}{1.442718in}}%
\pgfpathlineto{\pgfqpoint{3.038854in}{1.387293in}}%
\pgfpathlineto{\pgfqpoint{3.039171in}{2.208842in}}%
\pgfpathlineto{\pgfqpoint{3.039806in}{1.230652in}}%
\pgfpathlineto{\pgfqpoint{3.040123in}{2.061540in}}%
\pgfpathlineto{\pgfqpoint{3.040441in}{1.960566in}}%
\pgfpathlineto{\pgfqpoint{3.040758in}{1.197189in}}%
\pgfpathlineto{\pgfqpoint{3.041393in}{2.155464in}}%
\pgfpathlineto{\pgfqpoint{3.041711in}{1.302709in}}%
\pgfpathlineto{\pgfqpoint{3.042346in}{2.246669in}}%
\pgfpathlineto{\pgfqpoint{3.042663in}{1.521447in}}%
\pgfpathlineto{\pgfqpoint{3.042981in}{1.337101in}}%
\pgfpathlineto{\pgfqpoint{3.043298in}{2.186124in}}%
\pgfpathlineto{\pgfqpoint{3.043933in}{1.207788in}}%
\pgfpathlineto{\pgfqpoint{3.044250in}{1.997617in}}%
\pgfpathlineto{\pgfqpoint{3.044568in}{2.023437in}}%
\pgfpathlineto{\pgfqpoint{3.044885in}{1.210681in}}%
\pgfpathlineto{\pgfqpoint{3.045520in}{2.186358in}}%
\pgfpathlineto{\pgfqpoint{3.045838in}{1.350762in}}%
\pgfpathlineto{\pgfqpoint{3.046155in}{1.477966in}}%
\pgfpathlineto{\pgfqpoint{3.046473in}{2.228423in}}%
\pgfpathlineto{\pgfqpoint{3.047107in}{1.278187in}}%
\pgfpathlineto{\pgfqpoint{3.047425in}{2.134362in}}%
\pgfpathlineto{\pgfqpoint{3.048060in}{1.191798in}}%
\pgfpathlineto{\pgfqpoint{3.048377in}{1.925672in}}%
\pgfpathlineto{\pgfqpoint{3.048695in}{2.083304in}}%
\pgfpathlineto{\pgfqpoint{3.049012in}{1.239074in}}%
\pgfpathlineto{\pgfqpoint{3.049647in}{2.223533in}}%
\pgfpathlineto{\pgfqpoint{3.049965in}{1.418596in}}%
\pgfpathlineto{\pgfqpoint{3.050282in}{1.418926in}}%
\pgfpathlineto{\pgfqpoint{3.050599in}{2.225224in}}%
\pgfpathlineto{\pgfqpoint{3.051234in}{1.241949in}}%
\pgfpathlineto{\pgfqpoint{3.051552in}{2.084678in}}%
\pgfpathlineto{\pgfqpoint{3.051869in}{1.924477in}}%
\pgfpathlineto{\pgfqpoint{3.052187in}{1.189290in}}%
\pgfpathlineto{\pgfqpoint{3.052822in}{2.128376in}}%
\pgfpathlineto{\pgfqpoint{3.053139in}{1.274165in}}%
\pgfpathlineto{\pgfqpoint{3.053774in}{2.224386in}}%
\pgfpathlineto{\pgfqpoint{3.054092in}{1.476537in}}%
\pgfpathlineto{\pgfqpoint{3.054409in}{1.347417in}}%
\pgfpathlineto{\pgfqpoint{3.054726in}{2.185679in}}%
\pgfpathlineto{\pgfqpoint{3.055361in}{1.207692in}}%
\pgfpathlineto{\pgfqpoint{3.055679in}{2.020407in}}%
\pgfpathlineto{\pgfqpoint{3.055996in}{1.991869in}}%
\pgfpathlineto{\pgfqpoint{3.056314in}{1.200041in}}%
\pgfpathlineto{\pgfqpoint{3.056949in}{2.175635in}}%
\pgfpathlineto{\pgfqpoint{3.057266in}{1.327304in}}%
\pgfpathlineto{\pgfqpoint{3.057901in}{2.240887in}}%
\pgfpathlineto{\pgfqpoint{3.058218in}{1.556292in}}%
\pgfpathlineto{\pgfqpoint{3.058536in}{1.300562in}}%
\pgfpathlineto{\pgfqpoint{3.058853in}{2.153144in}}%
\pgfpathlineto{\pgfqpoint{3.059488in}{1.189381in}}%
\pgfpathlineto{\pgfqpoint{3.059806in}{1.950719in}}%
\pgfpathlineto{\pgfqpoint{3.060123in}{2.049355in}}%
\pgfpathlineto{\pgfqpoint{3.060441in}{1.219445in}}%
\pgfpathlineto{\pgfqpoint{3.061076in}{2.196914in}}%
\pgfpathlineto{\pgfqpoint{3.061393in}{1.377843in}}%
\pgfpathlineto{\pgfqpoint{3.061711in}{1.432279in}}%
\pgfpathlineto{\pgfqpoint{3.062028in}{2.215975in}}%
\pgfpathlineto{\pgfqpoint{3.062663in}{1.249214in}}%
\pgfpathlineto{\pgfqpoint{3.062980in}{2.098700in}}%
\pgfpathlineto{\pgfqpoint{3.063615in}{1.182067in}}%
\pgfpathlineto{\pgfqpoint{3.063933in}{1.878298in}}%
\pgfpathlineto{\pgfqpoint{3.064250in}{2.105721in}}%
\pgfpathlineto{\pgfqpoint{3.064568in}{1.254692in}}%
\pgfpathlineto{\pgfqpoint{3.065203in}{2.230045in}}%
\pgfpathlineto{\pgfqpoint{3.065520in}{1.450428in}}%
\pgfpathlineto{\pgfqpoint{3.065837in}{1.376286in}}%
\pgfpathlineto{\pgfqpoint{3.066155in}{2.203249in}}%
\pgfpathlineto{\pgfqpoint{3.066790in}{1.216857in}}%
\pgfpathlineto{\pgfqpoint{3.067107in}{2.041501in}}%
\pgfpathlineto{\pgfqpoint{3.067425in}{1.956682in}}%
\pgfpathlineto{\pgfqpoint{3.067742in}{1.185280in}}%
\pgfpathlineto{\pgfqpoint{3.068377in}{2.145306in}}%
\pgfpathlineto{\pgfqpoint{3.068695in}{1.294890in}}%
\pgfpathlineto{\pgfqpoint{3.069329in}{2.221484in}}%
\pgfpathlineto{\pgfqpoint{3.069647in}{1.510819in}}%
\pgfpathlineto{\pgfqpoint{3.069964in}{1.309766in}}%
\pgfpathlineto{\pgfqpoint{3.070282in}{2.160502in}}%
\pgfpathlineto{\pgfqpoint{3.070917in}{1.191260in}}%
\pgfpathlineto{\pgfqpoint{3.071234in}{1.975791in}}%
\pgfpathlineto{\pgfqpoint{3.071552in}{2.022319in}}%
\pgfpathlineto{\pgfqpoint{3.071869in}{1.203796in}}%
\pgfpathlineto{\pgfqpoint{3.072504in}{2.191227in}}%
\pgfpathlineto{\pgfqpoint{3.072822in}{1.354642in}}%
\pgfpathlineto{\pgfqpoint{3.073139in}{1.467208in}}%
\pgfpathlineto{\pgfqpoint{3.073456in}{2.229286in}}%
\pgfpathlineto{\pgfqpoint{3.074091in}{1.265513in}}%
\pgfpathlineto{\pgfqpoint{3.074409in}{2.119221in}}%
\pgfpathlineto{\pgfqpoint{3.075044in}{1.178196in}}%
\pgfpathlineto{\pgfqpoint{3.075361in}{1.901747in}}%
\pgfpathlineto{\pgfqpoint{3.075679in}{2.074629in}}%
\pgfpathlineto{\pgfqpoint{3.075996in}{1.229281in}}%
\pgfpathlineto{\pgfqpoint{3.076631in}{2.202636in}}%
\pgfpathlineto{\pgfqpoint{3.076948in}{1.407139in}}%
\pgfpathlineto{\pgfqpoint{3.077266in}{1.390083in}}%
\pgfpathlineto{\pgfqpoint{3.077583in}{2.199135in}}%
\pgfpathlineto{\pgfqpoint{3.078218in}{1.221822in}}%
\pgfpathlineto{\pgfqpoint{3.078536in}{2.062903in}}%
\pgfpathlineto{\pgfqpoint{3.078853in}{1.921515in}}%
\pgfpathlineto{\pgfqpoint{3.079171in}{1.178941in}}%
\pgfpathlineto{\pgfqpoint{3.079806in}{2.129405in}}%
\pgfpathlineto{\pgfqpoint{3.080123in}{1.272762in}}%
\pgfpathlineto{\pgfqpoint{3.080758in}{2.229807in}}%
\pgfpathlineto{\pgfqpoint{3.081075in}{1.481621in}}%
\pgfpathlineto{\pgfqpoint{3.081393in}{1.336706in}}%
\pgfpathlineto{\pgfqpoint{3.081710in}{2.176424in}}%
\pgfpathlineto{\pgfqpoint{3.082345in}{1.193428in}}%
\pgfpathlineto{\pgfqpoint{3.082663in}{1.999691in}}%
\pgfpathlineto{\pgfqpoint{3.082980in}{1.984463in}}%
\pgfpathlineto{\pgfqpoint{3.083298in}{1.188032in}}%
\pgfpathlineto{\pgfqpoint{3.083933in}{2.160494in}}%
\pgfpathlineto{\pgfqpoint{3.084250in}{1.316178in}}%
\pgfpathlineto{\pgfqpoint{3.084885in}{2.214668in}}%
\pgfpathlineto{\pgfqpoint{3.085202in}{1.544813in}}%
\pgfpathlineto{\pgfqpoint{3.085520in}{1.276942in}}%
\pgfpathlineto{\pgfqpoint{3.085837in}{2.130926in}}%
\pgfpathlineto{\pgfqpoint{3.086472in}{1.175974in}}%
\pgfpathlineto{\pgfqpoint{3.086790in}{1.932733in}}%
\pgfpathlineto{\pgfqpoint{3.087107in}{2.047203in}}%
\pgfpathlineto{\pgfqpoint{3.087425in}{1.213121in}}%
\pgfpathlineto{\pgfqpoint{3.088059in}{2.203288in}}%
\pgfpathlineto{\pgfqpoint{3.088377in}{1.381408in}}%
\pgfpathlineto{\pgfqpoint{3.088694in}{1.423569in}}%
\pgfpathlineto{\pgfqpoint{3.089012in}{2.213221in}}%
\pgfpathlineto{\pgfqpoint{3.089647in}{1.235919in}}%
\pgfpathlineto{\pgfqpoint{3.089964in}{2.081697in}}%
\pgfpathlineto{\pgfqpoint{3.090599in}{1.168363in}}%
\pgfpathlineto{\pgfqpoint{3.090917in}{1.855527in}}%
\pgfpathlineto{\pgfqpoint{3.091234in}{2.094945in}}%
\pgfpathlineto{\pgfqpoint{3.091552in}{1.244315in}}%
\pgfpathlineto{\pgfqpoint{3.092186in}{2.205429in}}%
\pgfpathlineto{\pgfqpoint{3.092504in}{1.436617in}}%
\pgfpathlineto{\pgfqpoint{3.092821in}{1.350239in}}%
\pgfpathlineto{\pgfqpoint{3.093139in}{2.179833in}}%
\pgfpathlineto{\pgfqpoint{3.093774in}{1.200325in}}%
\pgfpathlineto{\pgfqpoint{3.094091in}{2.023453in}}%
\pgfpathlineto{\pgfqpoint{3.094409in}{1.953308in}}%
\pgfpathlineto{\pgfqpoint{3.094726in}{1.176513in}}%
\pgfpathlineto{\pgfqpoint{3.095361in}{2.148816in}}%
\pgfpathlineto{\pgfqpoint{3.095678in}{1.294698in}}%
\pgfpathlineto{\pgfqpoint{3.096313in}{2.224508in}}%
\pgfpathlineto{\pgfqpoint{3.096631in}{1.512209in}}%
\pgfpathlineto{\pgfqpoint{3.096948in}{1.298742in}}%
\pgfpathlineto{\pgfqpoint{3.097266in}{2.148252in}}%
\pgfpathlineto{\pgfqpoint{3.097901in}{1.176796in}}%
\pgfpathlineto{\pgfqpoint{3.098218in}{1.955405in}}%
\pgfpathlineto{\pgfqpoint{3.098536in}{2.012040in}}%
\pgfpathlineto{\pgfqpoint{3.098853in}{1.191578in}}%
\pgfpathlineto{\pgfqpoint{3.099488in}{2.170711in}}%
\pgfpathlineto{\pgfqpoint{3.099805in}{1.340507in}}%
\pgfpathlineto{\pgfqpoint{3.100123in}{1.440481in}}%
\pgfpathlineto{\pgfqpoint{3.100440in}{2.204108in}}%
\pgfpathlineto{\pgfqpoint{3.101075in}{1.245816in}}%
\pgfpathlineto{\pgfqpoint{3.101393in}{2.100751in}}%
\pgfpathlineto{\pgfqpoint{3.102028in}{1.167030in}}%
\pgfpathlineto{\pgfqpoint{3.102345in}{1.887126in}}%
\pgfpathlineto{\pgfqpoint{3.102663in}{2.073147in}}%
\pgfpathlineto{\pgfqpoint{3.102980in}{1.224209in}}%
\pgfpathlineto{\pgfqpoint{3.103615in}{2.208518in}}%
\pgfpathlineto{\pgfqpoint{3.103932in}{1.408618in}}%
\pgfpathlineto{\pgfqpoint{3.104250in}{1.381913in}}%
\pgfpathlineto{\pgfqpoint{3.104567in}{2.192867in}}%
\pgfpathlineto{\pgfqpoint{3.105202in}{1.208061in}}%
\pgfpathlineto{\pgfqpoint{3.105520in}{2.045187in}}%
\pgfpathlineto{\pgfqpoint{3.105837in}{1.911687in}}%
\pgfpathlineto{\pgfqpoint{3.106155in}{1.165234in}}%
\pgfpathlineto{\pgfqpoint{3.106789in}{2.114098in}}%
\pgfpathlineto{\pgfqpoint{3.107107in}{1.259559in}}%
\pgfpathlineto{\pgfqpoint{3.107742in}{2.203936in}}%
\pgfpathlineto{\pgfqpoint{3.108059in}{1.466823in}}%
\pgfpathlineto{\pgfqpoint{3.108377in}{1.314204in}}%
\pgfpathlineto{\pgfqpoint{3.108694in}{2.156520in}}%
\pgfpathlineto{\pgfqpoint{3.109329in}{1.179981in}}%
\pgfpathlineto{\pgfqpoint{3.109647in}{1.985188in}}%
\pgfpathlineto{\pgfqpoint{3.109964in}{1.980051in}}%
\pgfpathlineto{\pgfqpoint{3.110282in}{1.179812in}}%
\pgfpathlineto{\pgfqpoint{3.110916in}{2.165737in}}%
\pgfpathlineto{\pgfqpoint{3.111234in}{1.315992in}}%
\pgfpathlineto{\pgfqpoint{3.111551in}{1.478444in}}%
\pgfpathlineto{\pgfqpoint{3.111869in}{2.214310in}}%
\pgfpathlineto{\pgfqpoint{3.112504in}{1.265279in}}%
\pgfpathlineto{\pgfqpoint{3.112821in}{2.116726in}}%
\pgfpathlineto{\pgfqpoint{3.113456in}{1.161464in}}%
\pgfpathlineto{\pgfqpoint{3.113774in}{1.913524in}}%
\pgfpathlineto{\pgfqpoint{3.114091in}{2.034927in}}%
\pgfpathlineto{\pgfqpoint{3.114408in}{1.200682in}}%
\pgfpathlineto{\pgfqpoint{3.115043in}{2.178726in}}%
\pgfpathlineto{\pgfqpoint{3.115361in}{1.364702in}}%
\pgfpathlineto{\pgfqpoint{3.115678in}{1.399281in}}%
\pgfpathlineto{\pgfqpoint{3.115996in}{2.190649in}}%
\pgfpathlineto{\pgfqpoint{3.116631in}{1.219789in}}%
\pgfpathlineto{\pgfqpoint{3.116948in}{2.066926in}}%
\pgfpathlineto{\pgfqpoint{3.117583in}{1.158676in}}%
\pgfpathlineto{\pgfqpoint{3.117901in}{1.843786in}}%
\pgfpathlineto{\pgfqpoint{3.118218in}{2.095391in}}%
\pgfpathlineto{\pgfqpoint{3.118535in}{1.240304in}}%
\pgfpathlineto{\pgfqpoint{3.119170in}{2.209598in}}%
\pgfpathlineto{\pgfqpoint{3.119488in}{1.434984in}}%
\pgfpathlineto{\pgfqpoint{3.119805in}{1.342036in}}%
\pgfpathlineto{\pgfqpoint{3.120123in}{2.170592in}}%
\pgfpathlineto{\pgfqpoint{3.120758in}{1.186310in}}%
\pgfpathlineto{\pgfqpoint{3.121075in}{2.006014in}}%
\pgfpathlineto{\pgfqpoint{3.121393in}{1.940718in}}%
\pgfpathlineto{\pgfqpoint{3.121710in}{1.162680in}}%
\pgfpathlineto{\pgfqpoint{3.122345in}{2.128228in}}%
\pgfpathlineto{\pgfqpoint{3.122662in}{1.278540in}}%
\pgfpathlineto{\pgfqpoint{3.123297in}{2.199393in}}%
\pgfpathlineto{\pgfqpoint{3.123615in}{1.496804in}}%
\pgfpathlineto{\pgfqpoint{3.123932in}{1.280077in}}%
\pgfpathlineto{\pgfqpoint{3.124250in}{2.132099in}}%
\pgfpathlineto{\pgfqpoint{3.124885in}{1.165526in}}%
\pgfpathlineto{\pgfqpoint{3.125202in}{1.944038in}}%
\pgfpathlineto{\pgfqpoint{3.125519in}{2.007496in}}%
\pgfpathlineto{\pgfqpoint{3.125837in}{1.183851in}}%
\pgfpathlineto{\pgfqpoint{3.126472in}{2.175914in}}%
\pgfpathlineto{\pgfqpoint{3.126789in}{1.338758in}}%
\pgfpathlineto{\pgfqpoint{3.127107in}{1.435629in}}%
\pgfpathlineto{\pgfqpoint{3.127424in}{2.200448in}}%
\pgfpathlineto{\pgfqpoint{3.128059in}{1.233696in}}%
\pgfpathlineto{\pgfqpoint{3.128377in}{2.085727in}}%
\pgfpathlineto{\pgfqpoint{3.129012in}{1.152562in}}%
\pgfpathlineto{\pgfqpoint{3.129329in}{1.869670in}}%
\pgfpathlineto{\pgfqpoint{3.129646in}{2.057146in}}%
\pgfpathlineto{\pgfqpoint{3.129964in}{1.209712in}}%
\pgfpathlineto{\pgfqpoint{3.130599in}{2.182184in}}%
\pgfpathlineto{\pgfqpoint{3.130916in}{1.390586in}}%
\pgfpathlineto{\pgfqpoint{3.131234in}{1.360857in}}%
\pgfpathlineto{\pgfqpoint{3.131551in}{2.173718in}}%
\pgfpathlineto{\pgfqpoint{3.132186in}{1.195026in}}%
\pgfpathlineto{\pgfqpoint{3.132504in}{2.033797in}}%
\pgfpathlineto{\pgfqpoint{3.132821in}{1.904504in}}%
\pgfpathlineto{\pgfqpoint{3.133138in}{1.156071in}}%
\pgfpathlineto{\pgfqpoint{3.133773in}{2.116385in}}%
\pgfpathlineto{\pgfqpoint{3.134091in}{1.255875in}}%
\pgfpathlineto{\pgfqpoint{3.134726in}{2.205329in}}%
\pgfpathlineto{\pgfqpoint{3.135043in}{1.461419in}}%
\pgfpathlineto{\pgfqpoint{3.135361in}{1.305508in}}%
\pgfpathlineto{\pgfqpoint{3.135678in}{2.145232in}}%
\pgfpathlineto{\pgfqpoint{3.136313in}{1.165881in}}%
\pgfpathlineto{\pgfqpoint{3.136631in}{1.968823in}}%
\pgfpathlineto{\pgfqpoint{3.136948in}{1.965452in}}%
\pgfpathlineto{\pgfqpoint{3.137265in}{1.165826in}}%
\pgfpathlineto{\pgfqpoint{3.137900in}{2.140840in}}%
\pgfpathlineto{\pgfqpoint{3.138218in}{1.297068in}}%
\pgfpathlineto{\pgfqpoint{3.138535in}{1.456732in}}%
\pgfpathlineto{\pgfqpoint{3.138853in}{2.191516in}}%
\pgfpathlineto{\pgfqpoint{3.139488in}{1.250138in}}%
\pgfpathlineto{\pgfqpoint{3.139805in}{2.104250in}}%
\pgfpathlineto{\pgfqpoint{3.140440in}{1.151681in}}%
\pgfpathlineto{\pgfqpoint{3.140757in}{1.904886in}}%
\pgfpathlineto{\pgfqpoint{3.141075in}{2.031905in}}%
\pgfpathlineto{\pgfqpoint{3.141392in}{1.193762in}}%
\pgfpathlineto{\pgfqpoint{3.142027in}{2.182887in}}%
\pgfpathlineto{\pgfqpoint{3.142345in}{1.360398in}}%
\pgfpathlineto{\pgfqpoint{3.142662in}{1.394782in}}%
\pgfpathlineto{\pgfqpoint{3.142980in}{2.183973in}}%
\pgfpathlineto{\pgfqpoint{3.143615in}{1.207320in}}%
\pgfpathlineto{\pgfqpoint{3.143932in}{2.052038in}}%
\pgfpathlineto{\pgfqpoint{3.144567in}{1.144155in}}%
\pgfpathlineto{\pgfqpoint{3.144884in}{1.828621in}}%
\pgfpathlineto{\pgfqpoint{3.145202in}{2.074497in}}%
\pgfpathlineto{\pgfqpoint{3.145519in}{1.223065in}}%
\pgfpathlineto{\pgfqpoint{3.146154in}{2.183349in}}%
\pgfpathlineto{\pgfqpoint{3.146472in}{1.415942in}}%
\pgfpathlineto{\pgfqpoint{3.146789in}{1.324642in}}%
\pgfpathlineto{\pgfqpoint{3.147107in}{2.155112in}}%
\pgfpathlineto{\pgfqpoint{3.147742in}{1.175487in}}%
\pgfpathlineto{\pgfqpoint{3.148059in}{1.997660in}}%
\pgfpathlineto{\pgfqpoint{3.148376in}{1.932001in}}%
\pgfpathlineto{\pgfqpoint{3.148694in}{1.153401in}}%
\pgfpathlineto{\pgfqpoint{3.149329in}{2.131116in}}%
\pgfpathlineto{\pgfqpoint{3.149646in}{1.273912in}}%
\pgfpathlineto{\pgfqpoint{3.150281in}{2.197862in}}%
\pgfpathlineto{\pgfqpoint{3.150599in}{1.487266in}}%
\pgfpathlineto{\pgfqpoint{3.150916in}{1.270941in}}%
\pgfpathlineto{\pgfqpoint{3.151234in}{2.119651in}}%
\pgfpathlineto{\pgfqpoint{3.151868in}{1.151384in}}%
\pgfpathlineto{\pgfqpoint{3.152186in}{1.929332in}}%
\pgfpathlineto{\pgfqpoint{3.152503in}{1.989898in}}%
\pgfpathlineto{\pgfqpoint{3.152821in}{1.168722in}}%
\pgfpathlineto{\pgfqpoint{3.153456in}{2.148761in}}%
\pgfpathlineto{\pgfqpoint{3.153773in}{1.318155in}}%
\pgfpathlineto{\pgfqpoint{3.154091in}{1.416740in}}%
\pgfpathlineto{\pgfqpoint{3.154408in}{2.180799in}}%
\pgfpathlineto{\pgfqpoint{3.155043in}{1.221708in}}%
\pgfpathlineto{\pgfqpoint{3.155361in}{2.076545in}}%
\pgfpathlineto{\pgfqpoint{3.155995in}{1.143359in}}%
\pgfpathlineto{\pgfqpoint{3.156313in}{1.863314in}}%
\pgfpathlineto{\pgfqpoint{3.156630in}{2.055974in}}%
\pgfpathlineto{\pgfqpoint{3.156948in}{1.203287in}}%
\pgfpathlineto{\pgfqpoint{3.157583in}{2.184136in}}%
\pgfpathlineto{\pgfqpoint{3.157900in}{1.382944in}}%
\pgfpathlineto{\pgfqpoint{3.158218in}{1.356045in}}%
\pgfpathlineto{\pgfqpoint{3.158535in}{2.164763in}}%
\pgfpathlineto{\pgfqpoint{3.159170in}{1.182407in}}%
\pgfpathlineto{\pgfqpoint{3.159487in}{2.019817in}}%
\pgfpathlineto{\pgfqpoint{3.159805in}{1.887020in}}%
\pgfpathlineto{\pgfqpoint{3.160122in}{1.141408in}}%
\pgfpathlineto{\pgfqpoint{3.160757in}{2.090962in}}%
\pgfpathlineto{\pgfqpoint{3.161075in}{1.235711in}}%
\pgfpathlineto{\pgfqpoint{3.161710in}{2.180795in}}%
\pgfpathlineto{\pgfqpoint{3.162027in}{1.441985in}}%
\pgfpathlineto{\pgfqpoint{3.162345in}{1.291596in}}%
\pgfpathlineto{\pgfqpoint{3.162662in}{2.133438in}}%
\pgfpathlineto{\pgfqpoint{3.163297in}{1.156701in}}%
\pgfpathlineto{\pgfqpoint{3.163614in}{1.963159in}}%
\pgfpathlineto{\pgfqpoint{3.163932in}{1.956988in}}%
\pgfpathlineto{\pgfqpoint{3.164249in}{1.157033in}}%
\pgfpathlineto{\pgfqpoint{3.164884in}{2.143494in}}%
\pgfpathlineto{\pgfqpoint{3.165202in}{1.290515in}}%
\pgfpathlineto{\pgfqpoint{3.165837in}{2.186901in}}%
\pgfpathlineto{\pgfqpoint{3.166154in}{1.512880in}}%
\pgfpathlineto{\pgfqpoint{3.166472in}{1.240495in}}%
\pgfpathlineto{\pgfqpoint{3.166789in}{2.091545in}}%
\pgfpathlineto{\pgfqpoint{3.167424in}{1.137489in}}%
\pgfpathlineto{\pgfqpoint{3.167741in}{1.892381in}}%
\pgfpathlineto{\pgfqpoint{3.168059in}{2.009802in}}%
\pgfpathlineto{\pgfqpoint{3.168376in}{1.176333in}}%
\pgfpathlineto{\pgfqpoint{3.169011in}{2.155095in}}%
\pgfpathlineto{\pgfqpoint{3.169329in}{1.338366in}}%
\pgfpathlineto{\pgfqpoint{3.169646in}{1.379290in}}%
\pgfpathlineto{\pgfqpoint{3.169964in}{2.167844in}}%
\pgfpathlineto{\pgfqpoint{3.170598in}{1.197693in}}%
\pgfpathlineto{\pgfqpoint{3.170916in}{2.045919in}}%
\pgfpathlineto{\pgfqpoint{3.171551in}{1.134895in}}%
\pgfpathlineto{\pgfqpoint{3.171868in}{1.825761in}}%
\pgfpathlineto{\pgfqpoint{3.172186in}{2.074266in}}%
\pgfpathlineto{\pgfqpoint{3.172503in}{1.216140in}}%
\pgfpathlineto{\pgfqpoint{3.173138in}{2.182730in}}%
\pgfpathlineto{\pgfqpoint{3.173456in}{1.404400in}}%
\pgfpathlineto{\pgfqpoint{3.173773in}{1.319429in}}%
\pgfpathlineto{\pgfqpoint{3.174091in}{2.144547in}}%
\pgfpathlineto{\pgfqpoint{3.174725in}{1.162750in}}%
\pgfpathlineto{\pgfqpoint{3.175043in}{1.985225in}}%
\pgfpathlineto{\pgfqpoint{3.175360in}{1.912672in}}%
\pgfpathlineto{\pgfqpoint{3.175678in}{1.138375in}}%
\pgfpathlineto{\pgfqpoint{3.176313in}{2.102737in}}%
\pgfpathlineto{\pgfqpoint{3.176630in}{1.251643in}}%
\pgfpathlineto{\pgfqpoint{3.177265in}{2.176060in}}%
\pgfpathlineto{\pgfqpoint{3.177583in}{1.467275in}}%
\pgfpathlineto{\pgfqpoint{3.177900in}{1.260376in}}%
\pgfpathlineto{\pgfqpoint{3.178217in}{2.111264in}}%
\pgfpathlineto{\pgfqpoint{3.178852in}{1.143004in}}%
\pgfpathlineto{\pgfqpoint{3.179170in}{1.926043in}}%
\pgfpathlineto{\pgfqpoint{3.179487in}{1.983085in}}%
\pgfpathlineto{\pgfqpoint{3.179805in}{1.160626in}}%
\pgfpathlineto{\pgfqpoint{3.180440in}{2.149810in}}%
\pgfpathlineto{\pgfqpoint{3.180757in}{1.308744in}}%
\pgfpathlineto{\pgfqpoint{3.181075in}{1.416425in}}%
\pgfpathlineto{\pgfqpoint{3.181392in}{2.173657in}}%
\pgfpathlineto{\pgfqpoint{3.182027in}{1.211809in}}%
\pgfpathlineto{\pgfqpoint{3.182344in}{2.064379in}}%
\pgfpathlineto{\pgfqpoint{3.182979in}{1.129039in}}%
\pgfpathlineto{\pgfqpoint{3.183297in}{1.853406in}}%
\pgfpathlineto{\pgfqpoint{3.183614in}{2.029225in}}%
\pgfpathlineto{\pgfqpoint{3.183932in}{1.183011in}}%
\pgfpathlineto{\pgfqpoint{3.184567in}{2.157338in}}%
\pgfpathlineto{\pgfqpoint{3.184884in}{1.360193in}}%
\pgfpathlineto{\pgfqpoint{3.185202in}{1.343957in}}%
\pgfpathlineto{\pgfqpoint{3.185519in}{2.152350in}}%
\pgfpathlineto{\pgfqpoint{3.186154in}{1.174714in}}%
\pgfpathlineto{\pgfqpoint{3.186471in}{2.016414in}}%
\pgfpathlineto{\pgfqpoint{3.186789in}{1.873878in}}%
\pgfpathlineto{\pgfqpoint{3.187106in}{1.132139in}}%
\pgfpathlineto{\pgfqpoint{3.187741in}{2.090891in}}%
\pgfpathlineto{\pgfqpoint{3.188059in}{1.227383in}}%
\pgfpathlineto{\pgfqpoint{3.188694in}{2.177211in}}%
\pgfpathlineto{\pgfqpoint{3.189011in}{1.426515in}}%
\pgfpathlineto{\pgfqpoint{3.189328in}{1.285838in}}%
\pgfpathlineto{\pgfqpoint{3.189646in}{2.122247in}}%
\pgfpathlineto{\pgfqpoint{3.190281in}{1.143967in}}%
\pgfpathlineto{\pgfqpoint{3.190598in}{1.952794in}}%
\pgfpathlineto{\pgfqpoint{3.190916in}{1.934239in}}%
\pgfpathlineto{\pgfqpoint{3.191233in}{1.140205in}}%
\pgfpathlineto{\pgfqpoint{3.191868in}{2.113513in}}%
\pgfpathlineto{\pgfqpoint{3.192186in}{1.266368in}}%
\pgfpathlineto{\pgfqpoint{3.192821in}{2.168514in}}%
\pgfpathlineto{\pgfqpoint{3.193138in}{1.492457in}}%
\pgfpathlineto{\pgfqpoint{3.193455in}{1.232617in}}%
\pgfpathlineto{\pgfqpoint{3.193773in}{2.086443in}}%
\pgfpathlineto{\pgfqpoint{3.194408in}{1.129407in}}%
\pgfpathlineto{\pgfqpoint{3.194725in}{1.891462in}}%
\pgfpathlineto{\pgfqpoint{3.195043in}{2.004296in}}%
\pgfpathlineto{\pgfqpoint{3.195360in}{1.168180in}}%
\pgfpathlineto{\pgfqpoint{3.195995in}{2.154021in}}%
\pgfpathlineto{\pgfqpoint{3.196313in}{1.325434in}}%
\pgfpathlineto{\pgfqpoint{3.196630in}{1.378852in}}%
\pgfpathlineto{\pgfqpoint{3.196947in}{2.158684in}}%
\pgfpathlineto{\pgfqpoint{3.197582in}{1.187571in}}%
\pgfpathlineto{\pgfqpoint{3.197900in}{2.035042in}}%
\pgfpathlineto{\pgfqpoint{3.198535in}{1.120429in}}%
\pgfpathlineto{\pgfqpoint{3.198852in}{1.817324in}}%
\pgfpathlineto{\pgfqpoint{3.199170in}{2.044077in}}%
\pgfpathlineto{\pgfqpoint{3.199487in}{1.193503in}}%
\pgfpathlineto{\pgfqpoint{3.200122in}{2.158058in}}%
\pgfpathlineto{\pgfqpoint{3.200440in}{1.380881in}}%
\pgfpathlineto{\pgfqpoint{3.200757in}{1.310868in}}%
\pgfpathlineto{\pgfqpoint{3.201074in}{2.135699in}}%
\pgfpathlineto{\pgfqpoint{3.201709in}{1.156164in}}%
\pgfpathlineto{\pgfqpoint{3.202027in}{1.984322in}}%
\pgfpathlineto{\pgfqpoint{3.202344in}{1.900390in}}%
\pgfpathlineto{\pgfqpoint{3.202662in}{1.129723in}}%
\pgfpathlineto{\pgfqpoint{3.203297in}{2.101499in}}%
\pgfpathlineto{\pgfqpoint{3.203614in}{1.240971in}}%
\pgfpathlineto{\pgfqpoint{3.204249in}{2.169870in}}%
\pgfpathlineto{\pgfqpoint{3.204566in}{1.447708in}}%
\pgfpathlineto{\pgfqpoint{3.204884in}{1.254332in}}%
\pgfpathlineto{\pgfqpoint{3.205201in}{2.100198in}}%
\pgfpathlineto{\pgfqpoint{3.205836in}{1.130170in}}%
\pgfpathlineto{\pgfqpoint{3.206154in}{1.918196in}}%
\pgfpathlineto{\pgfqpoint{3.206471in}{1.955584in}}%
\pgfpathlineto{\pgfqpoint{3.206789in}{1.141034in}}%
\pgfpathlineto{\pgfqpoint{3.207424in}{2.119987in}}%
\pgfpathlineto{\pgfqpoint{3.207741in}{1.283521in}}%
\pgfpathlineto{\pgfqpoint{3.208058in}{1.406560in}}%
\pgfpathlineto{\pgfqpoint{3.208376in}{2.159023in}}%
\pgfpathlineto{\pgfqpoint{3.209011in}{1.206285in}}%
\pgfpathlineto{\pgfqpoint{3.209328in}{2.062127in}}%
\pgfpathlineto{\pgfqpoint{3.209963in}{1.120495in}}%
\pgfpathlineto{\pgfqpoint{3.210281in}{1.856174in}}%
\pgfpathlineto{\pgfqpoint{3.210598in}{2.024531in}}%
\pgfpathlineto{\pgfqpoint{3.210916in}{1.174053in}}%
\pgfpathlineto{\pgfqpoint{3.211551in}{2.153538in}}%
\pgfpathlineto{\pgfqpoint{3.211868in}{1.343541in}}%
\pgfpathlineto{\pgfqpoint{3.212185in}{1.343096in}}%
\pgfpathlineto{\pgfqpoint{3.212503in}{2.142106in}}%
\pgfpathlineto{\pgfqpoint{3.213138in}{1.164603in}}%
\pgfpathlineto{\pgfqpoint{3.213455in}{2.007375in}}%
\pgfpathlineto{\pgfqpoint{3.213773in}{1.848638in}}%
\pgfpathlineto{\pgfqpoint{3.214090in}{1.116619in}}%
\pgfpathlineto{\pgfqpoint{3.214725in}{2.058364in}}%
\pgfpathlineto{\pgfqpoint{3.215043in}{1.202528in}}%
\pgfpathlineto{\pgfqpoint{3.215677in}{2.155472in}}%
\pgfpathlineto{\pgfqpoint{3.215995in}{1.402448in}}%
\pgfpathlineto{\pgfqpoint{3.216312in}{1.280231in}}%
\pgfpathlineto{\pgfqpoint{3.216630in}{2.116836in}}%
\pgfpathlineto{\pgfqpoint{3.217265in}{1.138061in}}%
\pgfpathlineto{\pgfqpoint{3.217582in}{1.953978in}}%
\pgfpathlineto{\pgfqpoint{3.217900in}{1.922953in}}%
\pgfpathlineto{\pgfqpoint{3.218217in}{1.131697in}}%
\pgfpathlineto{\pgfqpoint{3.218852in}{2.110623in}}%
\pgfpathlineto{\pgfqpoint{3.219170in}{1.252700in}}%
\pgfpathlineto{\pgfqpoint{3.219804in}{2.160081in}}%
\pgfpathlineto{\pgfqpoint{3.220122in}{1.469275in}}%
\pgfpathlineto{\pgfqpoint{3.220439in}{1.226315in}}%
\pgfpathlineto{\pgfqpoint{3.220757in}{2.076346in}}%
\pgfpathlineto{\pgfqpoint{3.221392in}{1.116495in}}%
\pgfpathlineto{\pgfqpoint{3.221709in}{1.886155in}}%
\pgfpathlineto{\pgfqpoint{3.222027in}{1.972650in}}%
\pgfpathlineto{\pgfqpoint{3.222344in}{1.145977in}}%
\pgfpathlineto{\pgfqpoint{3.222979in}{2.125527in}}%
\pgfpathlineto{\pgfqpoint{3.223296in}{1.299153in}}%
\pgfpathlineto{\pgfqpoint{3.223614in}{1.372596in}}%
\pgfpathlineto{\pgfqpoint{3.223931in}{2.147699in}}%
\pgfpathlineto{\pgfqpoint{3.224566in}{1.183549in}}%
\pgfpathlineto{\pgfqpoint{3.224884in}{2.035459in}}%
\pgfpathlineto{\pgfqpoint{3.225519in}{1.112027in}}%
\pgfpathlineto{\pgfqpoint{3.225836in}{1.824202in}}%
\pgfpathlineto{\pgfqpoint{3.226154in}{2.038965in}}%
\pgfpathlineto{\pgfqpoint{3.226471in}{1.182777in}}%
\pgfpathlineto{\pgfqpoint{3.227106in}{2.151747in}}%
\pgfpathlineto{\pgfqpoint{3.227423in}{1.360234in}}%
\pgfpathlineto{\pgfqpoint{3.227741in}{1.309812in}}%
\pgfpathlineto{\pgfqpoint{3.228058in}{2.125064in}}%
\pgfpathlineto{\pgfqpoint{3.228693in}{1.145970in}}%
\pgfpathlineto{\pgfqpoint{3.229011in}{1.977638in}}%
\pgfpathlineto{\pgfqpoint{3.229328in}{1.870788in}}%
\pgfpathlineto{\pgfqpoint{3.229646in}{1.111908in}}%
\pgfpathlineto{\pgfqpoint{3.230281in}{2.068306in}}%
\pgfpathlineto{\pgfqpoint{3.230598in}{1.214635in}}%
\pgfpathlineto{\pgfqpoint{3.231233in}{2.151570in}}%
\pgfpathlineto{\pgfqpoint{3.231550in}{1.422741in}}%
\pgfpathlineto{\pgfqpoint{3.231868in}{1.251510in}}%
\pgfpathlineto{\pgfqpoint{3.232185in}{2.097770in}}%
\pgfpathlineto{\pgfqpoint{3.232820in}{1.124149in}}%
\pgfpathlineto{\pgfqpoint{3.233138in}{1.922258in}}%
\pgfpathlineto{\pgfqpoint{3.233455in}{1.945198in}}%
\pgfpathlineto{\pgfqpoint{3.233773in}{1.132204in}}%
\pgfpathlineto{\pgfqpoint{3.234407in}{2.114733in}}%
\pgfpathlineto{\pgfqpoint{3.234725in}{1.266506in}}%
\pgfpathlineto{\pgfqpoint{3.235042in}{1.411320in}}%
\pgfpathlineto{\pgfqpoint{3.235360in}{2.149117in}}%
\pgfpathlineto{\pgfqpoint{3.235995in}{1.200021in}}%
\pgfpathlineto{\pgfqpoint{3.236312in}{2.053521in}}%
\pgfpathlineto{\pgfqpoint{3.236947in}{1.107330in}}%
\pgfpathlineto{\pgfqpoint{3.237265in}{1.852068in}}%
\pgfpathlineto{\pgfqpoint{3.237582in}{1.989444in}}%
\pgfpathlineto{\pgfqpoint{3.237900in}{1.149350in}}%
\pgfpathlineto{\pgfqpoint{3.238534in}{2.127378in}}%
\pgfpathlineto{\pgfqpoint{3.238852in}{1.316547in}}%
\pgfpathlineto{\pgfqpoint{3.239169in}{1.340053in}}%
\pgfpathlineto{\pgfqpoint{3.239487in}{2.134704in}}%
\pgfpathlineto{\pgfqpoint{3.240122in}{1.161741in}}%
\pgfpathlineto{\pgfqpoint{3.240439in}{2.010011in}}%
\pgfpathlineto{\pgfqpoint{3.241074in}{1.108449in}}%
\pgfpathlineto{\pgfqpoint{3.241392in}{1.790094in}}%
\pgfpathlineto{\pgfqpoint{3.241709in}{2.052328in}}%
\pgfpathlineto{\pgfqpoint{3.242026in}{1.189387in}}%
\pgfpathlineto{\pgfqpoint{3.242661in}{2.146783in}}%
\pgfpathlineto{\pgfqpoint{3.242979in}{1.378129in}}%
\pgfpathlineto{\pgfqpoint{3.243296in}{1.278921in}}%
\pgfpathlineto{\pgfqpoint{3.243614in}{2.106687in}}%
\pgfpathlineto{\pgfqpoint{3.244249in}{1.127914in}}%
\pgfpathlineto{\pgfqpoint{3.244566in}{1.950009in}}%
\pgfpathlineto{\pgfqpoint{3.244884in}{1.889186in}}%
\pgfpathlineto{\pgfqpoint{3.245201in}{1.111387in}}%
\pgfpathlineto{\pgfqpoint{3.245836in}{2.077770in}}%
\pgfpathlineto{\pgfqpoint{3.246153in}{1.224918in}}%
\pgfpathlineto{\pgfqpoint{3.246788in}{2.145217in}}%
\pgfpathlineto{\pgfqpoint{3.247106in}{1.443303in}}%
\pgfpathlineto{\pgfqpoint{3.247423in}{1.225432in}}%
\pgfpathlineto{\pgfqpoint{3.247741in}{2.076737in}}%
\pgfpathlineto{\pgfqpoint{3.248376in}{1.110156in}}%
\pgfpathlineto{\pgfqpoint{3.248693in}{1.893892in}}%
\pgfpathlineto{\pgfqpoint{3.249011in}{1.962255in}}%
\pgfpathlineto{\pgfqpoint{3.249328in}{1.135961in}}%
\pgfpathlineto{\pgfqpoint{3.249963in}{2.118014in}}%
\pgfpathlineto{\pgfqpoint{3.250280in}{1.278447in}}%
\pgfpathlineto{\pgfqpoint{3.250598in}{1.377435in}}%
\pgfpathlineto{\pgfqpoint{3.250915in}{2.136894in}}%
\pgfpathlineto{\pgfqpoint{3.251550in}{1.177226in}}%
\pgfpathlineto{\pgfqpoint{3.251868in}{2.028922in}}%
\pgfpathlineto{\pgfqpoint{3.252503in}{1.097557in}}%
\pgfpathlineto{\pgfqpoint{3.252820in}{1.820851in}}%
\pgfpathlineto{\pgfqpoint{3.253137in}{2.002043in}}%
\pgfpathlineto{\pgfqpoint{3.253455in}{1.156239in}}%
\pgfpathlineto{\pgfqpoint{3.254090in}{2.128483in}}%
\pgfpathlineto{\pgfqpoint{3.254407in}{1.332229in}}%
\pgfpathlineto{\pgfqpoint{3.254725in}{1.309947in}}%
\pgfpathlineto{\pgfqpoint{3.255042in}{2.120760in}}%
\pgfpathlineto{\pgfqpoint{3.255677in}{1.143452in}}%
\pgfpathlineto{\pgfqpoint{3.255995in}{1.982384in}}%
\pgfpathlineto{\pgfqpoint{3.256312in}{1.853453in}}%
\pgfpathlineto{\pgfqpoint{3.256630in}{1.103890in}}%
\pgfpathlineto{\pgfqpoint{3.257264in}{2.060496in}}%
\pgfpathlineto{\pgfqpoint{3.257582in}{1.198563in}}%
\pgfpathlineto{\pgfqpoint{3.258217in}{2.141104in}}%
\pgfpathlineto{\pgfqpoint{3.258534in}{1.394606in}}%
\pgfpathlineto{\pgfqpoint{3.258852in}{1.250276in}}%
\pgfpathlineto{\pgfqpoint{3.259169in}{2.088621in}}%
\pgfpathlineto{\pgfqpoint{3.259804in}{1.113858in}}%
\pgfpathlineto{\pgfqpoint{3.260122in}{1.920363in}}%
\pgfpathlineto{\pgfqpoint{3.260439in}{1.907388in}}%
\pgfpathlineto{\pgfqpoint{3.260756in}{1.109370in}}%
\pgfpathlineto{\pgfqpoint{3.261391in}{2.083337in}}%
\pgfpathlineto{\pgfqpoint{3.261709in}{1.237706in}}%
\pgfpathlineto{\pgfqpoint{3.262344in}{2.137786in}}%
\pgfpathlineto{\pgfqpoint{3.262661in}{1.462493in}}%
\pgfpathlineto{\pgfqpoint{3.262979in}{1.200860in}}%
\pgfpathlineto{\pgfqpoint{3.263296in}{2.056255in}}%
\pgfpathlineto{\pgfqpoint{3.263931in}{1.101036in}}%
\pgfpathlineto{\pgfqpoint{3.264248in}{1.863834in}}%
\pgfpathlineto{\pgfqpoint{3.264566in}{1.978681in}}%
\pgfpathlineto{\pgfqpoint{3.264883in}{1.137599in}}%
\pgfpathlineto{\pgfqpoint{3.265518in}{2.117367in}}%
\pgfpathlineto{\pgfqpoint{3.265836in}{1.292202in}}%
\pgfpathlineto{\pgfqpoint{3.266153in}{1.344713in}}%
\pgfpathlineto{\pgfqpoint{3.266471in}{2.123733in}}%
\pgfpathlineto{\pgfqpoint{3.267106in}{1.155582in}}%
\pgfpathlineto{\pgfqpoint{3.267423in}{2.005871in}}%
\pgfpathlineto{\pgfqpoint{3.268058in}{1.091838in}}%
\pgfpathlineto{\pgfqpoint{3.268375in}{1.787782in}}%
\pgfpathlineto{\pgfqpoint{3.268693in}{2.014474in}}%
\pgfpathlineto{\pgfqpoint{3.269010in}{1.161147in}}%
\pgfpathlineto{\pgfqpoint{3.269645in}{2.126593in}}%
\pgfpathlineto{\pgfqpoint{3.269963in}{1.349114in}}%
\pgfpathlineto{\pgfqpoint{3.270280in}{1.281484in}}%
\pgfpathlineto{\pgfqpoint{3.270598in}{2.105365in}}%
\pgfpathlineto{\pgfqpoint{3.271233in}{1.125494in}}%
\pgfpathlineto{\pgfqpoint{3.271550in}{1.957586in}}%
\pgfpathlineto{\pgfqpoint{3.271867in}{1.872181in}}%
\pgfpathlineto{\pgfqpoint{3.272185in}{1.102840in}}%
\pgfpathlineto{\pgfqpoint{3.272820in}{2.068057in}}%
\pgfpathlineto{\pgfqpoint{3.273137in}{1.205493in}}%
\pgfpathlineto{\pgfqpoint{3.273772in}{2.133324in}}%
\pgfpathlineto{\pgfqpoint{3.274090in}{1.411863in}}%
\pgfpathlineto{\pgfqpoint{3.274407in}{1.224152in}}%
\pgfpathlineto{\pgfqpoint{3.274725in}{2.069243in}}%
\pgfpathlineto{\pgfqpoint{3.275360in}{1.099697in}}%
\pgfpathlineto{\pgfqpoint{3.275677in}{1.893210in}}%
\pgfpathlineto{\pgfqpoint{3.275994in}{1.921782in}}%
\pgfpathlineto{\pgfqpoint{3.276312in}{1.111059in}}%
\pgfpathlineto{\pgfqpoint{3.276947in}{2.088617in}}%
\pgfpathlineto{\pgfqpoint{3.277264in}{1.248421in}}%
\pgfpathlineto{\pgfqpoint{3.277582in}{1.380452in}}%
\pgfpathlineto{\pgfqpoint{3.277899in}{2.128680in}}%
\pgfpathlineto{\pgfqpoint{3.278534in}{1.178959in}}%
\pgfpathlineto{\pgfqpoint{3.278852in}{2.033946in}}%
\pgfpathlineto{\pgfqpoint{3.279486in}{1.091689in}}%
\pgfpathlineto{\pgfqpoint{3.279804in}{1.836357in}}%
\pgfpathlineto{\pgfqpoint{3.280121in}{1.990011in}}%
\pgfpathlineto{\pgfqpoint{3.280439in}{1.142087in}}%
\pgfpathlineto{\pgfqpoint{3.281074in}{2.116373in}}%
\pgfpathlineto{\pgfqpoint{3.281391in}{1.304029in}}%
\pgfpathlineto{\pgfqpoint{3.281709in}{1.314748in}}%
\pgfpathlineto{\pgfqpoint{3.282026in}{2.110140in}}%
\pgfpathlineto{\pgfqpoint{3.282661in}{1.137252in}}%
\pgfpathlineto{\pgfqpoint{3.282978in}{1.981079in}}%
\pgfpathlineto{\pgfqpoint{3.283613in}{1.084903in}}%
\pgfpathlineto{\pgfqpoint{3.283931in}{1.757874in}}%
\pgfpathlineto{\pgfqpoint{3.284248in}{2.022983in}}%
\pgfpathlineto{\pgfqpoint{3.284566in}{1.169082in}}%
\pgfpathlineto{\pgfqpoint{3.285201in}{2.124174in}}%
\pgfpathlineto{\pgfqpoint{3.285518in}{1.364150in}}%
\pgfpathlineto{\pgfqpoint{3.285836in}{1.255155in}}%
\pgfpathlineto{\pgfqpoint{3.286153in}{2.089753in}}%
\pgfpathlineto{\pgfqpoint{3.286788in}{1.110972in}}%
\pgfpathlineto{\pgfqpoint{3.287105in}{1.931691in}}%
\pgfpathlineto{\pgfqpoint{3.287423in}{1.890450in}}%
\pgfpathlineto{\pgfqpoint{3.287740in}{1.099892in}}%
\pgfpathlineto{\pgfqpoint{3.288375in}{2.071331in}}%
\pgfpathlineto{\pgfqpoint{3.288693in}{1.214818in}}%
\pgfpathlineto{\pgfqpoint{3.289328in}{2.125149in}}%
\pgfpathlineto{\pgfqpoint{3.289645in}{1.427615in}}%
\pgfpathlineto{\pgfqpoint{3.289963in}{1.199858in}}%
\pgfpathlineto{\pgfqpoint{3.290280in}{2.050730in}}%
\pgfpathlineto{\pgfqpoint{3.290915in}{1.089181in}}%
\pgfpathlineto{\pgfqpoint{3.291232in}{1.864152in}}%
\pgfpathlineto{\pgfqpoint{3.291550in}{1.936065in}}%
\pgfpathlineto{\pgfqpoint{3.291867in}{1.110790in}}%
\pgfpathlineto{\pgfqpoint{3.292502in}{2.090528in}}%
\pgfpathlineto{\pgfqpoint{3.292820in}{1.261149in}}%
\pgfpathlineto{\pgfqpoint{3.293137in}{1.350661in}}%
\pgfpathlineto{\pgfqpoint{3.293455in}{2.118701in}}%
\pgfpathlineto{\pgfqpoint{3.294090in}{1.158079in}}%
\pgfpathlineto{\pgfqpoint{3.294407in}{2.012927in}}%
\pgfpathlineto{\pgfqpoint{3.295042in}{1.085905in}}%
\pgfpathlineto{\pgfqpoint{3.295359in}{1.806634in}}%
\pgfpathlineto{\pgfqpoint{3.295677in}{2.000870in}}%
\pgfpathlineto{\pgfqpoint{3.295994in}{1.144152in}}%
\pgfpathlineto{\pgfqpoint{3.296629in}{2.112593in}}%
\pgfpathlineto{\pgfqpoint{3.296947in}{1.317449in}}%
\pgfpathlineto{\pgfqpoint{3.297264in}{1.286286in}}%
\pgfpathlineto{\pgfqpoint{3.297582in}{2.095808in}}%
\pgfpathlineto{\pgfqpoint{3.298216in}{1.119489in}}%
\pgfpathlineto{\pgfqpoint{3.298534in}{1.958240in}}%
\pgfpathlineto{\pgfqpoint{3.298851in}{1.828725in}}%
\pgfpathlineto{\pgfqpoint{3.299169in}{1.081602in}}%
\pgfpathlineto{\pgfqpoint{3.299804in}{2.031474in}}%
\pgfpathlineto{\pgfqpoint{3.300121in}{1.174719in}}%
\pgfpathlineto{\pgfqpoint{3.300756in}{2.119426in}}%
\pgfpathlineto{\pgfqpoint{3.301074in}{1.379904in}}%
\pgfpathlineto{\pgfqpoint{3.301391in}{1.230554in}}%
\pgfpathlineto{\pgfqpoint{3.301708in}{2.072838in}}%
\pgfpathlineto{\pgfqpoint{3.302343in}{1.097051in}}%
\pgfpathlineto{\pgfqpoint{3.302661in}{1.908220in}}%
\pgfpathlineto{\pgfqpoint{3.302978in}{1.904113in}}%
\pgfpathlineto{\pgfqpoint{3.303296in}{1.099921in}}%
\pgfpathlineto{\pgfqpoint{3.303931in}{2.074550in}}%
\pgfpathlineto{\pgfqpoint{3.304248in}{1.221836in}}%
\pgfpathlineto{\pgfqpoint{3.304883in}{2.115762in}}%
\pgfpathlineto{\pgfqpoint{3.305201in}{1.443714in}}%
\pgfpathlineto{\pgfqpoint{3.305518in}{1.178034in}}%
\pgfpathlineto{\pgfqpoint{3.305835in}{2.030922in}}%
\pgfpathlineto{\pgfqpoint{3.306470in}{1.077950in}}%
\pgfpathlineto{\pgfqpoint{3.306788in}{1.837907in}}%
\pgfpathlineto{\pgfqpoint{3.307105in}{1.946638in}}%
\pgfpathlineto{\pgfqpoint{3.307423in}{1.113821in}}%
\pgfpathlineto{\pgfqpoint{3.308058in}{2.092361in}}%
\pgfpathlineto{\pgfqpoint{3.308375in}{1.271612in}}%
\pgfpathlineto{\pgfqpoint{3.308693in}{1.323627in}}%
\pgfpathlineto{\pgfqpoint{3.309010in}{2.107706in}}%
\pgfpathlineto{\pgfqpoint{3.309645in}{1.139737in}}%
\pgfpathlineto{\pgfqpoint{3.309962in}{1.991276in}}%
\pgfpathlineto{\pgfqpoint{3.310597in}{1.078687in}}%
\pgfpathlineto{\pgfqpoint{3.310915in}{1.779527in}}%
\pgfpathlineto{\pgfqpoint{3.311232in}{2.007336in}}%
\pgfpathlineto{\pgfqpoint{3.311550in}{1.148977in}}%
\pgfpathlineto{\pgfqpoint{3.312185in}{2.108911in}}%
\pgfpathlineto{\pgfqpoint{3.312502in}{1.328871in}}%
\pgfpathlineto{\pgfqpoint{3.312820in}{1.260361in}}%
\pgfpathlineto{\pgfqpoint{3.313137in}{2.081604in}}%
\pgfpathlineto{\pgfqpoint{3.313772in}{1.104759in}}%
\pgfpathlineto{\pgfqpoint{3.314089in}{1.933692in}}%
\pgfpathlineto{\pgfqpoint{3.314407in}{1.843695in}}%
\pgfpathlineto{\pgfqpoint{3.314724in}{1.076589in}}%
\pgfpathlineto{\pgfqpoint{3.315359in}{2.036421in}}%
\pgfpathlineto{\pgfqpoint{3.315677in}{1.182975in}}%
\pgfpathlineto{\pgfqpoint{3.316312in}{2.114352in}}%
\pgfpathlineto{\pgfqpoint{3.316629in}{1.393686in}}%
\pgfpathlineto{\pgfqpoint{3.316946in}{1.207720in}}%
\pgfpathlineto{\pgfqpoint{3.317264in}{2.056279in}}%
\pgfpathlineto{\pgfqpoint{3.317899in}{1.086621in}}%
\pgfpathlineto{\pgfqpoint{3.318216in}{1.882879in}}%
\pgfpathlineto{\pgfqpoint{3.318534in}{1.917395in}}%
\pgfpathlineto{\pgfqpoint{3.318851in}{1.097655in}}%
\pgfpathlineto{\pgfqpoint{3.319486in}{2.074428in}}%
\pgfpathlineto{\pgfqpoint{3.319804in}{1.231080in}}%
\pgfpathlineto{\pgfqpoint{3.320121in}{1.362316in}}%
\pgfpathlineto{\pgfqpoint{3.320438in}{2.106154in}}%
\pgfpathlineto{\pgfqpoint{3.321073in}{1.157546in}}%
\pgfpathlineto{\pgfqpoint{3.321391in}{2.012381in}}%
\pgfpathlineto{\pgfqpoint{3.322026in}{1.070032in}}%
\pgfpathlineto{\pgfqpoint{3.322343in}{1.809891in}}%
\pgfpathlineto{\pgfqpoint{3.322661in}{1.957251in}}%
\pgfpathlineto{\pgfqpoint{3.322978in}{1.114535in}}%
\pgfpathlineto{\pgfqpoint{3.323613in}{2.091357in}}%
\pgfpathlineto{\pgfqpoint{3.323931in}{1.283654in}}%
\pgfpathlineto{\pgfqpoint{3.324248in}{1.297381in}}%
\pgfpathlineto{\pgfqpoint{3.324565in}{2.095985in}}%
\pgfpathlineto{\pgfqpoint{3.325200in}{1.122075in}}%
\pgfpathlineto{\pgfqpoint{3.325518in}{1.971761in}}%
\pgfpathlineto{\pgfqpoint{3.326153in}{1.074394in}}%
\pgfpathlineto{\pgfqpoint{3.326470in}{1.750354in}}%
\pgfpathlineto{\pgfqpoint{3.326788in}{2.013803in}}%
\pgfpathlineto{\pgfqpoint{3.327105in}{1.151258in}}%
\pgfpathlineto{\pgfqpoint{3.327740in}{2.103260in}}%
\pgfpathlineto{\pgfqpoint{3.328057in}{1.341531in}}%
\pgfpathlineto{\pgfqpoint{3.328375in}{1.235972in}}%
\pgfpathlineto{\pgfqpoint{3.328692in}{2.066671in}}%
\pgfpathlineto{\pgfqpoint{3.329327in}{1.090026in}}%
\pgfpathlineto{\pgfqpoint{3.329645in}{1.911441in}}%
\pgfpathlineto{\pgfqpoint{3.329962in}{1.855419in}}%
\pgfpathlineto{\pgfqpoint{3.330280in}{1.074879in}}%
\pgfpathlineto{\pgfqpoint{3.330915in}{2.041555in}}%
\pgfpathlineto{\pgfqpoint{3.331232in}{1.188689in}}%
\pgfpathlineto{\pgfqpoint{3.331867in}{2.107523in}}%
\pgfpathlineto{\pgfqpoint{3.332184in}{1.407761in}}%
\pgfpathlineto{\pgfqpoint{3.332502in}{1.186572in}}%
\pgfpathlineto{\pgfqpoint{3.332819in}{2.038550in}}%
\pgfpathlineto{\pgfqpoint{3.333454in}{1.075611in}}%
\pgfpathlineto{\pgfqpoint{3.333772in}{1.859855in}}%
\pgfpathlineto{\pgfqpoint{3.334089in}{1.926460in}}%
\pgfpathlineto{\pgfqpoint{3.334407in}{1.098250in}}%
\pgfpathlineto{\pgfqpoint{3.335042in}{2.074660in}}%
\pgfpathlineto{\pgfqpoint{3.335359in}{1.237895in}}%
\pgfpathlineto{\pgfqpoint{3.335676in}{1.335889in}}%
\pgfpathlineto{\pgfqpoint{3.335994in}{2.095875in}}%
\pgfpathlineto{\pgfqpoint{3.336629in}{1.139325in}}%
\pgfpathlineto{\pgfqpoint{3.336946in}{1.992545in}}%
\pgfpathlineto{\pgfqpoint{3.337581in}{1.060901in}}%
\pgfpathlineto{\pgfqpoint{3.337899in}{1.785000in}}%
\pgfpathlineto{\pgfqpoint{3.338216in}{1.964386in}}%
\pgfpathlineto{\pgfqpoint{3.338534in}{1.118216in}}%
\pgfpathlineto{\pgfqpoint{3.339169in}{2.090471in}}%
\pgfpathlineto{\pgfqpoint{3.339486in}{1.293257in}}%
\pgfpathlineto{\pgfqpoint{3.339803in}{1.273633in}}%
\pgfpathlineto{\pgfqpoint{3.340121in}{2.083782in}}%
\pgfpathlineto{\pgfqpoint{3.340756in}{1.107408in}}%
\pgfpathlineto{\pgfqpoint{3.341073in}{1.950845in}}%
\pgfpathlineto{\pgfqpoint{3.341391in}{1.818568in}}%
\pgfpathlineto{\pgfqpoint{3.341708in}{1.068184in}}%
\pgfpathlineto{\pgfqpoint{3.342343in}{2.016700in}}%
\pgfpathlineto{\pgfqpoint{3.342661in}{1.156219in}}%
\pgfpathlineto{\pgfqpoint{3.343295in}{2.097898in}}%
\pgfpathlineto{\pgfqpoint{3.343613in}{1.352033in}}%
\pgfpathlineto{\pgfqpoint{3.343930in}{1.213748in}}%
\pgfpathlineto{\pgfqpoint{3.344248in}{2.052233in}}%
\pgfpathlineto{\pgfqpoint{3.344883in}{1.078051in}}%
\pgfpathlineto{\pgfqpoint{3.345200in}{1.887564in}}%
\pgfpathlineto{\pgfqpoint{3.345518in}{1.866986in}}%
\pgfpathlineto{\pgfqpoint{3.345835in}{1.071074in}}%
\pgfpathlineto{\pgfqpoint{3.346470in}{2.043545in}}%
\pgfpathlineto{\pgfqpoint{3.346787in}{1.196656in}}%
\pgfpathlineto{\pgfqpoint{3.347422in}{2.100528in}}%
\pgfpathlineto{\pgfqpoint{3.347740in}{1.419743in}}%
\pgfpathlineto{\pgfqpoint{3.348057in}{1.166794in}}%
\pgfpathlineto{\pgfqpoint{3.348375in}{2.022227in}}%
\pgfpathlineto{\pgfqpoint{3.349010in}{1.067340in}}%
\pgfpathlineto{\pgfqpoint{3.349327in}{1.834945in}}%
\pgfpathlineto{\pgfqpoint{3.349645in}{1.935504in}}%
\pgfpathlineto{\pgfqpoint{3.349962in}{1.096356in}}%
\pgfpathlineto{\pgfqpoint{3.350597in}{2.072288in}}%
\pgfpathlineto{\pgfqpoint{3.350914in}{1.246676in}}%
\pgfpathlineto{\pgfqpoint{3.351232in}{1.310055in}}%
\pgfpathlineto{\pgfqpoint{3.351549in}{2.085420in}}%
\pgfpathlineto{\pgfqpoint{3.352184in}{1.121949in}}%
\pgfpathlineto{\pgfqpoint{3.352502in}{1.974316in}}%
\pgfpathlineto{\pgfqpoint{3.353137in}{1.054812in}}%
\pgfpathlineto{\pgfqpoint{3.353454in}{1.758494in}}%
\pgfpathlineto{\pgfqpoint{3.353772in}{1.971746in}}%
\pgfpathlineto{\pgfqpoint{3.354089in}{1.119292in}}%
\pgfpathlineto{\pgfqpoint{3.354724in}{2.087216in}}%
\pgfpathlineto{\pgfqpoint{3.355041in}{1.304059in}}%
\pgfpathlineto{\pgfqpoint{3.355359in}{1.250667in}}%
\pgfpathlineto{\pgfqpoint{3.355676in}{2.070968in}}%
\pgfpathlineto{\pgfqpoint{3.356311in}{1.093161in}}%
\pgfpathlineto{\pgfqpoint{3.356629in}{1.931774in}}%
\pgfpathlineto{\pgfqpoint{3.356946in}{1.829014in}}%
\pgfpathlineto{\pgfqpoint{3.357264in}{1.064731in}}%
\pgfpathlineto{\pgfqpoint{3.357899in}{2.020008in}}%
\pgfpathlineto{\pgfqpoint{3.358216in}{1.158499in}}%
\pgfpathlineto{\pgfqpoint{3.358851in}{2.091098in}}%
\pgfpathlineto{\pgfqpoint{3.359168in}{1.363385in}}%
\pgfpathlineto{\pgfqpoint{3.359486in}{1.192933in}}%
\pgfpathlineto{\pgfqpoint{3.359803in}{2.037062in}}%
\pgfpathlineto{\pgfqpoint{3.360438in}{1.065598in}}%
\pgfpathlineto{\pgfqpoint{3.360756in}{1.866311in}}%
\pgfpathlineto{\pgfqpoint{3.361073in}{1.875407in}}%
\pgfpathlineto{\pgfqpoint{3.361391in}{1.070306in}}%
\pgfpathlineto{\pgfqpoint{3.362025in}{2.045915in}}%
\pgfpathlineto{\pgfqpoint{3.362343in}{1.201864in}}%
\pgfpathlineto{\pgfqpoint{3.362660in}{1.353990in}}%
\pgfpathlineto{\pgfqpoint{3.362978in}{2.092253in}}%
\pgfpathlineto{\pgfqpoint{3.363613in}{1.148599in}}%
\pgfpathlineto{\pgfqpoint{3.363930in}{2.005491in}}%
\pgfpathlineto{\pgfqpoint{3.364565in}{1.057960in}}%
\pgfpathlineto{\pgfqpoint{3.364883in}{1.812775in}}%
\pgfpathlineto{\pgfqpoint{3.365200in}{1.940759in}}%
\pgfpathlineto{\pgfqpoint{3.365517in}{1.097166in}}%
\pgfpathlineto{\pgfqpoint{3.366152in}{2.070307in}}%
\pgfpathlineto{\pgfqpoint{3.366470in}{1.252777in}}%
\pgfpathlineto{\pgfqpoint{3.366787in}{1.287072in}}%
\pgfpathlineto{\pgfqpoint{3.367105in}{2.074636in}}%
\pgfpathlineto{\pgfqpoint{3.367740in}{1.106644in}}%
\pgfpathlineto{\pgfqpoint{3.368057in}{1.954905in}}%
\pgfpathlineto{\pgfqpoint{3.368692in}{1.047141in}}%
\pgfpathlineto{\pgfqpoint{3.369010in}{1.735406in}}%
\pgfpathlineto{\pgfqpoint{3.369327in}{1.975900in}}%
\pgfpathlineto{\pgfqpoint{3.369644in}{1.123036in}}%
\pgfpathlineto{\pgfqpoint{3.370279in}{2.084245in}}%
\pgfpathlineto{\pgfqpoint{3.370597in}{1.312272in}}%
\pgfpathlineto{\pgfqpoint{3.370914in}{1.229929in}}%
\pgfpathlineto{\pgfqpoint{3.371232in}{2.058153in}}%
\pgfpathlineto{\pgfqpoint{3.371867in}{1.081306in}}%
\pgfpathlineto{\pgfqpoint{3.372184in}{1.911348in}}%
\pgfpathlineto{\pgfqpoint{3.372502in}{1.839159in}}%
\pgfpathlineto{\pgfqpoint{3.372819in}{1.059096in}}%
\pgfpathlineto{\pgfqpoint{3.373454in}{2.020176in}}%
\pgfpathlineto{\pgfqpoint{3.373771in}{1.163181in}}%
\pgfpathlineto{\pgfqpoint{3.374406in}{2.084560in}}%
\pgfpathlineto{\pgfqpoint{3.374724in}{1.372411in}}%
\pgfpathlineto{\pgfqpoint{3.375041in}{1.173888in}}%
\pgfpathlineto{\pgfqpoint{3.375359in}{2.022726in}}%
\pgfpathlineto{\pgfqpoint{3.375994in}{1.055703in}}%
\pgfpathlineto{\pgfqpoint{3.376311in}{1.843595in}}%
\pgfpathlineto{\pgfqpoint{3.376629in}{1.883838in}}%
\pgfpathlineto{\pgfqpoint{3.376946in}{1.067137in}}%
\pgfpathlineto{\pgfqpoint{3.377581in}{2.045502in}}%
\pgfpathlineto{\pgfqpoint{3.377898in}{1.209002in}}%
\pgfpathlineto{\pgfqpoint{3.378216in}{1.330283in}}%
\pgfpathlineto{\pgfqpoint{3.378533in}{2.083954in}}%
\pgfpathlineto{\pgfqpoint{3.379168in}{1.131677in}}%
\pgfpathlineto{\pgfqpoint{3.379486in}{1.990270in}}%
\pgfpathlineto{\pgfqpoint{3.380121in}{1.051034in}}%
\pgfpathlineto{\pgfqpoint{3.380438in}{1.788897in}}%
\pgfpathlineto{\pgfqpoint{3.380755in}{1.946154in}}%
\pgfpathlineto{\pgfqpoint{3.381073in}{1.095220in}}%
\pgfpathlineto{\pgfqpoint{3.381708in}{2.066153in}}%
\pgfpathlineto{\pgfqpoint{3.382025in}{1.260532in}}%
\pgfpathlineto{\pgfqpoint{3.382343in}{1.264621in}}%
\pgfpathlineto{\pgfqpoint{3.382660in}{2.063752in}}%
\pgfpathlineto{\pgfqpoint{3.383295in}{1.091795in}}%
\pgfpathlineto{\pgfqpoint{3.383613in}{1.937445in}}%
\pgfpathlineto{\pgfqpoint{3.384247in}{1.042292in}}%
\pgfpathlineto{\pgfqpoint{3.384565in}{1.710841in}}%
\pgfpathlineto{\pgfqpoint{3.384882in}{1.980448in}}%
\pgfpathlineto{\pgfqpoint{3.385200in}{1.123927in}}%
\pgfpathlineto{\pgfqpoint{3.385835in}{2.079310in}}%
\pgfpathlineto{\pgfqpoint{3.386152in}{1.321384in}}%
\pgfpathlineto{\pgfqpoint{3.386470in}{1.209916in}}%
\pgfpathlineto{\pgfqpoint{3.386787in}{2.045322in}}%
\pgfpathlineto{\pgfqpoint{3.387422in}{1.069249in}}%
\pgfpathlineto{\pgfqpoint{3.387740in}{1.893032in}}%
\pgfpathlineto{\pgfqpoint{3.388057in}{1.845868in}}%
\pgfpathlineto{\pgfqpoint{3.388374in}{1.056091in}}%
\pgfpathlineto{\pgfqpoint{3.389009in}{2.020898in}}%
\pgfpathlineto{\pgfqpoint{3.389327in}{1.164958in}}%
\pgfpathlineto{\pgfqpoint{3.389962in}{2.076940in}}%
\pgfpathlineto{\pgfqpoint{3.390279in}{1.381974in}}%
\pgfpathlineto{\pgfqpoint{3.390597in}{1.156100in}}%
\pgfpathlineto{\pgfqpoint{3.390914in}{2.007736in}}%
\pgfpathlineto{\pgfqpoint{3.391549in}{1.044971in}}%
\pgfpathlineto{\pgfqpoint{3.391866in}{1.823809in}}%
\pgfpathlineto{\pgfqpoint{3.392184in}{1.889284in}}%
\pgfpathlineto{\pgfqpoint{3.392501in}{1.066764in}}%
\pgfpathlineto{\pgfqpoint{3.393136in}{2.045634in}}%
\pgfpathlineto{\pgfqpoint{3.393454in}{1.213194in}}%
\pgfpathlineto{\pgfqpoint{3.393771in}{1.309569in}}%
\pgfpathlineto{\pgfqpoint{3.394089in}{2.074782in}}%
\pgfpathlineto{\pgfqpoint{3.394724in}{1.116658in}}%
\pgfpathlineto{\pgfqpoint{3.395041in}{1.974288in}}%
\pgfpathlineto{\pgfqpoint{3.395676in}{1.042542in}}%
\pgfpathlineto{\pgfqpoint{3.395993in}{1.768026in}}%
\pgfpathlineto{\pgfqpoint{3.396311in}{1.948303in}}%
\pgfpathlineto{\pgfqpoint{3.396628in}{1.095916in}}%
\pgfpathlineto{\pgfqpoint{3.397263in}{2.062667in}}%
\pgfpathlineto{\pgfqpoint{3.397581in}{1.265505in}}%
\pgfpathlineto{\pgfqpoint{3.397898in}{1.244851in}}%
\pgfpathlineto{\pgfqpoint{3.398216in}{2.052876in}}%
\pgfpathlineto{\pgfqpoint{3.398851in}{1.078809in}}%
\pgfpathlineto{\pgfqpoint{3.399168in}{1.918870in}}%
\pgfpathlineto{\pgfqpoint{3.399485in}{1.779837in}}%
\pgfpathlineto{\pgfqpoint{3.399803in}{1.035533in}}%
\pgfpathlineto{\pgfqpoint{3.400438in}{1.982058in}}%
\pgfpathlineto{\pgfqpoint{3.400755in}{1.127228in}}%
\pgfpathlineto{\pgfqpoint{3.401390in}{2.074808in}}%
\pgfpathlineto{\pgfqpoint{3.401708in}{1.327781in}}%
\pgfpathlineto{\pgfqpoint{3.402025in}{1.191865in}}%
\pgfpathlineto{\pgfqpoint{3.402343in}{2.033185in}}%
\pgfpathlineto{\pgfqpoint{3.402977in}{1.059141in}}%
\pgfpathlineto{\pgfqpoint{3.403295in}{1.873337in}}%
\pgfpathlineto{\pgfqpoint{3.403612in}{1.852469in}}%
\pgfpathlineto{\pgfqpoint{3.403930in}{1.050699in}}%
\pgfpathlineto{\pgfqpoint{3.404565in}{2.019076in}}%
\pgfpathlineto{\pgfqpoint{3.404882in}{1.168994in}}%
\pgfpathlineto{\pgfqpoint{3.405517in}{2.069721in}}%
\pgfpathlineto{\pgfqpoint{3.405835in}{1.389106in}}%
\pgfpathlineto{\pgfqpoint{3.406152in}{1.139763in}}%
\pgfpathlineto{\pgfqpoint{3.406470in}{1.993848in}}%
\pgfpathlineto{\pgfqpoint{3.407104in}{1.036584in}}%
\pgfpathlineto{\pgfqpoint{3.407422in}{1.802654in}}%
\pgfpathlineto{\pgfqpoint{3.407739in}{1.894874in}}%
\pgfpathlineto{\pgfqpoint{3.408057in}{1.063731in}}%
\pgfpathlineto{\pgfqpoint{3.408692in}{2.043301in}}%
\pgfpathlineto{\pgfqpoint{3.409009in}{1.219052in}}%
\pgfpathlineto{\pgfqpoint{3.409327in}{1.288772in}}%
\pgfpathlineto{\pgfqpoint{3.409644in}{2.065704in}}%
\pgfpathlineto{\pgfqpoint{3.410279in}{1.102603in}}%
\pgfpathlineto{\pgfqpoint{3.410596in}{1.959697in}}%
\pgfpathlineto{\pgfqpoint{3.411231in}{1.036327in}}%
\pgfpathlineto{\pgfqpoint{3.411549in}{1.745644in}}%
\pgfpathlineto{\pgfqpoint{3.411866in}{1.950885in}}%
\pgfpathlineto{\pgfqpoint{3.412184in}{1.093721in}}%
\pgfpathlineto{\pgfqpoint{3.412819in}{2.057459in}}%
\pgfpathlineto{\pgfqpoint{3.413136in}{1.271896in}}%
\pgfpathlineto{\pgfqpoint{3.413454in}{1.225524in}}%
\pgfpathlineto{\pgfqpoint{3.413771in}{2.041942in}}%
\pgfpathlineto{\pgfqpoint{3.414406in}{1.065940in}}%
\pgfpathlineto{\pgfqpoint{3.414723in}{1.902507in}}%
\pgfpathlineto{\pgfqpoint{3.415041in}{1.785702in}}%
\pgfpathlineto{\pgfqpoint{3.415358in}{1.031394in}}%
\pgfpathlineto{\pgfqpoint{3.415993in}{1.984217in}}%
\pgfpathlineto{\pgfqpoint{3.416311in}{1.127468in}}%
\pgfpathlineto{\pgfqpoint{3.416946in}{2.068671in}}%
\pgfpathlineto{\pgfqpoint{3.417263in}{1.335239in}}%
\pgfpathlineto{\pgfqpoint{3.417581in}{1.174494in}}%
\pgfpathlineto{\pgfqpoint{3.417898in}{2.020946in}}%
\pgfpathlineto{\pgfqpoint{3.418533in}{1.048410in}}%
\pgfpathlineto{\pgfqpoint{3.418850in}{1.856037in}}%
\pgfpathlineto{\pgfqpoint{3.419168in}{1.856036in}}%
\pgfpathlineto{\pgfqpoint{3.419485in}{1.047889in}}%
\pgfpathlineto{\pgfqpoint{3.420120in}{2.018018in}}%
\pgfpathlineto{\pgfqpoint{3.420438in}{1.169967in}}%
\pgfpathlineto{\pgfqpoint{3.421073in}{2.061696in}}%
\pgfpathlineto{\pgfqpoint{3.421390in}{1.396527in}}%
\pgfpathlineto{\pgfqpoint{3.421707in}{1.124500in}}%
\pgfpathlineto{\pgfqpoint{3.422025in}{1.979374in}}%
\pgfpathlineto{\pgfqpoint{3.422660in}{1.027069in}}%
\pgfpathlineto{\pgfqpoint{3.422977in}{1.784658in}}%
\pgfpathlineto{\pgfqpoint{3.423295in}{1.897659in}}%
\pgfpathlineto{\pgfqpoint{3.423612in}{1.063284in}}%
\pgfpathlineto{\pgfqpoint{3.424247in}{2.041657in}}%
\pgfpathlineto{\pgfqpoint{3.424565in}{1.221799in}}%
\pgfpathlineto{\pgfqpoint{3.424882in}{1.270811in}}%
\pgfpathlineto{\pgfqpoint{3.425200in}{2.056099in}}%
\pgfpathlineto{\pgfqpoint{3.425834in}{1.089904in}}%
\pgfpathlineto{\pgfqpoint{3.426152in}{1.944280in}}%
\pgfpathlineto{\pgfqpoint{3.426787in}{1.028349in}}%
\pgfpathlineto{\pgfqpoint{3.427104in}{1.726633in}}%
\pgfpathlineto{\pgfqpoint{3.427422in}{1.950717in}}%
\pgfpathlineto{\pgfqpoint{3.427739in}{1.094072in}}%
\pgfpathlineto{\pgfqpoint{3.428374in}{2.053021in}}%
\pgfpathlineto{\pgfqpoint{3.428692in}{1.275350in}}%
\pgfpathlineto{\pgfqpoint{3.429009in}{1.208644in}}%
\pgfpathlineto{\pgfqpoint{3.429326in}{2.031224in}}%
\pgfpathlineto{\pgfqpoint{3.429961in}{1.054744in}}%
\pgfpathlineto{\pgfqpoint{3.430279in}{1.885128in}}%
\pgfpathlineto{\pgfqpoint{3.430596in}{1.791201in}}%
\pgfpathlineto{\pgfqpoint{3.430914in}{1.025104in}}%
\pgfpathlineto{\pgfqpoint{3.431549in}{1.983679in}}%
\pgfpathlineto{\pgfqpoint{3.431866in}{1.129892in}}%
\pgfpathlineto{\pgfqpoint{3.432501in}{2.063091in}}%
\pgfpathlineto{\pgfqpoint{3.432819in}{1.340313in}}%
\pgfpathlineto{\pgfqpoint{3.433136in}{1.159101in}}%
\pgfpathlineto{\pgfqpoint{3.433453in}{2.009202in}}%
\pgfpathlineto{\pgfqpoint{3.434088in}{1.039449in}}%
\pgfpathlineto{\pgfqpoint{3.434406in}{1.837553in}}%
\pgfpathlineto{\pgfqpoint{3.434723in}{1.859746in}}%
\pgfpathlineto{\pgfqpoint{3.435041in}{1.042558in}}%
\pgfpathlineto{\pgfqpoint{3.435676in}{2.014792in}}%
\pgfpathlineto{\pgfqpoint{3.435993in}{1.173000in}}%
\pgfpathlineto{\pgfqpoint{3.436311in}{1.312880in}}%
\pgfpathlineto{\pgfqpoint{3.436628in}{2.054119in}}%
\pgfpathlineto{\pgfqpoint{3.437263in}{1.110416in}}%
\pgfpathlineto{\pgfqpoint{3.437580in}{1.966224in}}%
\pgfpathlineto{\pgfqpoint{3.438215in}{1.019722in}}%
\pgfpathlineto{\pgfqpoint{3.438533in}{1.765398in}}%
\pgfpathlineto{\pgfqpoint{3.438850in}{1.900711in}}%
\pgfpathlineto{\pgfqpoint{3.439168in}{1.059967in}}%
\pgfpathlineto{\pgfqpoint{3.439803in}{2.037808in}}%
\pgfpathlineto{\pgfqpoint{3.440120in}{1.226013in}}%
\pgfpathlineto{\pgfqpoint{3.440437in}{1.252694in}}%
\pgfpathlineto{\pgfqpoint{3.440755in}{2.046886in}}%
\pgfpathlineto{\pgfqpoint{3.441390in}{1.077693in}}%
\pgfpathlineto{\pgfqpoint{3.441707in}{1.930406in}}%
\pgfpathlineto{\pgfqpoint{3.442342in}{1.022602in}}%
\pgfpathlineto{\pgfqpoint{3.442660in}{1.706332in}}%
\pgfpathlineto{\pgfqpoint{3.442977in}{1.951187in}}%
\pgfpathlineto{\pgfqpoint{3.443295in}{1.091383in}}%
\pgfpathlineto{\pgfqpoint{3.443930in}{2.047112in}}%
\pgfpathlineto{\pgfqpoint{3.444247in}{1.279996in}}%
\pgfpathlineto{\pgfqpoint{3.444564in}{1.192044in}}%
\pgfpathlineto{\pgfqpoint{3.444882in}{2.020501in}}%
\pgfpathlineto{\pgfqpoint{3.445517in}{1.043418in}}%
\pgfpathlineto{\pgfqpoint{3.445834in}{1.870185in}}%
\pgfpathlineto{\pgfqpoint{3.446152in}{1.794319in}}%
\pgfpathlineto{\pgfqpoint{3.446469in}{1.021258in}}%
\pgfpathlineto{\pgfqpoint{3.447104in}{1.983813in}}%
\pgfpathlineto{\pgfqpoint{3.447422in}{1.129075in}}%
\pgfpathlineto{\pgfqpoint{3.448056in}{2.056152in}}%
\pgfpathlineto{\pgfqpoint{3.448374in}{1.345865in}}%
\pgfpathlineto{\pgfqpoint{3.448691in}{1.144410in}}%
\pgfpathlineto{\pgfqpoint{3.449009in}{1.997473in}}%
\pgfpathlineto{\pgfqpoint{3.449644in}{1.029705in}}%
\pgfpathlineto{\pgfqpoint{3.449961in}{1.821797in}}%
\pgfpathlineto{\pgfqpoint{3.450279in}{1.860667in}}%
\pgfpathlineto{\pgfqpoint{3.450596in}{1.039655in}}%
\pgfpathlineto{\pgfqpoint{3.451231in}{2.012313in}}%
\pgfpathlineto{\pgfqpoint{3.451549in}{1.172758in}}%
\pgfpathlineto{\pgfqpoint{3.451866in}{1.296346in}}%
\pgfpathlineto{\pgfqpoint{3.452183in}{2.045923in}}%
\pgfpathlineto{\pgfqpoint{3.452818in}{1.097254in}}%
\pgfpathlineto{\pgfqpoint{3.453136in}{1.952627in}}%
\pgfpathlineto{\pgfqpoint{3.453771in}{1.011050in}}%
\pgfpathlineto{\pgfqpoint{3.454088in}{1.749485in}}%
\pgfpathlineto{\pgfqpoint{3.454406in}{1.901135in}}%
\pgfpathlineto{\pgfqpoint{3.454723in}{1.059046in}}%
\pgfpathlineto{\pgfqpoint{3.455358in}{2.034773in}}%
\pgfpathlineto{\pgfqpoint{3.455675in}{1.226989in}}%
\pgfpathlineto{\pgfqpoint{3.455993in}{1.237300in}}%
\pgfpathlineto{\pgfqpoint{3.456310in}{2.037529in}}%
\pgfpathlineto{\pgfqpoint{3.456945in}{1.066655in}}%
\pgfpathlineto{\pgfqpoint{3.457263in}{1.915930in}}%
\pgfpathlineto{\pgfqpoint{3.457898in}{1.014949in}}%
\pgfpathlineto{\pgfqpoint{3.458215in}{1.689616in}}%
\pgfpathlineto{\pgfqpoint{3.458533in}{1.949139in}}%
\pgfpathlineto{\pgfqpoint{3.458850in}{1.091045in}}%
\pgfpathlineto{\pgfqpoint{3.459485in}{2.041989in}}%
\pgfpathlineto{\pgfqpoint{3.459802in}{1.281558in}}%
\pgfpathlineto{\pgfqpoint{3.460120in}{1.177701in}}%
\pgfpathlineto{\pgfqpoint{3.460437in}{2.010194in}}%
\pgfpathlineto{\pgfqpoint{3.461072in}{1.033618in}}%
\pgfpathlineto{\pgfqpoint{3.461390in}{1.854358in}}%
\pgfpathlineto{\pgfqpoint{3.461707in}{1.797153in}}%
\pgfpathlineto{\pgfqpoint{3.462025in}{1.015078in}}%
\pgfpathlineto{\pgfqpoint{3.462660in}{1.981454in}}%
\pgfpathlineto{\pgfqpoint{3.462977in}{1.130268in}}%
\pgfpathlineto{\pgfqpoint{3.463612in}{2.049901in}}%
\pgfpathlineto{\pgfqpoint{3.463929in}{1.348771in}}%
\pgfpathlineto{\pgfqpoint{3.464247in}{1.131482in}}%
\pgfpathlineto{\pgfqpoint{3.464564in}{1.986524in}}%
\pgfpathlineto{\pgfqpoint{3.465199in}{1.021646in}}%
\pgfpathlineto{\pgfqpoint{3.465517in}{1.805009in}}%
\pgfpathlineto{\pgfqpoint{3.465834in}{1.861677in}}%
\pgfpathlineto{\pgfqpoint{3.466152in}{1.033988in}}%
\pgfpathlineto{\pgfqpoint{3.466786in}{2.007785in}}%
\pgfpathlineto{\pgfqpoint{3.467104in}{1.174365in}}%
\pgfpathlineto{\pgfqpoint{3.467421in}{1.279373in}}%
\pgfpathlineto{\pgfqpoint{3.467739in}{2.038288in}}%
\pgfpathlineto{\pgfqpoint{3.468374in}{1.085119in}}%
\pgfpathlineto{\pgfqpoint{3.468691in}{1.940590in}}%
\pgfpathlineto{\pgfqpoint{3.469326in}{1.004397in}}%
\pgfpathlineto{\pgfqpoint{3.469644in}{1.732396in}}%
\pgfpathlineto{\pgfqpoint{3.469961in}{1.901914in}}%
\pgfpathlineto{\pgfqpoint{3.470279in}{1.055090in}}%
\pgfpathlineto{\pgfqpoint{3.470913in}{2.029760in}}%
\pgfpathlineto{\pgfqpoint{3.471231in}{1.229318in}}%
\pgfpathlineto{\pgfqpoint{3.471548in}{1.221670in}}%
\pgfpathlineto{\pgfqpoint{3.471866in}{2.028722in}}%
\pgfpathlineto{\pgfqpoint{3.472501in}{1.055981in}}%
\pgfpathlineto{\pgfqpoint{3.472818in}{1.903303in}}%
\pgfpathlineto{\pgfqpoint{3.473136in}{1.751216in}}%
\pgfpathlineto{\pgfqpoint{3.473453in}{1.009405in}}%
\pgfpathlineto{\pgfqpoint{3.474088in}{1.947681in}}%
\pgfpathlineto{\pgfqpoint{3.474405in}{1.087456in}}%
\pgfpathlineto{\pgfqpoint{3.475040in}{2.035531in}}%
\pgfpathlineto{\pgfqpoint{3.475358in}{1.284169in}}%
\pgfpathlineto{\pgfqpoint{3.475675in}{1.163495in}}%
\pgfpathlineto{\pgfqpoint{3.475993in}{2.000017in}}%
\pgfpathlineto{\pgfqpoint{3.476628in}{1.023549in}}%
\pgfpathlineto{\pgfqpoint{3.476945in}{1.841158in}}%
\pgfpathlineto{\pgfqpoint{3.477263in}{1.797740in}}%
\pgfpathlineto{\pgfqpoint{3.477580in}{1.011181in}}%
\pgfpathlineto{\pgfqpoint{3.478215in}{1.979868in}}%
\pgfpathlineto{\pgfqpoint{3.478532in}{1.128085in}}%
\pgfpathlineto{\pgfqpoint{3.479167in}{2.042524in}}%
\pgfpathlineto{\pgfqpoint{3.479485in}{1.351969in}}%
\pgfpathlineto{\pgfqpoint{3.479802in}{1.119041in}}%
\pgfpathlineto{\pgfqpoint{3.480120in}{1.975650in}}%
\pgfpathlineto{\pgfqpoint{3.480755in}{1.012633in}}%
\pgfpathlineto{\pgfqpoint{3.481072in}{1.791125in}}%
\pgfpathlineto{\pgfqpoint{3.481390in}{1.860188in}}%
\pgfpathlineto{\pgfqpoint{3.481707in}{1.030654in}}%
\pgfpathlineto{\pgfqpoint{3.482342in}{2.004121in}}%
\pgfpathlineto{\pgfqpoint{3.482659in}{1.172586in}}%
\pgfpathlineto{\pgfqpoint{3.482977in}{1.265587in}}%
\pgfpathlineto{\pgfqpoint{3.483294in}{2.030184in}}%
\pgfpathlineto{\pgfqpoint{3.483929in}{1.073747in}}%
\pgfpathlineto{\pgfqpoint{3.484247in}{1.928208in}}%
\pgfpathlineto{\pgfqpoint{3.484882in}{0.996273in}}%
\pgfpathlineto{\pgfqpoint{3.485199in}{1.718763in}}%
\pgfpathlineto{\pgfqpoint{3.485516in}{1.900229in}}%
\pgfpathlineto{\pgfqpoint{3.485834in}{1.053376in}}%
\pgfpathlineto{\pgfqpoint{3.486469in}{2.025669in}}%
\pgfpathlineto{\pgfqpoint{3.486786in}{1.228507in}}%
\pgfpathlineto{\pgfqpoint{3.487104in}{1.208694in}}%
\pgfpathlineto{\pgfqpoint{3.487421in}{2.019766in}}%
\pgfpathlineto{\pgfqpoint{3.488056in}{1.046300in}}%
\pgfpathlineto{\pgfqpoint{3.488374in}{1.890187in}}%
\pgfpathlineto{\pgfqpoint{3.488691in}{1.752076in}}%
\pgfpathlineto{\pgfqpoint{3.489009in}{1.001852in}}%
\pgfpathlineto{\pgfqpoint{3.489643in}{1.943980in}}%
\pgfpathlineto{\pgfqpoint{3.489961in}{1.086102in}}%
\pgfpathlineto{\pgfqpoint{3.490596in}{2.029947in}}%
\pgfpathlineto{\pgfqpoint{3.490913in}{1.283592in}}%
\pgfpathlineto{\pgfqpoint{3.491231in}{1.151436in}}%
\pgfpathlineto{\pgfqpoint{3.491548in}{1.990388in}}%
\pgfpathlineto{\pgfqpoint{3.492183in}{1.014847in}}%
\pgfpathlineto{\pgfqpoint{3.492501in}{1.827168in}}%
\pgfpathlineto{\pgfqpoint{3.492818in}{1.798082in}}%
\pgfpathlineto{\pgfqpoint{3.493135in}{1.004821in}}%
\pgfpathlineto{\pgfqpoint{3.493770in}{1.975962in}}%
\pgfpathlineto{\pgfqpoint{3.494088in}{1.127789in}}%
\pgfpathlineto{\pgfqpoint{3.494723in}{2.035950in}}%
\pgfpathlineto{\pgfqpoint{3.495040in}{1.352362in}}%
\pgfpathlineto{\pgfqpoint{3.495358in}{1.108100in}}%
\pgfpathlineto{\pgfqpoint{3.495675in}{1.965523in}}%
\pgfpathlineto{\pgfqpoint{3.496310in}{1.005187in}}%
\pgfpathlineto{\pgfqpoint{3.496627in}{1.776343in}}%
\pgfpathlineto{\pgfqpoint{3.496945in}{1.858950in}}%
\pgfpathlineto{\pgfqpoint{3.497262in}{1.024515in}}%
\pgfpathlineto{\pgfqpoint{3.497897in}{1.998667in}}%
\pgfpathlineto{\pgfqpoint{3.498215in}{1.172552in}}%
\pgfpathlineto{\pgfqpoint{3.498532in}{1.251259in}}%
\pgfpathlineto{\pgfqpoint{3.498850in}{2.022699in}}%
\pgfpathlineto{\pgfqpoint{3.499485in}{1.063279in}}%
\pgfpathlineto{\pgfqpoint{3.499802in}{1.917487in}}%
\pgfpathlineto{\pgfqpoint{3.500437in}{0.990009in}}%
\pgfpathlineto{\pgfqpoint{3.500754in}{1.704016in}}%
\pgfpathlineto{\pgfqpoint{3.501072in}{1.898957in}}%
\pgfpathlineto{\pgfqpoint{3.501389in}{1.048510in}}%
\pgfpathlineto{\pgfqpoint{3.502024in}{2.019765in}}%
\pgfpathlineto{\pgfqpoint{3.502342in}{1.229039in}}%
\pgfpathlineto{\pgfqpoint{3.502659in}{1.195432in}}%
\pgfpathlineto{\pgfqpoint{3.502977in}{2.011347in}}%
\pgfpathlineto{\pgfqpoint{3.503612in}{1.036825in}}%
\pgfpathlineto{\pgfqpoint{3.503929in}{1.879028in}}%
\pgfpathlineto{\pgfqpoint{3.504246in}{1.750862in}}%
\pgfpathlineto{\pgfqpoint{3.504564in}{0.996359in}}%
\pgfpathlineto{\pgfqpoint{3.505199in}{1.941007in}}%
\pgfpathlineto{\pgfqpoint{3.505516in}{1.081420in}}%
\pgfpathlineto{\pgfqpoint{3.506151in}{2.023160in}}%
\pgfpathlineto{\pgfqpoint{3.506469in}{1.283977in}}%
\pgfpathlineto{\pgfqpoint{3.506786in}{1.139349in}}%
\pgfpathlineto{\pgfqpoint{3.507104in}{1.980967in}}%
\pgfpathlineto{\pgfqpoint{3.507739in}{1.005782in}}%
\pgfpathlineto{\pgfqpoint{3.508056in}{1.815897in}}%
\pgfpathlineto{\pgfqpoint{3.508373in}{1.796314in}}%
\pgfpathlineto{\pgfqpoint{3.508691in}{1.000600in}}%
\pgfpathlineto{\pgfqpoint{3.509326in}{1.972896in}}%
\pgfpathlineto{\pgfqpoint{3.509643in}{1.124010in}}%
\pgfpathlineto{\pgfqpoint{3.510278in}{2.028416in}}%
\pgfpathlineto{\pgfqpoint{3.510596in}{1.352772in}}%
\pgfpathlineto{\pgfqpoint{3.510913in}{1.097363in}}%
\pgfpathlineto{\pgfqpoint{3.511231in}{1.955483in}}%
\pgfpathlineto{\pgfqpoint{3.511865in}{0.996757in}}%
\pgfpathlineto{\pgfqpoint{3.512183in}{1.764658in}}%
\pgfpathlineto{\pgfqpoint{3.512500in}{1.855588in}}%
\pgfpathlineto{\pgfqpoint{3.512818in}{1.020667in}}%
\pgfpathlineto{\pgfqpoint{3.513453in}{1.994177in}}%
\pgfpathlineto{\pgfqpoint{3.513770in}{1.169026in}}%
\pgfpathlineto{\pgfqpoint{3.514088in}{1.240061in}}%
\pgfpathlineto{\pgfqpoint{3.514405in}{2.014800in}}%
\pgfpathlineto{\pgfqpoint{3.515040in}{1.053395in}}%
\pgfpathlineto{\pgfqpoint{3.515357in}{1.906504in}}%
\pgfpathlineto{\pgfqpoint{3.515992in}{0.982202in}}%
\pgfpathlineto{\pgfqpoint{3.516310in}{1.692790in}}%
\pgfpathlineto{\pgfqpoint{3.516627in}{1.895369in}}%
\pgfpathlineto{\pgfqpoint{3.516945in}{1.045761in}}%
\pgfpathlineto{\pgfqpoint{3.517580in}{2.014862in}}%
\pgfpathlineto{\pgfqpoint{3.517897in}{1.226163in}}%
\pgfpathlineto{\pgfqpoint{3.518215in}{1.184865in}}%
\pgfpathlineto{\pgfqpoint{3.518532in}{2.002932in}}%
\pgfpathlineto{\pgfqpoint{3.519167in}{1.028282in}}%
\pgfpathlineto{\pgfqpoint{3.519484in}{1.867567in}}%
\pgfpathlineto{\pgfqpoint{3.519802in}{1.749104in}}%
\pgfpathlineto{\pgfqpoint{3.520119in}{0.988733in}}%
\pgfpathlineto{\pgfqpoint{3.520754in}{1.935843in}}%
\pgfpathlineto{\pgfqpoint{3.521072in}{1.078813in}}%
\pgfpathlineto{\pgfqpoint{3.521707in}{2.017326in}}%
\pgfpathlineto{\pgfqpoint{3.522024in}{1.281089in}}%
\pgfpathlineto{\pgfqpoint{3.522342in}{1.129375in}}%
\pgfpathlineto{\pgfqpoint{3.522659in}{1.972202in}}%
\pgfpathlineto{\pgfqpoint{3.523294in}{0.997943in}}%
\pgfpathlineto{\pgfqpoint{3.523611in}{1.803941in}}%
\pgfpathlineto{\pgfqpoint{3.523929in}{1.794305in}}%
\pgfpathlineto{\pgfqpoint{3.524246in}{0.993847in}}%
\pgfpathlineto{\pgfqpoint{3.524881in}{1.967653in}}%
\pgfpathlineto{\pgfqpoint{3.525199in}{1.122048in}}%
\pgfpathlineto{\pgfqpoint{3.525834in}{2.021795in}}%
\pgfpathlineto{\pgfqpoint{3.526151in}{1.350398in}}%
\pgfpathlineto{\pgfqpoint{3.526469in}{1.088114in}}%
\pgfpathlineto{\pgfqpoint{3.526786in}{1.946363in}}%
\pgfpathlineto{\pgfqpoint{3.527421in}{0.989801in}}%
\pgfpathlineto{\pgfqpoint{3.527738in}{1.752166in}}%
\pgfpathlineto{\pgfqpoint{3.528056in}{1.852399in}}%
\pgfpathlineto{\pgfqpoint{3.528373in}{1.013905in}}%
\pgfpathlineto{\pgfqpoint{3.529008in}{1.987960in}}%
\pgfpathlineto{\pgfqpoint{3.529326in}{1.167150in}}%
\pgfpathlineto{\pgfqpoint{3.529643in}{1.228183in}}%
\pgfpathlineto{\pgfqpoint{3.529961in}{2.007624in}}%
\pgfpathlineto{\pgfqpoint{3.530595in}{1.044411in}}%
\pgfpathlineto{\pgfqpoint{3.530913in}{1.897274in}}%
\pgfpathlineto{\pgfqpoint{3.531548in}{0.976116in}}%
\pgfpathlineto{\pgfqpoint{3.531865in}{1.680508in}}%
\pgfpathlineto{\pgfqpoint{3.532183in}{1.892214in}}%
\pgfpathlineto{\pgfqpoint{3.532500in}{1.039790in}}%
\pgfpathlineto{\pgfqpoint{3.533135in}{2.008275in}}%
\pgfpathlineto{\pgfqpoint{3.533453in}{1.224421in}}%
\pgfpathlineto{\pgfqpoint{3.533770in}{1.173912in}}%
\pgfpathlineto{\pgfqpoint{3.534088in}{1.995286in}}%
\pgfpathlineto{\pgfqpoint{3.534722in}{1.019978in}}%
\pgfpathlineto{\pgfqpoint{3.535040in}{1.858178in}}%
\pgfpathlineto{\pgfqpoint{3.535357in}{1.745389in}}%
\pgfpathlineto{\pgfqpoint{3.535675in}{0.983000in}}%
\pgfpathlineto{\pgfqpoint{3.536310in}{1.931309in}}%
\pgfpathlineto{\pgfqpoint{3.536627in}{1.072769in}}%
\pgfpathlineto{\pgfqpoint{3.537262in}{2.010348in}}%
\pgfpathlineto{\pgfqpoint{3.537580in}{1.279166in}}%
\pgfpathlineto{\pgfqpoint{3.537897in}{1.119249in}}%
\pgfpathlineto{\pgfqpoint{3.538214in}{1.963798in}}%
\pgfpathlineto{\pgfqpoint{3.538849in}{0.989749in}}%
\pgfpathlineto{\pgfqpoint{3.539167in}{1.794757in}}%
\pgfpathlineto{\pgfqpoint{3.539484in}{1.790323in}}%
\pgfpathlineto{\pgfqpoint{3.539802in}{0.989104in}}%
\pgfpathlineto{\pgfqpoint{3.540437in}{1.963297in}}%
\pgfpathlineto{\pgfqpoint{3.540754in}{1.116538in}}%
\pgfpathlineto{\pgfqpoint{3.541389in}{2.014308in}}%
\pgfpathlineto{\pgfqpoint{3.541706in}{1.348215in}}%
\pgfpathlineto{\pgfqpoint{3.542024in}{1.079004in}}%
\pgfpathlineto{\pgfqpoint{3.542341in}{1.937563in}}%
\pgfpathlineto{\pgfqpoint{3.542976in}{0.981864in}}%
\pgfpathlineto{\pgfqpoint{3.543294in}{1.742805in}}%
\pgfpathlineto{\pgfqpoint{3.543611in}{1.847170in}}%
\pgfpathlineto{\pgfqpoint{3.543929in}{1.009278in}}%
\pgfpathlineto{\pgfqpoint{3.544564in}{1.982686in}}%
\pgfpathlineto{\pgfqpoint{3.544881in}{1.161691in}}%
\pgfpathlineto{\pgfqpoint{3.545199in}{1.219430in}}%
\pgfpathlineto{\pgfqpoint{3.545516in}{2.000091in}}%
\pgfpathlineto{\pgfqpoint{3.546151in}{1.035879in}}%
\pgfpathlineto{\pgfqpoint{3.546468in}{1.887923in}}%
\pgfpathlineto{\pgfqpoint{3.547103in}{0.968487in}}%
\pgfpathlineto{\pgfqpoint{3.547421in}{1.671762in}}%
\pgfpathlineto{\pgfqpoint{3.547738in}{1.886874in}}%
\pgfpathlineto{\pgfqpoint{3.548056in}{1.035849in}}%
\pgfpathlineto{\pgfqpoint{3.548691in}{2.002762in}}%
\pgfpathlineto{\pgfqpoint{3.549008in}{1.219204in}}%
\pgfpathlineto{\pgfqpoint{3.549325in}{1.165574in}}%
\pgfpathlineto{\pgfqpoint{3.549643in}{1.987455in}}%
\pgfpathlineto{\pgfqpoint{3.550278in}{1.012393in}}%
\pgfpathlineto{\pgfqpoint{3.550595in}{1.848549in}}%
\pgfpathlineto{\pgfqpoint{3.550913in}{1.741221in}}%
\pgfpathlineto{\pgfqpoint{3.551230in}{0.975241in}}%
\pgfpathlineto{\pgfqpoint{3.551865in}{1.924906in}}%
\pgfpathlineto{\pgfqpoint{3.552183in}{1.068787in}}%
\pgfpathlineto{\pgfqpoint{3.552818in}{2.004365in}}%
\pgfpathlineto{\pgfqpoint{3.553135in}{1.273865in}}%
\pgfpathlineto{\pgfqpoint{3.553452in}{1.111233in}}%
\pgfpathlineto{\pgfqpoint{3.553770in}{1.956027in}}%
\pgfpathlineto{\pgfqpoint{3.554405in}{0.982616in}}%
\pgfpathlineto{\pgfqpoint{3.554722in}{1.784965in}}%
\pgfpathlineto{\pgfqpoint{3.555040in}{1.786063in}}%
\pgfpathlineto{\pgfqpoint{3.555357in}{0.981823in}}%
\pgfpathlineto{\pgfqpoint{3.555992in}{1.956866in}}%
\pgfpathlineto{\pgfqpoint{3.556310in}{1.112809in}}%
\pgfpathlineto{\pgfqpoint{3.556627in}{1.274059in}}%
\pgfpathlineto{\pgfqpoint{3.556944in}{2.007821in}}%
\pgfpathlineto{\pgfqpoint{3.557579in}{1.071365in}}%
\pgfpathlineto{\pgfqpoint{3.557897in}{1.929660in}}%
\pgfpathlineto{\pgfqpoint{3.558532in}{0.975299in}}%
\pgfpathlineto{\pgfqpoint{3.558849in}{1.732742in}}%
\pgfpathlineto{\pgfqpoint{3.559167in}{1.842123in}}%
\pgfpathlineto{\pgfqpoint{3.559484in}{1.001751in}}%
\pgfpathlineto{\pgfqpoint{3.560119in}{1.975790in}}%
\pgfpathlineto{\pgfqpoint{3.560436in}{1.157850in}}%
\pgfpathlineto{\pgfqpoint{3.560754in}{1.209869in}}%
\pgfpathlineto{\pgfqpoint{3.561071in}{1.993351in}}%
\pgfpathlineto{\pgfqpoint{3.561706in}{1.028282in}}%
\pgfpathlineto{\pgfqpoint{3.562024in}{1.880305in}}%
\pgfpathlineto{\pgfqpoint{3.562659in}{0.962441in}}%
\pgfpathlineto{\pgfqpoint{3.562976in}{1.662016in}}%
\pgfpathlineto{\pgfqpoint{3.563294in}{1.881948in}}%
\pgfpathlineto{\pgfqpoint{3.563611in}{1.028666in}}%
\pgfpathlineto{\pgfqpoint{3.564246in}{1.995630in}}%
\pgfpathlineto{\pgfqpoint{3.564563in}{1.214999in}}%
\pgfpathlineto{\pgfqpoint{3.564881in}{1.156624in}}%
\pgfpathlineto{\pgfqpoint{3.565198in}{1.980376in}}%
\pgfpathlineto{\pgfqpoint{3.565833in}{1.005085in}}%
\pgfpathlineto{\pgfqpoint{3.566151in}{1.841005in}}%
\pgfpathlineto{\pgfqpoint{3.566468in}{1.735417in}}%
\pgfpathlineto{\pgfqpoint{3.566786in}{0.969343in}}%
\pgfpathlineto{\pgfqpoint{3.567421in}{1.919196in}}%
\pgfpathlineto{\pgfqpoint{3.567738in}{1.061361in}}%
\pgfpathlineto{\pgfqpoint{3.568373in}{1.997234in}}%
\pgfpathlineto{\pgfqpoint{3.568690in}{1.269542in}}%
\pgfpathlineto{\pgfqpoint{3.569008in}{1.102901in}}%
\pgfpathlineto{\pgfqpoint{3.569325in}{1.948693in}}%
\pgfpathlineto{\pgfqpoint{3.569960in}{0.975189in}}%
\pgfpathlineto{\pgfqpoint{3.570278in}{1.777918in}}%
\pgfpathlineto{\pgfqpoint{3.570595in}{1.779971in}}%
\pgfpathlineto{\pgfqpoint{3.570913in}{0.976443in}}%
\pgfpathlineto{\pgfqpoint{3.571548in}{1.951325in}}%
\pgfpathlineto{\pgfqpoint{3.571865in}{1.105491in}}%
\pgfpathlineto{\pgfqpoint{3.572182in}{1.267358in}}%
\pgfpathlineto{\pgfqpoint{3.572500in}{2.000495in}}%
\pgfpathlineto{\pgfqpoint{3.573135in}{1.063857in}}%
\pgfpathlineto{\pgfqpoint{3.573452in}{1.922326in}}%
\pgfpathlineto{\pgfqpoint{3.574087in}{0.967772in}}%
\pgfpathlineto{\pgfqpoint{3.574405in}{1.725720in}}%
\pgfpathlineto{\pgfqpoint{3.574722in}{1.835024in}}%
\pgfpathlineto{\pgfqpoint{3.575040in}{0.996191in}}%
\pgfpathlineto{\pgfqpoint{3.575674in}{1.969828in}}%
\pgfpathlineto{\pgfqpoint{3.575992in}{1.150375in}}%
\pgfpathlineto{\pgfqpoint{3.576309in}{1.203485in}}%
\pgfpathlineto{\pgfqpoint{3.576627in}{1.986267in}}%
\pgfpathlineto{\pgfqpoint{3.577262in}{1.021006in}}%
\pgfpathlineto{\pgfqpoint{3.577579in}{1.872698in}}%
\pgfpathlineto{\pgfqpoint{3.578214in}{0.954925in}}%
\pgfpathlineto{\pgfqpoint{3.578532in}{1.655778in}}%
\pgfpathlineto{\pgfqpoint{3.578849in}{1.874955in}}%
\pgfpathlineto{\pgfqpoint{3.579166in}{1.023454in}}%
\pgfpathlineto{\pgfqpoint{3.579801in}{1.989619in}}%
\pgfpathlineto{\pgfqpoint{3.580119in}{1.207340in}}%
\pgfpathlineto{\pgfqpoint{3.580436in}{1.150367in}}%
\pgfpathlineto{\pgfqpoint{3.580754in}{1.973169in}}%
\pgfpathlineto{\pgfqpoint{3.581389in}{0.998410in}}%
\pgfpathlineto{\pgfqpoint{3.581706in}{1.833362in}}%
\pgfpathlineto{\pgfqpoint{3.582024in}{1.729000in}}%
\pgfpathlineto{\pgfqpoint{3.582341in}{0.961447in}}%
\pgfpathlineto{\pgfqpoint{3.582976in}{1.911657in}}%
\pgfpathlineto{\pgfqpoint{3.583293in}{1.055923in}}%
\pgfpathlineto{\pgfqpoint{3.583928in}{1.991137in}}%
\pgfpathlineto{\pgfqpoint{3.584246in}{1.261775in}}%
\pgfpathlineto{\pgfqpoint{3.584563in}{1.096782in}}%
\pgfpathlineto{\pgfqpoint{3.584881in}{1.941974in}}%
\pgfpathlineto{\pgfqpoint{3.585516in}{0.968681in}}%
\pgfpathlineto{\pgfqpoint{3.585833in}{1.770362in}}%
\pgfpathlineto{\pgfqpoint{3.586151in}{1.773520in}}%
\pgfpathlineto{\pgfqpoint{3.586468in}{0.968586in}}%
\pgfpathlineto{\pgfqpoint{3.587103in}{1.943794in}}%
\pgfpathlineto{\pgfqpoint{3.587420in}{1.099957in}}%
\pgfpathlineto{\pgfqpoint{3.587738in}{1.259582in}}%
\pgfpathlineto{\pgfqpoint{3.588055in}{1.994237in}}%
\pgfpathlineto{\pgfqpoint{3.588690in}{1.057926in}}%
\pgfpathlineto{\pgfqpoint{3.589008in}{1.915848in}}%
\pgfpathlineto{\pgfqpoint{3.589643in}{0.961490in}}%
\pgfpathlineto{\pgfqpoint{3.589960in}{1.718067in}}%
\pgfpathlineto{\pgfqpoint{3.590278in}{1.828007in}}%
\pgfpathlineto{\pgfqpoint{3.590595in}{0.987779in}}%
\pgfpathlineto{\pgfqpoint{3.591230in}{1.962258in}}%
\pgfpathlineto{\pgfqpoint{3.591547in}{1.144529in}}%
\pgfpathlineto{\pgfqpoint{3.591865in}{1.196184in}}%
\pgfpathlineto{\pgfqpoint{3.592182in}{1.980070in}}%
\pgfpathlineto{\pgfqpoint{3.592817in}{1.014798in}}%
\pgfpathlineto{\pgfqpoint{3.593135in}{1.866766in}}%
\pgfpathlineto{\pgfqpoint{3.593770in}{0.948857in}}%
\pgfpathlineto{\pgfqpoint{3.594087in}{1.648600in}}%
\pgfpathlineto{\pgfqpoint{3.594404in}{1.868316in}}%
\pgfpathlineto{\pgfqpoint{3.594722in}{1.015037in}}%
\pgfpathlineto{\pgfqpoint{3.595357in}{1.982018in}}%
\pgfpathlineto{\pgfqpoint{3.595674in}{1.200855in}}%
\pgfpathlineto{\pgfqpoint{3.595992in}{1.143398in}}%
\pgfpathlineto{\pgfqpoint{3.596309in}{1.966857in}}%
\pgfpathlineto{\pgfqpoint{3.596944in}{0.992164in}}%
\pgfpathlineto{\pgfqpoint{3.597262in}{1.827752in}}%
\pgfpathlineto{\pgfqpoint{3.597579in}{1.721099in}}%
\pgfpathlineto{\pgfqpoint{3.597896in}{0.955290in}}%
\pgfpathlineto{\pgfqpoint{3.598531in}{1.904746in}}%
\pgfpathlineto{\pgfqpoint{3.598849in}{1.047053in}}%
\pgfpathlineto{\pgfqpoint{3.599484in}{1.983862in}}%
\pgfpathlineto{\pgfqpoint{3.599801in}{1.255062in}}%
\pgfpathlineto{\pgfqpoint{3.600119in}{1.090205in}}%
\pgfpathlineto{\pgfqpoint{3.600436in}{1.935792in}}%
\pgfpathlineto{\pgfqpoint{3.601071in}{0.962022in}}%
\pgfpathlineto{\pgfqpoint{3.601389in}{1.765467in}}%
\pgfpathlineto{\pgfqpoint{3.601706in}{1.765378in}}%
\pgfpathlineto{\pgfqpoint{3.602023in}{0.962535in}}%
\pgfpathlineto{\pgfqpoint{3.602658in}{1.937127in}}%
\pgfpathlineto{\pgfqpoint{3.602976in}{1.090832in}}%
\pgfpathlineto{\pgfqpoint{3.603611in}{1.987107in}}%
\pgfpathlineto{\pgfqpoint{3.603928in}{1.323310in}}%
\pgfpathlineto{\pgfqpoint{3.604246in}{1.051750in}}%
\pgfpathlineto{\pgfqpoint{3.604563in}{1.909818in}}%
\pgfpathlineto{\pgfqpoint{3.605198in}{0.954516in}}%
\pgfpathlineto{\pgfqpoint{3.605515in}{1.713556in}}%
\pgfpathlineto{\pgfqpoint{3.605833in}{1.819444in}}%
\pgfpathlineto{\pgfqpoint{3.606150in}{0.981425in}}%
\pgfpathlineto{\pgfqpoint{3.606785in}{1.955662in}}%
\pgfpathlineto{\pgfqpoint{3.607103in}{1.135007in}}%
\pgfpathlineto{\pgfqpoint{3.607420in}{1.192121in}}%
\pgfpathlineto{\pgfqpoint{3.607738in}{1.973384in}}%
\pgfpathlineto{\pgfqpoint{3.608373in}{1.008707in}}%
\pgfpathlineto{\pgfqpoint{3.608690in}{1.860890in}}%
\pgfpathlineto{\pgfqpoint{3.609325in}{0.941465in}}%
\pgfpathlineto{\pgfqpoint{3.609642in}{1.644861in}}%
\pgfpathlineto{\pgfqpoint{3.609960in}{1.859723in}}%
\pgfpathlineto{\pgfqpoint{3.610277in}{1.008538in}}%
\pgfpathlineto{\pgfqpoint{3.610912in}{1.975553in}}%
\pgfpathlineto{\pgfqpoint{3.611230in}{1.190971in}}%
\pgfpathlineto{\pgfqpoint{3.611547in}{1.139316in}}%
\pgfpathlineto{\pgfqpoint{3.611865in}{1.960470in}}%
\pgfpathlineto{\pgfqpoint{3.612500in}{0.986440in}}%
\pgfpathlineto{\pgfqpoint{3.612817in}{1.822154in}}%
\pgfpathlineto{\pgfqpoint{3.613134in}{1.712316in}}%
\pgfpathlineto{\pgfqpoint{3.613452in}{0.947204in}}%
\pgfpathlineto{\pgfqpoint{3.614087in}{1.896033in}}%
\pgfpathlineto{\pgfqpoint{3.614404in}{1.040137in}}%
\pgfpathlineto{\pgfqpoint{3.615039in}{1.977651in}}%
\pgfpathlineto{\pgfqpoint{3.615357in}{1.244849in}}%
\pgfpathlineto{\pgfqpoint{3.615674in}{1.086003in}}%
\pgfpathlineto{\pgfqpoint{3.615992in}{1.930138in}}%
\pgfpathlineto{\pgfqpoint{3.616626in}{0.956135in}}%
\pgfpathlineto{\pgfqpoint{3.616944in}{1.760162in}}%
\pgfpathlineto{\pgfqpoint{3.617261in}{1.756748in}}%
\pgfpathlineto{\pgfqpoint{3.617579in}{0.954139in}}%
\pgfpathlineto{\pgfqpoint{3.618214in}{1.928532in}}%
\pgfpathlineto{\pgfqpoint{3.618531in}{1.083516in}}%
\pgfpathlineto{\pgfqpoint{3.619166in}{1.981088in}}%
\pgfpathlineto{\pgfqpoint{3.619484in}{1.312724in}}%
\pgfpathlineto{\pgfqpoint{3.619801in}{1.047334in}}%
\pgfpathlineto{\pgfqpoint{3.620119in}{1.904555in}}%
\pgfpathlineto{\pgfqpoint{3.620753in}{0.948667in}}%
\pgfpathlineto{\pgfqpoint{3.621071in}{1.708494in}}%
\pgfpathlineto{\pgfqpoint{3.621388in}{1.810818in}}%
\pgfpathlineto{\pgfqpoint{3.621706in}{0.972334in}}%
\pgfpathlineto{\pgfqpoint{3.622341in}{1.947454in}}%
\pgfpathlineto{\pgfqpoint{3.622658in}{1.127157in}}%
\pgfpathlineto{\pgfqpoint{3.622976in}{1.187036in}}%
\pgfpathlineto{\pgfqpoint{3.623293in}{1.967650in}}%
\pgfpathlineto{\pgfqpoint{3.623928in}{1.003877in}}%
\pgfpathlineto{\pgfqpoint{3.624245in}{1.856546in}}%
\pgfpathlineto{\pgfqpoint{3.624880in}{0.935389in}}%
\pgfpathlineto{\pgfqpoint{3.625198in}{1.640250in}}%
\pgfpathlineto{\pgfqpoint{3.625515in}{1.851382in}}%
\pgfpathlineto{\pgfqpoint{3.625833in}{0.998933in}}%
\pgfpathlineto{\pgfqpoint{3.626468in}{1.967484in}}%
\pgfpathlineto{\pgfqpoint{3.626785in}{1.182383in}}%
\pgfpathlineto{\pgfqpoint{3.627103in}{1.134386in}}%
\pgfpathlineto{\pgfqpoint{3.627420in}{1.955019in}}%
\pgfpathlineto{\pgfqpoint{3.628055in}{0.981343in}}%
\pgfpathlineto{\pgfqpoint{3.628372in}{1.818455in}}%
\pgfpathlineto{\pgfqpoint{3.628690in}{1.702274in}}%
\pgfpathlineto{\pgfqpoint{3.629007in}{0.940790in}}%
\pgfpathlineto{\pgfqpoint{3.629642in}{1.887868in}}%
\pgfpathlineto{\pgfqpoint{3.629960in}{1.029861in}}%
\pgfpathlineto{\pgfqpoint{3.630595in}{1.970203in}}%
\pgfpathlineto{\pgfqpoint{3.630912in}{1.235801in}}%
\pgfpathlineto{\pgfqpoint{3.631230in}{1.081206in}}%
\pgfpathlineto{\pgfqpoint{3.631547in}{1.925109in}}%
\pgfpathlineto{\pgfqpoint{3.632182in}{0.950313in}}%
\pgfpathlineto{\pgfqpoint{3.632499in}{1.757365in}}%
\pgfpathlineto{\pgfqpoint{3.632817in}{1.746570in}}%
\pgfpathlineto{\pgfqpoint{3.633134in}{0.947457in}}%
\pgfpathlineto{\pgfqpoint{3.633769in}{1.920734in}}%
\pgfpathlineto{\pgfqpoint{3.634087in}{1.072661in}}%
\pgfpathlineto{\pgfqpoint{3.634722in}{1.974106in}}%
\pgfpathlineto{\pgfqpoint{3.635039in}{1.302828in}}%
\pgfpathlineto{\pgfqpoint{3.635356in}{1.042730in}}%
\pgfpathlineto{\pgfqpoint{3.635674in}{1.900023in}}%
\pgfpathlineto{\pgfqpoint{3.636309in}{0.942254in}}%
\pgfpathlineto{\pgfqpoint{3.636626in}{1.706319in}}%
\pgfpathlineto{\pgfqpoint{3.636944in}{1.800567in}}%
\pgfpathlineto{\pgfqpoint{3.637261in}{0.965178in}}%
\pgfpathlineto{\pgfqpoint{3.637896in}{1.940127in}}%
\pgfpathlineto{\pgfqpoint{3.638214in}{1.115657in}}%
\pgfpathlineto{\pgfqpoint{3.638531in}{1.185337in}}%
\pgfpathlineto{\pgfqpoint{3.638849in}{1.961361in}}%
\pgfpathlineto{\pgfqpoint{3.639483in}{0.999036in}}%
\pgfpathlineto{\pgfqpoint{3.639801in}{1.852390in}}%
\pgfpathlineto{\pgfqpoint{3.640436in}{0.928211in}}%
\pgfpathlineto{\pgfqpoint{3.640753in}{1.638946in}}%
\pgfpathlineto{\pgfqpoint{3.641071in}{1.841189in}}%
\pgfpathlineto{\pgfqpoint{3.641388in}{0.991238in}}%
\pgfpathlineto{\pgfqpoint{3.642023in}{1.960508in}}%
\pgfpathlineto{\pgfqpoint{3.642341in}{1.170220in}}%
\pgfpathlineto{\pgfqpoint{3.642658in}{1.132357in}}%
\pgfpathlineto{\pgfqpoint{3.642975in}{1.949156in}}%
\pgfpathlineto{\pgfqpoint{3.643610in}{0.976590in}}%
\pgfpathlineto{\pgfqpoint{3.643928in}{1.814883in}}%
\pgfpathlineto{\pgfqpoint{3.644245in}{1.691406in}}%
\pgfpathlineto{\pgfqpoint{3.644563in}{0.932848in}}%
\pgfpathlineto{\pgfqpoint{3.645198in}{1.878126in}}%
\pgfpathlineto{\pgfqpoint{3.645515in}{1.021584in}}%
\pgfpathlineto{\pgfqpoint{3.646150in}{1.963804in}}%
\pgfpathlineto{\pgfqpoint{3.646468in}{1.223157in}}%
\pgfpathlineto{\pgfqpoint{3.646785in}{1.078945in}}%
\pgfpathlineto{\pgfqpoint{3.647102in}{1.920371in}}%
\pgfpathlineto{\pgfqpoint{3.647737in}{0.945085in}}%
\pgfpathlineto{\pgfqpoint{3.648055in}{1.754229in}}%
\pgfpathlineto{\pgfqpoint{3.648372in}{1.735719in}}%
\pgfpathlineto{\pgfqpoint{3.648690in}{0.938629in}}%
\pgfpathlineto{\pgfqpoint{3.649325in}{1.911042in}}%
\pgfpathlineto{\pgfqpoint{3.649642in}{1.063663in}}%
\pgfpathlineto{\pgfqpoint{3.650277in}{1.968261in}}%
\pgfpathlineto{\pgfqpoint{3.650594in}{1.289854in}}%
\pgfpathlineto{\pgfqpoint{3.650912in}{1.040171in}}%
\pgfpathlineto{\pgfqpoint{3.651229in}{1.896131in}}%
\pgfpathlineto{\pgfqpoint{3.651864in}{0.936801in}}%
\pgfpathlineto{\pgfqpoint{3.652182in}{1.703656in}}%
\pgfpathlineto{\pgfqpoint{3.652499in}{1.790022in}}%
\pgfpathlineto{\pgfqpoint{3.652817in}{0.955453in}}%
\pgfpathlineto{\pgfqpoint{3.653452in}{1.931167in}}%
\pgfpathlineto{\pgfqpoint{3.653769in}{1.105911in}}%
\pgfpathlineto{\pgfqpoint{3.654086in}{1.182534in}}%
\pgfpathlineto{\pgfqpoint{3.654404in}{1.956083in}}%
\pgfpathlineto{\pgfqpoint{3.655039in}{0.995719in}}%
\pgfpathlineto{\pgfqpoint{3.655356in}{1.849566in}}%
\pgfpathlineto{\pgfqpoint{3.655991in}{0.922209in}}%
\pgfpathlineto{\pgfqpoint{3.656309in}{1.636845in}}%
\pgfpathlineto{\pgfqpoint{3.656626in}{1.831082in}}%
\pgfpathlineto{\pgfqpoint{3.656944in}{0.980597in}}%
\pgfpathlineto{\pgfqpoint{3.657579in}{1.951892in}}%
\pgfpathlineto{\pgfqpoint{3.657896in}{1.159730in}}%
\pgfpathlineto{\pgfqpoint{3.658213in}{1.129452in}}%
\pgfpathlineto{\pgfqpoint{3.658531in}{1.944390in}}%
\pgfpathlineto{\pgfqpoint{3.659166in}{0.972749in}}%
\pgfpathlineto{\pgfqpoint{3.659483in}{1.812998in}}%
\pgfpathlineto{\pgfqpoint{3.659801in}{1.679352in}}%
\pgfpathlineto{\pgfqpoint{3.660118in}{0.926499in}}%
\pgfpathlineto{\pgfqpoint{3.660753in}{1.868814in}}%
\pgfpathlineto{\pgfqpoint{3.661071in}{1.010099in}}%
\pgfpathlineto{\pgfqpoint{3.661705in}{1.956111in}}%
\pgfpathlineto{\pgfqpoint{3.662023in}{1.211825in}}%
\pgfpathlineto{\pgfqpoint{3.662340in}{1.075965in}}%
\pgfpathlineto{\pgfqpoint{3.662658in}{1.916331in}}%
\pgfpathlineto{\pgfqpoint{3.663293in}{0.940203in}}%
\pgfpathlineto{\pgfqpoint{3.663610in}{1.753360in}}%
\pgfpathlineto{\pgfqpoint{3.663928in}{1.723444in}}%
\pgfpathlineto{\pgfqpoint{3.664245in}{0.931436in}}%
\pgfpathlineto{\pgfqpoint{3.664880in}{1.902028in}}%
\pgfpathlineto{\pgfqpoint{3.665198in}{1.051210in}}%
\pgfpathlineto{\pgfqpoint{3.665832in}{1.961308in}}%
\pgfpathlineto{\pgfqpoint{3.666150in}{1.277376in}}%
\pgfpathlineto{\pgfqpoint{3.666467in}{1.037095in}}%
\pgfpathlineto{\pgfqpoint{3.666785in}{1.892887in}}%
\pgfpathlineto{\pgfqpoint{3.667420in}{0.931219in}}%
\pgfpathlineto{\pgfqpoint{3.667737in}{1.703845in}}%
\pgfpathlineto{\pgfqpoint{3.668055in}{1.778250in}}%
\pgfpathlineto{\pgfqpoint{3.668372in}{0.947720in}}%
\pgfpathlineto{\pgfqpoint{3.669007in}{1.923072in}}%
\pgfpathlineto{\pgfqpoint{3.669324in}{1.092573in}}%
\pgfpathlineto{\pgfqpoint{3.669642in}{1.183264in}}%
\pgfpathlineto{\pgfqpoint{3.669959in}{1.950051in}}%
\pgfpathlineto{\pgfqpoint{3.670594in}{0.992193in}}%
\pgfpathlineto{\pgfqpoint{3.670912in}{1.846955in}}%
\pgfpathlineto{\pgfqpoint{3.671547in}{0.915379in}}%
\pgfpathlineto{\pgfqpoint{3.671864in}{1.637881in}}%
\pgfpathlineto{\pgfqpoint{3.672182in}{1.819204in}}%
\pgfpathlineto{\pgfqpoint{3.672499in}{0.971847in}}%
\pgfpathlineto{\pgfqpoint{3.673134in}{1.944312in}}%
\pgfpathlineto{\pgfqpoint{3.673451in}{1.145688in}}%
\pgfpathlineto{\pgfqpoint{3.673769in}{1.129799in}}%
\pgfpathlineto{\pgfqpoint{3.674086in}{1.939280in}}%
\pgfpathlineto{\pgfqpoint{3.674721in}{0.969145in}}%
\pgfpathlineto{\pgfqpoint{3.675039in}{1.811356in}}%
\pgfpathlineto{\pgfqpoint{3.675356in}{1.665996in}}%
\pgfpathlineto{\pgfqpoint{3.675674in}{0.918813in}}%
\pgfpathlineto{\pgfqpoint{3.676309in}{1.857850in}}%
\pgfpathlineto{\pgfqpoint{3.676626in}{1.000609in}}%
\pgfpathlineto{\pgfqpoint{3.677261in}{1.949463in}}%
\pgfpathlineto{\pgfqpoint{3.677578in}{1.196916in}}%
\pgfpathlineto{\pgfqpoint{3.677896in}{1.075813in}}%
\pgfpathlineto{\pgfqpoint{3.678213in}{1.912429in}}%
\pgfpathlineto{\pgfqpoint{3.678848in}{0.935764in}}%
\pgfpathlineto{\pgfqpoint{3.679166in}{1.752260in}}%
\pgfpathlineto{\pgfqpoint{3.679483in}{1.710236in}}%
\pgfpathlineto{\pgfqpoint{3.679801in}{0.922372in}}%
\pgfpathlineto{\pgfqpoint{3.680435in}{1.891148in}}%
\pgfpathlineto{\pgfqpoint{3.680753in}{1.040668in}}%
\pgfpathlineto{\pgfqpoint{3.681388in}{1.955473in}}%
\pgfpathlineto{\pgfqpoint{3.681705in}{1.261504in}}%
\pgfpathlineto{\pgfqpoint{3.682023in}{1.036204in}}%
\pgfpathlineto{\pgfqpoint{3.682340in}{1.889933in}}%
\pgfpathlineto{\pgfqpoint{3.682975in}{0.926578in}}%
\pgfpathlineto{\pgfqpoint{3.683293in}{1.703782in}}%
\pgfpathlineto{\pgfqpoint{3.683610in}{1.766216in}}%
\pgfpathlineto{\pgfqpoint{3.683928in}{0.937829in}}%
\pgfpathlineto{\pgfqpoint{3.684562in}{1.913445in}}%
\pgfpathlineto{\pgfqpoint{3.684880in}{1.081116in}}%
\pgfpathlineto{\pgfqpoint{3.685197in}{1.182827in}}%
\pgfpathlineto{\pgfqpoint{3.685515in}{1.945028in}}%
\pgfpathlineto{\pgfqpoint{3.686150in}{0.990479in}}%
\pgfpathlineto{\pgfqpoint{3.686467in}{1.845346in}}%
\pgfpathlineto{\pgfqpoint{3.687102in}{0.909554in}}%
\pgfpathlineto{\pgfqpoint{3.687420in}{1.638208in}}%
\pgfpathlineto{\pgfqpoint{3.687737in}{1.807195in}}%
\pgfpathlineto{\pgfqpoint{3.688054in}{0.960381in}}%
\pgfpathlineto{\pgfqpoint{3.688689in}{1.935019in}}%
\pgfpathlineto{\pgfqpoint{3.689007in}{1.133338in}}%
\pgfpathlineto{\pgfqpoint{3.689324in}{1.129095in}}%
\pgfpathlineto{\pgfqpoint{3.689642in}{1.935165in}}%
\pgfpathlineto{\pgfqpoint{3.690277in}{0.966736in}}%
\pgfpathlineto{\pgfqpoint{3.690594in}{1.811115in}}%
\pgfpathlineto{\pgfqpoint{3.690912in}{1.651821in}}%
\pgfpathlineto{\pgfqpoint{3.691229in}{0.912725in}}%
\pgfpathlineto{\pgfqpoint{3.691864in}{1.847271in}}%
\pgfpathlineto{\pgfqpoint{3.692181in}{0.988196in}}%
\pgfpathlineto{\pgfqpoint{3.692816in}{1.941481in}}%
\pgfpathlineto{\pgfqpoint{3.693134in}{1.183488in}}%
\pgfpathlineto{\pgfqpoint{3.693451in}{1.074818in}}%
\pgfpathlineto{\pgfqpoint{3.693769in}{1.909259in}}%
\pgfpathlineto{\pgfqpoint{3.694404in}{0.931972in}}%
\pgfpathlineto{\pgfqpoint{3.694721in}{1.753255in}}%
\pgfpathlineto{\pgfqpoint{3.695039in}{1.695795in}}%
\pgfpathlineto{\pgfqpoint{3.695356in}{0.914842in}}%
\pgfpathlineto{\pgfqpoint{3.695991in}{1.880755in}}%
\pgfpathlineto{\pgfqpoint{3.696308in}{1.026835in}}%
\pgfpathlineto{\pgfqpoint{3.696943in}{1.948387in}}%
\pgfpathlineto{\pgfqpoint{3.697261in}{1.247048in}}%
\pgfpathlineto{\pgfqpoint{3.697578in}{1.035046in}}%
\pgfpathlineto{\pgfqpoint{3.697896in}{1.888034in}}%
\pgfpathlineto{\pgfqpoint{3.698531in}{0.922001in}}%
\pgfpathlineto{\pgfqpoint{3.698848in}{1.706113in}}%
\pgfpathlineto{\pgfqpoint{3.699165in}{1.752669in}}%
\pgfpathlineto{\pgfqpoint{3.699483in}{0.929681in}}%
\pgfpathlineto{\pgfqpoint{3.700118in}{1.904431in}}%
\pgfpathlineto{\pgfqpoint{3.700435in}{1.066218in}}%
\pgfpathlineto{\pgfqpoint{3.700753in}{1.186150in}}%
\pgfpathlineto{\pgfqpoint{3.701070in}{1.939206in}}%
\pgfpathlineto{\pgfqpoint{3.701705in}{0.988518in}}%
\pgfpathlineto{\pgfqpoint{3.702023in}{1.844217in}}%
\pgfpathlineto{\pgfqpoint{3.702658in}{0.903289in}}%
\pgfpathlineto{\pgfqpoint{3.702975in}{1.641442in}}%
\pgfpathlineto{\pgfqpoint{3.703292in}{1.793513in}}%
\pgfpathlineto{\pgfqpoint{3.703610in}{0.950782in}}%
\pgfpathlineto{\pgfqpoint{3.704245in}{1.926639in}}%
\pgfpathlineto{\pgfqpoint{3.704562in}{1.117425in}}%
\pgfpathlineto{\pgfqpoint{3.704880in}{1.131846in}}%
\pgfpathlineto{\pgfqpoint{3.705197in}{1.930488in}}%
\pgfpathlineto{\pgfqpoint{3.705832in}{0.964372in}}%
\pgfpathlineto{\pgfqpoint{3.706150in}{1.811212in}}%
\pgfpathlineto{\pgfqpoint{3.706784in}{0.905729in}}%
\pgfpathlineto{\pgfqpoint{3.707102in}{1.583450in}}%
\pgfpathlineto{\pgfqpoint{3.707419in}{1.835158in}}%
\pgfpathlineto{\pgfqpoint{3.707737in}{0.977872in}}%
\pgfpathlineto{\pgfqpoint{3.708372in}{1.934535in}}%
\pgfpathlineto{\pgfqpoint{3.708689in}{1.166531in}}%
\pgfpathlineto{\pgfqpoint{3.709007in}{1.076965in}}%
\pgfpathlineto{\pgfqpoint{3.709324in}{1.906009in}}%
\pgfpathlineto{\pgfqpoint{3.709959in}{0.928419in}}%
\pgfpathlineto{\pgfqpoint{3.710277in}{1.754153in}}%
\pgfpathlineto{\pgfqpoint{3.710594in}{1.680157in}}%
\pgfpathlineto{\pgfqpoint{3.710911in}{0.905796in}}%
\pgfpathlineto{\pgfqpoint{3.711546in}{1.868511in}}%
\pgfpathlineto{\pgfqpoint{3.711864in}{1.014970in}}%
\pgfpathlineto{\pgfqpoint{3.712499in}{1.942368in}}%
\pgfpathlineto{\pgfqpoint{3.712816in}{1.229043in}}%
\pgfpathlineto{\pgfqpoint{3.713134in}{1.036329in}}%
\pgfpathlineto{\pgfqpoint{3.713451in}{1.886087in}}%
\pgfpathlineto{\pgfqpoint{3.714086in}{0.918203in}}%
\pgfpathlineto{\pgfqpoint{3.714403in}{1.708292in}}%
\pgfpathlineto{\pgfqpoint{3.714721in}{1.738569in}}%
\pgfpathlineto{\pgfqpoint{3.715038in}{0.919735in}}%
\pgfpathlineto{\pgfqpoint{3.715673in}{1.893898in}}%
\pgfpathlineto{\pgfqpoint{3.715991in}{1.053363in}}%
\pgfpathlineto{\pgfqpoint{3.716308in}{1.188287in}}%
\pgfpathlineto{\pgfqpoint{3.716626in}{1.934442in}}%
\pgfpathlineto{\pgfqpoint{3.717261in}{0.988662in}}%
\pgfpathlineto{\pgfqpoint{3.717578in}{1.844001in}}%
\pgfpathlineto{\pgfqpoint{3.718213in}{0.897951in}}%
\pgfpathlineto{\pgfqpoint{3.718530in}{1.644147in}}%
\pgfpathlineto{\pgfqpoint{3.718848in}{1.779449in}}%
\pgfpathlineto{\pgfqpoint{3.719165in}{0.938756in}}%
\pgfpathlineto{\pgfqpoint{3.719800in}{1.916470in}}%
\pgfpathlineto{\pgfqpoint{3.720118in}{1.103352in}}%
\pgfpathlineto{\pgfqpoint{3.720435in}{1.133595in}}%
\pgfpathlineto{\pgfqpoint{3.720753in}{1.927093in}}%
\pgfpathlineto{\pgfqpoint{3.721388in}{0.963692in}}%
\pgfpathlineto{\pgfqpoint{3.721705in}{1.812404in}}%
\pgfpathlineto{\pgfqpoint{3.722340in}{0.899795in}}%
\pgfpathlineto{\pgfqpoint{3.722657in}{1.587136in}}%
\pgfpathlineto{\pgfqpoint{3.722975in}{1.822674in}}%
\pgfpathlineto{\pgfqpoint{3.723292in}{0.964680in}}%
\pgfpathlineto{\pgfqpoint{3.723927in}{1.926079in}}%
\pgfpathlineto{\pgfqpoint{3.724245in}{1.151342in}}%
\pgfpathlineto{\pgfqpoint{3.724562in}{1.078261in}}%
\pgfpathlineto{\pgfqpoint{3.724880in}{1.903750in}}%
\pgfpathlineto{\pgfqpoint{3.725514in}{0.926052in}}%
\pgfpathlineto{\pgfqpoint{3.725832in}{1.756830in}}%
\pgfpathlineto{\pgfqpoint{3.726149in}{1.663497in}}%
\pgfpathlineto{\pgfqpoint{3.726467in}{0.898191in}}%
\pgfpathlineto{\pgfqpoint{3.727102in}{1.856525in}}%
\pgfpathlineto{\pgfqpoint{3.727419in}{1.000026in}}%
\pgfpathlineto{\pgfqpoint{3.728054in}{1.934903in}}%
\pgfpathlineto{\pgfqpoint{3.728372in}{1.212342in}}%
\pgfpathlineto{\pgfqpoint{3.728689in}{1.037013in}}%
\pgfpathlineto{\pgfqpoint{3.729007in}{1.885073in}}%
\pgfpathlineto{\pgfqpoint{3.729641in}{0.914950in}}%
\pgfpathlineto{\pgfqpoint{3.729959in}{1.712563in}}%
\pgfpathlineto{\pgfqpoint{3.730276in}{1.723171in}}%
\pgfpathlineto{\pgfqpoint{3.730594in}{0.911547in}}%
\pgfpathlineto{\pgfqpoint{3.731229in}{1.883856in}}%
\pgfpathlineto{\pgfqpoint{3.731546in}{1.037290in}}%
\pgfpathlineto{\pgfqpoint{3.731864in}{1.194377in}}%
\pgfpathlineto{\pgfqpoint{3.732181in}{1.928720in}}%
\pgfpathlineto{\pgfqpoint{3.732816in}{0.988343in}}%
\pgfpathlineto{\pgfqpoint{3.733133in}{1.844322in}}%
\pgfpathlineto{\pgfqpoint{3.733768in}{0.892640in}}%
\pgfpathlineto{\pgfqpoint{3.734086in}{1.649478in}}%
\pgfpathlineto{\pgfqpoint{3.734403in}{1.763854in}}%
\pgfpathlineto{\pgfqpoint{3.734721in}{0.928565in}}%
\pgfpathlineto{\pgfqpoint{3.735356in}{1.907029in}}%
\pgfpathlineto{\pgfqpoint{3.735673in}{1.085828in}}%
\pgfpathlineto{\pgfqpoint{3.735991in}{1.138948in}}%
\pgfpathlineto{\pgfqpoint{3.736308in}{1.922756in}}%
\pgfpathlineto{\pgfqpoint{3.736943in}{0.962921in}}%
\pgfpathlineto{\pgfqpoint{3.737260in}{1.814051in}}%
\pgfpathlineto{\pgfqpoint{3.737895in}{0.893405in}}%
\pgfpathlineto{\pgfqpoint{3.738213in}{1.593903in}}%
\pgfpathlineto{\pgfqpoint{3.738530in}{1.808705in}}%
\pgfpathlineto{\pgfqpoint{3.738848in}{0.953584in}}%
\pgfpathlineto{\pgfqpoint{3.739483in}{1.918510in}}%
\pgfpathlineto{\pgfqpoint{3.739800in}{1.132695in}}%
\pgfpathlineto{\pgfqpoint{3.740118in}{1.082806in}}%
\pgfpathlineto{\pgfqpoint{3.740435in}{1.901201in}}%
\pgfpathlineto{\pgfqpoint{3.741070in}{0.923918in}}%
\pgfpathlineto{\pgfqpoint{3.741387in}{1.759683in}}%
\pgfpathlineto{\pgfqpoint{3.741705in}{1.645486in}}%
\pgfpathlineto{\pgfqpoint{3.742022in}{0.889534in}}%
\pgfpathlineto{\pgfqpoint{3.742657in}{1.842700in}}%
\pgfpathlineto{\pgfqpoint{3.742975in}{0.987103in}}%
\pgfpathlineto{\pgfqpoint{3.743610in}{1.928427in}}%
\pgfpathlineto{\pgfqpoint{3.743927in}{1.192636in}}%
\pgfpathlineto{\pgfqpoint{3.744244in}{1.040837in}}%
\pgfpathlineto{\pgfqpoint{3.744562in}{1.883980in}}%
\pgfpathlineto{\pgfqpoint{3.745197in}{0.912129in}}%
\pgfpathlineto{\pgfqpoint{3.745514in}{1.716620in}}%
\pgfpathlineto{\pgfqpoint{3.745832in}{1.706430in}}%
\pgfpathlineto{\pgfqpoint{3.746149in}{0.901609in}}%
\pgfpathlineto{\pgfqpoint{3.746784in}{1.871930in}}%
\pgfpathlineto{\pgfqpoint{3.747102in}{1.023262in}}%
\pgfpathlineto{\pgfqpoint{3.747737in}{1.924007in}}%
\pgfpathlineto{\pgfqpoint{3.748054in}{1.244239in}}%
\pgfpathlineto{\pgfqpoint{3.748371in}{0.990693in}}%
\pgfpathlineto{\pgfqpoint{3.748689in}{1.845128in}}%
\pgfpathlineto{\pgfqpoint{3.749324in}{0.888099in}}%
\pgfpathlineto{\pgfqpoint{3.749641in}{1.654401in}}%
\pgfpathlineto{\pgfqpoint{3.749959in}{1.747571in}}%
\pgfpathlineto{\pgfqpoint{3.750276in}{0.916340in}}%
\pgfpathlineto{\pgfqpoint{3.750911in}{1.895717in}}%
\pgfpathlineto{\pgfqpoint{3.751229in}{1.070265in}}%
\pgfpathlineto{\pgfqpoint{3.751546in}{1.143148in}}%
\pgfpathlineto{\pgfqpoint{3.751863in}{1.919411in}}%
\pgfpathlineto{\pgfqpoint{3.752498in}{0.964151in}}%
\pgfpathlineto{\pgfqpoint{3.752816in}{1.816339in}}%
\pgfpathlineto{\pgfqpoint{3.753451in}{0.888233in}}%
\pgfpathlineto{\pgfqpoint{3.753768in}{1.600079in}}%
\pgfpathlineto{\pgfqpoint{3.754086in}{1.794440in}}%
\pgfpathlineto{\pgfqpoint{3.754403in}{0.940187in}}%
\pgfpathlineto{\pgfqpoint{3.755038in}{1.909437in}}%
\pgfpathlineto{\pgfqpoint{3.755355in}{1.116031in}}%
\pgfpathlineto{\pgfqpoint{3.755673in}{1.086337in}}%
\pgfpathlineto{\pgfqpoint{3.755990in}{1.899509in}}%
\pgfpathlineto{\pgfqpoint{3.756625in}{0.923395in}}%
\pgfpathlineto{\pgfqpoint{3.756943in}{1.763858in}}%
\pgfpathlineto{\pgfqpoint{3.757260in}{1.626651in}}%
\pgfpathlineto{\pgfqpoint{3.757578in}{0.882205in}}%
\pgfpathlineto{\pgfqpoint{3.758213in}{1.828856in}}%
\pgfpathlineto{\pgfqpoint{3.758530in}{0.971392in}}%
\pgfpathlineto{\pgfqpoint{3.759165in}{1.920282in}}%
\pgfpathlineto{\pgfqpoint{3.759482in}{1.174144in}}%
\pgfpathlineto{\pgfqpoint{3.759800in}{1.043696in}}%
\pgfpathlineto{\pgfqpoint{3.760117in}{1.883631in}}%
\pgfpathlineto{\pgfqpoint{3.760752in}{0.910446in}}%
\pgfpathlineto{\pgfqpoint{3.761070in}{1.722463in}}%
\pgfpathlineto{\pgfqpoint{3.761387in}{1.688816in}}%
\pgfpathlineto{\pgfqpoint{3.761705in}{0.893582in}}%
\pgfpathlineto{\pgfqpoint{3.762340in}{1.860445in}}%
\pgfpathlineto{\pgfqpoint{3.762657in}{1.006379in}}%
\pgfpathlineto{\pgfqpoint{3.763292in}{1.917950in}}%
\pgfpathlineto{\pgfqpoint{3.763609in}{1.225233in}}%
\pgfpathlineto{\pgfqpoint{3.763927in}{0.992502in}}%
\pgfpathlineto{\pgfqpoint{3.764244in}{1.846580in}}%
\pgfpathlineto{\pgfqpoint{3.764879in}{0.884152in}}%
\pgfpathlineto{\pgfqpoint{3.765197in}{1.661623in}}%
\pgfpathlineto{\pgfqpoint{3.765514in}{1.729943in}}%
\pgfpathlineto{\pgfqpoint{3.765832in}{0.905895in}}%
\pgfpathlineto{\pgfqpoint{3.766467in}{1.884891in}}%
\pgfpathlineto{\pgfqpoint{3.766784in}{1.051435in}}%
\pgfpathlineto{\pgfqpoint{3.767101in}{1.151164in}}%
\pgfpathlineto{\pgfqpoint{3.767419in}{1.914878in}}%
\pgfpathlineto{\pgfqpoint{3.768054in}{0.965188in}}%
\pgfpathlineto{\pgfqpoint{3.768371in}{1.819192in}}%
\pgfpathlineto{\pgfqpoint{3.769006in}{0.882997in}}%
\pgfpathlineto{\pgfqpoint{3.769324in}{1.608870in}}%
\pgfpathlineto{\pgfqpoint{3.769641in}{1.778590in}}%
\pgfpathlineto{\pgfqpoint{3.769959in}{0.928803in}}%
\pgfpathlineto{\pgfqpoint{3.770593in}{1.900968in}}%
\pgfpathlineto{\pgfqpoint{3.770911in}{1.096052in}}%
\pgfpathlineto{\pgfqpoint{3.771228in}{1.093585in}}%
\pgfpathlineto{\pgfqpoint{3.771546in}{1.897101in}}%
\pgfpathlineto{\pgfqpoint{3.772181in}{0.922851in}}%
\pgfpathlineto{\pgfqpoint{3.772498in}{1.768224in}}%
\pgfpathlineto{\pgfqpoint{3.772816in}{1.606047in}}%
\pgfpathlineto{\pgfqpoint{3.773133in}{0.874329in}}%
\pgfpathlineto{\pgfqpoint{3.773768in}{1.813248in}}%
\pgfpathlineto{\pgfqpoint{3.774085in}{0.957728in}}%
\pgfpathlineto{\pgfqpoint{3.774720in}{1.912924in}}%
\pgfpathlineto{\pgfqpoint{3.775038in}{1.152119in}}%
\pgfpathlineto{\pgfqpoint{3.775355in}{1.049517in}}%
\pgfpathlineto{\pgfqpoint{3.775673in}{1.882435in}}%
\pgfpathlineto{\pgfqpoint{3.776308in}{0.909183in}}%
\pgfpathlineto{\pgfqpoint{3.776625in}{1.728363in}}%
\pgfpathlineto{\pgfqpoint{3.776943in}{1.669961in}}%
\pgfpathlineto{\pgfqpoint{3.777260in}{0.884779in}}%
\pgfpathlineto{\pgfqpoint{3.777895in}{1.847517in}}%
\pgfpathlineto{\pgfqpoint{3.778212in}{0.991945in}}%
\pgfpathlineto{\pgfqpoint{3.778847in}{1.912791in}}%
\pgfpathlineto{\pgfqpoint{3.779165in}{1.202810in}}%
\pgfpathlineto{\pgfqpoint{3.779482in}{0.997313in}}%
\pgfpathlineto{\pgfqpoint{3.779800in}{1.848003in}}%
\pgfpathlineto{\pgfqpoint{3.780435in}{0.880754in}}%
\pgfpathlineto{\pgfqpoint{3.780752in}{1.668524in}}%
\pgfpathlineto{\pgfqpoint{3.781070in}{1.711240in}}%
\pgfpathlineto{\pgfqpoint{3.781387in}{0.893899in}}%
\pgfpathlineto{\pgfqpoint{3.782022in}{1.872128in}}%
\pgfpathlineto{\pgfqpoint{3.782339in}{1.034664in}}%
\pgfpathlineto{\pgfqpoint{3.782657in}{1.157970in}}%
\pgfpathlineto{\pgfqpoint{3.782974in}{1.911173in}}%
\pgfpathlineto{\pgfqpoint{3.783609in}{0.968435in}}%
\pgfpathlineto{\pgfqpoint{3.783927in}{1.822065in}}%
\pgfpathlineto{\pgfqpoint{3.784562in}{0.879104in}}%
\pgfpathlineto{\pgfqpoint{3.784879in}{1.617397in}}%
\pgfpathlineto{\pgfqpoint{3.785197in}{1.762467in}}%
\pgfpathlineto{\pgfqpoint{3.785514in}{0.915943in}}%
\pgfpathlineto{\pgfqpoint{3.786149in}{1.890632in}}%
\pgfpathlineto{\pgfqpoint{3.786466in}{1.078134in}}%
\pgfpathlineto{\pgfqpoint{3.786784in}{1.099721in}}%
\pgfpathlineto{\pgfqpoint{3.787101in}{1.895491in}}%
\pgfpathlineto{\pgfqpoint{3.787736in}{0.924489in}}%
\pgfpathlineto{\pgfqpoint{3.788054in}{1.773519in}}%
\pgfpathlineto{\pgfqpoint{3.788689in}{0.867658in}}%
\pgfpathlineto{\pgfqpoint{3.789006in}{1.552725in}}%
\pgfpathlineto{\pgfqpoint{3.789323in}{1.797264in}}%
\pgfpathlineto{\pgfqpoint{3.789641in}{0.941664in}}%
\pgfpathlineto{\pgfqpoint{3.790276in}{1.903706in}}%
\pgfpathlineto{\pgfqpoint{3.790593in}{1.132028in}}%
\pgfpathlineto{\pgfqpoint{3.790911in}{1.054417in}}%
\pgfpathlineto{\pgfqpoint{3.791228in}{1.882099in}}%
\pgfpathlineto{\pgfqpoint{3.791863in}{0.909409in}}%
\pgfpathlineto{\pgfqpoint{3.792181in}{1.735388in}}%
\pgfpathlineto{\pgfqpoint{3.792498in}{1.650065in}}%
\pgfpathlineto{\pgfqpoint{3.792815in}{0.877620in}}%
\pgfpathlineto{\pgfqpoint{3.793450in}{1.834288in}}%
\pgfpathlineto{\pgfqpoint{3.793768in}{0.974820in}}%
\pgfpathlineto{\pgfqpoint{3.794403in}{1.906050in}}%
\pgfpathlineto{\pgfqpoint{3.794720in}{1.182162in}}%
\pgfpathlineto{\pgfqpoint{3.795038in}{1.001418in}}%
\pgfpathlineto{\pgfqpoint{3.795355in}{1.850057in}}%
\pgfpathlineto{\pgfqpoint{3.795990in}{0.878485in}}%
\pgfpathlineto{\pgfqpoint{3.796308in}{1.677258in}}%
\pgfpathlineto{\pgfqpoint{3.796625in}{1.691369in}}%
\pgfpathlineto{\pgfqpoint{3.796942in}{0.883614in}}%
\pgfpathlineto{\pgfqpoint{3.797577in}{1.859552in}}%
\pgfpathlineto{\pgfqpoint{3.797895in}{1.014872in}}%
\pgfpathlineto{\pgfqpoint{3.798212in}{1.168763in}}%
\pgfpathlineto{\pgfqpoint{3.798530in}{1.906078in}}%
\pgfpathlineto{\pgfqpoint{3.799165in}{0.971011in}}%
\pgfpathlineto{\pgfqpoint{3.799482in}{1.825323in}}%
\pgfpathlineto{\pgfqpoint{3.800117in}{0.875888in}}%
\pgfpathlineto{\pgfqpoint{3.800434in}{1.628154in}}%
\pgfpathlineto{\pgfqpoint{3.800752in}{1.745096in}}%
\pgfpathlineto{\pgfqpoint{3.801069in}{0.905517in}}%
\pgfpathlineto{\pgfqpoint{3.801704in}{1.880885in}}%
\pgfpathlineto{\pgfqpoint{3.802022in}{1.057254in}}%
\pgfpathlineto{\pgfqpoint{3.802339in}{1.109866in}}%
\pgfpathlineto{\pgfqpoint{3.802657in}{1.892811in}}%
\pgfpathlineto{\pgfqpoint{3.803292in}{0.925849in}}%
\pgfpathlineto{\pgfqpoint{3.803609in}{1.779008in}}%
\pgfpathlineto{\pgfqpoint{3.804244in}{0.860990in}}%
\pgfpathlineto{\pgfqpoint{3.804561in}{1.564239in}}%
\pgfpathlineto{\pgfqpoint{3.804879in}{1.779608in}}%
\pgfpathlineto{\pgfqpoint{3.805196in}{0.927655in}}%
\pgfpathlineto{\pgfqpoint{3.805831in}{1.895029in}}%
\pgfpathlineto{\pgfqpoint{3.806149in}{1.108908in}}%
\pgfpathlineto{\pgfqpoint{3.806466in}{1.062163in}}%
\pgfpathlineto{\pgfqpoint{3.806784in}{1.880284in}}%
\pgfpathlineto{\pgfqpoint{3.807419in}{0.909659in}}%
\pgfpathlineto{\pgfqpoint{3.807736in}{1.742399in}}%
\pgfpathlineto{\pgfqpoint{3.808053in}{1.628461in}}%
\pgfpathlineto{\pgfqpoint{3.808371in}{0.870580in}}%
\pgfpathlineto{\pgfqpoint{3.809006in}{1.819371in}}%
\pgfpathlineto{\pgfqpoint{3.809323in}{0.960111in}}%
\pgfpathlineto{\pgfqpoint{3.809958in}{1.899980in}}%
\pgfpathlineto{\pgfqpoint{3.810276in}{1.158197in}}%
\pgfpathlineto{\pgfqpoint{3.810593in}{1.008950in}}%
\pgfpathlineto{\pgfqpoint{3.810911in}{1.851649in}}%
\pgfpathlineto{\pgfqpoint{3.811545in}{0.876581in}}%
\pgfpathlineto{\pgfqpoint{3.811863in}{1.685798in}}%
\pgfpathlineto{\pgfqpoint{3.812180in}{1.670008in}}%
\pgfpathlineto{\pgfqpoint{3.812498in}{0.872325in}}%
\pgfpathlineto{\pgfqpoint{3.813133in}{1.845006in}}%
\pgfpathlineto{\pgfqpoint{3.813450in}{0.997237in}}%
\pgfpathlineto{\pgfqpoint{3.814085in}{1.901704in}}%
\pgfpathlineto{\pgfqpoint{3.814403in}{1.219905in}}%
\pgfpathlineto{\pgfqpoint{3.814720in}{0.975932in}}%
\pgfpathlineto{\pgfqpoint{3.815038in}{1.827889in}}%
\pgfpathlineto{\pgfqpoint{3.815672in}{0.873467in}}%
\pgfpathlineto{\pgfqpoint{3.815990in}{1.638753in}}%
\pgfpathlineto{\pgfqpoint{3.816307in}{1.726195in}}%
\pgfpathlineto{\pgfqpoint{3.816625in}{0.893476in}}%
\pgfpathlineto{\pgfqpoint{3.817260in}{1.869109in}}%
\pgfpathlineto{\pgfqpoint{3.817577in}{1.038575in}}%
\pgfpathlineto{\pgfqpoint{3.817895in}{1.118839in}}%
\pgfpathlineto{\pgfqpoint{3.818212in}{1.890812in}}%
\pgfpathlineto{\pgfqpoint{3.818847in}{0.929898in}}%
\pgfpathlineto{\pgfqpoint{3.819164in}{1.784896in}}%
\pgfpathlineto{\pgfqpoint{3.819799in}{0.855330in}}%
\pgfpathlineto{\pgfqpoint{3.820117in}{1.575151in}}%
\pgfpathlineto{\pgfqpoint{3.820434in}{1.761156in}}%
\pgfpathlineto{\pgfqpoint{3.820752in}{0.911735in}}%
\pgfpathlineto{\pgfqpoint{3.821387in}{1.884323in}}%
\pgfpathlineto{\pgfqpoint{3.821704in}{1.088111in}}%
\pgfpathlineto{\pgfqpoint{3.822022in}{1.068903in}}%
\pgfpathlineto{\pgfqpoint{3.822339in}{1.879388in}}%
\pgfpathlineto{\pgfqpoint{3.822974in}{0.911469in}}%
\pgfpathlineto{\pgfqpoint{3.823291in}{1.749728in}}%
\pgfpathlineto{\pgfqpoint{3.823609in}{1.605475in}}%
\pgfpathlineto{\pgfqpoint{3.823926in}{0.864859in}}%
\pgfpathlineto{\pgfqpoint{3.824561in}{1.803959in}}%
\pgfpathlineto{\pgfqpoint{3.824879in}{0.943340in}}%
\pgfpathlineto{\pgfqpoint{3.825514in}{1.892192in}}%
\pgfpathlineto{\pgfqpoint{3.825831in}{1.136285in}}%
\pgfpathlineto{\pgfqpoint{3.826149in}{1.015647in}}%
\pgfpathlineto{\pgfqpoint{3.826466in}{1.853826in}}%
\pgfpathlineto{\pgfqpoint{3.827101in}{0.876304in}}%
\pgfpathlineto{\pgfqpoint{3.827418in}{1.695574in}}%
\pgfpathlineto{\pgfqpoint{3.827736in}{1.647624in}}%
\pgfpathlineto{\pgfqpoint{3.828053in}{0.862612in}}%
\pgfpathlineto{\pgfqpoint{3.828688in}{1.830280in}}%
\pgfpathlineto{\pgfqpoint{3.829006in}{0.976957in}}%
\pgfpathlineto{\pgfqpoint{3.829641in}{1.895462in}}%
\pgfpathlineto{\pgfqpoint{3.829958in}{1.194794in}}%
\pgfpathlineto{\pgfqpoint{3.830275in}{0.979459in}}%
\pgfpathlineto{\pgfqpoint{3.830593in}{1.830556in}}%
\pgfpathlineto{\pgfqpoint{3.831228in}{0.871727in}}%
\pgfpathlineto{\pgfqpoint{3.831545in}{1.651032in}}%
\pgfpathlineto{\pgfqpoint{3.831863in}{1.705810in}}%
\pgfpathlineto{\pgfqpoint{3.832180in}{0.883653in}}%
\pgfpathlineto{\pgfqpoint{3.832815in}{1.857553in}}%
\pgfpathlineto{\pgfqpoint{3.833133in}{1.017308in}}%
\pgfpathlineto{\pgfqpoint{3.833450in}{1.132068in}}%
\pgfpathlineto{\pgfqpoint{3.833768in}{1.887414in}}%
\pgfpathlineto{\pgfqpoint{3.834402in}{0.933497in}}%
\pgfpathlineto{\pgfqpoint{3.834720in}{1.791026in}}%
\pgfpathlineto{\pgfqpoint{3.835355in}{0.850261in}}%
\pgfpathlineto{\pgfqpoint{3.835672in}{1.588252in}}%
\pgfpathlineto{\pgfqpoint{3.835990in}{1.741116in}}%
\pgfpathlineto{\pgfqpoint{3.836307in}{0.897840in}}%
\pgfpathlineto{\pgfqpoint{3.836942in}{1.873888in}}%
\pgfpathlineto{\pgfqpoint{3.837260in}{1.064108in}}%
\pgfpathlineto{\pgfqpoint{3.837577in}{1.079746in}}%
\pgfpathlineto{\pgfqpoint{3.837894in}{1.877703in}}%
\pgfpathlineto{\pgfqpoint{3.838529in}{0.912260in}}%
\pgfpathlineto{\pgfqpoint{3.838847in}{1.757499in}}%
\pgfpathlineto{\pgfqpoint{3.839482in}{0.858734in}}%
\pgfpathlineto{\pgfqpoint{3.839799in}{1.535900in}}%
\pgfpathlineto{\pgfqpoint{3.840117in}{1.786466in}}%
\pgfpathlineto{\pgfqpoint{3.840434in}{0.928669in}}%
\pgfpathlineto{\pgfqpoint{3.841069in}{1.884645in}}%
\pgfpathlineto{\pgfqpoint{3.841387in}{1.111202in}}%
\pgfpathlineto{\pgfqpoint{3.841704in}{1.026167in}}%
\pgfpathlineto{\pgfqpoint{3.842021in}{1.855087in}}%
\pgfpathlineto{\pgfqpoint{3.842656in}{0.876208in}}%
\pgfpathlineto{\pgfqpoint{3.842974in}{1.705274in}}%
\pgfpathlineto{\pgfqpoint{3.843291in}{1.623291in}}%
\pgfpathlineto{\pgfqpoint{3.843609in}{0.852484in}}%
\pgfpathlineto{\pgfqpoint{3.844244in}{1.813568in}}%
\pgfpathlineto{\pgfqpoint{3.844561in}{0.958931in}}%
\pgfpathlineto{\pgfqpoint{3.845196in}{1.889731in}}%
\pgfpathlineto{\pgfqpoint{3.845513in}{1.167964in}}%
\pgfpathlineto{\pgfqpoint{3.845831in}{0.986009in}}%
\pgfpathlineto{\pgfqpoint{3.846148in}{1.832928in}}%
\pgfpathlineto{\pgfqpoint{3.846783in}{0.869858in}}%
\pgfpathlineto{\pgfqpoint{3.847101in}{1.662827in}}%
\pgfpathlineto{\pgfqpoint{3.847418in}{1.683339in}}%
\pgfpathlineto{\pgfqpoint{3.847736in}{0.872457in}}%
\pgfpathlineto{\pgfqpoint{3.848371in}{1.843646in}}%
\pgfpathlineto{\pgfqpoint{3.848688in}{0.998165in}}%
\pgfpathlineto{\pgfqpoint{3.849005in}{1.144018in}}%
\pgfpathlineto{\pgfqpoint{3.849323in}{1.884448in}}%
\pgfpathlineto{\pgfqpoint{3.849958in}{0.940305in}}%
\pgfpathlineto{\pgfqpoint{3.850275in}{1.796987in}}%
\pgfpathlineto{\pgfqpoint{3.850910in}{0.845987in}}%
\pgfpathlineto{\pgfqpoint{3.851228in}{1.600871in}}%
\pgfpathlineto{\pgfqpoint{3.851545in}{1.719772in}}%
\pgfpathlineto{\pgfqpoint{3.851863in}{0.882581in}}%
\pgfpathlineto{\pgfqpoint{3.852498in}{1.861306in}}%
\pgfpathlineto{\pgfqpoint{3.852815in}{1.042315in}}%
\pgfpathlineto{\pgfqpoint{3.853132in}{1.090535in}}%
\pgfpathlineto{\pgfqpoint{3.853450in}{1.876939in}}%
\pgfpathlineto{\pgfqpoint{3.854085in}{0.914943in}}%
\pgfpathlineto{\pgfqpoint{3.854402in}{1.764901in}}%
\pgfpathlineto{\pgfqpoint{3.855037in}{0.853466in}}%
\pgfpathlineto{\pgfqpoint{3.855355in}{1.550635in}}%
\pgfpathlineto{\pgfqpoint{3.855672in}{1.767861in}}%
\pgfpathlineto{\pgfqpoint{3.855990in}{0.912238in}}%
\pgfpathlineto{\pgfqpoint{3.856624in}{1.875054in}}%
\pgfpathlineto{\pgfqpoint{3.856942in}{1.088430in}}%
\pgfpathlineto{\pgfqpoint{3.857259in}{1.035800in}}%
\pgfpathlineto{\pgfqpoint{3.857577in}{1.856871in}}%
\pgfpathlineto{\pgfqpoint{3.858212in}{0.878394in}}%
\pgfpathlineto{\pgfqpoint{3.858529in}{1.715645in}}%
\pgfpathlineto{\pgfqpoint{3.858847in}{1.598201in}}%
\pgfpathlineto{\pgfqpoint{3.859164in}{0.843719in}}%
\pgfpathlineto{\pgfqpoint{3.859799in}{1.796180in}}%
\pgfpathlineto{\pgfqpoint{3.860117in}{0.938748in}}%
\pgfpathlineto{\pgfqpoint{3.860751in}{1.881924in}}%
\pgfpathlineto{\pgfqpoint{3.861069in}{1.143099in}}%
\pgfpathlineto{\pgfqpoint{3.861386in}{0.992319in}}%
\pgfpathlineto{\pgfqpoint{3.861704in}{1.836278in}}%
\pgfpathlineto{\pgfqpoint{3.862339in}{0.869094in}}%
\pgfpathlineto{\pgfqpoint{3.862656in}{1.675319in}}%
\pgfpathlineto{\pgfqpoint{3.862974in}{1.659359in}}%
\pgfpathlineto{\pgfqpoint{3.863291in}{0.862888in}}%
\pgfpathlineto{\pgfqpoint{3.863926in}{1.829247in}}%
\pgfpathlineto{\pgfqpoint{3.864243in}{0.976569in}}%
\pgfpathlineto{\pgfqpoint{3.864878in}{1.879638in}}%
\pgfpathlineto{\pgfqpoint{3.865196in}{1.194357in}}%
\pgfpathlineto{\pgfqpoint{3.865513in}{0.946613in}}%
\pgfpathlineto{\pgfqpoint{3.865831in}{1.803284in}}%
\pgfpathlineto{\pgfqpoint{3.866466in}{0.843013in}}%
\pgfpathlineto{\pgfqpoint{3.866783in}{1.615120in}}%
\pgfpathlineto{\pgfqpoint{3.867101in}{1.696980in}}%
\pgfpathlineto{\pgfqpoint{3.867418in}{0.869206in}}%
\pgfpathlineto{\pgfqpoint{3.868053in}{1.848591in}}%
\pgfpathlineto{\pgfqpoint{3.868370in}{1.017739in}}%
\pgfpathlineto{\pgfqpoint{3.868688in}{1.105371in}}%
\pgfpathlineto{\pgfqpoint{3.869005in}{1.874391in}}%
\pgfpathlineto{\pgfqpoint{3.869640in}{0.917969in}}%
\pgfpathlineto{\pgfqpoint{3.869958in}{1.772892in}}%
\pgfpathlineto{\pgfqpoint{3.870593in}{0.848489in}}%
\pgfpathlineto{\pgfqpoint{3.870910in}{1.567042in}}%
\pgfpathlineto{\pgfqpoint{3.871228in}{1.747089in}}%
\pgfpathlineto{\pgfqpoint{3.871545in}{0.897430in}}%
\pgfpathlineto{\pgfqpoint{3.872180in}{1.865115in}}%
\pgfpathlineto{\pgfqpoint{3.872497in}{1.062566in}}%
\pgfpathlineto{\pgfqpoint{3.872815in}{1.049552in}}%
\pgfpathlineto{\pgfqpoint{3.873132in}{1.857246in}}%
\pgfpathlineto{\pgfqpoint{3.873767in}{0.880654in}}%
\pgfpathlineto{\pgfqpoint{3.874085in}{1.726079in}}%
\pgfpathlineto{\pgfqpoint{3.874402in}{1.570775in}}%
\pgfpathlineto{\pgfqpoint{3.874720in}{0.835225in}}%
\pgfpathlineto{\pgfqpoint{3.875354in}{1.776802in}}%
\pgfpathlineto{\pgfqpoint{3.875672in}{0.920791in}}%
\pgfpathlineto{\pgfqpoint{3.876307in}{1.874350in}}%
\pgfpathlineto{\pgfqpoint{3.876624in}{1.115038in}}%
\pgfpathlineto{\pgfqpoint{3.876942in}{1.004088in}}%
\pgfpathlineto{\pgfqpoint{3.877259in}{1.839094in}}%
\pgfpathlineto{\pgfqpoint{3.877894in}{0.868894in}}%
\pgfpathlineto{\pgfqpoint{3.878212in}{1.687844in}}%
\pgfpathlineto{\pgfqpoint{3.878529in}{1.633251in}}%
\pgfpathlineto{\pgfqpoint{3.878847in}{0.852509in}}%
\pgfpathlineto{\pgfqpoint{3.879481in}{1.812317in}}%
\pgfpathlineto{\pgfqpoint{3.879799in}{0.956947in}}%
\pgfpathlineto{\pgfqpoint{3.880434in}{1.874943in}}%
\pgfpathlineto{\pgfqpoint{3.880751in}{1.165360in}}%
\pgfpathlineto{\pgfqpoint{3.881069in}{0.956650in}}%
\pgfpathlineto{\pgfqpoint{3.881386in}{1.808872in}}%
\pgfpathlineto{\pgfqpoint{3.882021in}{0.840724in}}%
\pgfpathlineto{\pgfqpoint{3.882339in}{1.629067in}}%
\pgfpathlineto{\pgfqpoint{3.882656in}{1.672421in}}%
\pgfpathlineto{\pgfqpoint{3.882973in}{0.855089in}}%
\pgfpathlineto{\pgfqpoint{3.883608in}{1.833573in}}%
\pgfpathlineto{\pgfqpoint{3.883926in}{0.995508in}}%
\pgfpathlineto{\pgfqpoint{3.884243in}{1.119141in}}%
\pgfpathlineto{\pgfqpoint{3.884561in}{1.872226in}}%
\pgfpathlineto{\pgfqpoint{3.885196in}{0.924532in}}%
\pgfpathlineto{\pgfqpoint{3.885513in}{1.780403in}}%
\pgfpathlineto{\pgfqpoint{3.886148in}{0.844421in}}%
\pgfpathlineto{\pgfqpoint{3.886466in}{1.582974in}}%
\pgfpathlineto{\pgfqpoint{3.886783in}{1.724728in}}%
\pgfpathlineto{\pgfqpoint{3.887100in}{0.881217in}}%
\pgfpathlineto{\pgfqpoint{3.887735in}{1.852817in}}%
\pgfpathlineto{\pgfqpoint{3.888053in}{1.038812in}}%
\pgfpathlineto{\pgfqpoint{3.888370in}{1.062254in}}%
\pgfpathlineto{\pgfqpoint{3.888688in}{1.857943in}}%
\pgfpathlineto{\pgfqpoint{3.889323in}{0.885981in}}%
\pgfpathlineto{\pgfqpoint{3.889640in}{1.736656in}}%
\pgfpathlineto{\pgfqpoint{3.890275in}{0.827990in}}%
\pgfpathlineto{\pgfqpoint{3.890592in}{1.517130in}}%
\pgfpathlineto{\pgfqpoint{3.890910in}{1.756228in}}%
\pgfpathlineto{\pgfqpoint{3.891227in}{0.901181in}}%
\pgfpathlineto{\pgfqpoint{3.891862in}{1.864401in}}%
\pgfpathlineto{\pgfqpoint{3.892180in}{1.089156in}}%
\pgfpathlineto{\pgfqpoint{3.892497in}{1.015328in}}%
\pgfpathlineto{\pgfqpoint{3.892815in}{1.842409in}}%
\pgfpathlineto{\pgfqpoint{3.893450in}{0.871463in}}%
\pgfpathlineto{\pgfqpoint{3.893767in}{1.700779in}}%
\pgfpathlineto{\pgfqpoint{3.894084in}{1.606287in}}%
\pgfpathlineto{\pgfqpoint{3.894402in}{0.843373in}}%
\pgfpathlineto{\pgfqpoint{3.895037in}{1.794138in}}%
\pgfpathlineto{\pgfqpoint{3.895354in}{0.934901in}}%
\pgfpathlineto{\pgfqpoint{3.895989in}{1.867862in}}%
\pgfpathlineto{\pgfqpoint{3.896307in}{1.138288in}}%
\pgfpathlineto{\pgfqpoint{3.896624in}{0.966094in}}%
\pgfpathlineto{\pgfqpoint{3.896942in}{1.814787in}}%
\pgfpathlineto{\pgfqpoint{3.897577in}{0.840619in}}%
\pgfpathlineto{\pgfqpoint{3.897894in}{1.644087in}}%
\pgfpathlineto{\pgfqpoint{3.898211in}{1.646758in}}%
\pgfpathlineto{\pgfqpoint{3.898529in}{0.842748in}}%
\pgfpathlineto{\pgfqpoint{3.899164in}{1.817887in}}%
\pgfpathlineto{\pgfqpoint{3.899481in}{0.970859in}}%
\pgfpathlineto{\pgfqpoint{3.900116in}{1.867893in}}%
\pgfpathlineto{\pgfqpoint{3.900434in}{1.196591in}}%
\pgfpathlineto{\pgfqpoint{3.900751in}{0.932063in}}%
\pgfpathlineto{\pgfqpoint{3.901069in}{1.789003in}}%
\pgfpathlineto{\pgfqpoint{3.901703in}{0.841856in}}%
\pgfpathlineto{\pgfqpoint{3.902021in}{1.599981in}}%
\pgfpathlineto{\pgfqpoint{3.902338in}{1.700451in}}%
\pgfpathlineto{\pgfqpoint{3.902656in}{0.866254in}}%
\pgfpathlineto{\pgfqpoint{3.903291in}{1.839647in}}%
\pgfpathlineto{\pgfqpoint{3.903608in}{1.011838in}}%
\pgfpathlineto{\pgfqpoint{3.903926in}{1.078996in}}%
\pgfpathlineto{\pgfqpoint{3.904243in}{1.856600in}}%
\pgfpathlineto{\pgfqpoint{3.904878in}{0.891397in}}%
\pgfpathlineto{\pgfqpoint{3.905196in}{1.747472in}}%
\pgfpathlineto{\pgfqpoint{3.905830in}{0.822023in}}%
\pgfpathlineto{\pgfqpoint{3.906148in}{1.535691in}}%
\pgfpathlineto{\pgfqpoint{3.906465in}{1.733888in}}%
\pgfpathlineto{\pgfqpoint{3.906783in}{0.883725in}}%
\pgfpathlineto{\pgfqpoint{3.907418in}{1.854212in}}%
\pgfpathlineto{\pgfqpoint{3.907735in}{1.060317in}}%
\pgfpathlineto{\pgfqpoint{3.908053in}{1.030703in}}%
\pgfpathlineto{\pgfqpoint{3.908370in}{1.843988in}}%
\pgfpathlineto{\pgfqpoint{3.909005in}{0.875318in}}%
\pgfpathlineto{\pgfqpoint{3.909322in}{1.714218in}}%
\pgfpathlineto{\pgfqpoint{3.909640in}{1.577107in}}%
\pgfpathlineto{\pgfqpoint{3.909957in}{0.834167in}}%
\pgfpathlineto{\pgfqpoint{3.910592in}{1.773381in}}%
\pgfpathlineto{\pgfqpoint{3.910910in}{0.914469in}}%
\pgfpathlineto{\pgfqpoint{3.911545in}{1.860363in}}%
\pgfpathlineto{\pgfqpoint{3.911862in}{1.107896in}}%
\pgfpathlineto{\pgfqpoint{3.912180in}{0.979644in}}%
\pgfpathlineto{\pgfqpoint{3.912497in}{1.819378in}}%
\pgfpathlineto{\pgfqpoint{3.913132in}{0.841164in}}%
\pgfpathlineto{\pgfqpoint{3.913449in}{1.659007in}}%
\pgfpathlineto{\pgfqpoint{3.913767in}{1.618946in}}%
\pgfpathlineto{\pgfqpoint{3.914084in}{0.830523in}}%
\pgfpathlineto{\pgfqpoint{3.914719in}{1.799881in}}%
\pgfpathlineto{\pgfqpoint{3.915037in}{0.948548in}}%
\pgfpathlineto{\pgfqpoint{3.915672in}{1.863633in}}%
\pgfpathlineto{\pgfqpoint{3.915989in}{1.164723in}}%
\pgfpathlineto{\pgfqpoint{3.916307in}{0.944000in}}%
\pgfpathlineto{\pgfqpoint{3.916624in}{1.796774in}}%
\pgfpathlineto{\pgfqpoint{3.917259in}{0.840238in}}%
\pgfpathlineto{\pgfqpoint{3.917576in}{1.616598in}}%
\pgfpathlineto{\pgfqpoint{3.917894in}{1.674018in}}%
\pgfpathlineto{\pgfqpoint{3.918211in}{0.850367in}}%
\pgfpathlineto{\pgfqpoint{3.918846in}{1.823823in}}%
\pgfpathlineto{\pgfqpoint{3.919164in}{0.986965in}}%
\pgfpathlineto{\pgfqpoint{3.919481in}{1.094615in}}%
\pgfpathlineto{\pgfqpoint{3.919799in}{1.855308in}}%
\pgfpathlineto{\pgfqpoint{3.920433in}{0.900488in}}%
\pgfpathlineto{\pgfqpoint{3.920751in}{1.757745in}}%
\pgfpathlineto{\pgfqpoint{3.921386in}{0.817233in}}%
\pgfpathlineto{\pgfqpoint{3.921703in}{1.553776in}}%
\pgfpathlineto{\pgfqpoint{3.922021in}{1.709893in}}%
\pgfpathlineto{\pgfqpoint{3.922338in}{0.865336in}}%
\pgfpathlineto{\pgfqpoint{3.922973in}{1.841434in}}%
\pgfpathlineto{\pgfqpoint{3.923291in}{1.033791in}}%
\pgfpathlineto{\pgfqpoint{3.923608in}{1.045239in}}%
\pgfpathlineto{\pgfqpoint{3.923926in}{1.845755in}}%
\pgfpathlineto{\pgfqpoint{3.924560in}{0.882818in}}%
\pgfpathlineto{\pgfqpoint{3.924878in}{1.727337in}}%
\pgfpathlineto{\pgfqpoint{3.925513in}{0.825878in}}%
\pgfpathlineto{\pgfqpoint{3.925830in}{1.501083in}}%
\pgfpathlineto{\pgfqpoint{3.926148in}{1.750836in}}%
\pgfpathlineto{\pgfqpoint{3.926465in}{0.892239in}}%
\pgfpathlineto{\pgfqpoint{3.927100in}{1.850231in}}%
\pgfpathlineto{\pgfqpoint{3.927418in}{1.079419in}}%
\pgfpathlineto{\pgfqpoint{3.927735in}{0.992549in}}%
\pgfpathlineto{\pgfqpoint{3.928052in}{1.824222in}}%
\pgfpathlineto{\pgfqpoint{3.928687in}{0.844825in}}%
\pgfpathlineto{\pgfqpoint{3.929005in}{1.674377in}}%
\pgfpathlineto{\pgfqpoint{3.929322in}{1.590468in}}%
\pgfpathlineto{\pgfqpoint{3.929640in}{0.820046in}}%
\pgfpathlineto{\pgfqpoint{3.930275in}{1.780681in}}%
\pgfpathlineto{\pgfqpoint{3.930592in}{0.924381in}}%
\pgfpathlineto{\pgfqpoint{3.931227in}{1.856837in}}%
\pgfpathlineto{\pgfqpoint{3.931544in}{1.134895in}}%
\pgfpathlineto{\pgfqpoint{3.931862in}{0.957069in}}%
\pgfpathlineto{\pgfqpoint{3.932179in}{1.805604in}}%
\pgfpathlineto{\pgfqpoint{3.932814in}{0.840656in}}%
\pgfpathlineto{\pgfqpoint{3.933132in}{1.633365in}}%
\pgfpathlineto{\pgfqpoint{3.933449in}{1.645837in}}%
\pgfpathlineto{\pgfqpoint{3.933767in}{0.835504in}}%
\pgfpathlineto{\pgfqpoint{3.934402in}{1.806566in}}%
\pgfpathlineto{\pgfqpoint{3.934719in}{0.959456in}}%
\pgfpathlineto{\pgfqpoint{3.935037in}{1.114506in}}%
\pgfpathlineto{\pgfqpoint{3.935354in}{1.851642in}}%
\pgfpathlineto{\pgfqpoint{3.935989in}{0.909562in}}%
\pgfpathlineto{\pgfqpoint{3.936306in}{1.768229in}}%
\pgfpathlineto{\pgfqpoint{3.936941in}{0.814554in}}%
\pgfpathlineto{\pgfqpoint{3.937259in}{1.573236in}}%
\pgfpathlineto{\pgfqpoint{3.937576in}{1.684312in}}%
\pgfpathlineto{\pgfqpoint{3.937894in}{0.849035in}}%
\pgfpathlineto{\pgfqpoint{3.938529in}{1.827946in}}%
\pgfpathlineto{\pgfqpoint{3.938846in}{1.004721in}}%
\pgfpathlineto{\pgfqpoint{3.939163in}{1.064395in}}%
\pgfpathlineto{\pgfqpoint{3.939481in}{1.845693in}}%
\pgfpathlineto{\pgfqpoint{3.940116in}{0.890748in}}%
\pgfpathlineto{\pgfqpoint{3.940433in}{1.740497in}}%
\pgfpathlineto{\pgfqpoint{3.941068in}{0.818405in}}%
\pgfpathlineto{\pgfqpoint{3.941386in}{1.520961in}}%
\pgfpathlineto{\pgfqpoint{3.941703in}{1.726124in}}%
\pgfpathlineto{\pgfqpoint{3.942021in}{0.871990in}}%
\pgfpathlineto{\pgfqpoint{3.942656in}{1.839539in}}%
\pgfpathlineto{\pgfqpoint{3.942973in}{1.048008in}}%
\pgfpathlineto{\pgfqpoint{3.943290in}{1.009992in}}%
\pgfpathlineto{\pgfqpoint{3.943608in}{1.827213in}}%
\pgfpathlineto{\pgfqpoint{3.944243in}{0.848984in}}%
\pgfpathlineto{\pgfqpoint{3.944560in}{1.689697in}}%
\pgfpathlineto{\pgfqpoint{3.944878in}{1.559305in}}%
\pgfpathlineto{\pgfqpoint{3.945195in}{0.810540in}}%
\pgfpathlineto{\pgfqpoint{3.945830in}{1.759213in}}%
\pgfpathlineto{\pgfqpoint{3.946148in}{0.902578in}}%
\pgfpathlineto{\pgfqpoint{3.946782in}{1.849720in}}%
\pgfpathlineto{\pgfqpoint{3.947100in}{1.101938in}}%
\pgfpathlineto{\pgfqpoint{3.947417in}{0.974875in}}%
\pgfpathlineto{\pgfqpoint{3.947735in}{1.812590in}}%
\pgfpathlineto{\pgfqpoint{3.948370in}{0.841224in}}%
\pgfpathlineto{\pgfqpoint{3.948687in}{1.649569in}}%
\pgfpathlineto{\pgfqpoint{3.949005in}{1.614815in}}%
\pgfpathlineto{\pgfqpoint{3.949322in}{0.820705in}}%
\pgfpathlineto{\pgfqpoint{3.949957in}{1.786894in}}%
\pgfpathlineto{\pgfqpoint{3.950274in}{0.934445in}}%
\pgfpathlineto{\pgfqpoint{3.950909in}{1.847878in}}%
\pgfpathlineto{\pgfqpoint{3.951227in}{1.152186in}}%
\pgfpathlineto{\pgfqpoint{3.951544in}{0.922931in}}%
\pgfpathlineto{\pgfqpoint{3.951862in}{1.777495in}}%
\pgfpathlineto{\pgfqpoint{3.952497in}{0.812877in}}%
\pgfpathlineto{\pgfqpoint{3.952814in}{1.592336in}}%
\pgfpathlineto{\pgfqpoint{3.953132in}{1.656414in}}%
\pgfpathlineto{\pgfqpoint{3.953449in}{0.832603in}}%
\pgfpathlineto{\pgfqpoint{3.954084in}{1.811805in}}%
\pgfpathlineto{\pgfqpoint{3.954401in}{0.978112in}}%
\pgfpathlineto{\pgfqpoint{3.954719in}{1.084227in}}%
\pgfpathlineto{\pgfqpoint{3.955036in}{1.846759in}}%
\pgfpathlineto{\pgfqpoint{3.955671in}{0.902036in}}%
\pgfpathlineto{\pgfqpoint{3.955989in}{1.751986in}}%
\pgfpathlineto{\pgfqpoint{3.956624in}{0.811213in}}%
\pgfpathlineto{\pgfqpoint{3.956941in}{1.539773in}}%
\pgfpathlineto{\pgfqpoint{3.957259in}{1.698934in}}%
\pgfpathlineto{\pgfqpoint{3.957576in}{0.850799in}}%
\pgfpathlineto{\pgfqpoint{3.958211in}{1.826198in}}%
\pgfpathlineto{\pgfqpoint{3.958528in}{1.019153in}}%
\pgfpathlineto{\pgfqpoint{3.958846in}{1.027052in}}%
\pgfpathlineto{\pgfqpoint{3.959163in}{1.830324in}}%
\pgfpathlineto{\pgfqpoint{3.959798in}{0.856781in}}%
\pgfpathlineto{\pgfqpoint{3.960116in}{1.704564in}}%
\pgfpathlineto{\pgfqpoint{3.960751in}{0.802499in}}%
\pgfpathlineto{\pgfqpoint{3.961068in}{1.481284in}}%
\pgfpathlineto{\pgfqpoint{3.961386in}{1.735946in}}%
\pgfpathlineto{\pgfqpoint{3.961703in}{0.879694in}}%
\pgfpathlineto{\pgfqpoint{3.962338in}{1.839719in}}%
\pgfpathlineto{\pgfqpoint{3.962655in}{1.071988in}}%
\pgfpathlineto{\pgfqpoint{3.962973in}{0.992331in}}%
\pgfpathlineto{\pgfqpoint{3.963290in}{1.819738in}}%
\pgfpathlineto{\pgfqpoint{3.963925in}{0.843882in}}%
\pgfpathlineto{\pgfqpoint{3.964243in}{1.665208in}}%
\pgfpathlineto{\pgfqpoint{3.964560in}{1.582529in}}%
\pgfpathlineto{\pgfqpoint{3.964878in}{0.807352in}}%
\pgfpathlineto{\pgfqpoint{3.965512in}{1.765737in}}%
\pgfpathlineto{\pgfqpoint{3.965830in}{0.908100in}}%
\pgfpathlineto{\pgfqpoint{3.966465in}{1.841476in}}%
\pgfpathlineto{\pgfqpoint{3.966782in}{1.120229in}}%
\pgfpathlineto{\pgfqpoint{3.967100in}{0.936194in}}%
\pgfpathlineto{\pgfqpoint{3.967417in}{1.786799in}}%
\pgfpathlineto{\pgfqpoint{3.968052in}{0.813969in}}%
\pgfpathlineto{\pgfqpoint{3.968370in}{1.611956in}}%
\pgfpathlineto{\pgfqpoint{3.968687in}{1.627132in}}%
\pgfpathlineto{\pgfqpoint{3.969004in}{0.818138in}}%
\pgfpathlineto{\pgfqpoint{3.969639in}{1.794427in}}%
\pgfpathlineto{\pgfqpoint{3.969957in}{0.949532in}}%
\pgfpathlineto{\pgfqpoint{3.970274in}{1.109495in}}%
\pgfpathlineto{\pgfqpoint{3.970592in}{1.845529in}}%
\pgfpathlineto{\pgfqpoint{3.971227in}{0.912197in}}%
\pgfpathlineto{\pgfqpoint{3.971544in}{1.762873in}}%
\pgfpathlineto{\pgfqpoint{3.972179in}{0.806001in}}%
\pgfpathlineto{\pgfqpoint{3.972497in}{1.559855in}}%
\pgfpathlineto{\pgfqpoint{3.972814in}{1.670258in}}%
\pgfpathlineto{\pgfqpoint{3.973131in}{0.832141in}}%
\pgfpathlineto{\pgfqpoint{3.973766in}{1.812028in}}%
\pgfpathlineto{\pgfqpoint{3.974084in}{0.988246in}}%
\pgfpathlineto{\pgfqpoint{3.974401in}{1.048683in}}%
\pgfpathlineto{\pgfqpoint{3.974719in}{1.830830in}}%
\pgfpathlineto{\pgfqpoint{3.975354in}{0.864765in}}%
\pgfpathlineto{\pgfqpoint{3.975671in}{1.719269in}}%
\pgfpathlineto{\pgfqpoint{3.976306in}{0.796246in}}%
\pgfpathlineto{\pgfqpoint{3.976623in}{1.504654in}}%
\pgfpathlineto{\pgfqpoint{3.976941in}{1.710614in}}%
\pgfpathlineto{\pgfqpoint{3.977258in}{0.859221in}}%
\pgfpathlineto{\pgfqpoint{3.977893in}{1.829624in}}%
\pgfpathlineto{\pgfqpoint{3.978211in}{1.040660in}}%
\pgfpathlineto{\pgfqpoint{3.978528in}{1.013423in}}%
\pgfpathlineto{\pgfqpoint{3.978846in}{1.823970in}}%
\pgfpathlineto{\pgfqpoint{3.979481in}{0.845697in}}%
\pgfpathlineto{\pgfqpoint{3.979798in}{1.680137in}}%
\pgfpathlineto{\pgfqpoint{3.980116in}{1.546782in}}%
\pgfpathlineto{\pgfqpoint{3.980433in}{0.795190in}}%
\pgfpathlineto{\pgfqpoint{3.981068in}{1.742453in}}%
\pgfpathlineto{\pgfqpoint{3.981385in}{0.884569in}}%
\pgfpathlineto{\pgfqpoint{3.982020in}{1.834579in}}%
\pgfpathlineto{\pgfqpoint{3.982338in}{1.085509in}}%
\pgfpathlineto{\pgfqpoint{3.982655in}{0.953892in}}%
\pgfpathlineto{\pgfqpoint{3.982973in}{1.793948in}}%
\pgfpathlineto{\pgfqpoint{3.983608in}{0.815669in}}%
\pgfpathlineto{\pgfqpoint{3.983925in}{1.631252in}}%
\pgfpathlineto{\pgfqpoint{3.984242in}{1.594946in}}%
\pgfpathlineto{\pgfqpoint{3.984560in}{0.804535in}}%
\pgfpathlineto{\pgfqpoint{3.985195in}{1.774392in}}%
\pgfpathlineto{\pgfqpoint{3.985512in}{0.923455in}}%
\pgfpathlineto{\pgfqpoint{3.986147in}{1.843644in}}%
\pgfpathlineto{\pgfqpoint{3.986465in}{1.147459in}}%
\pgfpathlineto{\pgfqpoint{3.986782in}{0.925300in}}%
\pgfpathlineto{\pgfqpoint{3.987100in}{1.771650in}}%
\pgfpathlineto{\pgfqpoint{3.987734in}{0.801096in}}%
\pgfpathlineto{\pgfqpoint{3.988052in}{1.579531in}}%
\pgfpathlineto{\pgfqpoint{3.988369in}{1.638956in}}%
\pgfpathlineto{\pgfqpoint{3.988687in}{0.814139in}}%
\pgfpathlineto{\pgfqpoint{3.989322in}{1.795279in}}%
\pgfpathlineto{\pgfqpoint{3.989639in}{0.960170in}}%
\pgfpathlineto{\pgfqpoint{3.989957in}{1.069798in}}%
\pgfpathlineto{\pgfqpoint{3.990274in}{1.830926in}}%
\pgfpathlineto{\pgfqpoint{3.990909in}{0.876713in}}%
\pgfpathlineto{\pgfqpoint{3.991227in}{1.732573in}}%
\pgfpathlineto{\pgfqpoint{3.991861in}{0.791247in}}%
\pgfpathlineto{\pgfqpoint{3.992179in}{1.527638in}}%
\pgfpathlineto{\pgfqpoint{3.992496in}{1.682944in}}%
\pgfpathlineto{\pgfqpoint{3.992814in}{0.838603in}}%
\pgfpathlineto{\pgfqpoint{3.993449in}{1.817857in}}%
\pgfpathlineto{\pgfqpoint{3.993766in}{1.012084in}}%
\pgfpathlineto{\pgfqpoint{3.994084in}{1.032867in}}%
\pgfpathlineto{\pgfqpoint{3.994401in}{1.826884in}}%
\pgfpathlineto{\pgfqpoint{3.995036in}{0.850874in}}%
\pgfpathlineto{\pgfqpoint{3.995353in}{1.694345in}}%
\pgfpathlineto{\pgfqpoint{3.995988in}{0.784973in}}%
\pgfpathlineto{\pgfqpoint{3.996306in}{1.468463in}}%
\pgfpathlineto{\pgfqpoint{3.996623in}{1.717387in}}%
\pgfpathlineto{\pgfqpoint{3.996941in}{0.860862in}}%
\pgfpathlineto{\pgfqpoint{3.997576in}{1.824613in}}%
\pgfpathlineto{\pgfqpoint{3.997893in}{1.053089in}}%
\pgfpathlineto{\pgfqpoint{3.998211in}{0.970992in}}%
\pgfpathlineto{\pgfqpoint{3.998528in}{1.800643in}}%
\pgfpathlineto{\pgfqpoint{3.999163in}{0.820790in}}%
\pgfpathlineto{\pgfqpoint{3.999480in}{1.650262in}}%
\pgfpathlineto{\pgfqpoint{3.999798in}{1.561815in}}%
\pgfpathlineto{\pgfqpoint{4.000115in}{0.792898in}}%
\pgfpathlineto{\pgfqpoint{4.000750in}{1.752638in}}%
\pgfpathlineto{\pgfqpoint{4.001068in}{0.897991in}}%
\pgfpathlineto{\pgfqpoint{4.001703in}{1.837777in}}%
\pgfpathlineto{\pgfqpoint{4.002020in}{1.112891in}}%
\pgfpathlineto{\pgfqpoint{4.002338in}{0.937001in}}%
\pgfpathlineto{\pgfqpoint{4.002655in}{1.779441in}}%
\pgfpathlineto{\pgfqpoint{4.003290in}{0.799760in}}%
\pgfpathlineto{\pgfqpoint{4.003607in}{1.600042in}}%
\pgfpathlineto{\pgfqpoint{4.003925in}{1.606752in}}%
\pgfpathlineto{\pgfqpoint{4.004242in}{0.798922in}}%
\pgfpathlineto{\pgfqpoint{4.004877in}{1.777078in}}%
\pgfpathlineto{\pgfqpoint{4.005195in}{0.930629in}}%
\pgfpathlineto{\pgfqpoint{4.005830in}{1.827763in}}%
\pgfpathlineto{\pgfqpoint{4.006147in}{1.157503in}}%
\pgfpathlineto{\pgfqpoint{4.006464in}{0.888445in}}%
\pgfpathlineto{\pgfqpoint{4.006782in}{1.745582in}}%
\pgfpathlineto{\pgfqpoint{4.007417in}{0.789134in}}%
\pgfpathlineto{\pgfqpoint{4.007734in}{1.551523in}}%
\pgfpathlineto{\pgfqpoint{4.008052in}{1.653536in}}%
\pgfpathlineto{\pgfqpoint{4.008369in}{0.820324in}}%
\pgfpathlineto{\pgfqpoint{4.009004in}{1.805167in}}%
\pgfpathlineto{\pgfqpoint{4.009322in}{0.980945in}}%
\pgfpathlineto{\pgfqpoint{4.009639in}{1.054902in}}%
\pgfpathlineto{\pgfqpoint{4.009957in}{1.825562in}}%
\pgfpathlineto{\pgfqpoint{4.010591in}{0.856148in}}%
\pgfpathlineto{\pgfqpoint{4.010909in}{1.708425in}}%
\pgfpathlineto{\pgfqpoint{4.011544in}{0.778047in}}%
\pgfpathlineto{\pgfqpoint{4.011861in}{1.494505in}}%
\pgfpathlineto{\pgfqpoint{4.012179in}{1.690585in}}%
\pgfpathlineto{\pgfqpoint{4.012496in}{0.840034in}}%
\pgfpathlineto{\pgfqpoint{4.013131in}{1.813299in}}%
\pgfpathlineto{\pgfqpoint{4.013449in}{1.017990in}}%
\pgfpathlineto{\pgfqpoint{4.013766in}{0.992448in}}%
\pgfpathlineto{\pgfqpoint{4.014083in}{1.804460in}}%
\pgfpathlineto{\pgfqpoint{4.014718in}{0.826519in}}%
\pgfpathlineto{\pgfqpoint{4.015036in}{1.669148in}}%
\pgfpathlineto{\pgfqpoint{4.015353in}{1.525463in}}%
\pgfpathlineto{\pgfqpoint{4.015671in}{0.783235in}}%
\pgfpathlineto{\pgfqpoint{4.016306in}{1.730372in}}%
\pgfpathlineto{\pgfqpoint{4.016623in}{0.876077in}}%
\pgfpathlineto{\pgfqpoint{4.017258in}{1.829668in}}%
\pgfpathlineto{\pgfqpoint{4.017576in}{1.074292in}}%
\pgfpathlineto{\pgfqpoint{4.017893in}{0.952743in}}%
\pgfpathlineto{\pgfqpoint{4.018210in}{1.784783in}}%
\pgfpathlineto{\pgfqpoint{4.018845in}{0.800365in}}%
\pgfpathlineto{\pgfqpoint{4.019163in}{1.620762in}}%
\pgfpathlineto{\pgfqpoint{4.019480in}{1.571961in}}%
\pgfpathlineto{\pgfqpoint{4.019798in}{0.785820in}}%
\pgfpathlineto{\pgfqpoint{4.020433in}{1.756016in}}%
\pgfpathlineto{\pgfqpoint{4.020750in}{0.903542in}}%
\pgfpathlineto{\pgfqpoint{4.021385in}{1.823482in}}%
\pgfpathlineto{\pgfqpoint{4.021702in}{1.118095in}}%
\pgfpathlineto{\pgfqpoint{4.022020in}{0.904759in}}%
\pgfpathlineto{\pgfqpoint{4.022337in}{1.756604in}}%
\pgfpathlineto{\pgfqpoint{4.022972in}{0.788397in}}%
\pgfpathlineto{\pgfqpoint{4.023290in}{1.576368in}}%
\pgfpathlineto{\pgfqpoint{4.023607in}{1.621834in}}%
\pgfpathlineto{\pgfqpoint{4.023925in}{0.804799in}}%
\pgfpathlineto{\pgfqpoint{4.024560in}{1.787945in}}%
\pgfpathlineto{\pgfqpoint{4.024877in}{0.950888in}}%
\pgfpathlineto{\pgfqpoint{4.025194in}{1.075616in}}%
\pgfpathlineto{\pgfqpoint{4.025512in}{1.823254in}}%
\pgfpathlineto{\pgfqpoint{4.026147in}{0.867133in}}%
\pgfpathlineto{\pgfqpoint{4.026464in}{1.721800in}}%
\pgfpathlineto{\pgfqpoint{4.027099in}{0.773524in}}%
\pgfpathlineto{\pgfqpoint{4.027417in}{1.520173in}}%
\pgfpathlineto{\pgfqpoint{4.027734in}{1.661167in}}%
\pgfpathlineto{\pgfqpoint{4.028052in}{0.819536in}}%
\pgfpathlineto{\pgfqpoint{4.028687in}{1.798410in}}%
\pgfpathlineto{\pgfqpoint{4.029004in}{0.985063in}}%
\pgfpathlineto{\pgfqpoint{4.029321in}{1.013385in}}%
\pgfpathlineto{\pgfqpoint{4.029639in}{1.807646in}}%
\pgfpathlineto{\pgfqpoint{4.030274in}{0.836785in}}%
\pgfpathlineto{\pgfqpoint{4.030591in}{1.686850in}}%
\pgfpathlineto{\pgfqpoint{4.031226in}{0.776657in}}%
\pgfpathlineto{\pgfqpoint{4.031544in}{1.475984in}}%
\pgfpathlineto{\pgfqpoint{4.031861in}{1.706069in}}%
\pgfpathlineto{\pgfqpoint{4.032179in}{0.852990in}}%
\pgfpathlineto{\pgfqpoint{4.032813in}{1.816782in}}%
\pgfpathlineto{\pgfqpoint{4.033131in}{1.037148in}}%
\pgfpathlineto{\pgfqpoint{4.033448in}{0.968217in}}%
\pgfpathlineto{\pgfqpoint{4.033766in}{1.789843in}}%
\pgfpathlineto{\pgfqpoint{4.034401in}{0.806084in}}%
\pgfpathlineto{\pgfqpoint{4.034718in}{1.641458in}}%
\pgfpathlineto{\pgfqpoint{4.035036in}{1.536649in}}%
\pgfpathlineto{\pgfqpoint{4.035353in}{0.774948in}}%
\pgfpathlineto{\pgfqpoint{4.035988in}{1.732443in}}%
\pgfpathlineto{\pgfqpoint{4.036306in}{0.875403in}}%
\pgfpathlineto{\pgfqpoint{4.036940in}{1.815604in}}%
\pgfpathlineto{\pgfqpoint{4.037258in}{1.080769in}}%
\pgfpathlineto{\pgfqpoint{4.037575in}{0.921255in}}%
\pgfpathlineto{\pgfqpoint{4.037893in}{1.767280in}}%
\pgfpathlineto{\pgfqpoint{4.038528in}{0.792211in}}%
\pgfpathlineto{\pgfqpoint{4.038845in}{1.604410in}}%
\pgfpathlineto{\pgfqpoint{4.039163in}{1.590687in}}%
\pgfpathlineto{\pgfqpoint{4.039480in}{0.793604in}}%
\pgfpathlineto{\pgfqpoint{4.040115in}{1.767154in}}%
\pgfpathlineto{\pgfqpoint{4.040432in}{0.917897in}}%
\pgfpathlineto{\pgfqpoint{4.041067in}{1.816770in}}%
\pgfpathlineto{\pgfqpoint{4.041385in}{1.133550in}}%
\pgfpathlineto{\pgfqpoint{4.041702in}{0.878511in}}%
\pgfpathlineto{\pgfqpoint{4.042020in}{1.735128in}}%
\pgfpathlineto{\pgfqpoint{4.042655in}{0.773061in}}%
\pgfpathlineto{\pgfqpoint{4.042972in}{1.546540in}}%
\pgfpathlineto{\pgfqpoint{4.043290in}{1.629881in}}%
\pgfpathlineto{\pgfqpoint{4.043607in}{0.801266in}}%
\pgfpathlineto{\pgfqpoint{4.044242in}{1.781376in}}%
\pgfpathlineto{\pgfqpoint{4.044559in}{0.950155in}}%
\pgfpathlineto{\pgfqpoint{4.044877in}{1.039091in}}%
\pgfpathlineto{\pgfqpoint{4.045194in}{1.807627in}}%
\pgfpathlineto{\pgfqpoint{4.045829in}{0.847747in}}%
\pgfpathlineto{\pgfqpoint{4.046147in}{1.707633in}}%
\pgfpathlineto{\pgfqpoint{4.046782in}{0.776599in}}%
\pgfpathlineto{\pgfqpoint{4.047099in}{1.506069in}}%
\pgfpathlineto{\pgfqpoint{4.047417in}{1.677441in}}%
\pgfpathlineto{\pgfqpoint{4.047734in}{0.831009in}}%
\pgfpathlineto{\pgfqpoint{4.048369in}{1.800887in}}%
\pgfpathlineto{\pgfqpoint{4.048686in}{0.996892in}}%
\pgfpathlineto{\pgfqpoint{4.049004in}{0.988591in}}%
\pgfpathlineto{\pgfqpoint{4.049321in}{1.792198in}}%
\pgfpathlineto{\pgfqpoint{4.049956in}{0.813308in}}%
\pgfpathlineto{\pgfqpoint{4.050274in}{1.662176in}}%
\pgfpathlineto{\pgfqpoint{4.050909in}{0.766610in}}%
\pgfpathlineto{\pgfqpoint{4.051226in}{1.439370in}}%
\pgfpathlineto{\pgfqpoint{4.051543in}{1.705875in}}%
\pgfpathlineto{\pgfqpoint{4.051861in}{0.849538in}}%
\pgfpathlineto{\pgfqpoint{4.052496in}{1.806244in}}%
\pgfpathlineto{\pgfqpoint{4.052813in}{1.040977in}}%
\pgfpathlineto{\pgfqpoint{4.053131in}{0.942875in}}%
\pgfpathlineto{\pgfqpoint{4.053448in}{1.775969in}}%
\pgfpathlineto{\pgfqpoint{4.054083in}{0.801443in}}%
\pgfpathlineto{\pgfqpoint{4.054401in}{1.630797in}}%
\pgfpathlineto{\pgfqpoint{4.054718in}{1.555553in}}%
\pgfpathlineto{\pgfqpoint{4.055036in}{0.782270in}}%
\pgfpathlineto{\pgfqpoint{4.055670in}{1.741091in}}%
\pgfpathlineto{\pgfqpoint{4.055988in}{0.886070in}}%
\pgfpathlineto{\pgfqpoint{4.056623in}{1.809029in}}%
\pgfpathlineto{\pgfqpoint{4.056940in}{1.089870in}}%
\pgfpathlineto{\pgfqpoint{4.057258in}{0.896011in}}%
\pgfpathlineto{\pgfqpoint{4.057575in}{1.746633in}}%
\pgfpathlineto{\pgfqpoint{4.058210in}{0.774270in}}%
\pgfpathlineto{\pgfqpoint{4.058528in}{1.572469in}}%
\pgfpathlineto{\pgfqpoint{4.058845in}{1.594939in}}%
\pgfpathlineto{\pgfqpoint{4.059162in}{0.784031in}}%
\pgfpathlineto{\pgfqpoint{4.059797in}{1.760891in}}%
\pgfpathlineto{\pgfqpoint{4.060115in}{0.917901in}}%
\pgfpathlineto{\pgfqpoint{4.060432in}{1.064747in}}%
\pgfpathlineto{\pgfqpoint{4.060750in}{1.806526in}}%
\pgfpathlineto{\pgfqpoint{4.061385in}{0.867854in}}%
\pgfpathlineto{\pgfqpoint{4.061702in}{1.726453in}}%
\pgfpathlineto{\pgfqpoint{4.062337in}{0.776013in}}%
\pgfpathlineto{\pgfqpoint{4.062655in}{1.532086in}}%
\pgfpathlineto{\pgfqpoint{4.062972in}{1.643538in}}%
\pgfpathlineto{\pgfqpoint{4.063289in}{0.807664in}}%
\pgfpathlineto{\pgfqpoint{4.063924in}{1.781185in}}%
\pgfpathlineto{\pgfqpoint{4.064242in}{0.959475in}}%
\pgfpathlineto{\pgfqpoint{4.064559in}{1.009964in}}%
\pgfpathlineto{\pgfqpoint{4.064877in}{1.794313in}}%
\pgfpathlineto{\pgfqpoint{4.065512in}{0.826021in}}%
\pgfpathlineto{\pgfqpoint{4.065829in}{1.681313in}}%
\pgfpathlineto{\pgfqpoint{4.066464in}{0.759845in}}%
\pgfpathlineto{\pgfqpoint{4.066781in}{1.468479in}}%
\pgfpathlineto{\pgfqpoint{4.067099in}{1.676002in}}%
\pgfpathlineto{\pgfqpoint{4.067416in}{0.823831in}}%
\pgfpathlineto{\pgfqpoint{4.068051in}{1.793295in}}%
\pgfpathlineto{\pgfqpoint{4.068369in}{1.003816in}}%
\pgfpathlineto{\pgfqpoint{4.068686in}{0.966754in}}%
\pgfpathlineto{\pgfqpoint{4.069004in}{1.785991in}}%
\pgfpathlineto{\pgfqpoint{4.069639in}{0.813443in}}%
\pgfpathlineto{\pgfqpoint{4.069956in}{1.652454in}}%
\pgfpathlineto{\pgfqpoint{4.070273in}{1.517136in}}%
\pgfpathlineto{\pgfqpoint{4.070591in}{0.769938in}}%
\pgfpathlineto{\pgfqpoint{4.071226in}{1.712097in}}%
\pgfpathlineto{\pgfqpoint{4.071543in}{0.854082in}}%
\pgfpathlineto{\pgfqpoint{4.072178in}{1.798473in}}%
\pgfpathlineto{\pgfqpoint{4.072496in}{1.049429in}}%
\pgfpathlineto{\pgfqpoint{4.072813in}{0.914746in}}%
\pgfpathlineto{\pgfqpoint{4.073131in}{1.757744in}}%
\pgfpathlineto{\pgfqpoint{4.073766in}{0.779709in}}%
\pgfpathlineto{\pgfqpoint{4.074083in}{1.597513in}}%
\pgfpathlineto{\pgfqpoint{4.074400in}{1.558283in}}%
\pgfpathlineto{\pgfqpoint{4.074718in}{0.768859in}}%
\pgfpathlineto{\pgfqpoint{4.075353in}{1.737798in}}%
\pgfpathlineto{\pgfqpoint{4.075670in}{0.885022in}}%
\pgfpathlineto{\pgfqpoint{4.076305in}{1.803021in}}%
\pgfpathlineto{\pgfqpoint{4.076623in}{1.104476in}}%
\pgfpathlineto{\pgfqpoint{4.076940in}{0.888952in}}%
\pgfpathlineto{\pgfqpoint{4.077258in}{1.742325in}}%
\pgfpathlineto{\pgfqpoint{4.077892in}{0.776076in}}%
\pgfpathlineto{\pgfqpoint{4.078210in}{1.556176in}}%
\pgfpathlineto{\pgfqpoint{4.078527in}{1.606827in}}%
\pgfpathlineto{\pgfqpoint{4.078845in}{0.785432in}}%
\pgfpathlineto{\pgfqpoint{4.079480in}{1.759857in}}%
\pgfpathlineto{\pgfqpoint{4.079797in}{0.921196in}}%
\pgfpathlineto{\pgfqpoint{4.080115in}{1.037117in}}%
\pgfpathlineto{\pgfqpoint{4.080432in}{1.793398in}}%
\pgfpathlineto{\pgfqpoint{4.081067in}{0.839969in}}%
\pgfpathlineto{\pgfqpoint{4.081385in}{1.699999in}}%
\pgfpathlineto{\pgfqpoint{4.082019in}{0.756547in}}%
\pgfpathlineto{\pgfqpoint{4.082337in}{1.498019in}}%
\pgfpathlineto{\pgfqpoint{4.082654in}{1.643665in}}%
\pgfpathlineto{\pgfqpoint{4.082972in}{0.800789in}}%
\pgfpathlineto{\pgfqpoint{4.083607in}{1.778130in}}%
\pgfpathlineto{\pgfqpoint{4.083924in}{0.964887in}}%
\pgfpathlineto{\pgfqpoint{4.084242in}{0.998453in}}%
\pgfpathlineto{\pgfqpoint{4.084559in}{1.792776in}}%
\pgfpathlineto{\pgfqpoint{4.085194in}{0.823998in}}%
\pgfpathlineto{\pgfqpoint{4.085511in}{1.671293in}}%
\pgfpathlineto{\pgfqpoint{4.086146in}{0.758310in}}%
\pgfpathlineto{\pgfqpoint{4.086464in}{1.445175in}}%
\pgfpathlineto{\pgfqpoint{4.086781in}{1.680137in}}%
\pgfpathlineto{\pgfqpoint{4.087099in}{0.824424in}}%
\pgfpathlineto{\pgfqpoint{4.087734in}{1.786704in}}%
\pgfpathlineto{\pgfqpoint{4.088051in}{1.007259in}}%
\pgfpathlineto{\pgfqpoint{4.088369in}{0.939691in}}%
\pgfpathlineto{\pgfqpoint{4.088686in}{1.765955in}}%
\pgfpathlineto{\pgfqpoint{4.089321in}{0.786551in}}%
\pgfpathlineto{\pgfqpoint{4.089638in}{1.622021in}}%
\pgfpathlineto{\pgfqpoint{4.089956in}{1.517575in}}%
\pgfpathlineto{\pgfqpoint{4.090273in}{0.756224in}}%
\pgfpathlineto{\pgfqpoint{4.090908in}{1.711585in}}%
\pgfpathlineto{\pgfqpoint{4.091226in}{0.855077in}}%
\pgfpathlineto{\pgfqpoint{4.091861in}{1.799878in}}%
\pgfpathlineto{\pgfqpoint{4.092178in}{1.063684in}}%
\pgfpathlineto{\pgfqpoint{4.092496in}{0.912754in}}%
\pgfpathlineto{\pgfqpoint{4.092813in}{1.753226in}}%
\pgfpathlineto{\pgfqpoint{4.093448in}{0.774946in}}%
\pgfpathlineto{\pgfqpoint{4.093765in}{1.578891in}}%
\pgfpathlineto{\pgfqpoint{4.094083in}{1.565491in}}%
\pgfpathlineto{\pgfqpoint{4.094400in}{0.764702in}}%
\pgfpathlineto{\pgfqpoint{4.095035in}{1.735911in}}%
\pgfpathlineto{\pgfqpoint{4.095353in}{0.886522in}}%
\pgfpathlineto{\pgfqpoint{4.095988in}{1.791534in}}%
\pgfpathlineto{\pgfqpoint{4.096305in}{1.107365in}}%
\pgfpathlineto{\pgfqpoint{4.096622in}{0.859556in}}%
\pgfpathlineto{\pgfqpoint{4.096940in}{1.715894in}}%
\pgfpathlineto{\pgfqpoint{4.097575in}{0.754800in}}%
\pgfpathlineto{\pgfqpoint{4.097892in}{1.527123in}}%
\pgfpathlineto{\pgfqpoint{4.098210in}{1.607586in}}%
\pgfpathlineto{\pgfqpoint{4.098527in}{0.779413in}}%
\pgfpathlineto{\pgfqpoint{4.099162in}{1.760083in}}%
\pgfpathlineto{\pgfqpoint{4.099480in}{0.930323in}}%
\pgfpathlineto{\pgfqpoint{4.099797in}{1.029210in}}%
\pgfpathlineto{\pgfqpoint{4.100115in}{1.796092in}}%
\pgfpathlineto{\pgfqpoint{4.100749in}{0.836728in}}%
\pgfpathlineto{\pgfqpoint{4.101067in}{1.687109in}}%
\pgfpathlineto{\pgfqpoint{4.101702in}{0.748305in}}%
\pgfpathlineto{\pgfqpoint{4.102019in}{1.473282in}}%
\pgfpathlineto{\pgfqpoint{4.102337in}{1.645354in}}%
\pgfpathlineto{\pgfqpoint{4.102654in}{0.796853in}}%
\pgfpathlineto{\pgfqpoint{4.103289in}{1.771891in}}%
\pgfpathlineto{\pgfqpoint{4.103607in}{0.968412in}}%
\pgfpathlineto{\pgfqpoint{4.103924in}{0.965590in}}%
\pgfpathlineto{\pgfqpoint{4.104241in}{1.772946in}}%
\pgfpathlineto{\pgfqpoint{4.104876in}{0.798271in}}%
\pgfpathlineto{\pgfqpoint{4.105194in}{1.644588in}}%
\pgfpathlineto{\pgfqpoint{4.105829in}{0.745823in}}%
\pgfpathlineto{\pgfqpoint{4.106146in}{1.420637in}}%
\pgfpathlineto{\pgfqpoint{4.106464in}{1.681975in}}%
\pgfpathlineto{\pgfqpoint{4.106781in}{0.825556in}}%
\pgfpathlineto{\pgfqpoint{4.107416in}{1.791246in}}%
\pgfpathlineto{\pgfqpoint{4.107733in}{1.023971in}}%
\pgfpathlineto{\pgfqpoint{4.108051in}{0.934669in}}%
\pgfpathlineto{\pgfqpoint{4.108368in}{1.761890in}}%
\pgfpathlineto{\pgfqpoint{4.109003in}{0.778213in}}%
\pgfpathlineto{\pgfqpoint{4.109321in}{1.601788in}}%
\pgfpathlineto{\pgfqpoint{4.109638in}{1.523541in}}%
\pgfpathlineto{\pgfqpoint{4.109956in}{0.747881in}}%
\pgfpathlineto{\pgfqpoint{4.110591in}{1.709605in}}%
\pgfpathlineto{\pgfqpoint{4.110908in}{0.852844in}}%
\pgfpathlineto{\pgfqpoint{4.111543in}{1.785588in}}%
\pgfpathlineto{\pgfqpoint{4.111860in}{1.064613in}}%
\pgfpathlineto{\pgfqpoint{4.112178in}{0.879695in}}%
\pgfpathlineto{\pgfqpoint{4.112495in}{1.730587in}}%
\pgfpathlineto{\pgfqpoint{4.113130in}{0.757520in}}%
\pgfpathlineto{\pgfqpoint{4.113448in}{1.555648in}}%
\pgfpathlineto{\pgfqpoint{4.113765in}{1.569620in}}%
\pgfpathlineto{\pgfqpoint{4.114083in}{0.760805in}}%
\pgfpathlineto{\pgfqpoint{4.114718in}{1.739915in}}%
\pgfpathlineto{\pgfqpoint{4.115035in}{0.895037in}}%
\pgfpathlineto{\pgfqpoint{4.115670in}{1.793479in}}%
\pgfpathlineto{\pgfqpoint{4.115987in}{1.123639in}}%
\pgfpathlineto{\pgfqpoint{4.116305in}{0.850190in}}%
\pgfpathlineto{\pgfqpoint{4.116622in}{1.702853in}}%
\pgfpathlineto{\pgfqpoint{4.117257in}{0.743786in}}%
\pgfpathlineto{\pgfqpoint{4.117575in}{1.503226in}}%
\pgfpathlineto{\pgfqpoint{4.117892in}{1.609168in}}%
\pgfpathlineto{\pgfqpoint{4.118210in}{0.773447in}}%
\pgfpathlineto{\pgfqpoint{4.118845in}{1.754532in}}%
\pgfpathlineto{\pgfqpoint{4.119162in}{0.928676in}}%
\pgfpathlineto{\pgfqpoint{4.119479in}{0.996421in}}%
\pgfpathlineto{\pgfqpoint{4.119797in}{1.775459in}}%
\pgfpathlineto{\pgfqpoint{4.120432in}{0.810876in}}%
\pgfpathlineto{\pgfqpoint{4.120749in}{1.666333in}}%
\pgfpathlineto{\pgfqpoint{4.121384in}{0.739386in}}%
\pgfpathlineto{\pgfqpoint{4.121702in}{1.453928in}}%
\pgfpathlineto{\pgfqpoint{4.122019in}{1.649438in}}%
\pgfpathlineto{\pgfqpoint{4.122337in}{0.799018in}}%
\pgfpathlineto{\pgfqpoint{4.122971in}{1.778851in}}%
\pgfpathlineto{\pgfqpoint{4.123289in}{0.981363in}}%
\pgfpathlineto{\pgfqpoint{4.123606in}{0.960955in}}%
\pgfpathlineto{\pgfqpoint{4.123924in}{1.766476in}}%
\pgfpathlineto{\pgfqpoint{4.124559in}{0.784035in}}%
\pgfpathlineto{\pgfqpoint{4.124876in}{1.624958in}}%
\pgfpathlineto{\pgfqpoint{4.125194in}{1.477865in}}%
\pgfpathlineto{\pgfqpoint{4.125511in}{0.735600in}}%
\pgfpathlineto{\pgfqpoint{4.126146in}{1.680463in}}%
\pgfpathlineto{\pgfqpoint{4.126463in}{0.822826in}}%
\pgfpathlineto{\pgfqpoint{4.127098in}{1.777205in}}%
\pgfpathlineto{\pgfqpoint{4.127416in}{1.019621in}}%
\pgfpathlineto{\pgfqpoint{4.127733in}{0.905099in}}%
\pgfpathlineto{\pgfqpoint{4.128051in}{1.741406in}}%
\pgfpathlineto{\pgfqpoint{4.128686in}{0.761857in}}%
\pgfpathlineto{\pgfqpoint{4.129003in}{1.583710in}}%
\pgfpathlineto{\pgfqpoint{4.129321in}{1.527444in}}%
\pgfpathlineto{\pgfqpoint{4.129638in}{0.745241in}}%
\pgfpathlineto{\pgfqpoint{4.130273in}{1.716273in}}%
\pgfpathlineto{\pgfqpoint{4.130590in}{0.862775in}}%
\pgfpathlineto{\pgfqpoint{4.131225in}{1.787930in}}%
\pgfpathlineto{\pgfqpoint{4.131543in}{1.074822in}}%
\pgfpathlineto{\pgfqpoint{4.131860in}{0.869055in}}%
\pgfpathlineto{\pgfqpoint{4.132178in}{1.715588in}}%
\pgfpathlineto{\pgfqpoint{4.132813in}{0.742175in}}%
\pgfpathlineto{\pgfqpoint{4.133130in}{1.533324in}}%
\pgfpathlineto{\pgfqpoint{4.133448in}{1.569334in}}%
\pgfpathlineto{\pgfqpoint{4.133765in}{0.753372in}}%
\pgfpathlineto{\pgfqpoint{4.134400in}{1.733250in}}%
\pgfpathlineto{\pgfqpoint{4.134717in}{0.892209in}}%
\pgfpathlineto{\pgfqpoint{4.135035in}{1.027662in}}%
\pgfpathlineto{\pgfqpoint{4.135352in}{1.775879in}}%
\pgfpathlineto{\pgfqpoint{4.135987in}{0.829006in}}%
\pgfpathlineto{\pgfqpoint{4.136305in}{1.685165in}}%
\pgfpathlineto{\pgfqpoint{4.136940in}{0.735272in}}%
\pgfpathlineto{\pgfqpoint{4.137257in}{1.486875in}}%
\pgfpathlineto{\pgfqpoint{4.137575in}{1.613559in}}%
\pgfpathlineto{\pgfqpoint{4.137892in}{0.775562in}}%
\pgfpathlineto{\pgfqpoint{4.138527in}{1.761301in}}%
\pgfpathlineto{\pgfqpoint{4.138844in}{0.941022in}}%
\pgfpathlineto{\pgfqpoint{4.139162in}{0.987072in}}%
\pgfpathlineto{\pgfqpoint{4.139479in}{1.768985in}}%
\pgfpathlineto{\pgfqpoint{4.140114in}{0.795967in}}%
\pgfpathlineto{\pgfqpoint{4.140432in}{1.646545in}}%
\pgfpathlineto{\pgfqpoint{4.141067in}{0.726792in}}%
\pgfpathlineto{\pgfqpoint{4.141384in}{1.431511in}}%
\pgfpathlineto{\pgfqpoint{4.141701in}{1.647675in}}%
\pgfpathlineto{\pgfqpoint{4.142019in}{0.794523in}}%
\pgfpathlineto{\pgfqpoint{4.142654in}{1.764110in}}%
\pgfpathlineto{\pgfqpoint{4.142971in}{0.977112in}}%
\pgfpathlineto{\pgfqpoint{4.143289in}{0.930901in}}%
\pgfpathlineto{\pgfqpoint{4.143606in}{1.750534in}}%
\pgfpathlineto{\pgfqpoint{4.144241in}{0.771800in}}%
\pgfpathlineto{\pgfqpoint{4.144559in}{1.609998in}}%
\pgfpathlineto{\pgfqpoint{4.144876in}{1.483950in}}%
\pgfpathlineto{\pgfqpoint{4.145193in}{0.732351in}}%
\pgfpathlineto{\pgfqpoint{4.145828in}{1.688273in}}%
\pgfpathlineto{\pgfqpoint{4.146146in}{0.830405in}}%
\pgfpathlineto{\pgfqpoint{4.146781in}{1.777498in}}%
\pgfpathlineto{\pgfqpoint{4.147098in}{1.028438in}}%
\pgfpathlineto{\pgfqpoint{4.147416in}{0.889457in}}%
\pgfpathlineto{\pgfqpoint{4.147733in}{1.727719in}}%
\pgfpathlineto{\pgfqpoint{4.148368in}{0.747056in}}%
\pgfpathlineto{\pgfqpoint{4.148686in}{1.562807in}}%
\pgfpathlineto{\pgfqpoint{4.149003in}{1.528066in}}%
\pgfpathlineto{\pgfqpoint{4.149320in}{0.736581in}}%
\pgfpathlineto{\pgfqpoint{4.149955in}{1.708082in}}%
\pgfpathlineto{\pgfqpoint{4.150273in}{0.855556in}}%
\pgfpathlineto{\pgfqpoint{4.150908in}{1.771404in}}%
\pgfpathlineto{\pgfqpoint{4.151225in}{1.073301in}}%
\pgfpathlineto{\pgfqpoint{4.151543in}{0.848224in}}%
\pgfpathlineto{\pgfqpoint{4.151860in}{1.702879in}}%
\pgfpathlineto{\pgfqpoint{4.152495in}{0.736379in}}%
\pgfpathlineto{\pgfqpoint{4.152812in}{1.520370in}}%
\pgfpathlineto{\pgfqpoint{4.153130in}{1.576151in}}%
\pgfpathlineto{\pgfqpoint{4.153447in}{0.755151in}}%
\pgfpathlineto{\pgfqpoint{4.154082in}{1.739750in}}%
\pgfpathlineto{\pgfqpoint{4.154400in}{0.899789in}}%
\pgfpathlineto{\pgfqpoint{4.154717in}{1.018978in}}%
\pgfpathlineto{\pgfqpoint{4.155035in}{1.767892in}}%
\pgfpathlineto{\pgfqpoint{4.155670in}{0.810511in}}%
\pgfpathlineto{\pgfqpoint{4.155987in}{1.667862in}}%
\pgfpathlineto{\pgfqpoint{4.156622in}{0.723220in}}%
\pgfpathlineto{\pgfqpoint{4.156939in}{1.466664in}}%
\pgfpathlineto{\pgfqpoint{4.157257in}{1.611865in}}%
\pgfpathlineto{\pgfqpoint{4.157574in}{0.769103in}}%
\pgfpathlineto{\pgfqpoint{4.158209in}{1.747437in}}%
\pgfpathlineto{\pgfqpoint{4.158527in}{0.933243in}}%
\pgfpathlineto{\pgfqpoint{4.158844in}{0.962034in}}%
\pgfpathlineto{\pgfqpoint{4.159162in}{1.755197in}}%
\pgfpathlineto{\pgfqpoint{4.159797in}{0.783565in}}%
\pgfpathlineto{\pgfqpoint{4.160114in}{1.635409in}}%
\pgfpathlineto{\pgfqpoint{4.160749in}{0.724103in}}%
\pgfpathlineto{\pgfqpoint{4.161066in}{1.425700in}}%
\pgfpathlineto{\pgfqpoint{4.161384in}{1.656219in}}%
\pgfpathlineto{\pgfqpoint{4.161701in}{0.800783in}}%
\pgfpathlineto{\pgfqpoint{4.162336in}{1.764126in}}%
\pgfpathlineto{\pgfqpoint{4.162654in}{0.980586in}}%
\pgfpathlineto{\pgfqpoint{4.162971in}{0.916449in}}%
\pgfpathlineto{\pgfqpoint{4.163289in}{1.736274in}}%
\pgfpathlineto{\pgfqpoint{4.163923in}{0.754333in}}%
\pgfpathlineto{\pgfqpoint{4.164241in}{1.591769in}}%
\pgfpathlineto{\pgfqpoint{4.164558in}{1.482195in}}%
\pgfpathlineto{\pgfqpoint{4.164876in}{0.723605in}}%
\pgfpathlineto{\pgfqpoint{4.165511in}{1.678972in}}%
\pgfpathlineto{\pgfqpoint{4.165828in}{0.821917in}}%
\pgfpathlineto{\pgfqpoint{4.166463in}{1.764005in}}%
\pgfpathlineto{\pgfqpoint{4.166781in}{1.023955in}}%
\pgfpathlineto{\pgfqpoint{4.167098in}{0.873548in}}%
\pgfpathlineto{\pgfqpoint{4.167416in}{1.716695in}}%
\pgfpathlineto{\pgfqpoint{4.168050in}{0.739738in}}%
\pgfpathlineto{\pgfqpoint{4.168368in}{1.554933in}}%
\pgfpathlineto{\pgfqpoint{4.168685in}{1.534131in}}%
\pgfpathlineto{\pgfqpoint{4.169003in}{0.738683in}}%
\pgfpathlineto{\pgfqpoint{4.169638in}{1.713381in}}%
\pgfpathlineto{\pgfqpoint{4.169955in}{0.861347in}}%
\pgfpathlineto{\pgfqpoint{4.170590in}{1.764018in}}%
\pgfpathlineto{\pgfqpoint{4.170908in}{1.072485in}}%
\pgfpathlineto{\pgfqpoint{4.171225in}{0.831490in}}%
\pgfpathlineto{\pgfqpoint{4.171542in}{1.685894in}}%
\pgfpathlineto{\pgfqpoint{4.172177in}{0.722296in}}%
\pgfpathlineto{\pgfqpoint{4.172495in}{1.501483in}}%
\pgfpathlineto{\pgfqpoint{4.172812in}{1.571354in}}%
\pgfpathlineto{\pgfqpoint{4.173130in}{0.746367in}}%
\pgfpathlineto{\pgfqpoint{4.173765in}{1.726111in}}%
\pgfpathlineto{\pgfqpoint{4.174082in}{0.892256in}}%
\pgfpathlineto{\pgfqpoint{4.174400in}{0.993917in}}%
\pgfpathlineto{\pgfqpoint{4.174717in}{1.757536in}}%
\pgfpathlineto{\pgfqpoint{4.175352in}{0.801709in}}%
\pgfpathlineto{\pgfqpoint{4.175669in}{1.658014in}}%
\pgfpathlineto{\pgfqpoint{4.176304in}{0.720276in}}%
\pgfpathlineto{\pgfqpoint{4.176622in}{1.464291in}}%
\pgfpathlineto{\pgfqpoint{4.176939in}{1.618972in}}%
\pgfpathlineto{\pgfqpoint{4.177257in}{0.773042in}}%
\pgfpathlineto{\pgfqpoint{4.177892in}{1.745760in}}%
\pgfpathlineto{\pgfqpoint{4.178209in}{0.935542in}}%
\pgfpathlineto{\pgfqpoint{4.178527in}{0.944459in}}%
\pgfpathlineto{\pgfqpoint{4.178844in}{1.742850in}}%
\pgfpathlineto{\pgfqpoint{4.179479in}{0.768148in}}%
\pgfpathlineto{\pgfqpoint{4.179796in}{1.618237in}}%
\pgfpathlineto{\pgfqpoint{4.180431in}{0.713490in}}%
\pgfpathlineto{\pgfqpoint{4.180749in}{1.399606in}}%
\pgfpathlineto{\pgfqpoint{4.181066in}{1.645180in}}%
\pgfpathlineto{\pgfqpoint{4.181384in}{0.789623in}}%
\pgfpathlineto{\pgfqpoint{4.182019in}{1.751614in}}%
\pgfpathlineto{\pgfqpoint{4.182336in}{0.977247in}}%
\pgfpathlineto{\pgfqpoint{4.182653in}{0.900248in}}%
\pgfpathlineto{\pgfqpoint{4.182971in}{1.728636in}}%
\pgfpathlineto{\pgfqpoint{4.183606in}{0.749365in}}%
\pgfpathlineto{\pgfqpoint{4.183923in}{1.587196in}}%
\pgfpathlineto{\pgfqpoint{4.184241in}{1.489908in}}%
\pgfpathlineto{\pgfqpoint{4.184558in}{0.725221in}}%
\pgfpathlineto{\pgfqpoint{4.185193in}{1.682332in}}%
\pgfpathlineto{\pgfqpoint{4.185511in}{0.823682in}}%
\pgfpathlineto{\pgfqpoint{4.186146in}{1.755642in}}%
\pgfpathlineto{\pgfqpoint{4.186463in}{1.023128in}}%
\pgfpathlineto{\pgfqpoint{4.186780in}{0.854345in}}%
\pgfpathlineto{\pgfqpoint{4.187098in}{1.702366in}}%
\pgfpathlineto{\pgfqpoint{4.187733in}{0.727235in}}%
\pgfpathlineto{\pgfqpoint{4.188050in}{1.534642in}}%
\pgfpathlineto{\pgfqpoint{4.188368in}{1.528320in}}%
\pgfpathlineto{\pgfqpoint{4.188685in}{0.726611in}}%
\pgfpathlineto{\pgfqpoint{4.189320in}{1.700520in}}%
\pgfpathlineto{\pgfqpoint{4.189638in}{0.851538in}}%
\pgfpathlineto{\pgfqpoint{4.190272in}{1.754819in}}%
\pgfpathlineto{\pgfqpoint{4.190590in}{1.073319in}}%
\pgfpathlineto{\pgfqpoint{4.190907in}{0.821597in}}%
\pgfpathlineto{\pgfqpoint{4.191225in}{1.680402in}}%
\pgfpathlineto{\pgfqpoint{4.191860in}{0.721798in}}%
\pgfpathlineto{\pgfqpoint{4.192177in}{1.500304in}}%
\pgfpathlineto{\pgfqpoint{4.192495in}{1.578294in}}%
\pgfpathlineto{\pgfqpoint{4.192812in}{0.748718in}}%
\pgfpathlineto{\pgfqpoint{4.193447in}{1.723686in}}%
\pgfpathlineto{\pgfqpoint{4.193765in}{0.890550in}}%
\pgfpathlineto{\pgfqpoint{4.194082in}{0.978857in}}%
\pgfpathlineto{\pgfqpoint{4.194399in}{1.744794in}}%
\pgfpathlineto{\pgfqpoint{4.195034in}{0.783973in}}%
\pgfpathlineto{\pgfqpoint{4.195352in}{1.643301in}}%
\pgfpathlineto{\pgfqpoint{4.195987in}{0.708428in}}%
\pgfpathlineto{\pgfqpoint{4.196304in}{1.437791in}}%
\pgfpathlineto{\pgfqpoint{4.196622in}{1.607895in}}%
\pgfpathlineto{\pgfqpoint{4.196939in}{0.760644in}}%
\pgfpathlineto{\pgfqpoint{4.197574in}{1.735461in}}%
\pgfpathlineto{\pgfqpoint{4.197891in}{0.929685in}}%
\pgfpathlineto{\pgfqpoint{4.198209in}{0.933027in}}%
\pgfpathlineto{\pgfqpoint{4.198526in}{1.735672in}}%
\pgfpathlineto{\pgfqpoint{4.199161in}{0.762569in}}%
\pgfpathlineto{\pgfqpoint{4.199479in}{1.617512in}}%
\pgfpathlineto{\pgfqpoint{4.200114in}{0.715700in}}%
\pgfpathlineto{\pgfqpoint{4.200431in}{1.397389in}}%
\pgfpathlineto{\pgfqpoint{4.200749in}{1.647655in}}%
\pgfpathlineto{\pgfqpoint{4.201066in}{0.789850in}}%
\pgfpathlineto{\pgfqpoint{4.201701in}{1.743712in}}%
\pgfpathlineto{\pgfqpoint{4.202018in}{0.972372in}}%
\pgfpathlineto{\pgfqpoint{4.202336in}{0.883788in}}%
\pgfpathlineto{\pgfqpoint{4.202653in}{1.714276in}}%
\pgfpathlineto{\pgfqpoint{4.203288in}{0.734644in}}%
\pgfpathlineto{\pgfqpoint{4.203606in}{1.566911in}}%
\pgfpathlineto{\pgfqpoint{4.203923in}{1.480165in}}%
\pgfpathlineto{\pgfqpoint{4.204241in}{0.711185in}}%
\pgfpathlineto{\pgfqpoint{4.204876in}{1.670611in}}%
\pgfpathlineto{\pgfqpoint{4.205193in}{0.814241in}}%
\pgfpathlineto{\pgfqpoint{4.205828in}{1.748788in}}%
\pgfpathlineto{\pgfqpoint{4.206145in}{1.020637in}}%
\pgfpathlineto{\pgfqpoint{4.206463in}{0.848428in}}%
\pgfpathlineto{\pgfqpoint{4.206780in}{1.699402in}}%
\pgfpathlineto{\pgfqpoint{4.207415in}{0.726646in}}%
\pgfpathlineto{\pgfqpoint{4.207733in}{1.535346in}}%
\pgfpathlineto{\pgfqpoint{4.208050in}{1.532137in}}%
\pgfpathlineto{\pgfqpoint{4.208368in}{0.727748in}}%
\pgfpathlineto{\pgfqpoint{4.209002in}{1.696834in}}%
\pgfpathlineto{\pgfqpoint{4.209320in}{0.848677in}}%
\pgfpathlineto{\pgfqpoint{4.209955in}{1.743636in}}%
\pgfpathlineto{\pgfqpoint{4.210272in}{1.066787in}}%
\pgfpathlineto{\pgfqpoint{4.210590in}{0.806707in}}%
\pgfpathlineto{\pgfqpoint{4.210907in}{1.664489in}}%
\pgfpathlineto{\pgfqpoint{4.211542in}{0.706372in}}%
\pgfpathlineto{\pgfqpoint{4.211860in}{1.475535in}}%
\pgfpathlineto{\pgfqpoint{4.212177in}{1.565442in}}%
\pgfpathlineto{\pgfqpoint{4.212495in}{0.734959in}}%
\pgfpathlineto{\pgfqpoint{4.213129in}{1.714308in}}%
\pgfpathlineto{\pgfqpoint{4.213447in}{0.885354in}}%
\pgfpathlineto{\pgfqpoint{4.213764in}{0.967090in}}%
\pgfpathlineto{\pgfqpoint{4.214082in}{1.741220in}}%
\pgfpathlineto{\pgfqpoint{4.214717in}{0.783554in}}%
\pgfpathlineto{\pgfqpoint{4.215034in}{1.643605in}}%
\pgfpathlineto{\pgfqpoint{4.215669in}{0.709166in}}%
\pgfpathlineto{\pgfqpoint{4.215987in}{1.435731in}}%
\pgfpathlineto{\pgfqpoint{4.216304in}{1.607743in}}%
\pgfpathlineto{\pgfqpoint{4.216621in}{0.758538in}}%
\pgfpathlineto{\pgfqpoint{4.217256in}{1.726717in}}%
\pgfpathlineto{\pgfqpoint{4.217574in}{0.924663in}}%
\pgfpathlineto{\pgfqpoint{4.217891in}{0.914856in}}%
\pgfpathlineto{\pgfqpoint{4.218209in}{1.723609in}}%
\pgfpathlineto{\pgfqpoint{4.218844in}{0.748785in}}%
\pgfpathlineto{\pgfqpoint{4.219161in}{1.596116in}}%
\pgfpathlineto{\pgfqpoint{4.219796in}{0.699061in}}%
\pgfpathlineto{\pgfqpoint{4.220113in}{1.373096in}}%
\pgfpathlineto{\pgfqpoint{4.220431in}{1.635683in}}%
\pgfpathlineto{\pgfqpoint{4.220748in}{0.779056in}}%
\pgfpathlineto{\pgfqpoint{4.221383in}{1.737215in}}%
\pgfpathlineto{\pgfqpoint{4.221701in}{0.970588in}}%
\pgfpathlineto{\pgfqpoint{4.222018in}{0.878858in}}%
\pgfpathlineto{\pgfqpoint{4.222336in}{1.715116in}}%
\pgfpathlineto{\pgfqpoint{4.222971in}{0.737033in}}%
\pgfpathlineto{\pgfqpoint{4.223288in}{1.567299in}}%
\pgfpathlineto{\pgfqpoint{4.223606in}{1.483897in}}%
\pgfpathlineto{\pgfqpoint{4.223923in}{0.710392in}}%
\pgfpathlineto{\pgfqpoint{4.224558in}{1.665729in}}%
\pgfpathlineto{\pgfqpoint{4.224875in}{0.808682in}}%
\pgfpathlineto{\pgfqpoint{4.225510in}{1.737082in}}%
\pgfpathlineto{\pgfqpoint{4.225828in}{1.014538in}}%
\pgfpathlineto{\pgfqpoint{4.226145in}{0.831335in}}%
\pgfpathlineto{\pgfqpoint{4.226463in}{1.683428in}}%
\pgfpathlineto{\pgfqpoint{4.227098in}{0.710648in}}%
\pgfpathlineto{\pgfqpoint{4.227415in}{1.511513in}}%
\pgfpathlineto{\pgfqpoint{4.227732in}{1.520265in}}%
\pgfpathlineto{\pgfqpoint{4.228050in}{0.712926in}}%
\pgfpathlineto{\pgfqpoint{4.228685in}{1.688365in}}%
\pgfpathlineto{\pgfqpoint{4.229002in}{0.841740in}}%
\pgfpathlineto{\pgfqpoint{4.229637in}{1.741464in}}%
\pgfpathlineto{\pgfqpoint{4.229955in}{1.067855in}}%
\pgfpathlineto{\pgfqpoint{4.230272in}{0.806831in}}%
\pgfpathlineto{\pgfqpoint{4.230590in}{1.667210in}}%
\pgfpathlineto{\pgfqpoint{4.231225in}{0.708352in}}%
\pgfpathlineto{\pgfqpoint{4.231542in}{1.473120in}}%
\pgfpathlineto{\pgfqpoint{4.231859in}{1.564825in}}%
\pgfpathlineto{\pgfqpoint{4.232177in}{0.731137in}}%
\pgfpathlineto{\pgfqpoint{4.232812in}{1.705383in}}%
\pgfpathlineto{\pgfqpoint{4.233129in}{0.876997in}}%
\pgfpathlineto{\pgfqpoint{4.233447in}{0.952109in}}%
\pgfpathlineto{\pgfqpoint{4.233764in}{1.727427in}}%
\pgfpathlineto{\pgfqpoint{4.234399in}{0.765446in}}%
\pgfpathlineto{\pgfqpoint{4.234717in}{1.623788in}}%
\pgfpathlineto{\pgfqpoint{4.235351in}{0.692867in}}%
\pgfpathlineto{\pgfqpoint{4.235669in}{1.414317in}}%
\pgfpathlineto{\pgfqpoint{4.235986in}{1.596922in}}%
\pgfpathlineto{\pgfqpoint{4.236304in}{0.747623in}}%
\pgfpathlineto{\pgfqpoint{4.236939in}{1.721492in}}%
\pgfpathlineto{\pgfqpoint{4.237256in}{0.920071in}}%
\pgfpathlineto{\pgfqpoint{4.237574in}{0.915491in}}%
\pgfpathlineto{\pgfqpoint{4.237891in}{1.725425in}}%
\pgfpathlineto{\pgfqpoint{4.238526in}{0.750326in}}%
\pgfpathlineto{\pgfqpoint{4.238843in}{1.597622in}}%
\pgfpathlineto{\pgfqpoint{4.239478in}{0.698165in}}%
\pgfpathlineto{\pgfqpoint{4.239796in}{1.366571in}}%
\pgfpathlineto{\pgfqpoint{4.240113in}{1.630151in}}%
\pgfpathlineto{\pgfqpoint{4.240431in}{0.772217in}}%
\pgfpathlineto{\pgfqpoint{4.241066in}{1.726356in}}%
\pgfpathlineto{\pgfqpoint{4.241383in}{0.960928in}}%
\pgfpathlineto{\pgfqpoint{4.241701in}{0.862951in}}%
\pgfpathlineto{\pgfqpoint{4.242018in}{1.697302in}}%
\pgfpathlineto{\pgfqpoint{4.242653in}{0.718042in}}%
\pgfpathlineto{\pgfqpoint{4.242970in}{1.546522in}}%
\pgfpathlineto{\pgfqpoint{4.243288in}{1.469585in}}%
\pgfpathlineto{\pgfqpoint{4.243605in}{0.695963in}}%
\pgfpathlineto{\pgfqpoint{4.244240in}{1.657606in}}%
\pgfpathlineto{\pgfqpoint{4.244558in}{0.801780in}}%
\pgfpathlineto{\pgfqpoint{4.245193in}{1.738040in}}%
\pgfpathlineto{\pgfqpoint{4.245510in}{1.013198in}}%
\pgfpathlineto{\pgfqpoint{4.245828in}{0.836503in}}%
\pgfpathlineto{\pgfqpoint{4.246145in}{1.685148in}}%
\pgfpathlineto{\pgfqpoint{4.246780in}{0.710689in}}%
\pgfpathlineto{\pgfqpoint{4.247097in}{1.510011in}}%
\pgfpathlineto{\pgfqpoint{4.247415in}{1.516461in}}%
\pgfpathlineto{\pgfqpoint{4.247732in}{0.708446in}}%
\pgfpathlineto{\pgfqpoint{4.248367in}{1.679008in}}%
\pgfpathlineto{\pgfqpoint{4.248685in}{0.832911in}}%
\pgfpathlineto{\pgfqpoint{4.249002in}{0.990925in}}%
\pgfpathlineto{\pgfqpoint{4.249320in}{1.727495in}}%
\pgfpathlineto{\pgfqpoint{4.249955in}{0.789216in}}%
\pgfpathlineto{\pgfqpoint{4.250272in}{1.646914in}}%
\pgfpathlineto{\pgfqpoint{4.250907in}{0.690133in}}%
\pgfpathlineto{\pgfqpoint{4.251224in}{1.455112in}}%
\pgfpathlineto{\pgfqpoint{4.251542in}{1.552494in}}%
\pgfpathlineto{\pgfqpoint{4.251859in}{0.720155in}}%
\pgfpathlineto{\pgfqpoint{4.252494in}{1.700061in}}%
\pgfpathlineto{\pgfqpoint{4.252812in}{0.872856in}}%
\pgfpathlineto{\pgfqpoint{4.253129in}{0.952873in}}%
\pgfpathlineto{\pgfqpoint{4.253447in}{1.731714in}}%
\pgfpathlineto{\pgfqpoint{4.254081in}{0.770141in}}%
\pgfpathlineto{\pgfqpoint{4.254399in}{1.623994in}}%
\pgfpathlineto{\pgfqpoint{4.255034in}{0.689959in}}%
\pgfpathlineto{\pgfqpoint{4.255351in}{1.408875in}}%
\pgfpathlineto{\pgfqpoint{4.255669in}{1.589358in}}%
\pgfpathlineto{\pgfqpoint{4.255986in}{0.739548in}}%
\pgfpathlineto{\pgfqpoint{4.256621in}{1.709830in}}%
\pgfpathlineto{\pgfqpoint{4.256939in}{0.910284in}}%
\pgfpathlineto{\pgfqpoint{4.257256in}{0.896501in}}%
\pgfpathlineto{\pgfqpoint{4.257574in}{1.707923in}}%
\pgfpathlineto{\pgfqpoint{4.258208in}{0.732706in}}%
\pgfpathlineto{\pgfqpoint{4.258526in}{1.578133in}}%
\pgfpathlineto{\pgfqpoint{4.258843in}{1.417066in}}%
\pgfpathlineto{\pgfqpoint{4.259161in}{0.682864in}}%
\pgfpathlineto{\pgfqpoint{4.259796in}{1.621158in}}%
\pgfpathlineto{\pgfqpoint{4.260113in}{0.764429in}}%
\pgfpathlineto{\pgfqpoint{4.260748in}{1.726894in}}%
\pgfpathlineto{\pgfqpoint{4.261066in}{0.960321in}}%
\pgfpathlineto{\pgfqpoint{4.261383in}{0.867401in}}%
\pgfpathlineto{\pgfqpoint{4.261700in}{1.701368in}}%
\pgfpathlineto{\pgfqpoint{4.262335in}{0.720050in}}%
\pgfpathlineto{\pgfqpoint{4.262653in}{1.544054in}}%
\pgfpathlineto{\pgfqpoint{4.262970in}{1.465641in}}%
\pgfpathlineto{\pgfqpoint{4.263288in}{0.689672in}}%
\pgfpathlineto{\pgfqpoint{4.263923in}{1.647076in}}%
\pgfpathlineto{\pgfqpoint{4.264240in}{0.790867in}}%
\pgfpathlineto{\pgfqpoint{4.264875in}{1.721354in}}%
\pgfpathlineto{\pgfqpoint{4.265192in}{1.000430in}}%
\pgfpathlineto{\pgfqpoint{4.265510in}{0.815733in}}%
\pgfpathlineto{\pgfqpoint{4.265827in}{1.667635in}}%
\pgfpathlineto{\pgfqpoint{4.266462in}{0.694650in}}%
\pgfpathlineto{\pgfqpoint{4.266780in}{1.493844in}}%
\pgfpathlineto{\pgfqpoint{4.267097in}{1.504958in}}%
\pgfpathlineto{\pgfqpoint{4.267415in}{0.696729in}}%
\pgfpathlineto{\pgfqpoint{4.268050in}{1.673338in}}%
\pgfpathlineto{\pgfqpoint{4.268367in}{0.826969in}}%
\pgfpathlineto{\pgfqpoint{4.269002in}{1.731459in}}%
\pgfpathlineto{\pgfqpoint{4.269319in}{1.056393in}}%
\pgfpathlineto{\pgfqpoint{4.269637in}{0.792666in}}%
\pgfpathlineto{\pgfqpoint{4.269954in}{1.648286in}}%
\pgfpathlineto{\pgfqpoint{4.270589in}{0.688496in}}%
\pgfpathlineto{\pgfqpoint{4.270907in}{1.449718in}}%
\pgfpathlineto{\pgfqpoint{4.271224in}{1.544778in}}%
\pgfpathlineto{\pgfqpoint{4.271542in}{0.710819in}}%
\pgfpathlineto{\pgfqpoint{4.272177in}{1.687992in}}%
\pgfpathlineto{\pgfqpoint{4.272494in}{0.860187in}}%
\pgfpathlineto{\pgfqpoint{4.272811in}{0.936384in}}%
\pgfpathlineto{\pgfqpoint{4.273129in}{1.712380in}}%
\pgfpathlineto{\pgfqpoint{4.273764in}{0.750533in}}%
\pgfpathlineto{\pgfqpoint{4.274081in}{1.607909in}}%
\pgfpathlineto{\pgfqpoint{4.274716in}{0.676473in}}%
\pgfpathlineto{\pgfqpoint{4.275034in}{1.398008in}}%
\pgfpathlineto{\pgfqpoint{4.275351in}{1.580427in}}%
\pgfpathlineto{\pgfqpoint{4.275669in}{0.731161in}}%
\pgfpathlineto{\pgfqpoint{4.276304in}{1.710588in}}%
\pgfpathlineto{\pgfqpoint{4.276621in}{0.907033in}}%
\pgfpathlineto{\pgfqpoint{4.276938in}{0.904723in}}%
\pgfpathlineto{\pgfqpoint{4.277256in}{1.711139in}}%
\pgfpathlineto{\pgfqpoint{4.277891in}{0.732666in}}%
\pgfpathlineto{\pgfqpoint{4.278208in}{1.576693in}}%
\pgfpathlineto{\pgfqpoint{4.278843in}{0.677364in}}%
\pgfpathlineto{\pgfqpoint{4.279161in}{1.345914in}}%
\pgfpathlineto{\pgfqpoint{4.279478in}{1.610415in}}%
\pgfpathlineto{\pgfqpoint{4.279796in}{0.752856in}}%
\pgfpathlineto{\pgfqpoint{4.280430in}{1.710277in}}%
\pgfpathlineto{\pgfqpoint{4.280748in}{0.943844in}}%
\pgfpathlineto{\pgfqpoint{4.281065in}{0.849366in}}%
\pgfpathlineto{\pgfqpoint{4.281383in}{1.682524in}}%
\pgfpathlineto{\pgfqpoint{4.282018in}{0.702756in}}%
\pgfpathlineto{\pgfqpoint{4.282335in}{1.531412in}}%
\pgfpathlineto{\pgfqpoint{4.282653in}{1.451487in}}%
\pgfpathlineto{\pgfqpoint{4.282970in}{0.679159in}}%
\pgfpathlineto{\pgfqpoint{4.283605in}{1.640688in}}%
\pgfpathlineto{\pgfqpoint{4.283922in}{0.784563in}}%
\pgfpathlineto{\pgfqpoint{4.284557in}{1.725996in}}%
\pgfpathlineto{\pgfqpoint{4.284875in}{0.997372in}}%
\pgfpathlineto{\pgfqpoint{4.285192in}{0.822646in}}%
\pgfpathlineto{\pgfqpoint{4.285510in}{1.668140in}}%
\pgfpathlineto{\pgfqpoint{4.286145in}{0.691053in}}%
\pgfpathlineto{\pgfqpoint{4.286462in}{1.490040in}}%
\pgfpathlineto{\pgfqpoint{4.286780in}{1.494293in}}%
\pgfpathlineto{\pgfqpoint{4.287097in}{0.687689in}}%
\pgfpathlineto{\pgfqpoint{4.287732in}{1.660420in}}%
\pgfpathlineto{\pgfqpoint{4.288049in}{0.813790in}}%
\pgfpathlineto{\pgfqpoint{4.288684in}{1.712383in}}%
\pgfpathlineto{\pgfqpoint{4.289002in}{1.037452in}}%
\pgfpathlineto{\pgfqpoint{4.289319in}{0.776147in}}%
\pgfpathlineto{\pgfqpoint{4.289637in}{1.632679in}}%
\pgfpathlineto{\pgfqpoint{4.290272in}{0.674115in}}%
\pgfpathlineto{\pgfqpoint{4.290589in}{1.441658in}}%
\pgfpathlineto{\pgfqpoint{4.290907in}{1.533434in}}%
\pgfpathlineto{\pgfqpoint{4.291224in}{0.702602in}}%
\pgfpathlineto{\pgfqpoint{4.291859in}{1.687579in}}%
\pgfpathlineto{\pgfqpoint{4.292176in}{0.857047in}}%
\pgfpathlineto{\pgfqpoint{4.292494in}{0.944358in}}%
\pgfpathlineto{\pgfqpoint{4.292811in}{1.716928in}}%
\pgfpathlineto{\pgfqpoint{4.293446in}{0.753565in}}%
\pgfpathlineto{\pgfqpoint{4.293764in}{1.605457in}}%
\pgfpathlineto{\pgfqpoint{4.294399in}{0.669238in}}%
\pgfpathlineto{\pgfqpoint{4.294716in}{1.391720in}}%
\pgfpathlineto{\pgfqpoint{4.295034in}{1.567474in}}%
\pgfpathlineto{\pgfqpoint{4.295351in}{0.719020in}}%
\pgfpathlineto{\pgfqpoint{4.295986in}{1.692744in}}%
\pgfpathlineto{\pgfqpoint{4.296303in}{0.890434in}}%
\pgfpathlineto{\pgfqpoint{4.296621in}{0.885521in}}%
\pgfpathlineto{\pgfqpoint{4.296938in}{1.693659in}}%
\pgfpathlineto{\pgfqpoint{4.297573in}{0.718969in}}%
\pgfpathlineto{\pgfqpoint{4.297891in}{1.565055in}}%
\pgfpathlineto{\pgfqpoint{4.298526in}{0.665924in}}%
\pgfpathlineto{\pgfqpoint{4.298843in}{1.342493in}}%
\pgfpathlineto{\pgfqpoint{4.299160in}{1.601825in}}%
\pgfpathlineto{\pgfqpoint{4.299478in}{0.745554in}}%
\pgfpathlineto{\pgfqpoint{4.300113in}{1.713760in}}%
\pgfpathlineto{\pgfqpoint{4.300430in}{0.941354in}}%
\pgfpathlineto{\pgfqpoint{4.300748in}{0.855666in}}%
\pgfpathlineto{\pgfqpoint{4.301065in}{1.684529in}}%
\pgfpathlineto{\pgfqpoint{4.301700in}{0.701451in}}%
\pgfpathlineto{\pgfqpoint{4.302018in}{1.526972in}}%
\pgfpathlineto{\pgfqpoint{4.302335in}{1.440930in}}%
\pgfpathlineto{\pgfqpoint{4.302652in}{0.668837in}}%
\pgfpathlineto{\pgfqpoint{4.303287in}{1.626559in}}%
\pgfpathlineto{\pgfqpoint{4.303605in}{0.770097in}}%
\pgfpathlineto{\pgfqpoint{4.304240in}{1.705608in}}%
\pgfpathlineto{\pgfqpoint{4.304557in}{0.978780in}}%
\pgfpathlineto{\pgfqpoint{4.304875in}{0.804943in}}%
\pgfpathlineto{\pgfqpoint{4.305192in}{1.654429in}}%
\pgfpathlineto{\pgfqpoint{4.305827in}{0.679786in}}%
\pgfpathlineto{\pgfqpoint{4.306145in}{1.482757in}}%
\pgfpathlineto{\pgfqpoint{4.306462in}{1.482998in}}%
\pgfpathlineto{\pgfqpoint{4.306779in}{0.678523in}}%
\pgfpathlineto{\pgfqpoint{4.307414in}{1.657781in}}%
\pgfpathlineto{\pgfqpoint{4.307732in}{0.808398in}}%
\pgfpathlineto{\pgfqpoint{4.308367in}{1.715946in}}%
\pgfpathlineto{\pgfqpoint{4.308684in}{1.032869in}}%
\pgfpathlineto{\pgfqpoint{4.309002in}{0.778487in}}%
\pgfpathlineto{\pgfqpoint{4.309319in}{1.632186in}}%
\pgfpathlineto{\pgfqpoint{4.309954in}{0.669305in}}%
\pgfpathlineto{\pgfqpoint{4.310271in}{1.436115in}}%
\pgfpathlineto{\pgfqpoint{4.310589in}{1.520587in}}%
\pgfpathlineto{\pgfqpoint{4.310906in}{0.689755in}}%
\pgfpathlineto{\pgfqpoint{4.311541in}{1.669067in}}%
\pgfpathlineto{\pgfqpoint{4.311859in}{0.838095in}}%
\pgfpathlineto{\pgfqpoint{4.312176in}{0.928021in}}%
\pgfpathlineto{\pgfqpoint{4.312494in}{1.697916in}}%
\pgfpathlineto{\pgfqpoint{4.313129in}{0.738870in}}%
\pgfpathlineto{\pgfqpoint{4.313446in}{1.596436in}}%
\pgfpathlineto{\pgfqpoint{4.314081in}{0.660254in}}%
\pgfpathlineto{\pgfqpoint{4.314398in}{1.389289in}}%
\pgfpathlineto{\pgfqpoint{4.314716in}{1.558179in}}%
\pgfpathlineto{\pgfqpoint{4.315033in}{0.710906in}}%
\pgfpathlineto{\pgfqpoint{4.315668in}{1.695250in}}%
\pgfpathlineto{\pgfqpoint{4.315986in}{0.885329in}}%
\pgfpathlineto{\pgfqpoint{4.316303in}{0.895790in}}%
\pgfpathlineto{\pgfqpoint{4.316621in}{1.694877in}}%
\pgfpathlineto{\pgfqpoint{4.317256in}{0.716660in}}%
\pgfpathlineto{\pgfqpoint{4.317573in}{1.562564in}}%
\pgfpathlineto{\pgfqpoint{4.318208in}{0.657393in}}%
\pgfpathlineto{\pgfqpoint{4.318525in}{1.336034in}}%
\pgfpathlineto{\pgfqpoint{4.318843in}{1.587458in}}%
\pgfpathlineto{\pgfqpoint{4.319160in}{0.730756in}}%
\pgfpathlineto{\pgfqpoint{4.319795in}{1.693077in}}%
\pgfpathlineto{\pgfqpoint{4.320113in}{0.919594in}}%
\pgfpathlineto{\pgfqpoint{4.320430in}{0.841222in}}%
\pgfpathlineto{\pgfqpoint{4.320748in}{1.669732in}}%
\pgfpathlineto{\pgfqpoint{4.321382in}{0.689652in}}%
\pgfpathlineto{\pgfqpoint{4.321700in}{1.522361in}}%
\pgfpathlineto{\pgfqpoint{4.322017in}{1.426188in}}%
\pgfpathlineto{\pgfqpoint{4.322335in}{0.661197in}}%
\pgfpathlineto{\pgfqpoint{4.322970in}{1.622246in}}%
\pgfpathlineto{\pgfqpoint{4.323287in}{0.764159in}}%
\pgfpathlineto{\pgfqpoint{4.323922in}{1.709482in}}%
\pgfpathlineto{\pgfqpoint{4.324240in}{0.971260in}}%
\pgfpathlineto{\pgfqpoint{4.324557in}{0.811376in}}%
\pgfpathlineto{\pgfqpoint{4.324875in}{1.652846in}}%
\pgfpathlineto{\pgfqpoint{4.325509in}{0.673433in}}%
\pgfpathlineto{\pgfqpoint{4.325827in}{1.479148in}}%
\pgfpathlineto{\pgfqpoint{4.326144in}{1.467017in}}%
\pgfpathlineto{\pgfqpoint{4.326462in}{0.666752in}}%
\pgfpathlineto{\pgfqpoint{4.327097in}{1.639192in}}%
\pgfpathlineto{\pgfqpoint{4.327414in}{0.789810in}}%
\pgfpathlineto{\pgfqpoint{4.328049in}{1.697044in}}%
\pgfpathlineto{\pgfqpoint{4.328367in}{1.010576in}}%
\pgfpathlineto{\pgfqpoint{4.328684in}{0.767123in}}%
\pgfpathlineto{\pgfqpoint{4.329001in}{1.622142in}}%
\pgfpathlineto{\pgfqpoint{4.329636in}{0.659207in}}%
\pgfpathlineto{\pgfqpoint{4.329954in}{1.435383in}}%
\pgfpathlineto{\pgfqpoint{4.330271in}{1.507968in}}%
\pgfpathlineto{\pgfqpoint{4.330589in}{0.682130in}}%
\pgfpathlineto{\pgfqpoint{4.331224in}{1.669689in}}%
\pgfpathlineto{\pgfqpoint{4.331541in}{0.833018in}}%
\pgfpathlineto{\pgfqpoint{4.331859in}{0.938890in}}%
\pgfpathlineto{\pgfqpoint{4.332176in}{1.700652in}}%
\pgfpathlineto{\pgfqpoint{4.332811in}{0.740936in}}%
\pgfpathlineto{\pgfqpoint{4.333128in}{1.593346in}}%
\pgfpathlineto{\pgfqpoint{4.333763in}{0.650793in}}%
\pgfpathlineto{\pgfqpoint{4.334081in}{1.384987in}}%
\pgfpathlineto{\pgfqpoint{4.334398in}{1.541450in}}%
\pgfpathlineto{\pgfqpoint{4.334716in}{0.696496in}}%
\pgfpathlineto{\pgfqpoint{4.335351in}{1.673490in}}%
\pgfpathlineto{\pgfqpoint{4.335668in}{0.863803in}}%
\pgfpathlineto{\pgfqpoint{4.335986in}{0.880530in}}%
\pgfpathlineto{\pgfqpoint{4.336303in}{1.680534in}}%
\pgfpathlineto{\pgfqpoint{4.336938in}{0.708315in}}%
\pgfpathlineto{\pgfqpoint{4.337255in}{1.557382in}}%
\pgfpathlineto{\pgfqpoint{4.337890in}{0.648748in}}%
\pgfpathlineto{\pgfqpoint{4.338208in}{1.338750in}}%
\pgfpathlineto{\pgfqpoint{4.338525in}{1.579044in}}%
\pgfpathlineto{\pgfqpoint{4.338843in}{0.723575in}}%
\pgfpathlineto{\pgfqpoint{4.339478in}{1.695718in}}%
\pgfpathlineto{\pgfqpoint{4.339795in}{0.912906in}}%
\pgfpathlineto{\pgfqpoint{4.340112in}{0.848311in}}%
\pgfpathlineto{\pgfqpoint{4.340430in}{1.670196in}}%
\pgfpathlineto{\pgfqpoint{4.341065in}{0.687259in}}%
\pgfpathlineto{\pgfqpoint{4.341382in}{1.518508in}}%
\pgfpathlineto{\pgfqpoint{4.341700in}{1.410450in}}%
\pgfpathlineto{\pgfqpoint{4.342017in}{0.648685in}}%
\pgfpathlineto{\pgfqpoint{4.342652in}{1.602268in}}%
\pgfpathlineto{\pgfqpoint{4.342970in}{0.745072in}}%
\pgfpathlineto{\pgfqpoint{4.343605in}{1.688700in}}%
\pgfpathlineto{\pgfqpoint{4.343922in}{0.949189in}}%
\pgfpathlineto{\pgfqpoint{4.344239in}{0.799085in}}%
\pgfpathlineto{\pgfqpoint{4.344557in}{1.644152in}}%
\pgfpathlineto{\pgfqpoint{4.345192in}{0.667002in}}%
\pgfpathlineto{\pgfqpoint{4.345509in}{1.478278in}}%
\pgfpathlineto{\pgfqpoint{4.345827in}{1.454054in}}%
\pgfpathlineto{\pgfqpoint{4.346144in}{0.658395in}}%
\pgfpathlineto{\pgfqpoint{4.346779in}{1.637029in}}%
\pgfpathlineto{\pgfqpoint{4.347097in}{0.783227in}}%
\pgfpathlineto{\pgfqpoint{4.347731in}{1.698771in}}%
\pgfpathlineto{\pgfqpoint{4.348049in}{1.000498in}}%
\pgfpathlineto{\pgfqpoint{4.348366in}{0.769216in}}%
\pgfpathlineto{\pgfqpoint{4.348684in}{1.620818in}}%
\pgfpathlineto{\pgfqpoint{4.349319in}{0.652950in}}%
\pgfpathlineto{\pgfqpoint{4.349636in}{1.431592in}}%
\pgfpathlineto{\pgfqpoint{4.349954in}{1.491094in}}%
\pgfpathlineto{\pgfqpoint{4.350271in}{0.667227in}}%
\pgfpathlineto{\pgfqpoint{4.350906in}{1.646995in}}%
\pgfpathlineto{\pgfqpoint{4.351224in}{0.809926in}}%
\pgfpathlineto{\pgfqpoint{4.351541in}{0.926415in}}%
\pgfpathlineto{\pgfqpoint{4.351858in}{1.683805in}}%
\pgfpathlineto{\pgfqpoint{4.352493in}{0.731288in}}%
\pgfpathlineto{\pgfqpoint{4.352811in}{1.589573in}}%
\pgfpathlineto{\pgfqpoint{4.353446in}{0.644803in}}%
\pgfpathlineto{\pgfqpoint{4.353763in}{1.387845in}}%
\pgfpathlineto{\pgfqpoint{4.354081in}{1.530779in}}%
\pgfpathlineto{\pgfqpoint{4.354398in}{0.687942in}}%
\pgfpathlineto{\pgfqpoint{4.355033in}{1.674870in}}%
\pgfpathlineto{\pgfqpoint{4.355350in}{0.855379in}}%
\pgfpathlineto{\pgfqpoint{4.355668in}{0.892562in}}%
\pgfpathlineto{\pgfqpoint{4.355985in}{1.680550in}}%
\pgfpathlineto{\pgfqpoint{4.356620in}{0.706045in}}%
\pgfpathlineto{\pgfqpoint{4.356938in}{1.555690in}}%
\pgfpathlineto{\pgfqpoint{4.357573in}{0.639037in}}%
\pgfpathlineto{\pgfqpoint{4.357890in}{1.335438in}}%
\pgfpathlineto{\pgfqpoint{4.358208in}{1.559716in}}%
\pgfpathlineto{\pgfqpoint{4.358525in}{0.705127in}}%
\pgfpathlineto{\pgfqpoint{4.359160in}{1.673683in}}%
\pgfpathlineto{\pgfqpoint{4.359477in}{0.887953in}}%
\pgfpathlineto{\pgfqpoint{4.359795in}{0.838824in}}%
\pgfpathlineto{\pgfqpoint{4.360112in}{1.658971in}}%
\pgfpathlineto{\pgfqpoint{4.360747in}{0.679648in}}%
\pgfpathlineto{\pgfqpoint{4.361065in}{1.519230in}}%
\pgfpathlineto{\pgfqpoint{4.361382in}{1.393362in}}%
\pgfpathlineto{\pgfqpoint{4.361700in}{0.642379in}}%
\pgfpathlineto{\pgfqpoint{4.362335in}{1.597355in}}%
\pgfpathlineto{\pgfqpoint{4.362652in}{0.737688in}}%
\pgfpathlineto{\pgfqpoint{4.363287in}{1.690841in}}%
\pgfpathlineto{\pgfqpoint{4.363604in}{0.937169in}}%
\pgfpathlineto{\pgfqpoint{4.363922in}{0.806641in}}%
\pgfpathlineto{\pgfqpoint{4.364239in}{1.642323in}}%
\pgfpathlineto{\pgfqpoint{4.364874in}{0.660549in}}%
\pgfpathlineto{\pgfqpoint{4.365192in}{1.476635in}}%
\pgfpathlineto{\pgfqpoint{4.365509in}{1.433691in}}%
\pgfpathlineto{\pgfqpoint{4.365827in}{0.645356in}}%
\pgfpathlineto{\pgfqpoint{4.366461in}{1.613798in}}%
\pgfpathlineto{\pgfqpoint{4.366779in}{0.760458in}}%
\pgfpathlineto{\pgfqpoint{4.367414in}{1.680968in}}%
\pgfpathlineto{\pgfqpoint{4.367731in}{0.975632in}}%
\pgfpathlineto{\pgfqpoint{4.368049in}{0.763072in}}%
\pgfpathlineto{\pgfqpoint{4.368366in}{1.615300in}}%
\pgfpathlineto{\pgfqpoint{4.369001in}{0.646120in}}%
\pgfpathlineto{\pgfqpoint{4.369319in}{1.435909in}}%
\pgfpathlineto{\pgfqpoint{4.369636in}{1.475786in}}%
\pgfpathlineto{\pgfqpoint{4.369954in}{0.659363in}}%
\pgfpathlineto{\pgfqpoint{4.370588in}{1.646482in}}%
\pgfpathlineto{\pgfqpoint{4.370906in}{0.801904in}}%
\pgfpathlineto{\pgfqpoint{4.371223in}{0.939988in}}%
\pgfpathlineto{\pgfqpoint{4.371541in}{1.685009in}}%
\pgfpathlineto{\pgfqpoint{4.372176in}{0.734095in}}%
\pgfpathlineto{\pgfqpoint{4.372493in}{1.586998in}}%
\pgfpathlineto{\pgfqpoint{4.373128in}{0.634766in}}%
\pgfpathlineto{\pgfqpoint{4.373446in}{1.386705in}}%
\pgfpathlineto{\pgfqpoint{4.373763in}{1.509656in}}%
\pgfpathlineto{\pgfqpoint{4.374080in}{0.671355in}}%
\pgfpathlineto{\pgfqpoint{4.374715in}{1.651074in}}%
\pgfpathlineto{\pgfqpoint{4.375033in}{0.830457in}}%
\pgfpathlineto{\pgfqpoint{4.375350in}{0.882100in}}%
\pgfpathlineto{\pgfqpoint{4.375668in}{1.668388in}}%
\pgfpathlineto{\pgfqpoint{4.376303in}{0.701767in}}%
\pgfpathlineto{\pgfqpoint{4.376620in}{1.554788in}}%
\pgfpathlineto{\pgfqpoint{4.377255in}{0.631844in}}%
\pgfpathlineto{\pgfqpoint{4.377572in}{1.342682in}}%
\pgfpathlineto{\pgfqpoint{4.377890in}{1.549318in}}%
\pgfpathlineto{\pgfqpoint{4.378207in}{0.696875in}}%
\pgfpathlineto{\pgfqpoint{4.378842in}{1.674811in}}%
\pgfpathlineto{\pgfqpoint{4.379160in}{0.877242in}}%
\pgfpathlineto{\pgfqpoint{4.379477in}{0.848386in}}%
\pgfpathlineto{\pgfqpoint{4.379795in}{1.659399in}}%
\pgfpathlineto{\pgfqpoint{4.380430in}{0.678413in}}%
\pgfpathlineto{\pgfqpoint{4.380747in}{1.517241in}}%
\pgfpathlineto{\pgfqpoint{4.381065in}{1.373254in}}%
\pgfpathlineto{\pgfqpoint{4.381382in}{0.629131in}}%
\pgfpathlineto{\pgfqpoint{4.382017in}{1.572763in}}%
\pgfpathlineto{\pgfqpoint{4.382334in}{0.715579in}}%
\pgfpathlineto{\pgfqpoint{4.382969in}{1.669949in}}%
\pgfpathlineto{\pgfqpoint{4.383287in}{0.911983in}}%
\pgfpathlineto{\pgfqpoint{4.383604in}{0.799091in}}%
\pgfpathlineto{\pgfqpoint{4.383922in}{1.636513in}}%
\pgfpathlineto{\pgfqpoint{4.384557in}{0.657155in}}%
\pgfpathlineto{\pgfqpoint{4.384874in}{1.479908in}}%
\pgfpathlineto{\pgfqpoint{4.385191in}{1.417615in}}%
\pgfpathlineto{\pgfqpoint{4.385509in}{0.636944in}}%
\pgfpathlineto{\pgfqpoint{4.386144in}{1.609459in}}%
\pgfpathlineto{\pgfqpoint{4.386461in}{0.751584in}}%
\pgfpathlineto{\pgfqpoint{4.387096in}{1.681676in}}%
\pgfpathlineto{\pgfqpoint{4.387414in}{0.961137in}}%
\pgfpathlineto{\pgfqpoint{4.387731in}{0.767456in}}%
\pgfpathlineto{\pgfqpoint{4.388049in}{1.614947in}}%
\pgfpathlineto{\pgfqpoint{4.388684in}{0.640633in}}%
\pgfpathlineto{\pgfqpoint{4.389001in}{1.434862in}}%
\pgfpathlineto{\pgfqpoint{4.389318in}{1.455102in}}%
\pgfpathlineto{\pgfqpoint{4.389636in}{0.643218in}}%
\pgfpathlineto{\pgfqpoint{4.390271in}{1.620597in}}%
\pgfpathlineto{\pgfqpoint{4.390588in}{0.775832in}}%
\pgfpathlineto{\pgfqpoint{4.390906in}{0.931821in}}%
\pgfpathlineto{\pgfqpoint{4.391223in}{1.669477in}}%
\pgfpathlineto{\pgfqpoint{4.391858in}{0.728806in}}%
\pgfpathlineto{\pgfqpoint{4.392176in}{1.586802in}}%
\pgfpathlineto{\pgfqpoint{4.392810in}{0.630832in}}%
\pgfpathlineto{\pgfqpoint{4.393128in}{1.393354in}}%
\pgfpathlineto{\pgfqpoint{4.393445in}{1.496309in}}%
\pgfpathlineto{\pgfqpoint{4.393763in}{0.661799in}}%
\pgfpathlineto{\pgfqpoint{4.394398in}{1.650405in}}%
\pgfpathlineto{\pgfqpoint{4.394715in}{0.818909in}}%
\pgfpathlineto{\pgfqpoint{4.395033in}{0.897133in}}%
\pgfpathlineto{\pgfqpoint{4.395350in}{1.668257in}}%
\pgfpathlineto{\pgfqpoint{4.395985in}{0.701512in}}%
\pgfpathlineto{\pgfqpoint{4.396302in}{1.554882in}}%
\pgfpathlineto{\pgfqpoint{4.396937in}{0.622380in}}%
\pgfpathlineto{\pgfqpoint{4.397255in}{1.342705in}}%
\pgfpathlineto{\pgfqpoint{4.397572in}{1.525877in}}%
\pgfpathlineto{\pgfqpoint{4.397890in}{0.676099in}}%
\pgfpathlineto{\pgfqpoint{4.398525in}{1.651273in}}%
\pgfpathlineto{\pgfqpoint{4.398842in}{0.849396in}}%
\pgfpathlineto{\pgfqpoint{4.399160in}{0.843039in}}%
\pgfpathlineto{\pgfqpoint{4.399477in}{1.649685in}}%
\pgfpathlineto{\pgfqpoint{4.400112in}{0.673712in}}%
\pgfpathlineto{\pgfqpoint{4.400429in}{1.521365in}}%
\pgfpathlineto{\pgfqpoint{4.401064in}{0.623490in}}%
\pgfpathlineto{\pgfqpoint{4.401382in}{1.299666in}}%
\pgfpathlineto{\pgfqpoint{4.401699in}{1.564861in}}%
\pgfpathlineto{\pgfqpoint{4.402017in}{0.705854in}}%
\pgfpathlineto{\pgfqpoint{4.402652in}{1.670868in}}%
\pgfpathlineto{\pgfqpoint{4.402969in}{0.896541in}}%
\pgfpathlineto{\pgfqpoint{4.403287in}{0.810009in}}%
\pgfpathlineto{\pgfqpoint{4.403604in}{1.635921in}}%
\pgfpathlineto{\pgfqpoint{4.404239in}{0.652650in}}%
\pgfpathlineto{\pgfqpoint{4.404556in}{1.481046in}}%
\pgfpathlineto{\pgfqpoint{4.404874in}{1.393230in}}%
\pgfpathlineto{\pgfqpoint{4.405191in}{0.623748in}}%
\pgfpathlineto{\pgfqpoint{4.405826in}{1.583045in}}%
\pgfpathlineto{\pgfqpoint{4.406144in}{0.726129in}}%
\pgfpathlineto{\pgfqpoint{4.406779in}{1.663369in}}%
\pgfpathlineto{\pgfqpoint{4.407096in}{0.933064in}}%
\pgfpathlineto{\pgfqpoint{4.407414in}{0.765003in}}%
\pgfpathlineto{\pgfqpoint{4.407731in}{1.611452in}}%
\pgfpathlineto{\pgfqpoint{4.408366in}{0.635774in}}%
\pgfpathlineto{\pgfqpoint{4.408683in}{1.442678in}}%
\pgfpathlineto{\pgfqpoint{4.409001in}{1.436036in}}%
\pgfpathlineto{\pgfqpoint{4.409318in}{0.634935in}}%
\pgfpathlineto{\pgfqpoint{4.409953in}{1.617609in}}%
\pgfpathlineto{\pgfqpoint{4.410271in}{0.764856in}}%
\pgfpathlineto{\pgfqpoint{4.410906in}{1.670507in}}%
\pgfpathlineto{\pgfqpoint{4.411223in}{0.981863in}}%
\pgfpathlineto{\pgfqpoint{4.411540in}{0.734832in}}%
\pgfpathlineto{\pgfqpoint{4.411858in}{1.586145in}}%
\pgfpathlineto{\pgfqpoint{4.412493in}{0.621970in}}%
\pgfpathlineto{\pgfqpoint{4.412810in}{1.395770in}}%
\pgfpathlineto{\pgfqpoint{4.413128in}{1.470986in}}%
\pgfpathlineto{\pgfqpoint{4.413445in}{0.644100in}}%
\pgfpathlineto{\pgfqpoint{4.414080in}{1.624391in}}%
\pgfpathlineto{\pgfqpoint{4.414398in}{0.790926in}}%
\pgfpathlineto{\pgfqpoint{4.414715in}{0.890833in}}%
\pgfpathlineto{\pgfqpoint{4.415032in}{1.656357in}}%
\pgfpathlineto{\pgfqpoint{4.415667in}{0.700350in}}%
\pgfpathlineto{\pgfqpoint{4.415985in}{1.556480in}}%
\pgfpathlineto{\pgfqpoint{4.416620in}{0.616200in}}%
\pgfpathlineto{\pgfqpoint{4.416937in}{1.353416in}}%
\pgfpathlineto{\pgfqpoint{4.417255in}{1.511318in}}%
\pgfpathlineto{\pgfqpoint{4.417572in}{0.666168in}}%
\pgfpathlineto{\pgfqpoint{4.418207in}{1.650858in}}%
\pgfpathlineto{\pgfqpoint{4.418525in}{0.835451in}}%
\pgfpathlineto{\pgfqpoint{4.418842in}{0.856954in}}%
\pgfpathlineto{\pgfqpoint{4.419159in}{1.651161in}}%
\pgfpathlineto{\pgfqpoint{4.419794in}{0.675521in}}%
\pgfpathlineto{\pgfqpoint{4.420112in}{1.522022in}}%
\pgfpathlineto{\pgfqpoint{4.420747in}{0.610744in}}%
\pgfpathlineto{\pgfqpoint{4.421064in}{1.301687in}}%
\pgfpathlineto{\pgfqpoint{4.421382in}{1.536952in}}%
\pgfpathlineto{\pgfqpoint{4.421699in}{0.682333in}}%
\pgfpathlineto{\pgfqpoint{4.422334in}{1.648378in}}%
\pgfpathlineto{\pgfqpoint{4.422651in}{0.867747in}}%
\pgfpathlineto{\pgfqpoint{4.422969in}{0.805906in}}%
\pgfpathlineto{\pgfqpoint{4.423286in}{1.630571in}}%
\pgfpathlineto{\pgfqpoint{4.423921in}{0.651290in}}%
\pgfpathlineto{\pgfqpoint{4.424239in}{1.486932in}}%
\pgfpathlineto{\pgfqpoint{4.424556in}{1.373150in}}%
\pgfpathlineto{\pgfqpoint{4.424874in}{0.615329in}}%
\pgfpathlineto{\pgfqpoint{4.425509in}{1.574746in}}%
\pgfpathlineto{\pgfqpoint{4.425826in}{0.714838in}}%
\pgfpathlineto{\pgfqpoint{4.426461in}{1.663993in}}%
\pgfpathlineto{\pgfqpoint{4.426778in}{0.915412in}}%
\pgfpathlineto{\pgfqpoint{4.427096in}{0.774037in}}%
\pgfpathlineto{\pgfqpoint{4.427413in}{1.613249in}}%
\pgfpathlineto{\pgfqpoint{4.428048in}{0.632936in}}%
\pgfpathlineto{\pgfqpoint{4.428366in}{1.444814in}}%
\pgfpathlineto{\pgfqpoint{4.428683in}{1.411635in}}%
\pgfpathlineto{\pgfqpoint{4.429001in}{0.618546in}}%
\pgfpathlineto{\pgfqpoint{4.429636in}{1.588790in}}%
\pgfpathlineto{\pgfqpoint{4.429953in}{0.736625in}}%
\pgfpathlineto{\pgfqpoint{4.430588in}{1.653865in}}%
\pgfpathlineto{\pgfqpoint{4.430905in}{0.953383in}}%
\pgfpathlineto{\pgfqpoint{4.431223in}{0.732472in}}%
\pgfpathlineto{\pgfqpoint{4.431540in}{1.587131in}}%
\pgfpathlineto{\pgfqpoint{4.432175in}{0.619482in}}%
\pgfpathlineto{\pgfqpoint{4.432493in}{1.405292in}}%
\pgfpathlineto{\pgfqpoint{4.432810in}{1.452940in}}%
\pgfpathlineto{\pgfqpoint{4.433128in}{0.633042in}}%
\pgfpathlineto{\pgfqpoint{4.433763in}{1.621010in}}%
\pgfpathlineto{\pgfqpoint{4.434080in}{0.776890in}}%
\pgfpathlineto{\pgfqpoint{4.434397in}{0.910512in}}%
\pgfpathlineto{\pgfqpoint{4.434715in}{1.657349in}}%
\pgfpathlineto{\pgfqpoint{4.435350in}{0.704437in}}%
\pgfpathlineto{\pgfqpoint{4.435667in}{1.559356in}}%
\pgfpathlineto{\pgfqpoint{4.436302in}{0.608525in}}%
\pgfpathlineto{\pgfqpoint{4.436620in}{1.357124in}}%
\pgfpathlineto{\pgfqpoint{4.436937in}{1.484923in}}%
\pgfpathlineto{\pgfqpoint{4.437255in}{0.644680in}}%
\pgfpathlineto{\pgfqpoint{4.437889in}{1.624627in}}%
\pgfpathlineto{\pgfqpoint{4.438207in}{0.804630in}}%
\pgfpathlineto{\pgfqpoint{4.438524in}{0.854616in}}%
\pgfpathlineto{\pgfqpoint{4.438842in}{1.640693in}}%
\pgfpathlineto{\pgfqpoint{4.439477in}{0.673003in}}%
\pgfpathlineto{\pgfqpoint{4.439794in}{1.527838in}}%
\pgfpathlineto{\pgfqpoint{4.440429in}{0.605811in}}%
\pgfpathlineto{\pgfqpoint{4.440747in}{1.314498in}}%
\pgfpathlineto{\pgfqpoint{4.441064in}{1.523975in}}%
\pgfpathlineto{\pgfqpoint{4.441381in}{0.669940in}}%
\pgfpathlineto{\pgfqpoint{4.442016in}{1.648152in}}%
\pgfpathlineto{\pgfqpoint{4.442334in}{0.849871in}}%
\pgfpathlineto{\pgfqpoint{4.442651in}{0.822020in}}%
\pgfpathlineto{\pgfqpoint{4.442969in}{1.632215in}}%
\pgfpathlineto{\pgfqpoint{4.443604in}{0.650715in}}%
\pgfpathlineto{\pgfqpoint{4.443921in}{1.491522in}}%
\pgfpathlineto{\pgfqpoint{4.444239in}{1.344957in}}%
\pgfpathlineto{\pgfqpoint{4.444556in}{0.603147in}}%
\pgfpathlineto{\pgfqpoint{4.445191in}{1.545847in}}%
\pgfpathlineto{\pgfqpoint{4.445508in}{0.687905in}}%
\pgfpathlineto{\pgfqpoint{4.446143in}{1.642944in}}%
\pgfpathlineto{\pgfqpoint{4.446461in}{0.883436in}}%
\pgfpathlineto{\pgfqpoint{4.446778in}{0.773881in}}%
\pgfpathlineto{\pgfqpoint{4.447096in}{1.609394in}}%
\pgfpathlineto{\pgfqpoint{4.447731in}{0.629478in}}%
\pgfpathlineto{\pgfqpoint{4.448048in}{1.454914in}}%
\pgfpathlineto{\pgfqpoint{4.448366in}{1.388105in}}%
\pgfpathlineto{\pgfqpoint{4.448683in}{0.610360in}}%
\pgfpathlineto{\pgfqpoint{4.449318in}{1.581795in}}%
\pgfpathlineto{\pgfqpoint{4.449635in}{0.722822in}}%
\pgfpathlineto{\pgfqpoint{4.449953in}{0.956466in}}%
\pgfpathlineto{\pgfqpoint{4.449953in}{0.956466in}}%
\pgfusepath{stroke}%
\end{pgfscope}%
\begin{pgfscope}%
\pgfpathrectangle{\pgfqpoint{0.450320in}{0.472202in}}{\pgfqpoint{4.190092in}{2.880788in}}%
\pgfusepath{clip}%
\pgfsetrectcap%
\pgfsetroundjoin%
\pgfsetlinewidth{1.505625pt}%
\definecolor{currentstroke}{rgb}{1.000000,0.498039,0.054902}%
\pgfsetstrokecolor{currentstroke}%
\pgfsetdash{}{0pt}%
\pgfpathmoveto{\pgfqpoint{0.640778in}{2.701518in}}%
\pgfpathlineto{\pgfqpoint{0.641413in}{2.700511in}}%
\pgfpathlineto{\pgfqpoint{0.641731in}{2.702074in}}%
\pgfpathlineto{\pgfqpoint{0.642048in}{2.711497in}}%
\pgfpathlineto{\pgfqpoint{0.643001in}{2.711370in}}%
\pgfpathlineto{\pgfqpoint{0.644588in}{2.696463in}}%
\pgfpathlineto{\pgfqpoint{0.644905in}{2.695228in}}%
\pgfpathlineto{\pgfqpoint{0.645223in}{2.697729in}}%
\pgfpathlineto{\pgfqpoint{0.645540in}{2.697024in}}%
\pgfpathlineto{\pgfqpoint{0.646810in}{2.699028in}}%
\pgfpathlineto{\pgfqpoint{0.647445in}{2.697646in}}%
\pgfpathlineto{\pgfqpoint{0.648080in}{2.698432in}}%
\pgfpathlineto{\pgfqpoint{0.649032in}{2.702400in}}%
\pgfpathlineto{\pgfqpoint{0.649350in}{2.710408in}}%
\pgfpathlineto{\pgfqpoint{0.649985in}{2.706109in}}%
\pgfpathlineto{\pgfqpoint{0.650937in}{2.697414in}}%
\pgfpathlineto{\pgfqpoint{0.651255in}{2.692733in}}%
\pgfpathlineto{\pgfqpoint{0.652207in}{2.692965in}}%
\pgfpathlineto{\pgfqpoint{0.653477in}{2.696156in}}%
\pgfpathlineto{\pgfqpoint{0.654747in}{2.694611in}}%
\pgfpathlineto{\pgfqpoint{0.655381in}{2.696397in}}%
\pgfpathlineto{\pgfqpoint{0.656651in}{2.706872in}}%
\pgfpathlineto{\pgfqpoint{0.658239in}{2.692842in}}%
\pgfpathlineto{\pgfqpoint{0.658556in}{2.688804in}}%
\pgfpathlineto{\pgfqpoint{0.659508in}{2.690822in}}%
\pgfpathlineto{\pgfqpoint{0.660461in}{2.693147in}}%
\pgfpathlineto{\pgfqpoint{0.660778in}{2.693013in}}%
\pgfpathlineto{\pgfqpoint{0.661096in}{2.691408in}}%
\pgfpathlineto{\pgfqpoint{0.662048in}{2.691878in}}%
\pgfpathlineto{\pgfqpoint{0.662683in}{2.698285in}}%
\pgfpathlineto{\pgfqpoint{0.663000in}{2.701349in}}%
\pgfpathlineto{\pgfqpoint{0.663318in}{2.697073in}}%
\pgfpathlineto{\pgfqpoint{0.663953in}{2.700735in}}%
\pgfpathlineto{\pgfqpoint{0.665223in}{2.688404in}}%
\pgfpathlineto{\pgfqpoint{0.665540in}{2.688668in}}%
\pgfpathlineto{\pgfqpoint{0.665858in}{2.685807in}}%
\pgfpathlineto{\pgfqpoint{0.666492in}{2.688316in}}%
\pgfpathlineto{\pgfqpoint{0.668080in}{2.689936in}}%
\pgfpathlineto{\pgfqpoint{0.668397in}{2.688382in}}%
\pgfpathlineto{\pgfqpoint{0.669032in}{2.690691in}}%
\pgfpathlineto{\pgfqpoint{0.669667in}{2.692887in}}%
\pgfpathlineto{\pgfqpoint{0.669985in}{2.699356in}}%
\pgfpathlineto{\pgfqpoint{0.670937in}{2.696637in}}%
\pgfpathlineto{\pgfqpoint{0.673159in}{2.683288in}}%
\pgfpathlineto{\pgfqpoint{0.674429in}{2.686988in}}%
\pgfpathlineto{\pgfqpoint{0.675699in}{2.685432in}}%
\pgfpathlineto{\pgfqpoint{0.677286in}{2.697779in}}%
\pgfpathlineto{\pgfqpoint{0.677604in}{2.692944in}}%
\pgfpathlineto{\pgfqpoint{0.679508in}{2.681198in}}%
\pgfpathlineto{\pgfqpoint{0.679826in}{2.681316in}}%
\pgfpathlineto{\pgfqpoint{0.681730in}{2.683875in}}%
\pgfpathlineto{\pgfqpoint{0.683000in}{2.682491in}}%
\pgfpathlineto{\pgfqpoint{0.683635in}{2.694644in}}%
\pgfpathlineto{\pgfqpoint{0.684588in}{2.693198in}}%
\pgfpathlineto{\pgfqpoint{0.686175in}{2.679428in}}%
\pgfpathlineto{\pgfqpoint{0.686810in}{2.678234in}}%
\pgfpathlineto{\pgfqpoint{0.687445in}{2.679696in}}%
\pgfpathlineto{\pgfqpoint{0.688080in}{2.680146in}}%
\pgfpathlineto{\pgfqpoint{0.688715in}{2.681158in}}%
\pgfpathlineto{\pgfqpoint{0.689032in}{2.680771in}}%
\pgfpathlineto{\pgfqpoint{0.689349in}{2.680068in}}%
\pgfpathlineto{\pgfqpoint{0.689984in}{2.681128in}}%
\pgfpathlineto{\pgfqpoint{0.690302in}{2.680683in}}%
\pgfpathlineto{\pgfqpoint{0.690937in}{2.694028in}}%
\pgfpathlineto{\pgfqpoint{0.691889in}{2.686942in}}%
\pgfpathlineto{\pgfqpoint{0.693476in}{2.675759in}}%
\pgfpathlineto{\pgfqpoint{0.694429in}{2.676327in}}%
\pgfpathlineto{\pgfqpoint{0.696016in}{2.678144in}}%
\pgfpathlineto{\pgfqpoint{0.696651in}{2.677087in}}%
\pgfpathlineto{\pgfqpoint{0.696968in}{2.677889in}}%
\pgfpathlineto{\pgfqpoint{0.698238in}{2.691853in}}%
\pgfpathlineto{\pgfqpoint{0.698556in}{2.682641in}}%
\pgfpathlineto{\pgfqpoint{0.698873in}{2.682771in}}%
\pgfpathlineto{\pgfqpoint{0.700143in}{2.672467in}}%
\pgfpathlineto{\pgfqpoint{0.701095in}{2.673051in}}%
\pgfpathlineto{\pgfqpoint{0.702365in}{2.675390in}}%
\pgfpathlineto{\pgfqpoint{0.703635in}{2.674608in}}%
\pgfpathlineto{\pgfqpoint{0.703952in}{2.674192in}}%
\pgfpathlineto{\pgfqpoint{0.704270in}{2.674989in}}%
\pgfpathlineto{\pgfqpoint{0.705540in}{2.687655in}}%
\pgfpathlineto{\pgfqpoint{0.707127in}{2.670848in}}%
\pgfpathlineto{\pgfqpoint{0.707445in}{2.668774in}}%
\pgfpathlineto{\pgfqpoint{0.708079in}{2.670187in}}%
\pgfpathlineto{\pgfqpoint{0.709667in}{2.672536in}}%
\pgfpathlineto{\pgfqpoint{0.709984in}{2.671247in}}%
\pgfpathlineto{\pgfqpoint{0.710937in}{2.671619in}}%
\pgfpathlineto{\pgfqpoint{0.711254in}{2.671328in}}%
\pgfpathlineto{\pgfqpoint{0.711571in}{2.672457in}}%
\pgfpathlineto{\pgfqpoint{0.712841in}{2.682160in}}%
\pgfpathlineto{\pgfqpoint{0.714111in}{2.667091in}}%
\pgfpathlineto{\pgfqpoint{0.714429in}{2.667301in}}%
\pgfpathlineto{\pgfqpoint{0.714746in}{2.665883in}}%
\pgfpathlineto{\pgfqpoint{0.715064in}{2.668238in}}%
\pgfpathlineto{\pgfqpoint{0.715381in}{2.667843in}}%
\pgfpathlineto{\pgfqpoint{0.716651in}{2.669732in}}%
\pgfpathlineto{\pgfqpoint{0.716968in}{2.669405in}}%
\pgfpathlineto{\pgfqpoint{0.717286in}{2.668135in}}%
\pgfpathlineto{\pgfqpoint{0.717921in}{2.669075in}}%
\pgfpathlineto{\pgfqpoint{0.718873in}{2.672036in}}%
\pgfpathlineto{\pgfqpoint{0.719190in}{2.680795in}}%
\pgfpathlineto{\pgfqpoint{0.719825in}{2.676618in}}%
\pgfpathlineto{\pgfqpoint{0.720143in}{2.676334in}}%
\pgfpathlineto{\pgfqpoint{0.721095in}{2.663714in}}%
\pgfpathlineto{\pgfqpoint{0.721730in}{2.664377in}}%
\pgfpathlineto{\pgfqpoint{0.722048in}{2.663546in}}%
\pgfpathlineto{\pgfqpoint{0.722365in}{2.665785in}}%
\pgfpathlineto{\pgfqpoint{0.722682in}{2.665431in}}%
\pgfpathlineto{\pgfqpoint{0.723000in}{2.666512in}}%
\pgfpathlineto{\pgfqpoint{0.723317in}{2.666815in}}%
\pgfpathlineto{\pgfqpoint{0.724587in}{2.665153in}}%
\pgfpathlineto{\pgfqpoint{0.726175in}{2.672586in}}%
\pgfpathlineto{\pgfqpoint{0.726492in}{2.677829in}}%
\pgfpathlineto{\pgfqpoint{0.726809in}{2.670862in}}%
\pgfpathlineto{\pgfqpoint{0.727127in}{2.672212in}}%
\pgfpathlineto{\pgfqpoint{0.728397in}{2.659791in}}%
\pgfpathlineto{\pgfqpoint{0.729349in}{2.661402in}}%
\pgfpathlineto{\pgfqpoint{0.730301in}{2.663825in}}%
\pgfpathlineto{\pgfqpoint{0.730619in}{2.663754in}}%
\pgfpathlineto{\pgfqpoint{0.731889in}{2.662153in}}%
\pgfpathlineto{\pgfqpoint{0.732524in}{2.667059in}}%
\pgfpathlineto{\pgfqpoint{0.733794in}{2.672495in}}%
\pgfpathlineto{\pgfqpoint{0.735063in}{2.659471in}}%
\pgfpathlineto{\pgfqpoint{0.735381in}{2.659895in}}%
\pgfpathlineto{\pgfqpoint{0.735698in}{2.656472in}}%
\pgfpathlineto{\pgfqpoint{0.736333in}{2.659171in}}%
\pgfpathlineto{\pgfqpoint{0.736651in}{2.659178in}}%
\pgfpathlineto{\pgfqpoint{0.737603in}{2.660851in}}%
\pgfpathlineto{\pgfqpoint{0.737920in}{2.660650in}}%
\pgfpathlineto{\pgfqpoint{0.738238in}{2.659211in}}%
\pgfpathlineto{\pgfqpoint{0.738873in}{2.660673in}}%
\pgfpathlineto{\pgfqpoint{0.739190in}{2.660835in}}%
\pgfpathlineto{\pgfqpoint{0.740143in}{2.668849in}}%
\pgfpathlineto{\pgfqpoint{0.741095in}{2.665993in}}%
\pgfpathlineto{\pgfqpoint{0.742365in}{2.655457in}}%
\pgfpathlineto{\pgfqpoint{0.742682in}{2.656243in}}%
\pgfpathlineto{\pgfqpoint{0.743000in}{2.653901in}}%
\pgfpathlineto{\pgfqpoint{0.743635in}{2.656756in}}%
\pgfpathlineto{\pgfqpoint{0.745222in}{2.657668in}}%
\pgfpathlineto{\pgfqpoint{0.745539in}{2.656332in}}%
\pgfpathlineto{\pgfqpoint{0.745857in}{2.657753in}}%
\pgfpathlineto{\pgfqpoint{0.747127in}{2.668220in}}%
\pgfpathlineto{\pgfqpoint{0.747444in}{2.664572in}}%
\pgfpathlineto{\pgfqpoint{0.749666in}{2.652033in}}%
\pgfpathlineto{\pgfqpoint{0.749984in}{2.653080in}}%
\pgfpathlineto{\pgfqpoint{0.750301in}{2.651700in}}%
\pgfpathlineto{\pgfqpoint{0.750619in}{2.653117in}}%
\pgfpathlineto{\pgfqpoint{0.751571in}{2.654590in}}%
\pgfpathlineto{\pgfqpoint{0.751889in}{2.654236in}}%
\pgfpathlineto{\pgfqpoint{0.752524in}{2.654719in}}%
\pgfpathlineto{\pgfqpoint{0.752841in}{2.653452in}}%
\pgfpathlineto{\pgfqpoint{0.753158in}{2.654918in}}%
\pgfpathlineto{\pgfqpoint{0.754428in}{2.665092in}}%
\pgfpathlineto{\pgfqpoint{0.756016in}{2.651484in}}%
\pgfpathlineto{\pgfqpoint{0.756650in}{2.648925in}}%
\pgfpathlineto{\pgfqpoint{0.757603in}{2.649832in}}%
\pgfpathlineto{\pgfqpoint{0.759508in}{2.651687in}}%
\pgfpathlineto{\pgfqpoint{0.760142in}{2.651008in}}%
\pgfpathlineto{\pgfqpoint{0.760777in}{2.663442in}}%
\pgfpathlineto{\pgfqpoint{0.761730in}{2.659777in}}%
\pgfpathlineto{\pgfqpoint{0.763317in}{2.647479in}}%
\pgfpathlineto{\pgfqpoint{0.763952in}{2.646336in}}%
\pgfpathlineto{\pgfqpoint{0.764269in}{2.646723in}}%
\pgfpathlineto{\pgfqpoint{0.765857in}{2.649173in}}%
\pgfpathlineto{\pgfqpoint{0.766492in}{2.648331in}}%
\pgfpathlineto{\pgfqpoint{0.766809in}{2.648736in}}%
\pgfpathlineto{\pgfqpoint{0.767761in}{2.652641in}}%
\pgfpathlineto{\pgfqpoint{0.768079in}{2.662070in}}%
\pgfpathlineto{\pgfqpoint{0.768714in}{2.654912in}}%
\pgfpathlineto{\pgfqpoint{0.769031in}{2.653855in}}%
\pgfpathlineto{\pgfqpoint{0.770619in}{2.644017in}}%
\pgfpathlineto{\pgfqpoint{0.770936in}{2.643481in}}%
\pgfpathlineto{\pgfqpoint{0.771571in}{2.644408in}}%
\pgfpathlineto{\pgfqpoint{0.772841in}{2.646409in}}%
\pgfpathlineto{\pgfqpoint{0.773158in}{2.646177in}}%
\pgfpathlineto{\pgfqpoint{0.773793in}{2.645465in}}%
\pgfpathlineto{\pgfqpoint{0.774111in}{2.645912in}}%
\pgfpathlineto{\pgfqpoint{0.775380in}{2.659492in}}%
\pgfpathlineto{\pgfqpoint{0.776650in}{2.642805in}}%
\pgfpathlineto{\pgfqpoint{0.776968in}{2.642861in}}%
\pgfpathlineto{\pgfqpoint{0.777285in}{2.640253in}}%
\pgfpathlineto{\pgfqpoint{0.778238in}{2.641159in}}%
\pgfpathlineto{\pgfqpoint{0.779507in}{2.643546in}}%
\pgfpathlineto{\pgfqpoint{0.780777in}{2.642439in}}%
\pgfpathlineto{\pgfqpoint{0.781412in}{2.643056in}}%
\pgfpathlineto{\pgfqpoint{0.782682in}{2.655533in}}%
\pgfpathlineto{\pgfqpoint{0.783952in}{2.638624in}}%
\pgfpathlineto{\pgfqpoint{0.784269in}{2.639210in}}%
\pgfpathlineto{\pgfqpoint{0.784587in}{2.636992in}}%
\pgfpathlineto{\pgfqpoint{0.785222in}{2.638781in}}%
\pgfpathlineto{\pgfqpoint{0.786809in}{2.640766in}}%
\pgfpathlineto{\pgfqpoint{0.787126in}{2.639237in}}%
\pgfpathlineto{\pgfqpoint{0.788079in}{2.639510in}}%
\pgfpathlineto{\pgfqpoint{0.788714in}{2.640820in}}%
\pgfpathlineto{\pgfqpoint{0.789031in}{2.650259in}}%
\pgfpathlineto{\pgfqpoint{0.789984in}{2.650149in}}%
\pgfpathlineto{\pgfqpoint{0.791253in}{2.634858in}}%
\pgfpathlineto{\pgfqpoint{0.791571in}{2.635871in}}%
\pgfpathlineto{\pgfqpoint{0.791888in}{2.634228in}}%
\pgfpathlineto{\pgfqpoint{0.792523in}{2.636541in}}%
\pgfpathlineto{\pgfqpoint{0.793793in}{2.637978in}}%
\pgfpathlineto{\pgfqpoint{0.794110in}{2.637733in}}%
\pgfpathlineto{\pgfqpoint{0.794428in}{2.636228in}}%
\pgfpathlineto{\pgfqpoint{0.795063in}{2.637374in}}%
\pgfpathlineto{\pgfqpoint{0.795380in}{2.637108in}}%
\pgfpathlineto{\pgfqpoint{0.795698in}{2.637936in}}%
\pgfpathlineto{\pgfqpoint{0.796333in}{2.649056in}}%
\pgfpathlineto{\pgfqpoint{0.797285in}{2.644542in}}%
\pgfpathlineto{\pgfqpoint{0.798555in}{2.631685in}}%
\pgfpathlineto{\pgfqpoint{0.800460in}{2.635016in}}%
\pgfpathlineto{\pgfqpoint{0.801729in}{2.633333in}}%
\pgfpathlineto{\pgfqpoint{0.803317in}{2.640801in}}%
\pgfpathlineto{\pgfqpoint{0.803634in}{2.646057in}}%
\pgfpathlineto{\pgfqpoint{0.803952in}{2.639834in}}%
\pgfpathlineto{\pgfqpoint{0.804269in}{2.640464in}}%
\pgfpathlineto{\pgfqpoint{0.805539in}{2.628438in}}%
\pgfpathlineto{\pgfqpoint{0.806491in}{2.629736in}}%
\pgfpathlineto{\pgfqpoint{0.807444in}{2.632156in}}%
\pgfpathlineto{\pgfqpoint{0.807761in}{2.632033in}}%
\pgfpathlineto{\pgfqpoint{0.809031in}{2.630473in}}%
\pgfpathlineto{\pgfqpoint{0.810936in}{2.641346in}}%
\pgfpathlineto{\pgfqpoint{0.811253in}{2.635471in}}%
\pgfpathlineto{\pgfqpoint{0.811571in}{2.635793in}}%
\pgfpathlineto{\pgfqpoint{0.812840in}{2.625115in}}%
\pgfpathlineto{\pgfqpoint{0.813158in}{2.626214in}}%
\pgfpathlineto{\pgfqpoint{0.814745in}{2.629306in}}%
\pgfpathlineto{\pgfqpoint{0.816333in}{2.628869in}}%
\pgfpathlineto{\pgfqpoint{0.817920in}{2.637711in}}%
\pgfpathlineto{\pgfqpoint{0.818237in}{2.635730in}}%
\pgfpathlineto{\pgfqpoint{0.820142in}{2.622473in}}%
\pgfpathlineto{\pgfqpoint{0.822047in}{2.626379in}}%
\pgfpathlineto{\pgfqpoint{0.822682in}{2.625162in}}%
\pgfpathlineto{\pgfqpoint{0.822999in}{2.626271in}}%
\pgfpathlineto{\pgfqpoint{0.824269in}{2.636375in}}%
\pgfpathlineto{\pgfqpoint{0.825221in}{2.633588in}}%
\pgfpathlineto{\pgfqpoint{0.826809in}{2.620763in}}%
\pgfpathlineto{\pgfqpoint{0.827126in}{2.621593in}}%
\pgfpathlineto{\pgfqpoint{0.827444in}{2.620216in}}%
\pgfpathlineto{\pgfqpoint{0.827761in}{2.621440in}}%
\pgfpathlineto{\pgfqpoint{0.829031in}{2.623315in}}%
\pgfpathlineto{\pgfqpoint{0.830301in}{2.623470in}}%
\pgfpathlineto{\pgfqpoint{0.831570in}{2.634122in}}%
\pgfpathlineto{\pgfqpoint{0.833158in}{2.621461in}}%
\pgfpathlineto{\pgfqpoint{0.834110in}{2.617894in}}%
\pgfpathlineto{\pgfqpoint{0.835063in}{2.619080in}}%
\pgfpathlineto{\pgfqpoint{0.835697in}{2.620751in}}%
\pgfpathlineto{\pgfqpoint{0.836332in}{2.620459in}}%
\pgfpathlineto{\pgfqpoint{0.837602in}{2.621427in}}%
\pgfpathlineto{\pgfqpoint{0.837920in}{2.630087in}}%
\pgfpathlineto{\pgfqpoint{0.838872in}{2.629731in}}%
\pgfpathlineto{\pgfqpoint{0.840142in}{2.617360in}}%
\pgfpathlineto{\pgfqpoint{0.840459in}{2.617382in}}%
\pgfpathlineto{\pgfqpoint{0.841094in}{2.614891in}}%
\pgfpathlineto{\pgfqpoint{0.841729in}{2.616174in}}%
\pgfpathlineto{\pgfqpoint{0.844586in}{2.618514in}}%
\pgfpathlineto{\pgfqpoint{0.844904in}{2.620216in}}%
\pgfpathlineto{\pgfqpoint{0.845221in}{2.629421in}}%
\pgfpathlineto{\pgfqpoint{0.845856in}{2.624212in}}%
\pgfpathlineto{\pgfqpoint{0.846174in}{2.624718in}}%
\pgfpathlineto{\pgfqpoint{0.847443in}{2.612952in}}%
\pgfpathlineto{\pgfqpoint{0.847761in}{2.613666in}}%
\pgfpathlineto{\pgfqpoint{0.848078in}{2.612447in}}%
\pgfpathlineto{\pgfqpoint{0.848713in}{2.613039in}}%
\pgfpathlineto{\pgfqpoint{0.849983in}{2.615541in}}%
\pgfpathlineto{\pgfqpoint{0.850300in}{2.615299in}}%
\pgfpathlineto{\pgfqpoint{0.850618in}{2.614063in}}%
\pgfpathlineto{\pgfqpoint{0.851253in}{2.614883in}}%
\pgfpathlineto{\pgfqpoint{0.852523in}{2.627812in}}%
\pgfpathlineto{\pgfqpoint{0.853158in}{2.620085in}}%
\pgfpathlineto{\pgfqpoint{0.853475in}{2.618796in}}%
\pgfpathlineto{\pgfqpoint{0.854745in}{2.609428in}}%
\pgfpathlineto{\pgfqpoint{0.856332in}{2.611838in}}%
\pgfpathlineto{\pgfqpoint{0.857285in}{2.612771in}}%
\pgfpathlineto{\pgfqpoint{0.856967in}{2.611299in}}%
\pgfpathlineto{\pgfqpoint{0.857602in}{2.612429in}}%
\pgfpathlineto{\pgfqpoint{0.857919in}{2.611205in}}%
\pgfpathlineto{\pgfqpoint{0.858554in}{2.612176in}}%
\pgfpathlineto{\pgfqpoint{0.859824in}{2.624500in}}%
\pgfpathlineto{\pgfqpoint{0.861094in}{2.608118in}}%
\pgfpathlineto{\pgfqpoint{0.861411in}{2.609171in}}%
\pgfpathlineto{\pgfqpoint{0.862046in}{2.606693in}}%
\pgfpathlineto{\pgfqpoint{0.862681in}{2.607851in}}%
\pgfpathlineto{\pgfqpoint{0.863951in}{2.609970in}}%
\pgfpathlineto{\pgfqpoint{0.865221in}{2.608400in}}%
\pgfpathlineto{\pgfqpoint{0.865856in}{2.609531in}}%
\pgfpathlineto{\pgfqpoint{0.867126in}{2.619782in}}%
\pgfpathlineto{\pgfqpoint{0.868396in}{2.604109in}}%
\pgfpathlineto{\pgfqpoint{0.868713in}{2.605716in}}%
\pgfpathlineto{\pgfqpoint{0.869030in}{2.603632in}}%
\pgfpathlineto{\pgfqpoint{0.869665in}{2.606050in}}%
\pgfpathlineto{\pgfqpoint{0.870618in}{2.606324in}}%
\pgfpathlineto{\pgfqpoint{0.871253in}{2.607175in}}%
\pgfpathlineto{\pgfqpoint{0.872523in}{2.605629in}}%
\pgfpathlineto{\pgfqpoint{0.873157in}{2.608542in}}%
\pgfpathlineto{\pgfqpoint{0.873475in}{2.617126in}}%
\pgfpathlineto{\pgfqpoint{0.874427in}{2.614395in}}%
\pgfpathlineto{\pgfqpoint{0.875697in}{2.600825in}}%
\pgfpathlineto{\pgfqpoint{0.877284in}{2.603900in}}%
\pgfpathlineto{\pgfqpoint{0.878554in}{2.604274in}}%
\pgfpathlineto{\pgfqpoint{0.878872in}{2.602822in}}%
\pgfpathlineto{\pgfqpoint{0.879507in}{2.604148in}}%
\pgfpathlineto{\pgfqpoint{0.879824in}{2.603849in}}%
\pgfpathlineto{\pgfqpoint{0.880776in}{2.615172in}}%
\pgfpathlineto{\pgfqpoint{0.881411in}{2.610564in}}%
\pgfpathlineto{\pgfqpoint{0.882999in}{2.598001in}}%
\pgfpathlineto{\pgfqpoint{0.883951in}{2.600111in}}%
\pgfpathlineto{\pgfqpoint{0.884586in}{2.601620in}}%
\pgfpathlineto{\pgfqpoint{0.885221in}{2.600859in}}%
\pgfpathlineto{\pgfqpoint{0.885856in}{2.601447in}}%
\pgfpathlineto{\pgfqpoint{0.886173in}{2.600175in}}%
\pgfpathlineto{\pgfqpoint{0.886808in}{2.601879in}}%
\pgfpathlineto{\pgfqpoint{0.888078in}{2.611417in}}%
\pgfpathlineto{\pgfqpoint{0.888395in}{2.606911in}}%
\pgfpathlineto{\pgfqpoint{0.888713in}{2.606484in}}%
\pgfpathlineto{\pgfqpoint{0.889983in}{2.595142in}}%
\pgfpathlineto{\pgfqpoint{0.890300in}{2.595580in}}%
\pgfpathlineto{\pgfqpoint{0.891887in}{2.599036in}}%
\pgfpathlineto{\pgfqpoint{0.893475in}{2.597871in}}%
\pgfpathlineto{\pgfqpoint{0.895062in}{2.607437in}}%
\pgfpathlineto{\pgfqpoint{0.895697in}{2.602554in}}%
\pgfpathlineto{\pgfqpoint{0.896014in}{2.601684in}}%
\pgfpathlineto{\pgfqpoint{0.897284in}{2.592373in}}%
\pgfpathlineto{\pgfqpoint{0.897602in}{2.593318in}}%
\pgfpathlineto{\pgfqpoint{0.898871in}{2.596441in}}%
\pgfpathlineto{\pgfqpoint{0.900459in}{2.596076in}}%
\pgfpathlineto{\pgfqpoint{0.901729in}{2.604094in}}%
\pgfpathlineto{\pgfqpoint{0.902364in}{2.603990in}}%
\pgfpathlineto{\pgfqpoint{0.904586in}{2.590050in}}%
\pgfpathlineto{\pgfqpoint{0.906173in}{2.593715in}}%
\pgfpathlineto{\pgfqpoint{0.906490in}{2.593492in}}%
\pgfpathlineto{\pgfqpoint{0.907125in}{2.592882in}}%
\pgfpathlineto{\pgfqpoint{0.907443in}{2.593410in}}%
\pgfpathlineto{\pgfqpoint{0.908713in}{2.603497in}}%
\pgfpathlineto{\pgfqpoint{0.909030in}{2.600492in}}%
\pgfpathlineto{\pgfqpoint{0.911887in}{2.588112in}}%
\pgfpathlineto{\pgfqpoint{0.913475in}{2.590987in}}%
\pgfpathlineto{\pgfqpoint{0.913792in}{2.590720in}}%
\pgfpathlineto{\pgfqpoint{0.914109in}{2.590115in}}%
\pgfpathlineto{\pgfqpoint{0.914744in}{2.590783in}}%
\pgfpathlineto{\pgfqpoint{0.916014in}{2.600741in}}%
\pgfpathlineto{\pgfqpoint{0.917919in}{2.586958in}}%
\pgfpathlineto{\pgfqpoint{0.918236in}{2.585247in}}%
\pgfpathlineto{\pgfqpoint{0.918871in}{2.586178in}}%
\pgfpathlineto{\pgfqpoint{0.919506in}{2.586643in}}%
\pgfpathlineto{\pgfqpoint{0.920141in}{2.588463in}}%
\pgfpathlineto{\pgfqpoint{0.920776in}{2.588258in}}%
\pgfpathlineto{\pgfqpoint{0.921411in}{2.587401in}}%
\pgfpathlineto{\pgfqpoint{0.921728in}{2.588855in}}%
\pgfpathlineto{\pgfqpoint{0.922046in}{2.589496in}}%
\pgfpathlineto{\pgfqpoint{0.922363in}{2.597130in}}%
\pgfpathlineto{\pgfqpoint{0.923316in}{2.596653in}}%
\pgfpathlineto{\pgfqpoint{0.924586in}{2.583586in}}%
\pgfpathlineto{\pgfqpoint{0.924903in}{2.584627in}}%
\pgfpathlineto{\pgfqpoint{0.925538in}{2.582518in}}%
\pgfpathlineto{\pgfqpoint{0.925855in}{2.583448in}}%
\pgfpathlineto{\pgfqpoint{0.927443in}{2.585844in}}%
\pgfpathlineto{\pgfqpoint{0.927760in}{2.584265in}}%
\pgfpathlineto{\pgfqpoint{0.928713in}{2.585151in}}%
\pgfpathlineto{\pgfqpoint{0.929665in}{2.596438in}}%
\pgfpathlineto{\pgfqpoint{0.930300in}{2.590821in}}%
\pgfpathlineto{\pgfqpoint{0.930617in}{2.591574in}}%
\pgfpathlineto{\pgfqpoint{0.931887in}{2.579574in}}%
\pgfpathlineto{\pgfqpoint{0.932205in}{2.581687in}}%
\pgfpathlineto{\pgfqpoint{0.932839in}{2.580193in}}%
\pgfpathlineto{\pgfqpoint{0.933157in}{2.581514in}}%
\pgfpathlineto{\pgfqpoint{0.934427in}{2.583367in}}%
\pgfpathlineto{\pgfqpoint{0.934744in}{2.583163in}}%
\pgfpathlineto{\pgfqpoint{0.935062in}{2.581538in}}%
\pgfpathlineto{\pgfqpoint{0.935697in}{2.582807in}}%
\pgfpathlineto{\pgfqpoint{0.936966in}{2.594138in}}%
\pgfpathlineto{\pgfqpoint{0.937601in}{2.587186in}}%
\pgfpathlineto{\pgfqpoint{0.937919in}{2.586179in}}%
\pgfpathlineto{\pgfqpoint{0.939189in}{2.576395in}}%
\pgfpathlineto{\pgfqpoint{0.940458in}{2.579512in}}%
\pgfpathlineto{\pgfqpoint{0.941728in}{2.580656in}}%
\pgfpathlineto{\pgfqpoint{0.942046in}{2.580400in}}%
\pgfpathlineto{\pgfqpoint{0.942363in}{2.578882in}}%
\pgfpathlineto{\pgfqpoint{0.942998in}{2.580263in}}%
\pgfpathlineto{\pgfqpoint{0.944268in}{2.590714in}}%
\pgfpathlineto{\pgfqpoint{0.946173in}{2.575198in}}%
\pgfpathlineto{\pgfqpoint{0.946490in}{2.573878in}}%
\pgfpathlineto{\pgfqpoint{0.946808in}{2.576831in}}%
\pgfpathlineto{\pgfqpoint{0.947125in}{2.575867in}}%
\pgfpathlineto{\pgfqpoint{0.949347in}{2.577726in}}%
\pgfpathlineto{\pgfqpoint{0.949665in}{2.576289in}}%
\pgfpathlineto{\pgfqpoint{0.950300in}{2.577911in}}%
\pgfpathlineto{\pgfqpoint{0.951569in}{2.586721in}}%
\pgfpathlineto{\pgfqpoint{0.952839in}{2.572275in}}%
\pgfpathlineto{\pgfqpoint{0.953157in}{2.574228in}}%
\pgfpathlineto{\pgfqpoint{0.953792in}{2.571678in}}%
\pgfpathlineto{\pgfqpoint{0.954109in}{2.574613in}}%
\pgfpathlineto{\pgfqpoint{0.954427in}{2.573927in}}%
\pgfpathlineto{\pgfqpoint{0.955061in}{2.575058in}}%
\pgfpathlineto{\pgfqpoint{0.955696in}{2.575147in}}%
\pgfpathlineto{\pgfqpoint{0.956966in}{2.573724in}}%
\pgfpathlineto{\pgfqpoint{0.957919in}{2.583722in}}%
\pgfpathlineto{\pgfqpoint{0.958871in}{2.581798in}}%
\pgfpathlineto{\pgfqpoint{0.960141in}{2.569014in}}%
\pgfpathlineto{\pgfqpoint{0.960458in}{2.571239in}}%
\pgfpathlineto{\pgfqpoint{0.961093in}{2.569759in}}%
\pgfpathlineto{\pgfqpoint{0.962363in}{2.572573in}}%
\pgfpathlineto{\pgfqpoint{0.962998in}{2.572354in}}%
\pgfpathlineto{\pgfqpoint{0.963315in}{2.571557in}}%
\pgfpathlineto{\pgfqpoint{0.963950in}{2.572584in}}%
\pgfpathlineto{\pgfqpoint{0.964268in}{2.572752in}}%
\pgfpathlineto{\pgfqpoint{0.965220in}{2.581092in}}%
\pgfpathlineto{\pgfqpoint{0.965855in}{2.579001in}}%
\pgfpathlineto{\pgfqpoint{0.967442in}{2.566346in}}%
\pgfpathlineto{\pgfqpoint{0.968395in}{2.567700in}}%
\pgfpathlineto{\pgfqpoint{0.968712in}{2.570200in}}%
\pgfpathlineto{\pgfqpoint{0.969665in}{2.570011in}}%
\pgfpathlineto{\pgfqpoint{0.970299in}{2.569616in}}%
\pgfpathlineto{\pgfqpoint{0.970617in}{2.569040in}}%
\pgfpathlineto{\pgfqpoint{0.971252in}{2.569979in}}%
\pgfpathlineto{\pgfqpoint{0.972522in}{2.578777in}}%
\pgfpathlineto{\pgfqpoint{0.972839in}{2.576488in}}%
\pgfpathlineto{\pgfqpoint{0.974744in}{2.563841in}}%
\pgfpathlineto{\pgfqpoint{0.975696in}{2.565632in}}%
\pgfpathlineto{\pgfqpoint{0.976014in}{2.567870in}}%
\pgfpathlineto{\pgfqpoint{0.976966in}{2.567385in}}%
\pgfpathlineto{\pgfqpoint{0.978553in}{2.568425in}}%
\pgfpathlineto{\pgfqpoint{0.979506in}{2.575820in}}%
\pgfpathlineto{\pgfqpoint{0.979823in}{2.575121in}}%
\pgfpathlineto{\pgfqpoint{0.982045in}{2.561669in}}%
\pgfpathlineto{\pgfqpoint{0.983315in}{2.565415in}}%
\pgfpathlineto{\pgfqpoint{0.983633in}{2.565204in}}%
\pgfpathlineto{\pgfqpoint{0.983950in}{2.563948in}}%
\pgfpathlineto{\pgfqpoint{0.984903in}{2.564343in}}%
\pgfpathlineto{\pgfqpoint{0.985537in}{2.565072in}}%
\pgfpathlineto{\pgfqpoint{0.986172in}{2.573688in}}%
\pgfpathlineto{\pgfqpoint{0.987125in}{2.570609in}}%
\pgfpathlineto{\pgfqpoint{0.989029in}{2.559221in}}%
\pgfpathlineto{\pgfqpoint{0.991887in}{2.562181in}}%
\pgfpathlineto{\pgfqpoint{0.992204in}{2.561775in}}%
\pgfpathlineto{\pgfqpoint{0.993474in}{2.572244in}}%
\pgfpathlineto{\pgfqpoint{0.993791in}{2.568817in}}%
\pgfpathlineto{\pgfqpoint{0.994109in}{2.569608in}}%
\pgfpathlineto{\pgfqpoint{0.995379in}{2.556938in}}%
\pgfpathlineto{\pgfqpoint{0.996014in}{2.557736in}}%
\pgfpathlineto{\pgfqpoint{0.996331in}{2.557310in}}%
\pgfpathlineto{\pgfqpoint{0.996648in}{2.557797in}}%
\pgfpathlineto{\pgfqpoint{0.997918in}{2.560281in}}%
\pgfpathlineto{\pgfqpoint{0.998553in}{2.558546in}}%
\pgfpathlineto{\pgfqpoint{0.999506in}{2.559692in}}%
\pgfpathlineto{\pgfqpoint{1.000775in}{2.569062in}}%
\pgfpathlineto{\pgfqpoint{1.002680in}{2.553415in}}%
\pgfpathlineto{\pgfqpoint{1.002998in}{2.555107in}}%
\pgfpathlineto{\pgfqpoint{1.005537in}{2.557365in}}%
\pgfpathlineto{\pgfqpoint{1.005855in}{2.555968in}}%
\pgfpathlineto{\pgfqpoint{1.006490in}{2.557179in}}%
\pgfpathlineto{\pgfqpoint{1.007759in}{2.566418in}}%
\pgfpathlineto{\pgfqpoint{1.008077in}{2.564561in}}%
\pgfpathlineto{\pgfqpoint{1.009982in}{2.550302in}}%
\pgfpathlineto{\pgfqpoint{1.011569in}{2.555139in}}%
\pgfpathlineto{\pgfqpoint{1.011886in}{2.555094in}}%
\pgfpathlineto{\pgfqpoint{1.013156in}{2.553447in}}%
\pgfpathlineto{\pgfqpoint{1.014426in}{2.564473in}}%
\pgfpathlineto{\pgfqpoint{1.015061in}{2.563152in}}%
\pgfpathlineto{\pgfqpoint{1.017283in}{2.547868in}}%
\pgfpathlineto{\pgfqpoint{1.018870in}{2.552604in}}%
\pgfpathlineto{\pgfqpoint{1.020458in}{2.551026in}}%
\pgfpathlineto{\pgfqpoint{1.021728in}{2.561561in}}%
\pgfpathlineto{\pgfqpoint{1.023632in}{2.545923in}}%
\pgfpathlineto{\pgfqpoint{1.023950in}{2.548615in}}%
\pgfpathlineto{\pgfqpoint{1.024585in}{2.545850in}}%
\pgfpathlineto{\pgfqpoint{1.024902in}{2.549114in}}%
\pgfpathlineto{\pgfqpoint{1.026489in}{2.549706in}}%
\pgfpathlineto{\pgfqpoint{1.026807in}{2.548503in}}%
\pgfpathlineto{\pgfqpoint{1.027124in}{2.549911in}}%
\pgfpathlineto{\pgfqpoint{1.027442in}{2.549832in}}%
\pgfpathlineto{\pgfqpoint{1.028712in}{2.558365in}}%
\pgfpathlineto{\pgfqpoint{1.029347in}{2.554019in}}%
\pgfpathlineto{\pgfqpoint{1.030616in}{2.546182in}}%
\pgfpathlineto{\pgfqpoint{1.030934in}{2.542667in}}%
\pgfpathlineto{\pgfqpoint{1.031886in}{2.544212in}}%
\pgfpathlineto{\pgfqpoint{1.032521in}{2.547201in}}%
\pgfpathlineto{\pgfqpoint{1.033156in}{2.546886in}}%
\pgfpathlineto{\pgfqpoint{1.033474in}{2.547302in}}%
\pgfpathlineto{\pgfqpoint{1.033791in}{2.547009in}}%
\pgfpathlineto{\pgfqpoint{1.034108in}{2.546115in}}%
\pgfpathlineto{\pgfqpoint{1.034426in}{2.547550in}}%
\pgfpathlineto{\pgfqpoint{1.035061in}{2.553900in}}%
\pgfpathlineto{\pgfqpoint{1.035378in}{2.556088in}}%
\pgfpathlineto{\pgfqpoint{1.036013in}{2.553973in}}%
\pgfpathlineto{\pgfqpoint{1.038235in}{2.540408in}}%
\pgfpathlineto{\pgfqpoint{1.038870in}{2.543368in}}%
\pgfpathlineto{\pgfqpoint{1.039188in}{2.542637in}}%
\pgfpathlineto{\pgfqpoint{1.039505in}{2.545064in}}%
\pgfpathlineto{\pgfqpoint{1.040458in}{2.544345in}}%
\pgfpathlineto{\pgfqpoint{1.041727in}{2.546977in}}%
\pgfpathlineto{\pgfqpoint{1.042362in}{2.554296in}}%
\pgfpathlineto{\pgfqpoint{1.042997in}{2.551083in}}%
\pgfpathlineto{\pgfqpoint{1.045537in}{2.538726in}}%
\pgfpathlineto{\pgfqpoint{1.046807in}{2.542697in}}%
\pgfpathlineto{\pgfqpoint{1.048394in}{2.541949in}}%
\pgfpathlineto{\pgfqpoint{1.049664in}{2.552707in}}%
\pgfpathlineto{\pgfqpoint{1.049981in}{2.548687in}}%
\pgfpathlineto{\pgfqpoint{1.051886in}{2.536418in}}%
\pgfpathlineto{\pgfqpoint{1.052521in}{2.537795in}}%
\pgfpathlineto{\pgfqpoint{1.052838in}{2.537340in}}%
\pgfpathlineto{\pgfqpoint{1.054108in}{2.540114in}}%
\pgfpathlineto{\pgfqpoint{1.055696in}{2.539628in}}%
\pgfpathlineto{\pgfqpoint{1.056965in}{2.548294in}}%
\pgfpathlineto{\pgfqpoint{1.058870in}{2.533961in}}%
\pgfpathlineto{\pgfqpoint{1.060775in}{2.537639in}}%
\pgfpathlineto{\pgfqpoint{1.062045in}{2.536135in}}%
\pgfpathlineto{\pgfqpoint{1.062997in}{2.540227in}}%
\pgfpathlineto{\pgfqpoint{1.063315in}{2.548226in}}%
\pgfpathlineto{\pgfqpoint{1.063949in}{2.544896in}}%
\pgfpathlineto{\pgfqpoint{1.066172in}{2.531602in}}%
\pgfpathlineto{\pgfqpoint{1.066489in}{2.533368in}}%
\pgfpathlineto{\pgfqpoint{1.067759in}{2.535250in}}%
\pgfpathlineto{\pgfqpoint{1.068076in}{2.535095in}}%
\pgfpathlineto{\pgfqpoint{1.068394in}{2.533363in}}%
\pgfpathlineto{\pgfqpoint{1.069346in}{2.533656in}}%
\pgfpathlineto{\pgfqpoint{1.070616in}{2.545923in}}%
\pgfpathlineto{\pgfqpoint{1.072521in}{2.527413in}}%
\pgfpathlineto{\pgfqpoint{1.074108in}{2.532580in}}%
\pgfpathlineto{\pgfqpoint{1.075695in}{2.530841in}}%
\pgfpathlineto{\pgfqpoint{1.076330in}{2.532298in}}%
\pgfpathlineto{\pgfqpoint{1.076648in}{2.533065in}}%
\pgfpathlineto{\pgfqpoint{1.076965in}{2.541226in}}%
\pgfpathlineto{\pgfqpoint{1.077918in}{2.540781in}}%
\pgfpathlineto{\pgfqpoint{1.079822in}{2.524710in}}%
\pgfpathlineto{\pgfqpoint{1.081092in}{2.529796in}}%
\pgfpathlineto{\pgfqpoint{1.081410in}{2.530129in}}%
\pgfpathlineto{\pgfqpoint{1.081727in}{2.529733in}}%
\pgfpathlineto{\pgfqpoint{1.082997in}{2.528496in}}%
\pgfpathlineto{\pgfqpoint{1.084267in}{2.540133in}}%
\pgfpathlineto{\pgfqpoint{1.086172in}{2.524001in}}%
\pgfpathlineto{\pgfqpoint{1.086489in}{2.525880in}}%
\pgfpathlineto{\pgfqpoint{1.086806in}{2.525243in}}%
\pgfpathlineto{\pgfqpoint{1.087124in}{2.522907in}}%
\pgfpathlineto{\pgfqpoint{1.087441in}{2.526943in}}%
\pgfpathlineto{\pgfqpoint{1.089981in}{2.527356in}}%
\pgfpathlineto{\pgfqpoint{1.091568in}{2.535233in}}%
\pgfpathlineto{\pgfqpoint{1.093473in}{2.520233in}}%
\pgfpathlineto{\pgfqpoint{1.095060in}{2.525027in}}%
\pgfpathlineto{\pgfqpoint{1.096330in}{2.524239in}}%
\pgfpathlineto{\pgfqpoint{1.096648in}{2.524046in}}%
\pgfpathlineto{\pgfqpoint{1.097917in}{2.533863in}}%
\pgfpathlineto{\pgfqpoint{1.098552in}{2.530255in}}%
\pgfpathlineto{\pgfqpoint{1.100775in}{2.518125in}}%
\pgfpathlineto{\pgfqpoint{1.102044in}{2.522804in}}%
\pgfpathlineto{\pgfqpoint{1.102362in}{2.522661in}}%
\pgfpathlineto{\pgfqpoint{1.102679in}{2.521294in}}%
\pgfpathlineto{\pgfqpoint{1.103632in}{2.521737in}}%
\pgfpathlineto{\pgfqpoint{1.104902in}{2.532997in}}%
\pgfpathlineto{\pgfqpoint{1.105219in}{2.529218in}}%
\pgfpathlineto{\pgfqpoint{1.107124in}{2.516772in}}%
\pgfpathlineto{\pgfqpoint{1.108076in}{2.517086in}}%
\pgfpathlineto{\pgfqpoint{1.109346in}{2.520304in}}%
\pgfpathlineto{\pgfqpoint{1.110933in}{2.519253in}}%
\pgfpathlineto{\pgfqpoint{1.111251in}{2.528771in}}%
\pgfpathlineto{\pgfqpoint{1.112203in}{2.528611in}}%
\pgfpathlineto{\pgfqpoint{1.114108in}{2.513536in}}%
\pgfpathlineto{\pgfqpoint{1.115695in}{2.517767in}}%
\pgfpathlineto{\pgfqpoint{1.116013in}{2.517679in}}%
\pgfpathlineto{\pgfqpoint{1.117282in}{2.515887in}}%
\pgfpathlineto{\pgfqpoint{1.118235in}{2.519833in}}%
\pgfpathlineto{\pgfqpoint{1.118552in}{2.529557in}}%
\pgfpathlineto{\pgfqpoint{1.119187in}{2.524013in}}%
\pgfpathlineto{\pgfqpoint{1.120457in}{2.510780in}}%
\pgfpathlineto{\pgfqpoint{1.121409in}{2.511446in}}%
\pgfpathlineto{\pgfqpoint{1.122044in}{2.514827in}}%
\pgfpathlineto{\pgfqpoint{1.122679in}{2.514315in}}%
\pgfpathlineto{\pgfqpoint{1.122997in}{2.515141in}}%
\pgfpathlineto{\pgfqpoint{1.123314in}{2.514997in}}%
\pgfpathlineto{\pgfqpoint{1.124584in}{2.513448in}}%
\pgfpathlineto{\pgfqpoint{1.125854in}{2.526369in}}%
\pgfpathlineto{\pgfqpoint{1.127758in}{2.507020in}}%
\pgfpathlineto{\pgfqpoint{1.129028in}{2.512371in}}%
\pgfpathlineto{\pgfqpoint{1.130616in}{2.512257in}}%
\pgfpathlineto{\pgfqpoint{1.130933in}{2.511022in}}%
\pgfpathlineto{\pgfqpoint{1.131250in}{2.513294in}}%
\pgfpathlineto{\pgfqpoint{1.131568in}{2.513029in}}%
\pgfpathlineto{\pgfqpoint{1.131885in}{2.513821in}}%
\pgfpathlineto{\pgfqpoint{1.132203in}{2.522238in}}%
\pgfpathlineto{\pgfqpoint{1.133155in}{2.520418in}}%
\pgfpathlineto{\pgfqpoint{1.135060in}{2.504773in}}%
\pgfpathlineto{\pgfqpoint{1.136330in}{2.510137in}}%
\pgfpathlineto{\pgfqpoint{1.138235in}{2.510559in}}%
\pgfpathlineto{\pgfqpoint{1.139504in}{2.519505in}}%
\pgfpathlineto{\pgfqpoint{1.141409in}{2.503417in}}%
\pgfpathlineto{\pgfqpoint{1.142996in}{2.507411in}}%
\pgfpathlineto{\pgfqpoint{1.143314in}{2.506152in}}%
\pgfpathlineto{\pgfqpoint{1.144266in}{2.506447in}}%
\pgfpathlineto{\pgfqpoint{1.145219in}{2.508313in}}%
\pgfpathlineto{\pgfqpoint{1.145854in}{2.516243in}}%
\pgfpathlineto{\pgfqpoint{1.146488in}{2.514075in}}%
\pgfpathlineto{\pgfqpoint{1.148711in}{2.500443in}}%
\pgfpathlineto{\pgfqpoint{1.149346in}{2.503389in}}%
\pgfpathlineto{\pgfqpoint{1.149663in}{2.502116in}}%
\pgfpathlineto{\pgfqpoint{1.149980in}{2.505109in}}%
\pgfpathlineto{\pgfqpoint{1.150298in}{2.505104in}}%
\pgfpathlineto{\pgfqpoint{1.150615in}{2.503579in}}%
\pgfpathlineto{\pgfqpoint{1.151568in}{2.503851in}}%
\pgfpathlineto{\pgfqpoint{1.152838in}{2.515616in}}%
\pgfpathlineto{\pgfqpoint{1.153155in}{2.511945in}}%
\pgfpathlineto{\pgfqpoint{1.156012in}{2.499303in}}%
\pgfpathlineto{\pgfqpoint{1.157282in}{2.502682in}}%
\pgfpathlineto{\pgfqpoint{1.157599in}{2.502523in}}%
\pgfpathlineto{\pgfqpoint{1.157917in}{2.500782in}}%
\pgfpathlineto{\pgfqpoint{1.158869in}{2.501238in}}%
\pgfpathlineto{\pgfqpoint{1.160139in}{2.511292in}}%
\pgfpathlineto{\pgfqpoint{1.162044in}{2.495609in}}%
\pgfpathlineto{\pgfqpoint{1.163631in}{2.500024in}}%
\pgfpathlineto{\pgfqpoint{1.165218in}{2.498074in}}%
\pgfpathlineto{\pgfqpoint{1.165853in}{2.499990in}}%
\pgfpathlineto{\pgfqpoint{1.166171in}{2.500791in}}%
\pgfpathlineto{\pgfqpoint{1.166488in}{2.512017in}}%
\pgfpathlineto{\pgfqpoint{1.167123in}{2.506067in}}%
\pgfpathlineto{\pgfqpoint{1.168076in}{2.498204in}}%
\pgfpathlineto{\pgfqpoint{1.168393in}{2.493029in}}%
\pgfpathlineto{\pgfqpoint{1.169345in}{2.493464in}}%
\pgfpathlineto{\pgfqpoint{1.169980in}{2.497096in}}%
\pgfpathlineto{\pgfqpoint{1.170615in}{2.496887in}}%
\pgfpathlineto{\pgfqpoint{1.170933in}{2.497396in}}%
\pgfpathlineto{\pgfqpoint{1.171250in}{2.497157in}}%
\pgfpathlineto{\pgfqpoint{1.172520in}{2.495625in}}%
\pgfpathlineto{\pgfqpoint{1.173790in}{2.509411in}}%
\pgfpathlineto{\pgfqpoint{1.175695in}{2.489329in}}%
\pgfpathlineto{\pgfqpoint{1.177282in}{2.494912in}}%
\pgfpathlineto{\pgfqpoint{1.178869in}{2.493438in}}%
\pgfpathlineto{\pgfqpoint{1.180139in}{2.504655in}}%
\pgfpathlineto{\pgfqpoint{1.181091in}{2.503602in}}%
\pgfpathlineto{\pgfqpoint{1.182996in}{2.486845in}}%
\pgfpathlineto{\pgfqpoint{1.184266in}{2.492474in}}%
\pgfpathlineto{\pgfqpoint{1.185853in}{2.491700in}}%
\pgfpathlineto{\pgfqpoint{1.187441in}{2.501788in}}%
\pgfpathlineto{\pgfqpoint{1.188075in}{2.496188in}}%
\pgfpathlineto{\pgfqpoint{1.190298in}{2.485337in}}%
\pgfpathlineto{\pgfqpoint{1.190615in}{2.489185in}}%
\pgfpathlineto{\pgfqpoint{1.190933in}{2.489561in}}%
\pgfpathlineto{\pgfqpoint{1.191567in}{2.489740in}}%
\pgfpathlineto{\pgfqpoint{1.192202in}{2.488301in}}%
\pgfpathlineto{\pgfqpoint{1.193155in}{2.490528in}}%
\pgfpathlineto{\pgfqpoint{1.193472in}{2.498057in}}%
\pgfpathlineto{\pgfqpoint{1.194425in}{2.496351in}}%
\pgfpathlineto{\pgfqpoint{1.196647in}{2.482879in}}%
\pgfpathlineto{\pgfqpoint{1.196964in}{2.485367in}}%
\pgfpathlineto{\pgfqpoint{1.197282in}{2.485501in}}%
\pgfpathlineto{\pgfqpoint{1.197599in}{2.484284in}}%
\pgfpathlineto{\pgfqpoint{1.197917in}{2.487115in}}%
\pgfpathlineto{\pgfqpoint{1.198234in}{2.487204in}}%
\pgfpathlineto{\pgfqpoint{1.199504in}{2.485640in}}%
\pgfpathlineto{\pgfqpoint{1.200774in}{2.498144in}}%
\pgfpathlineto{\pgfqpoint{1.201091in}{2.493810in}}%
\pgfpathlineto{\pgfqpoint{1.203631in}{2.481381in}}%
\pgfpathlineto{\pgfqpoint{1.205536in}{2.484559in}}%
\pgfpathlineto{\pgfqpoint{1.205853in}{2.482779in}}%
\pgfpathlineto{\pgfqpoint{1.206805in}{2.483010in}}%
\pgfpathlineto{\pgfqpoint{1.208075in}{2.493978in}}%
\pgfpathlineto{\pgfqpoint{1.209980in}{2.477322in}}%
\pgfpathlineto{\pgfqpoint{1.211567in}{2.481989in}}%
\pgfpathlineto{\pgfqpoint{1.213155in}{2.480050in}}%
\pgfpathlineto{\pgfqpoint{1.214107in}{2.481925in}}%
\pgfpathlineto{\pgfqpoint{1.214424in}{2.493729in}}%
\pgfpathlineto{\pgfqpoint{1.215059in}{2.487713in}}%
\pgfpathlineto{\pgfqpoint{1.215377in}{2.487775in}}%
\pgfpathlineto{\pgfqpoint{1.216329in}{2.475220in}}%
\pgfpathlineto{\pgfqpoint{1.216964in}{2.477613in}}%
\pgfpathlineto{\pgfqpoint{1.217282in}{2.475311in}}%
\pgfpathlineto{\pgfqpoint{1.217599in}{2.478698in}}%
\pgfpathlineto{\pgfqpoint{1.218234in}{2.478083in}}%
\pgfpathlineto{\pgfqpoint{1.218869in}{2.479299in}}%
\pgfpathlineto{\pgfqpoint{1.220456in}{2.477602in}}%
\pgfpathlineto{\pgfqpoint{1.221726in}{2.491430in}}%
\pgfpathlineto{\pgfqpoint{1.223631in}{2.471459in}}%
\pgfpathlineto{\pgfqpoint{1.225218in}{2.476858in}}%
\pgfpathlineto{\pgfqpoint{1.225535in}{2.475639in}}%
\pgfpathlineto{\pgfqpoint{1.226488in}{2.475906in}}%
\pgfpathlineto{\pgfqpoint{1.226805in}{2.475597in}}%
\pgfpathlineto{\pgfqpoint{1.227123in}{2.476807in}}%
\pgfpathlineto{\pgfqpoint{1.227440in}{2.476636in}}%
\pgfpathlineto{\pgfqpoint{1.228075in}{2.486272in}}%
\pgfpathlineto{\pgfqpoint{1.229027in}{2.485283in}}%
\pgfpathlineto{\pgfqpoint{1.230932in}{2.468710in}}%
\pgfpathlineto{\pgfqpoint{1.232202in}{2.474353in}}%
\pgfpathlineto{\pgfqpoint{1.232519in}{2.474191in}}%
\pgfpathlineto{\pgfqpoint{1.232837in}{2.472732in}}%
\pgfpathlineto{\pgfqpoint{1.233789in}{2.473233in}}%
\pgfpathlineto{\pgfqpoint{1.235377in}{2.483165in}}%
\pgfpathlineto{\pgfqpoint{1.237281in}{2.468261in}}%
\pgfpathlineto{\pgfqpoint{1.237916in}{2.469187in}}%
\pgfpathlineto{\pgfqpoint{1.238234in}{2.467125in}}%
\pgfpathlineto{\pgfqpoint{1.238551in}{2.470668in}}%
\pgfpathlineto{\pgfqpoint{1.239504in}{2.471591in}}%
\pgfpathlineto{\pgfqpoint{1.239186in}{2.469751in}}%
\pgfpathlineto{\pgfqpoint{1.239821in}{2.471360in}}%
\pgfpathlineto{\pgfqpoint{1.240138in}{2.469815in}}%
\pgfpathlineto{\pgfqpoint{1.240773in}{2.471472in}}%
\pgfpathlineto{\pgfqpoint{1.241091in}{2.471562in}}%
\pgfpathlineto{\pgfqpoint{1.241408in}{2.480087in}}%
\pgfpathlineto{\pgfqpoint{1.242361in}{2.478440in}}%
\pgfpathlineto{\pgfqpoint{1.244583in}{2.465020in}}%
\pgfpathlineto{\pgfqpoint{1.246170in}{2.468827in}}%
\pgfpathlineto{\pgfqpoint{1.247440in}{2.467048in}}%
\pgfpathlineto{\pgfqpoint{1.248710in}{2.480410in}}%
\pgfpathlineto{\pgfqpoint{1.249345in}{2.473401in}}%
\pgfpathlineto{\pgfqpoint{1.251567in}{2.462558in}}%
\pgfpathlineto{\pgfqpoint{1.253154in}{2.466260in}}%
\pgfpathlineto{\pgfqpoint{1.253472in}{2.466154in}}%
\pgfpathlineto{\pgfqpoint{1.254742in}{2.464430in}}%
\pgfpathlineto{\pgfqpoint{1.256011in}{2.477514in}}%
\pgfpathlineto{\pgfqpoint{1.257916in}{2.458740in}}%
\pgfpathlineto{\pgfqpoint{1.259503in}{2.463547in}}%
\pgfpathlineto{\pgfqpoint{1.261091in}{2.461799in}}%
\pgfpathlineto{\pgfqpoint{1.261726in}{2.463294in}}%
\pgfpathlineto{\pgfqpoint{1.262043in}{2.462242in}}%
\pgfpathlineto{\pgfqpoint{1.262361in}{2.473605in}}%
\pgfpathlineto{\pgfqpoint{1.263313in}{2.472163in}}%
\pgfpathlineto{\pgfqpoint{1.264265in}{2.458281in}}%
\pgfpathlineto{\pgfqpoint{1.264900in}{2.459080in}}%
\pgfpathlineto{\pgfqpoint{1.265218in}{2.456345in}}%
\pgfpathlineto{\pgfqpoint{1.265535in}{2.459849in}}%
\pgfpathlineto{\pgfqpoint{1.266170in}{2.459220in}}%
\pgfpathlineto{\pgfqpoint{1.266805in}{2.460912in}}%
\pgfpathlineto{\pgfqpoint{1.268392in}{2.459290in}}%
\pgfpathlineto{\pgfqpoint{1.269662in}{2.472902in}}%
\pgfpathlineto{\pgfqpoint{1.269979in}{2.467812in}}%
\pgfpathlineto{\pgfqpoint{1.271567in}{2.453946in}}%
\pgfpathlineto{\pgfqpoint{1.272519in}{2.454369in}}%
\pgfpathlineto{\pgfqpoint{1.273154in}{2.458303in}}%
\pgfpathlineto{\pgfqpoint{1.273789in}{2.458206in}}%
\pgfpathlineto{\pgfqpoint{1.275376in}{2.457651in}}%
\pgfpathlineto{\pgfqpoint{1.276964in}{2.467947in}}%
\pgfpathlineto{\pgfqpoint{1.277281in}{2.462381in}}%
\pgfpathlineto{\pgfqpoint{1.278868in}{2.450445in}}%
\pgfpathlineto{\pgfqpoint{1.279186in}{2.453361in}}%
\pgfpathlineto{\pgfqpoint{1.279821in}{2.452635in}}%
\pgfpathlineto{\pgfqpoint{1.280456in}{2.455722in}}%
\pgfpathlineto{\pgfqpoint{1.280773in}{2.454078in}}%
\pgfpathlineto{\pgfqpoint{1.281725in}{2.454274in}}%
\pgfpathlineto{\pgfqpoint{1.283313in}{2.464374in}}%
\pgfpathlineto{\pgfqpoint{1.283948in}{2.460801in}}%
\pgfpathlineto{\pgfqpoint{1.284900in}{2.453538in}}%
\pgfpathlineto{\pgfqpoint{1.286170in}{2.448418in}}%
\pgfpathlineto{\pgfqpoint{1.287440in}{2.453015in}}%
\pgfpathlineto{\pgfqpoint{1.287757in}{2.452907in}}%
\pgfpathlineto{\pgfqpoint{1.288075in}{2.451158in}}%
\pgfpathlineto{\pgfqpoint{1.289027in}{2.451512in}}%
\pgfpathlineto{\pgfqpoint{1.290297in}{2.461620in}}%
\pgfpathlineto{\pgfqpoint{1.292519in}{2.446869in}}%
\pgfpathlineto{\pgfqpoint{1.294106in}{2.449812in}}%
\pgfpathlineto{\pgfqpoint{1.294741in}{2.450191in}}%
\pgfpathlineto{\pgfqpoint{1.295376in}{2.448254in}}%
\pgfpathlineto{\pgfqpoint{1.296646in}{2.460468in}}%
\pgfpathlineto{\pgfqpoint{1.297598in}{2.456328in}}%
\pgfpathlineto{\pgfqpoint{1.299503in}{2.443350in}}%
\pgfpathlineto{\pgfqpoint{1.301090in}{2.447420in}}%
\pgfpathlineto{\pgfqpoint{1.301408in}{2.447245in}}%
\pgfpathlineto{\pgfqpoint{1.302678in}{2.445520in}}%
\pgfpathlineto{\pgfqpoint{1.303947in}{2.460207in}}%
\pgfpathlineto{\pgfqpoint{1.304582in}{2.451621in}}%
\pgfpathlineto{\pgfqpoint{1.305535in}{2.444717in}}%
\pgfpathlineto{\pgfqpoint{1.305852in}{2.440673in}}%
\pgfpathlineto{\pgfqpoint{1.306805in}{2.441100in}}%
\pgfpathlineto{\pgfqpoint{1.307439in}{2.444491in}}%
\pgfpathlineto{\pgfqpoint{1.308074in}{2.444324in}}%
\pgfpathlineto{\pgfqpoint{1.308392in}{2.444715in}}%
\pgfpathlineto{\pgfqpoint{1.308709in}{2.444352in}}%
\pgfpathlineto{\pgfqpoint{1.309979in}{2.442931in}}%
\pgfpathlineto{\pgfqpoint{1.311249in}{2.457067in}}%
\pgfpathlineto{\pgfqpoint{1.313154in}{2.437177in}}%
\pgfpathlineto{\pgfqpoint{1.314741in}{2.442175in}}%
\pgfpathlineto{\pgfqpoint{1.315058in}{2.441020in}}%
\pgfpathlineto{\pgfqpoint{1.316011in}{2.441311in}}%
\pgfpathlineto{\pgfqpoint{1.316328in}{2.440684in}}%
\pgfpathlineto{\pgfqpoint{1.316646in}{2.441709in}}%
\pgfpathlineto{\pgfqpoint{1.316963in}{2.441654in}}%
\pgfpathlineto{\pgfqpoint{1.317281in}{2.441452in}}%
\pgfpathlineto{\pgfqpoint{1.317598in}{2.451607in}}%
\pgfpathlineto{\pgfqpoint{1.318551in}{2.451443in}}%
\pgfpathlineto{\pgfqpoint{1.319820in}{2.437084in}}%
\pgfpathlineto{\pgfqpoint{1.320138in}{2.437340in}}%
\pgfpathlineto{\pgfqpoint{1.320455in}{2.434442in}}%
\pgfpathlineto{\pgfqpoint{1.320773in}{2.437917in}}%
\pgfpathlineto{\pgfqpoint{1.321725in}{2.439561in}}%
\pgfpathlineto{\pgfqpoint{1.321408in}{2.437348in}}%
\pgfpathlineto{\pgfqpoint{1.322043in}{2.439404in}}%
\pgfpathlineto{\pgfqpoint{1.322360in}{2.438044in}}%
\pgfpathlineto{\pgfqpoint{1.323312in}{2.438440in}}%
\pgfpathlineto{\pgfqpoint{1.323630in}{2.438217in}}%
\pgfpathlineto{\pgfqpoint{1.324900in}{2.450292in}}%
\pgfpathlineto{\pgfqpoint{1.325217in}{2.445592in}}%
\pgfpathlineto{\pgfqpoint{1.327757in}{2.432413in}}%
\pgfpathlineto{\pgfqpoint{1.329027in}{2.436718in}}%
\pgfpathlineto{\pgfqpoint{1.330614in}{2.435933in}}%
\pgfpathlineto{\pgfqpoint{1.332201in}{2.444434in}}%
\pgfpathlineto{\pgfqpoint{1.334106in}{2.429647in}}%
\pgfpathlineto{\pgfqpoint{1.335693in}{2.433930in}}%
\pgfpathlineto{\pgfqpoint{1.336963in}{2.432135in}}%
\pgfpathlineto{\pgfqpoint{1.338233in}{2.443108in}}%
\pgfpathlineto{\pgfqpoint{1.339185in}{2.438901in}}%
\pgfpathlineto{\pgfqpoint{1.341407in}{2.427266in}}%
\pgfpathlineto{\pgfqpoint{1.342995in}{2.431161in}}%
\pgfpathlineto{\pgfqpoint{1.344265in}{2.429358in}}%
\pgfpathlineto{\pgfqpoint{1.345534in}{2.441875in}}%
\pgfpathlineto{\pgfqpoint{1.345852in}{2.438157in}}%
\pgfpathlineto{\pgfqpoint{1.347439in}{2.425248in}}%
\pgfpathlineto{\pgfqpoint{1.347757in}{2.425763in}}%
\pgfpathlineto{\pgfqpoint{1.349026in}{2.427640in}}%
\pgfpathlineto{\pgfqpoint{1.348392in}{2.424874in}}%
\pgfpathlineto{\pgfqpoint{1.349344in}{2.427320in}}%
\pgfpathlineto{\pgfqpoint{1.351566in}{2.426568in}}%
\pgfpathlineto{\pgfqpoint{1.352836in}{2.438098in}}%
\pgfpathlineto{\pgfqpoint{1.354741in}{2.421209in}}%
\pgfpathlineto{\pgfqpoint{1.356328in}{2.425529in}}%
\pgfpathlineto{\pgfqpoint{1.357915in}{2.423871in}}%
\pgfpathlineto{\pgfqpoint{1.358550in}{2.425784in}}%
\pgfpathlineto{\pgfqpoint{1.358868in}{2.424822in}}%
\pgfpathlineto{\pgfqpoint{1.359185in}{2.436895in}}%
\pgfpathlineto{\pgfqpoint{1.360137in}{2.432928in}}%
\pgfpathlineto{\pgfqpoint{1.361407in}{2.420186in}}%
\pgfpathlineto{\pgfqpoint{1.361725in}{2.421191in}}%
\pgfpathlineto{\pgfqpoint{1.362042in}{2.418527in}}%
\pgfpathlineto{\pgfqpoint{1.362360in}{2.421571in}}%
\pgfpathlineto{\pgfqpoint{1.362995in}{2.421055in}}%
\pgfpathlineto{\pgfqpoint{1.363630in}{2.422929in}}%
\pgfpathlineto{\pgfqpoint{1.365217in}{2.421271in}}%
\pgfpathlineto{\pgfqpoint{1.366487in}{2.435502in}}%
\pgfpathlineto{\pgfqpoint{1.366804in}{2.430033in}}%
\pgfpathlineto{\pgfqpoint{1.368391in}{2.416144in}}%
\pgfpathlineto{\pgfqpoint{1.368709in}{2.417296in}}%
\pgfpathlineto{\pgfqpoint{1.369344in}{2.416394in}}%
\pgfpathlineto{\pgfqpoint{1.369979in}{2.420194in}}%
\pgfpathlineto{\pgfqpoint{1.370296in}{2.418644in}}%
\pgfpathlineto{\pgfqpoint{1.371248in}{2.418937in}}%
\pgfpathlineto{\pgfqpoint{1.371883in}{2.419715in}}%
\pgfpathlineto{\pgfqpoint{1.372518in}{2.418844in}}%
\pgfpathlineto{\pgfqpoint{1.373788in}{2.431296in}}%
\pgfpathlineto{\pgfqpoint{1.375693in}{2.412674in}}%
\pgfpathlineto{\pgfqpoint{1.377280in}{2.417520in}}%
\pgfpathlineto{\pgfqpoint{1.378550in}{2.415924in}}%
\pgfpathlineto{\pgfqpoint{1.379820in}{2.421245in}}%
\pgfpathlineto{\pgfqpoint{1.380137in}{2.426196in}}%
\pgfpathlineto{\pgfqpoint{1.381090in}{2.424925in}}%
\pgfpathlineto{\pgfqpoint{1.382994in}{2.409813in}}%
\pgfpathlineto{\pgfqpoint{1.384582in}{2.414690in}}%
\pgfpathlineto{\pgfqpoint{1.385852in}{2.413088in}}%
\pgfpathlineto{\pgfqpoint{1.387121in}{2.423113in}}%
\pgfpathlineto{\pgfqpoint{1.387439in}{2.422987in}}%
\pgfpathlineto{\pgfqpoint{1.389344in}{2.409413in}}%
\pgfpathlineto{\pgfqpoint{1.389661in}{2.409675in}}%
\pgfpathlineto{\pgfqpoint{1.389978in}{2.408935in}}%
\pgfpathlineto{\pgfqpoint{1.390296in}{2.407927in}}%
\pgfpathlineto{\pgfqpoint{1.391566in}{2.411849in}}%
\pgfpathlineto{\pgfqpoint{1.391883in}{2.411750in}}%
\pgfpathlineto{\pgfqpoint{1.392201in}{2.410003in}}%
\pgfpathlineto{\pgfqpoint{1.393153in}{2.410186in}}%
\pgfpathlineto{\pgfqpoint{1.394423in}{2.420463in}}%
\pgfpathlineto{\pgfqpoint{1.396645in}{2.405849in}}%
\pgfpathlineto{\pgfqpoint{1.398232in}{2.408488in}}%
\pgfpathlineto{\pgfqpoint{1.398867in}{2.408918in}}%
\pgfpathlineto{\pgfqpoint{1.399502in}{2.407072in}}%
\pgfpathlineto{\pgfqpoint{1.401090in}{2.418190in}}%
\pgfpathlineto{\pgfqpoint{1.401724in}{2.416241in}}%
\pgfpathlineto{\pgfqpoint{1.403629in}{2.402094in}}%
\pgfpathlineto{\pgfqpoint{1.405216in}{2.406035in}}%
\pgfpathlineto{\pgfqpoint{1.405534in}{2.405732in}}%
\pgfpathlineto{\pgfqpoint{1.406804in}{2.404283in}}%
\pgfpathlineto{\pgfqpoint{1.408074in}{2.418107in}}%
\pgfpathlineto{\pgfqpoint{1.408708in}{2.410386in}}%
\pgfpathlineto{\pgfqpoint{1.409026in}{2.410781in}}%
\pgfpathlineto{\pgfqpoint{1.409978in}{2.400407in}}%
\pgfpathlineto{\pgfqpoint{1.410613in}{2.401946in}}%
\pgfpathlineto{\pgfqpoint{1.410931in}{2.399387in}}%
\pgfpathlineto{\pgfqpoint{1.411566in}{2.402991in}}%
\pgfpathlineto{\pgfqpoint{1.411883in}{2.401511in}}%
\pgfpathlineto{\pgfqpoint{1.412518in}{2.403358in}}%
\pgfpathlineto{\pgfqpoint{1.414105in}{2.401590in}}%
\pgfpathlineto{\pgfqpoint{1.415375in}{2.416066in}}%
\pgfpathlineto{\pgfqpoint{1.415693in}{2.410107in}}%
\pgfpathlineto{\pgfqpoint{1.417280in}{2.396445in}}%
\pgfpathlineto{\pgfqpoint{1.417597in}{2.397784in}}%
\pgfpathlineto{\pgfqpoint{1.418232in}{2.397208in}}%
\pgfpathlineto{\pgfqpoint{1.418867in}{2.400664in}}%
\pgfpathlineto{\pgfqpoint{1.419185in}{2.399243in}}%
\pgfpathlineto{\pgfqpoint{1.420137in}{2.399710in}}%
\pgfpathlineto{\pgfqpoint{1.421407in}{2.398887in}}%
\pgfpathlineto{\pgfqpoint{1.422677in}{2.411995in}}%
\pgfpathlineto{\pgfqpoint{1.424581in}{2.393085in}}%
\pgfpathlineto{\pgfqpoint{1.426169in}{2.397970in}}%
\pgfpathlineto{\pgfqpoint{1.427438in}{2.396598in}}%
\pgfpathlineto{\pgfqpoint{1.428708in}{2.398566in}}%
\pgfpathlineto{\pgfqpoint{1.429026in}{2.407072in}}%
\pgfpathlineto{\pgfqpoint{1.429978in}{2.406390in}}%
\pgfpathlineto{\pgfqpoint{1.431883in}{2.390364in}}%
\pgfpathlineto{\pgfqpoint{1.433470in}{2.395086in}}%
\pgfpathlineto{\pgfqpoint{1.433788in}{2.393452in}}%
\pgfpathlineto{\pgfqpoint{1.434740in}{2.393662in}}%
\pgfpathlineto{\pgfqpoint{1.435057in}{2.394290in}}%
\pgfpathlineto{\pgfqpoint{1.436327in}{2.404853in}}%
\pgfpathlineto{\pgfqpoint{1.436645in}{2.401173in}}%
\pgfpathlineto{\pgfqpoint{1.437915in}{2.392496in}}%
\pgfpathlineto{\pgfqpoint{1.439184in}{2.387960in}}%
\pgfpathlineto{\pgfqpoint{1.440454in}{2.392284in}}%
\pgfpathlineto{\pgfqpoint{1.440772in}{2.392129in}}%
\pgfpathlineto{\pgfqpoint{1.441089in}{2.390393in}}%
\pgfpathlineto{\pgfqpoint{1.442042in}{2.390754in}}%
\pgfpathlineto{\pgfqpoint{1.443311in}{2.400168in}}%
\pgfpathlineto{\pgfqpoint{1.443629in}{2.399789in}}%
\pgfpathlineto{\pgfqpoint{1.445534in}{2.385868in}}%
\pgfpathlineto{\pgfqpoint{1.445851in}{2.386971in}}%
\pgfpathlineto{\pgfqpoint{1.446486in}{2.386069in}}%
\pgfpathlineto{\pgfqpoint{1.447756in}{2.389293in}}%
\pgfpathlineto{\pgfqpoint{1.448073in}{2.389167in}}%
\pgfpathlineto{\pgfqpoint{1.448391in}{2.387434in}}%
\pgfpathlineto{\pgfqpoint{1.449026in}{2.389575in}}%
\pgfpathlineto{\pgfqpoint{1.449343in}{2.388569in}}%
\pgfpathlineto{\pgfqpoint{1.449978in}{2.398814in}}%
\pgfpathlineto{\pgfqpoint{1.450613in}{2.396635in}}%
\pgfpathlineto{\pgfqpoint{1.452835in}{2.382991in}}%
\pgfpathlineto{\pgfqpoint{1.454422in}{2.386238in}}%
\pgfpathlineto{\pgfqpoint{1.455692in}{2.384516in}}%
\pgfpathlineto{\pgfqpoint{1.456962in}{2.396386in}}%
\pgfpathlineto{\pgfqpoint{1.457914in}{2.391915in}}%
\pgfpathlineto{\pgfqpoint{1.459819in}{2.379691in}}%
\pgfpathlineto{\pgfqpoint{1.461406in}{2.383560in}}%
\pgfpathlineto{\pgfqpoint{1.461724in}{2.383300in}}%
\pgfpathlineto{\pgfqpoint{1.462994in}{2.381737in}}%
\pgfpathlineto{\pgfqpoint{1.464264in}{2.395275in}}%
\pgfpathlineto{\pgfqpoint{1.464581in}{2.391420in}}%
\pgfpathlineto{\pgfqpoint{1.466168in}{2.377509in}}%
\pgfpathlineto{\pgfqpoint{1.466486in}{2.377838in}}%
\pgfpathlineto{\pgfqpoint{1.467121in}{2.377000in}}%
\pgfpathlineto{\pgfqpoint{1.467756in}{2.380293in}}%
\pgfpathlineto{\pgfqpoint{1.468073in}{2.379114in}}%
\pgfpathlineto{\pgfqpoint{1.468708in}{2.380736in}}%
\pgfpathlineto{\pgfqpoint{1.470295in}{2.378977in}}%
\pgfpathlineto{\pgfqpoint{1.471565in}{2.392496in}}%
\pgfpathlineto{\pgfqpoint{1.473470in}{2.373822in}}%
\pgfpathlineto{\pgfqpoint{1.475057in}{2.378003in}}%
\pgfpathlineto{\pgfqpoint{1.475375in}{2.376626in}}%
\pgfpathlineto{\pgfqpoint{1.476327in}{2.377152in}}%
\pgfpathlineto{\pgfqpoint{1.477597in}{2.376236in}}%
\pgfpathlineto{\pgfqpoint{1.478867in}{2.388630in}}%
\pgfpathlineto{\pgfqpoint{1.480771in}{2.370571in}}%
\pgfpathlineto{\pgfqpoint{1.482359in}{2.375277in}}%
\pgfpathlineto{\pgfqpoint{1.482676in}{2.373811in}}%
\pgfpathlineto{\pgfqpoint{1.483628in}{2.374021in}}%
\pgfpathlineto{\pgfqpoint{1.484581in}{2.374698in}}%
\pgfpathlineto{\pgfqpoint{1.484898in}{2.375353in}}%
\pgfpathlineto{\pgfqpoint{1.485216in}{2.384950in}}%
\pgfpathlineto{\pgfqpoint{1.486168in}{2.383487in}}%
\pgfpathlineto{\pgfqpoint{1.487438in}{2.369786in}}%
\pgfpathlineto{\pgfqpoint{1.487755in}{2.370286in}}%
\pgfpathlineto{\pgfqpoint{1.488073in}{2.367865in}}%
\pgfpathlineto{\pgfqpoint{1.488390in}{2.370523in}}%
\pgfpathlineto{\pgfqpoint{1.489025in}{2.370162in}}%
\pgfpathlineto{\pgfqpoint{1.489660in}{2.372374in}}%
\pgfpathlineto{\pgfqpoint{1.489978in}{2.370800in}}%
\pgfpathlineto{\pgfqpoint{1.490930in}{2.370999in}}%
\pgfpathlineto{\pgfqpoint{1.491247in}{2.371332in}}%
\pgfpathlineto{\pgfqpoint{1.492517in}{2.382723in}}%
\pgfpathlineto{\pgfqpoint{1.493152in}{2.377021in}}%
\pgfpathlineto{\pgfqpoint{1.493470in}{2.377861in}}%
\pgfpathlineto{\pgfqpoint{1.494740in}{2.366717in}}%
\pgfpathlineto{\pgfqpoint{1.495057in}{2.367230in}}%
\pgfpathlineto{\pgfqpoint{1.495374in}{2.365299in}}%
\pgfpathlineto{\pgfqpoint{1.495692in}{2.368202in}}%
\pgfpathlineto{\pgfqpoint{1.496644in}{2.369561in}}%
\pgfpathlineto{\pgfqpoint{1.496327in}{2.367674in}}%
\pgfpathlineto{\pgfqpoint{1.496962in}{2.369406in}}%
\pgfpathlineto{\pgfqpoint{1.497279in}{2.367715in}}%
\pgfpathlineto{\pgfqpoint{1.498232in}{2.368069in}}%
\pgfpathlineto{\pgfqpoint{1.499819in}{2.378489in}}%
\pgfpathlineto{\pgfqpoint{1.500136in}{2.374730in}}%
\pgfpathlineto{\pgfqpoint{1.502676in}{2.363096in}}%
\pgfpathlineto{\pgfqpoint{1.503946in}{2.366569in}}%
\pgfpathlineto{\pgfqpoint{1.504263in}{2.366424in}}%
\pgfpathlineto{\pgfqpoint{1.504581in}{2.364706in}}%
\pgfpathlineto{\pgfqpoint{1.505533in}{2.365197in}}%
\pgfpathlineto{\pgfqpoint{1.506168in}{2.375659in}}%
\pgfpathlineto{\pgfqpoint{1.506803in}{2.374224in}}%
\pgfpathlineto{\pgfqpoint{1.507120in}{2.373474in}}%
\pgfpathlineto{\pgfqpoint{1.509025in}{2.360004in}}%
\pgfpathlineto{\pgfqpoint{1.510612in}{2.363486in}}%
\pgfpathlineto{\pgfqpoint{1.511882in}{2.361721in}}%
\pgfpathlineto{\pgfqpoint{1.513470in}{2.373549in}}%
\pgfpathlineto{\pgfqpoint{1.514104in}{2.370245in}}%
\pgfpathlineto{\pgfqpoint{1.516327in}{2.357357in}}%
\pgfpathlineto{\pgfqpoint{1.517596in}{2.360598in}}%
\pgfpathlineto{\pgfqpoint{1.517914in}{2.360516in}}%
\pgfpathlineto{\pgfqpoint{1.519184in}{2.358844in}}%
\pgfpathlineto{\pgfqpoint{1.520454in}{2.370917in}}%
\pgfpathlineto{\pgfqpoint{1.521406in}{2.365822in}}%
\pgfpathlineto{\pgfqpoint{1.523311in}{2.354002in}}%
\pgfpathlineto{\pgfqpoint{1.524898in}{2.357821in}}%
\pgfpathlineto{\pgfqpoint{1.525215in}{2.357505in}}%
\pgfpathlineto{\pgfqpoint{1.526485in}{2.356015in}}%
\pgfpathlineto{\pgfqpoint{1.527755in}{2.369188in}}%
\pgfpathlineto{\pgfqpoint{1.528073in}{2.365850in}}%
\pgfpathlineto{\pgfqpoint{1.529977in}{2.351951in}}%
\pgfpathlineto{\pgfqpoint{1.530612in}{2.351149in}}%
\pgfpathlineto{\pgfqpoint{1.531247in}{2.354500in}}%
\pgfpathlineto{\pgfqpoint{1.531565in}{2.353298in}}%
\pgfpathlineto{\pgfqpoint{1.532200in}{2.354960in}}%
\pgfpathlineto{\pgfqpoint{1.533787in}{2.353237in}}%
\pgfpathlineto{\pgfqpoint{1.535057in}{2.366600in}}%
\pgfpathlineto{\pgfqpoint{1.536961in}{2.348167in}}%
\pgfpathlineto{\pgfqpoint{1.539184in}{2.352017in}}%
\pgfpathlineto{\pgfqpoint{1.540771in}{2.351619in}}%
\pgfpathlineto{\pgfqpoint{1.541088in}{2.350383in}}%
\pgfpathlineto{\pgfqpoint{1.542358in}{2.363018in}}%
\pgfpathlineto{\pgfqpoint{1.544263in}{2.345003in}}%
\pgfpathlineto{\pgfqpoint{1.545850in}{2.349416in}}%
\pgfpathlineto{\pgfqpoint{1.546168in}{2.347900in}}%
\pgfpathlineto{\pgfqpoint{1.547120in}{2.348110in}}%
\pgfpathlineto{\pgfqpoint{1.548390in}{2.348636in}}%
\pgfpathlineto{\pgfqpoint{1.549660in}{2.358578in}}%
\pgfpathlineto{\pgfqpoint{1.550930in}{2.344126in}}%
\pgfpathlineto{\pgfqpoint{1.551247in}{2.344448in}}%
\pgfpathlineto{\pgfqpoint{1.551564in}{2.341962in}}%
\pgfpathlineto{\pgfqpoint{1.552199in}{2.345669in}}%
\pgfpathlineto{\pgfqpoint{1.552517in}{2.344027in}}%
\pgfpathlineto{\pgfqpoint{1.553152in}{2.346536in}}%
\pgfpathlineto{\pgfqpoint{1.553469in}{2.344916in}}%
\pgfpathlineto{\pgfqpoint{1.553787in}{2.346127in}}%
\pgfpathlineto{\pgfqpoint{1.554739in}{2.345499in}}%
\pgfpathlineto{\pgfqpoint{1.555691in}{2.348332in}}%
\pgfpathlineto{\pgfqpoint{1.556009in}{2.356033in}}%
\pgfpathlineto{\pgfqpoint{1.556961in}{2.353810in}}%
\pgfpathlineto{\pgfqpoint{1.558231in}{2.340985in}}%
\pgfpathlineto{\pgfqpoint{1.558549in}{2.341316in}}%
\pgfpathlineto{\pgfqpoint{1.558866in}{2.339266in}}%
\pgfpathlineto{\pgfqpoint{1.559183in}{2.341793in}}%
\pgfpathlineto{\pgfqpoint{1.559818in}{2.341476in}}%
\pgfpathlineto{\pgfqpoint{1.560453in}{2.343573in}}%
\pgfpathlineto{\pgfqpoint{1.561723in}{2.342000in}}%
\pgfpathlineto{\pgfqpoint{1.562041in}{2.342745in}}%
\pgfpathlineto{\pgfqpoint{1.563310in}{2.353033in}}%
\pgfpathlineto{\pgfqpoint{1.563628in}{2.349700in}}%
\pgfpathlineto{\pgfqpoint{1.566167in}{2.336729in}}%
\pgfpathlineto{\pgfqpoint{1.566485in}{2.339437in}}%
\pgfpathlineto{\pgfqpoint{1.567120in}{2.338920in}}%
\pgfpathlineto{\pgfqpoint{1.567755in}{2.340568in}}%
\pgfpathlineto{\pgfqpoint{1.568072in}{2.338830in}}%
\pgfpathlineto{\pgfqpoint{1.569025in}{2.339024in}}%
\pgfpathlineto{\pgfqpoint{1.570612in}{2.349501in}}%
\pgfpathlineto{\pgfqpoint{1.572517in}{2.334657in}}%
\pgfpathlineto{\pgfqpoint{1.575056in}{2.337548in}}%
\pgfpathlineto{\pgfqpoint{1.575374in}{2.335834in}}%
\pgfpathlineto{\pgfqpoint{1.576326in}{2.336010in}}%
\pgfpathlineto{\pgfqpoint{1.576961in}{2.346812in}}%
\pgfpathlineto{\pgfqpoint{1.577596in}{2.345832in}}%
\pgfpathlineto{\pgfqpoint{1.579183in}{2.335264in}}%
\pgfpathlineto{\pgfqpoint{1.579818in}{2.331518in}}%
\pgfpathlineto{\pgfqpoint{1.580453in}{2.331935in}}%
\pgfpathlineto{\pgfqpoint{1.580771in}{2.332036in}}%
\pgfpathlineto{\pgfqpoint{1.582040in}{2.334636in}}%
\pgfpathlineto{\pgfqpoint{1.582358in}{2.334524in}}%
\pgfpathlineto{\pgfqpoint{1.582675in}{2.332849in}}%
\pgfpathlineto{\pgfqpoint{1.583310in}{2.334860in}}%
\pgfpathlineto{\pgfqpoint{1.583628in}{2.333450in}}%
\pgfpathlineto{\pgfqpoint{1.584263in}{2.344535in}}%
\pgfpathlineto{\pgfqpoint{1.584897in}{2.342648in}}%
\pgfpathlineto{\pgfqpoint{1.586802in}{2.328405in}}%
\pgfpathlineto{\pgfqpoint{1.587120in}{2.328780in}}%
\pgfpathlineto{\pgfqpoint{1.588390in}{2.331509in}}%
\pgfpathlineto{\pgfqpoint{1.588707in}{2.331298in}}%
\pgfpathlineto{\pgfqpoint{1.589977in}{2.329961in}}%
\pgfpathlineto{\pgfqpoint{1.591564in}{2.341349in}}%
\pgfpathlineto{\pgfqpoint{1.592199in}{2.338615in}}%
\pgfpathlineto{\pgfqpoint{1.594104in}{2.324963in}}%
\pgfpathlineto{\pgfqpoint{1.595691in}{2.328748in}}%
\pgfpathlineto{\pgfqpoint{1.596009in}{2.328272in}}%
\pgfpathlineto{\pgfqpoint{1.597278in}{2.327083in}}%
\pgfpathlineto{\pgfqpoint{1.598548in}{2.339771in}}%
\pgfpathlineto{\pgfqpoint{1.599501in}{2.334477in}}%
\pgfpathlineto{\pgfqpoint{1.600770in}{2.323051in}}%
\pgfpathlineto{\pgfqpoint{1.601088in}{2.324591in}}%
\pgfpathlineto{\pgfqpoint{1.601405in}{2.321872in}}%
\pgfpathlineto{\pgfqpoint{1.602040in}{2.325387in}}%
\pgfpathlineto{\pgfqpoint{1.602358in}{2.323840in}}%
\pgfpathlineto{\pgfqpoint{1.602993in}{2.325962in}}%
\pgfpathlineto{\pgfqpoint{1.604580in}{2.324300in}}%
\pgfpathlineto{\pgfqpoint{1.605850in}{2.338005in}}%
\pgfpathlineto{\pgfqpoint{1.606167in}{2.333723in}}%
\pgfpathlineto{\pgfqpoint{1.608072in}{2.319905in}}%
\pgfpathlineto{\pgfqpoint{1.608707in}{2.319169in}}%
\pgfpathlineto{\pgfqpoint{1.609342in}{2.322875in}}%
\pgfpathlineto{\pgfqpoint{1.609659in}{2.321132in}}%
\pgfpathlineto{\pgfqpoint{1.610294in}{2.323052in}}%
\pgfpathlineto{\pgfqpoint{1.610612in}{2.322089in}}%
\pgfpathlineto{\pgfqpoint{1.611564in}{2.322340in}}%
\pgfpathlineto{\pgfqpoint{1.611881in}{2.321493in}}%
\pgfpathlineto{\pgfqpoint{1.613151in}{2.335371in}}%
\pgfpathlineto{\pgfqpoint{1.613469in}{2.329887in}}%
\pgfpathlineto{\pgfqpoint{1.615056in}{2.316816in}}%
\pgfpathlineto{\pgfqpoint{1.615373in}{2.316992in}}%
\pgfpathlineto{\pgfqpoint{1.616008in}{2.316561in}}%
\pgfpathlineto{\pgfqpoint{1.616643in}{2.320181in}}%
\pgfpathlineto{\pgfqpoint{1.616961in}{2.318504in}}%
\pgfpathlineto{\pgfqpoint{1.617913in}{2.318884in}}%
\pgfpathlineto{\pgfqpoint{1.618548in}{2.319630in}}%
\pgfpathlineto{\pgfqpoint{1.618865in}{2.319301in}}%
\pgfpathlineto{\pgfqpoint{1.619183in}{2.318720in}}%
\pgfpathlineto{\pgfqpoint{1.620453in}{2.331891in}}%
\pgfpathlineto{\pgfqpoint{1.622357in}{2.313422in}}%
\pgfpathlineto{\pgfqpoint{1.624580in}{2.317281in}}%
\pgfpathlineto{\pgfqpoint{1.624897in}{2.317056in}}%
\pgfpathlineto{\pgfqpoint{1.625215in}{2.315751in}}%
\pgfpathlineto{\pgfqpoint{1.626167in}{2.316238in}}%
\pgfpathlineto{\pgfqpoint{1.626484in}{2.316579in}}%
\pgfpathlineto{\pgfqpoint{1.627754in}{2.327825in}}%
\pgfpathlineto{\pgfqpoint{1.629659in}{2.310194in}}%
\pgfpathlineto{\pgfqpoint{1.631246in}{2.314425in}}%
\pgfpathlineto{\pgfqpoint{1.632516in}{2.312600in}}%
\pgfpathlineto{\pgfqpoint{1.635056in}{2.322964in}}%
\pgfpathlineto{\pgfqpoint{1.636961in}{2.307124in}}%
\pgfpathlineto{\pgfqpoint{1.638548in}{2.311450in}}%
\pgfpathlineto{\pgfqpoint{1.639818in}{2.309562in}}%
\pgfpathlineto{\pgfqpoint{1.641405in}{2.321124in}}%
\pgfpathlineto{\pgfqpoint{1.642357in}{2.317513in}}%
\pgfpathlineto{\pgfqpoint{1.644262in}{2.304322in}}%
\pgfpathlineto{\pgfqpoint{1.645849in}{2.308442in}}%
\pgfpathlineto{\pgfqpoint{1.647119in}{2.306525in}}%
\pgfpathlineto{\pgfqpoint{1.648706in}{2.318720in}}%
\pgfpathlineto{\pgfqpoint{1.649341in}{2.313880in}}%
\pgfpathlineto{\pgfqpoint{1.651564in}{2.301732in}}%
\pgfpathlineto{\pgfqpoint{1.652199in}{2.304102in}}%
\pgfpathlineto{\pgfqpoint{1.652516in}{2.303926in}}%
\pgfpathlineto{\pgfqpoint{1.652833in}{2.305442in}}%
\pgfpathlineto{\pgfqpoint{1.653786in}{2.305163in}}%
\pgfpathlineto{\pgfqpoint{1.654103in}{2.305193in}}%
\pgfpathlineto{\pgfqpoint{1.654421in}{2.303577in}}%
\pgfpathlineto{\pgfqpoint{1.654738in}{2.305918in}}%
\pgfpathlineto{\pgfqpoint{1.656008in}{2.314819in}}%
\pgfpathlineto{\pgfqpoint{1.657913in}{2.300646in}}%
\pgfpathlineto{\pgfqpoint{1.658230in}{2.300788in}}%
\pgfpathlineto{\pgfqpoint{1.658548in}{2.299038in}}%
\pgfpathlineto{\pgfqpoint{1.659183in}{2.301232in}}%
\pgfpathlineto{\pgfqpoint{1.659817in}{2.301320in}}%
\pgfpathlineto{\pgfqpoint{1.660135in}{2.302465in}}%
\pgfpathlineto{\pgfqpoint{1.661087in}{2.302174in}}%
\pgfpathlineto{\pgfqpoint{1.661405in}{2.302229in}}%
\pgfpathlineto{\pgfqpoint{1.661722in}{2.300595in}}%
\pgfpathlineto{\pgfqpoint{1.662040in}{2.304251in}}%
\pgfpathlineto{\pgfqpoint{1.662992in}{2.312313in}}%
\pgfpathlineto{\pgfqpoint{1.663310in}{2.310980in}}%
\pgfpathlineto{\pgfqpoint{1.665214in}{2.296853in}}%
\pgfpathlineto{\pgfqpoint{1.665849in}{2.295878in}}%
\pgfpathlineto{\pgfqpoint{1.666484in}{2.298622in}}%
\pgfpathlineto{\pgfqpoint{1.666802in}{2.298089in}}%
\pgfpathlineto{\pgfqpoint{1.667119in}{2.298730in}}%
\pgfpathlineto{\pgfqpoint{1.668071in}{2.298148in}}%
\pgfpathlineto{\pgfqpoint{1.668389in}{2.299168in}}%
\pgfpathlineto{\pgfqpoint{1.668706in}{2.299240in}}%
\pgfpathlineto{\pgfqpoint{1.669024in}{2.297691in}}%
\pgfpathlineto{\pgfqpoint{1.670294in}{2.309847in}}%
\pgfpathlineto{\pgfqpoint{1.672516in}{2.293391in}}%
\pgfpathlineto{\pgfqpoint{1.673151in}{2.293077in}}%
\pgfpathlineto{\pgfqpoint{1.673786in}{2.296042in}}%
\pgfpathlineto{\pgfqpoint{1.674103in}{2.295142in}}%
\pgfpathlineto{\pgfqpoint{1.674738in}{2.296486in}}%
\pgfpathlineto{\pgfqpoint{1.676325in}{2.294737in}}%
\pgfpathlineto{\pgfqpoint{1.677595in}{2.306988in}}%
\pgfpathlineto{\pgfqpoint{1.679500in}{2.290077in}}%
\pgfpathlineto{\pgfqpoint{1.679817in}{2.290403in}}%
\pgfpathlineto{\pgfqpoint{1.680452in}{2.290201in}}%
\pgfpathlineto{\pgfqpoint{1.681087in}{2.293447in}}%
\pgfpathlineto{\pgfqpoint{1.681405in}{2.292261in}}%
\pgfpathlineto{\pgfqpoint{1.682357in}{2.292860in}}%
\pgfpathlineto{\pgfqpoint{1.683627in}{2.291901in}}%
\pgfpathlineto{\pgfqpoint{1.684897in}{2.303682in}}%
\pgfpathlineto{\pgfqpoint{1.686801in}{2.286711in}}%
\pgfpathlineto{\pgfqpoint{1.689024in}{2.290561in}}%
\pgfpathlineto{\pgfqpoint{1.690928in}{2.289456in}}%
\pgfpathlineto{\pgfqpoint{1.692198in}{2.300161in}}%
\pgfpathlineto{\pgfqpoint{1.694103in}{2.283654in}}%
\pgfpathlineto{\pgfqpoint{1.695690in}{2.287902in}}%
\pgfpathlineto{\pgfqpoint{1.696960in}{2.286479in}}%
\pgfpathlineto{\pgfqpoint{1.698230in}{2.287439in}}%
\pgfpathlineto{\pgfqpoint{1.698547in}{2.296900in}}%
\pgfpathlineto{\pgfqpoint{1.699500in}{2.296559in}}%
\pgfpathlineto{\pgfqpoint{1.700770in}{2.282457in}}%
\pgfpathlineto{\pgfqpoint{1.701087in}{2.282917in}}%
\pgfpathlineto{\pgfqpoint{1.701404in}{2.280618in}}%
\pgfpathlineto{\pgfqpoint{1.702039in}{2.283985in}}%
\pgfpathlineto{\pgfqpoint{1.702357in}{2.282313in}}%
\pgfpathlineto{\pgfqpoint{1.702674in}{2.284643in}}%
\pgfpathlineto{\pgfqpoint{1.702992in}{2.284939in}}%
\pgfpathlineto{\pgfqpoint{1.704262in}{2.283341in}}%
\pgfpathlineto{\pgfqpoint{1.705531in}{2.286369in}}%
\pgfpathlineto{\pgfqpoint{1.705849in}{2.294059in}}%
\pgfpathlineto{\pgfqpoint{1.706801in}{2.292531in}}%
\pgfpathlineto{\pgfqpoint{1.708071in}{2.279398in}}%
\pgfpathlineto{\pgfqpoint{1.708389in}{2.279673in}}%
\pgfpathlineto{\pgfqpoint{1.708706in}{2.277763in}}%
\pgfpathlineto{\pgfqpoint{1.709023in}{2.279782in}}%
\pgfpathlineto{\pgfqpoint{1.710293in}{2.281950in}}%
\pgfpathlineto{\pgfqpoint{1.711563in}{2.280203in}}%
\pgfpathlineto{\pgfqpoint{1.712833in}{2.285768in}}%
\pgfpathlineto{\pgfqpoint{1.713150in}{2.291051in}}%
\pgfpathlineto{\pgfqpoint{1.713785in}{2.286944in}}%
\pgfpathlineto{\pgfqpoint{1.714103in}{2.288214in}}%
\pgfpathlineto{\pgfqpoint{1.716008in}{2.274916in}}%
\pgfpathlineto{\pgfqpoint{1.717595in}{2.278914in}}%
\pgfpathlineto{\pgfqpoint{1.718865in}{2.277137in}}%
\pgfpathlineto{\pgfqpoint{1.720134in}{2.284471in}}%
\pgfpathlineto{\pgfqpoint{1.720452in}{2.288670in}}%
\pgfpathlineto{\pgfqpoint{1.721087in}{2.283643in}}%
\pgfpathlineto{\pgfqpoint{1.721404in}{2.283447in}}%
\pgfpathlineto{\pgfqpoint{1.723309in}{2.272150in}}%
\pgfpathlineto{\pgfqpoint{1.724896in}{2.275866in}}%
\pgfpathlineto{\pgfqpoint{1.726166in}{2.274076in}}%
\pgfpathlineto{\pgfqpoint{1.727753in}{2.285939in}}%
\pgfpathlineto{\pgfqpoint{1.728071in}{2.280641in}}%
\pgfpathlineto{\pgfqpoint{1.729023in}{2.275583in}}%
\pgfpathlineto{\pgfqpoint{1.730611in}{2.269352in}}%
\pgfpathlineto{\pgfqpoint{1.731880in}{2.272883in}}%
\pgfpathlineto{\pgfqpoint{1.732198in}{2.272784in}}%
\pgfpathlineto{\pgfqpoint{1.733468in}{2.271042in}}%
\pgfpathlineto{\pgfqpoint{1.735055in}{2.282235in}}%
\pgfpathlineto{\pgfqpoint{1.736960in}{2.267537in}}%
\pgfpathlineto{\pgfqpoint{1.737277in}{2.267816in}}%
\pgfpathlineto{\pgfqpoint{1.737595in}{2.266668in}}%
\pgfpathlineto{\pgfqpoint{1.737912in}{2.266798in}}%
\pgfpathlineto{\pgfqpoint{1.739182in}{2.269839in}}%
\pgfpathlineto{\pgfqpoint{1.739499in}{2.269689in}}%
\pgfpathlineto{\pgfqpoint{1.740769in}{2.268001in}}%
\pgfpathlineto{\pgfqpoint{1.742039in}{2.279014in}}%
\pgfpathlineto{\pgfqpoint{1.742356in}{2.278048in}}%
\pgfpathlineto{\pgfqpoint{1.744261in}{2.264337in}}%
\pgfpathlineto{\pgfqpoint{1.744896in}{2.263479in}}%
\pgfpathlineto{\pgfqpoint{1.746483in}{2.266810in}}%
\pgfpathlineto{\pgfqpoint{1.748071in}{2.265008in}}%
\pgfpathlineto{\pgfqpoint{1.749341in}{2.276457in}}%
\pgfpathlineto{\pgfqpoint{1.749658in}{2.274090in}}%
\pgfpathlineto{\pgfqpoint{1.751563in}{2.260814in}}%
\pgfpathlineto{\pgfqpoint{1.752198in}{2.260514in}}%
\pgfpathlineto{\pgfqpoint{1.752833in}{2.263291in}}%
\pgfpathlineto{\pgfqpoint{1.753150in}{2.262689in}}%
\pgfpathlineto{\pgfqpoint{1.753785in}{2.263773in}}%
\pgfpathlineto{\pgfqpoint{1.755372in}{2.261999in}}%
\pgfpathlineto{\pgfqpoint{1.756642in}{2.273624in}}%
\pgfpathlineto{\pgfqpoint{1.758864in}{2.257583in}}%
\pgfpathlineto{\pgfqpoint{1.760134in}{2.260555in}}%
\pgfpathlineto{\pgfqpoint{1.760452in}{2.259643in}}%
\pgfpathlineto{\pgfqpoint{1.761404in}{2.260260in}}%
\pgfpathlineto{\pgfqpoint{1.762674in}{2.259197in}}%
\pgfpathlineto{\pgfqpoint{1.763944in}{2.270648in}}%
\pgfpathlineto{\pgfqpoint{1.765848in}{2.254225in}}%
\pgfpathlineto{\pgfqpoint{1.768388in}{2.257712in}}%
\pgfpathlineto{\pgfqpoint{1.768705in}{2.257098in}}%
\pgfpathlineto{\pgfqpoint{1.769975in}{2.256482in}}%
\pgfpathlineto{\pgfqpoint{1.771245in}{2.267379in}}%
\pgfpathlineto{\pgfqpoint{1.773150in}{2.251081in}}%
\pgfpathlineto{\pgfqpoint{1.775372in}{2.254824in}}%
\pgfpathlineto{\pgfqpoint{1.775690in}{2.254664in}}%
\pgfpathlineto{\pgfqpoint{1.776324in}{2.253791in}}%
\pgfpathlineto{\pgfqpoint{1.776959in}{2.254338in}}%
\pgfpathlineto{\pgfqpoint{1.777277in}{2.253925in}}%
\pgfpathlineto{\pgfqpoint{1.778547in}{2.264042in}}%
\pgfpathlineto{\pgfqpoint{1.780451in}{2.247947in}}%
\pgfpathlineto{\pgfqpoint{1.782039in}{2.252105in}}%
\pgfpathlineto{\pgfqpoint{1.783309in}{2.250685in}}%
\pgfpathlineto{\pgfqpoint{1.784578in}{2.251680in}}%
\pgfpathlineto{\pgfqpoint{1.785848in}{2.260812in}}%
\pgfpathlineto{\pgfqpoint{1.787753in}{2.244968in}}%
\pgfpathlineto{\pgfqpoint{1.789340in}{2.249132in}}%
\pgfpathlineto{\pgfqpoint{1.790610in}{2.247480in}}%
\pgfpathlineto{\pgfqpoint{1.791880in}{2.249976in}}%
\pgfpathlineto{\pgfqpoint{1.792197in}{2.257965in}}%
\pgfpathlineto{\pgfqpoint{1.793150in}{2.257199in}}%
\pgfpathlineto{\pgfqpoint{1.795054in}{2.241956in}}%
\pgfpathlineto{\pgfqpoint{1.796642in}{2.246112in}}%
\pgfpathlineto{\pgfqpoint{1.797912in}{2.244342in}}%
\pgfpathlineto{\pgfqpoint{1.799181in}{2.248880in}}%
\pgfpathlineto{\pgfqpoint{1.799499in}{2.254493in}}%
\pgfpathlineto{\pgfqpoint{1.800451in}{2.253528in}}%
\pgfpathlineto{\pgfqpoint{1.802356in}{2.239029in}}%
\pgfpathlineto{\pgfqpoint{1.803943in}{2.243074in}}%
\pgfpathlineto{\pgfqpoint{1.805213in}{2.241180in}}%
\pgfpathlineto{\pgfqpoint{1.806483in}{2.246951in}}%
\pgfpathlineto{\pgfqpoint{1.806800in}{2.252699in}}%
\pgfpathlineto{\pgfqpoint{1.807435in}{2.248391in}}%
\pgfpathlineto{\pgfqpoint{1.807753in}{2.249168in}}%
\pgfpathlineto{\pgfqpoint{1.809658in}{2.236082in}}%
\pgfpathlineto{\pgfqpoint{1.811245in}{2.239994in}}%
\pgfpathlineto{\pgfqpoint{1.812515in}{2.238110in}}%
\pgfpathlineto{\pgfqpoint{1.813784in}{2.245397in}}%
\pgfpathlineto{\pgfqpoint{1.814102in}{2.250400in}}%
\pgfpathlineto{\pgfqpoint{1.814737in}{2.245210in}}%
\pgfpathlineto{\pgfqpoint{1.815054in}{2.244718in}}%
\pgfpathlineto{\pgfqpoint{1.816959in}{2.233135in}}%
\pgfpathlineto{\pgfqpoint{1.818546in}{2.236922in}}%
\pgfpathlineto{\pgfqpoint{1.819816in}{2.234995in}}%
\pgfpathlineto{\pgfqpoint{1.821403in}{2.247723in}}%
\pgfpathlineto{\pgfqpoint{1.822038in}{2.242295in}}%
\pgfpathlineto{\pgfqpoint{1.824261in}{2.230316in}}%
\pgfpathlineto{\pgfqpoint{1.825213in}{2.232477in}}%
\pgfpathlineto{\pgfqpoint{1.825530in}{2.233860in}}%
\pgfpathlineto{\pgfqpoint{1.826483in}{2.233600in}}%
\pgfpathlineto{\pgfqpoint{1.826800in}{2.233628in}}%
\pgfpathlineto{\pgfqpoint{1.827118in}{2.231982in}}%
\pgfpathlineto{\pgfqpoint{1.827435in}{2.233682in}}%
\pgfpathlineto{\pgfqpoint{1.828705in}{2.244182in}}%
\pgfpathlineto{\pgfqpoint{1.830292in}{2.230495in}}%
\pgfpathlineto{\pgfqpoint{1.831562in}{2.227554in}}%
\pgfpathlineto{\pgfqpoint{1.832832in}{2.230813in}}%
\pgfpathlineto{\pgfqpoint{1.833149in}{2.230651in}}%
\pgfpathlineto{\pgfqpoint{1.834419in}{2.228915in}}%
\pgfpathlineto{\pgfqpoint{1.836006in}{2.241087in}}%
\pgfpathlineto{\pgfqpoint{1.836641in}{2.236113in}}%
\pgfpathlineto{\pgfqpoint{1.838546in}{2.224213in}}%
\pgfpathlineto{\pgfqpoint{1.838864in}{2.225012in}}%
\pgfpathlineto{\pgfqpoint{1.840133in}{2.227783in}}%
\pgfpathlineto{\pgfqpoint{1.840451in}{2.227464in}}%
\pgfpathlineto{\pgfqpoint{1.841721in}{2.225965in}}%
\pgfpathlineto{\pgfqpoint{1.843308in}{2.237283in}}%
\pgfpathlineto{\pgfqpoint{1.845213in}{2.222167in}}%
\pgfpathlineto{\pgfqpoint{1.845848in}{2.221083in}}%
\pgfpathlineto{\pgfqpoint{1.846483in}{2.223685in}}%
\pgfpathlineto{\pgfqpoint{1.846800in}{2.222872in}}%
\pgfpathlineto{\pgfqpoint{1.847118in}{2.223969in}}%
\pgfpathlineto{\pgfqpoint{1.848387in}{2.224402in}}%
\pgfpathlineto{\pgfqpoint{1.848705in}{2.224405in}}%
\pgfpathlineto{\pgfqpoint{1.849022in}{2.222983in}}%
\pgfpathlineto{\pgfqpoint{1.849340in}{2.225757in}}%
\pgfpathlineto{\pgfqpoint{1.850292in}{2.234725in}}%
\pgfpathlineto{\pgfqpoint{1.850610in}{2.234174in}}%
\pgfpathlineto{\pgfqpoint{1.852514in}{2.218829in}}%
\pgfpathlineto{\pgfqpoint{1.853149in}{2.217915in}}%
\pgfpathlineto{\pgfqpoint{1.853784in}{2.220960in}}%
\pgfpathlineto{\pgfqpoint{1.854102in}{2.219834in}}%
\pgfpathlineto{\pgfqpoint{1.854419in}{2.221275in}}%
\pgfpathlineto{\pgfqpoint{1.855689in}{2.221336in}}%
\pgfpathlineto{\pgfqpoint{1.856006in}{2.221289in}}%
\pgfpathlineto{\pgfqpoint{1.856324in}{2.220069in}}%
\pgfpathlineto{\pgfqpoint{1.857594in}{2.232407in}}%
\pgfpathlineto{\pgfqpoint{1.857911in}{2.230311in}}%
\pgfpathlineto{\pgfqpoint{1.859816in}{2.215535in}}%
\pgfpathlineto{\pgfqpoint{1.860451in}{2.214898in}}%
\pgfpathlineto{\pgfqpoint{1.861086in}{2.218221in}}%
\pgfpathlineto{\pgfqpoint{1.861403in}{2.216805in}}%
\pgfpathlineto{\pgfqpoint{1.862038in}{2.218747in}}%
\pgfpathlineto{\pgfqpoint{1.863625in}{2.217127in}}%
\pgfpathlineto{\pgfqpoint{1.864895in}{2.229928in}}%
\pgfpathlineto{\pgfqpoint{1.865213in}{2.226940in}}%
\pgfpathlineto{\pgfqpoint{1.867117in}{2.212428in}}%
\pgfpathlineto{\pgfqpoint{1.867752in}{2.211816in}}%
\pgfpathlineto{\pgfqpoint{1.868387in}{2.215336in}}%
\pgfpathlineto{\pgfqpoint{1.868705in}{2.213706in}}%
\pgfpathlineto{\pgfqpoint{1.869340in}{2.215740in}}%
\pgfpathlineto{\pgfqpoint{1.870927in}{2.214251in}}%
\pgfpathlineto{\pgfqpoint{1.872197in}{2.227162in}}%
\pgfpathlineto{\pgfqpoint{1.872514in}{2.223393in}}%
\pgfpathlineto{\pgfqpoint{1.874419in}{2.209433in}}%
\pgfpathlineto{\pgfqpoint{1.875054in}{2.208928in}}%
\pgfpathlineto{\pgfqpoint{1.875689in}{2.212497in}}%
\pgfpathlineto{\pgfqpoint{1.876006in}{2.210686in}}%
\pgfpathlineto{\pgfqpoint{1.876641in}{2.212736in}}%
\pgfpathlineto{\pgfqpoint{1.876959in}{2.211484in}}%
\pgfpathlineto{\pgfqpoint{1.877593in}{2.212193in}}%
\pgfpathlineto{\pgfqpoint{1.877911in}{2.211744in}}%
\pgfpathlineto{\pgfqpoint{1.878228in}{2.211363in}}%
\pgfpathlineto{\pgfqpoint{1.879498in}{2.224453in}}%
\pgfpathlineto{\pgfqpoint{1.879816in}{2.220240in}}%
\pgfpathlineto{\pgfqpoint{1.881720in}{2.206419in}}%
\pgfpathlineto{\pgfqpoint{1.882355in}{2.205894in}}%
\pgfpathlineto{\pgfqpoint{1.882990in}{2.209468in}}%
\pgfpathlineto{\pgfqpoint{1.883308in}{2.207675in}}%
\pgfpathlineto{\pgfqpoint{1.883625in}{2.209694in}}%
\pgfpathlineto{\pgfqpoint{1.884260in}{2.208285in}}%
\pgfpathlineto{\pgfqpoint{1.884895in}{2.209175in}}%
\pgfpathlineto{\pgfqpoint{1.885212in}{2.208567in}}%
\pgfpathlineto{\pgfqpoint{1.885530in}{2.208519in}}%
\pgfpathlineto{\pgfqpoint{1.886800in}{2.221107in}}%
\pgfpathlineto{\pgfqpoint{1.887117in}{2.216846in}}%
\pgfpathlineto{\pgfqpoint{1.889022in}{2.203531in}}%
\pgfpathlineto{\pgfqpoint{1.889657in}{2.203024in}}%
\pgfpathlineto{\pgfqpoint{1.890292in}{2.206442in}}%
\pgfpathlineto{\pgfqpoint{1.890609in}{2.204777in}}%
\pgfpathlineto{\pgfqpoint{1.890927in}{2.206733in}}%
\pgfpathlineto{\pgfqpoint{1.891562in}{2.205151in}}%
\pgfpathlineto{\pgfqpoint{1.894101in}{2.217666in}}%
\pgfpathlineto{\pgfqpoint{1.894419in}{2.213664in}}%
\pgfpathlineto{\pgfqpoint{1.896323in}{2.200563in}}%
\pgfpathlineto{\pgfqpoint{1.896958in}{2.200011in}}%
\pgfpathlineto{\pgfqpoint{1.897593in}{2.203326in}}%
\pgfpathlineto{\pgfqpoint{1.898228in}{2.203701in}}%
\pgfpathlineto{\pgfqpoint{1.898863in}{2.202001in}}%
\pgfpathlineto{\pgfqpoint{1.901403in}{2.213575in}}%
\pgfpathlineto{\pgfqpoint{1.901720in}{2.210123in}}%
\pgfpathlineto{\pgfqpoint{1.903625in}{2.197730in}}%
\pgfpathlineto{\pgfqpoint{1.903942in}{2.198444in}}%
\pgfpathlineto{\pgfqpoint{1.904260in}{2.197163in}}%
\pgfpathlineto{\pgfqpoint{1.904577in}{2.199218in}}%
\pgfpathlineto{\pgfqpoint{1.905212in}{2.198905in}}%
\pgfpathlineto{\pgfqpoint{1.905847in}{2.200610in}}%
\pgfpathlineto{\pgfqpoint{1.907117in}{2.198958in}}%
\pgfpathlineto{\pgfqpoint{1.907434in}{2.199826in}}%
\pgfpathlineto{\pgfqpoint{1.908704in}{2.210234in}}%
\pgfpathlineto{\pgfqpoint{1.910927in}{2.194785in}}%
\pgfpathlineto{\pgfqpoint{1.914101in}{2.197185in}}%
\pgfpathlineto{\pgfqpoint{1.914419in}{2.195809in}}%
\pgfpathlineto{\pgfqpoint{1.914736in}{2.196986in}}%
\pgfpathlineto{\pgfqpoint{1.916006in}{2.207653in}}%
\pgfpathlineto{\pgfqpoint{1.917911in}{2.192053in}}%
\pgfpathlineto{\pgfqpoint{1.918863in}{2.191266in}}%
\pgfpathlineto{\pgfqpoint{1.920133in}{2.194502in}}%
\pgfpathlineto{\pgfqpoint{1.920450in}{2.194446in}}%
\pgfpathlineto{\pgfqpoint{1.921720in}{2.192658in}}%
\pgfpathlineto{\pgfqpoint{1.923307in}{2.204850in}}%
\pgfpathlineto{\pgfqpoint{1.923625in}{2.199114in}}%
\pgfpathlineto{\pgfqpoint{1.926164in}{2.188262in}}%
\pgfpathlineto{\pgfqpoint{1.926799in}{2.190645in}}%
\pgfpathlineto{\pgfqpoint{1.927117in}{2.190253in}}%
\pgfpathlineto{\pgfqpoint{1.927434in}{2.191423in}}%
\pgfpathlineto{\pgfqpoint{1.927752in}{2.191311in}}%
\pgfpathlineto{\pgfqpoint{1.929022in}{2.189557in}}%
\pgfpathlineto{\pgfqpoint{1.930609in}{2.201547in}}%
\pgfpathlineto{\pgfqpoint{1.930926in}{2.195517in}}%
\pgfpathlineto{\pgfqpoint{1.931561in}{2.192846in}}%
\pgfpathlineto{\pgfqpoint{1.933149in}{2.185187in}}%
\pgfpathlineto{\pgfqpoint{1.933466in}{2.185346in}}%
\pgfpathlineto{\pgfqpoint{1.934736in}{2.188338in}}%
\pgfpathlineto{\pgfqpoint{1.935053in}{2.188177in}}%
\pgfpathlineto{\pgfqpoint{1.936323in}{2.186450in}}%
\pgfpathlineto{\pgfqpoint{1.937910in}{2.198122in}}%
\pgfpathlineto{\pgfqpoint{1.939498in}{2.183819in}}%
\pgfpathlineto{\pgfqpoint{1.940450in}{2.181896in}}%
\pgfpathlineto{\pgfqpoint{1.940768in}{2.182576in}}%
\pgfpathlineto{\pgfqpoint{1.942037in}{2.185262in}}%
\pgfpathlineto{\pgfqpoint{1.942355in}{2.184988in}}%
\pgfpathlineto{\pgfqpoint{1.943625in}{2.183410in}}%
\pgfpathlineto{\pgfqpoint{1.945212in}{2.194701in}}%
\pgfpathlineto{\pgfqpoint{1.946799in}{2.180749in}}%
\pgfpathlineto{\pgfqpoint{1.947752in}{2.178721in}}%
\pgfpathlineto{\pgfqpoint{1.948069in}{2.179907in}}%
\pgfpathlineto{\pgfqpoint{1.949339in}{2.182198in}}%
\pgfpathlineto{\pgfqpoint{1.950926in}{2.180380in}}%
\pgfpathlineto{\pgfqpoint{1.952196in}{2.191899in}}%
\pgfpathlineto{\pgfqpoint{1.952513in}{2.191351in}}%
\pgfpathlineto{\pgfqpoint{1.954418in}{2.176077in}}%
\pgfpathlineto{\pgfqpoint{1.955053in}{2.175517in}}%
\pgfpathlineto{\pgfqpoint{1.955688in}{2.178369in}}%
\pgfpathlineto{\pgfqpoint{1.956005in}{2.177508in}}%
\pgfpathlineto{\pgfqpoint{1.956323in}{2.178648in}}%
\pgfpathlineto{\pgfqpoint{1.957593in}{2.178747in}}%
\pgfpathlineto{\pgfqpoint{1.957910in}{2.178781in}}%
\pgfpathlineto{\pgfqpoint{1.958228in}{2.177416in}}%
\pgfpathlineto{\pgfqpoint{1.959498in}{2.189180in}}%
\pgfpathlineto{\pgfqpoint{1.959815in}{2.187989in}}%
\pgfpathlineto{\pgfqpoint{1.961720in}{2.172930in}}%
\pgfpathlineto{\pgfqpoint{1.962355in}{2.172382in}}%
\pgfpathlineto{\pgfqpoint{1.962990in}{2.175511in}}%
\pgfpathlineto{\pgfqpoint{1.963307in}{2.174342in}}%
\pgfpathlineto{\pgfqpoint{1.963942in}{2.176099in}}%
\pgfpathlineto{\pgfqpoint{1.965529in}{2.174432in}}%
\pgfpathlineto{\pgfqpoint{1.966799in}{2.186739in}}%
\pgfpathlineto{\pgfqpoint{1.967117in}{2.184562in}}%
\pgfpathlineto{\pgfqpoint{1.969021in}{2.169689in}}%
\pgfpathlineto{\pgfqpoint{1.969656in}{2.169241in}}%
\pgfpathlineto{\pgfqpoint{1.970291in}{2.172601in}}%
\pgfpathlineto{\pgfqpoint{1.970609in}{2.171184in}}%
\pgfpathlineto{\pgfqpoint{1.971243in}{2.173059in}}%
\pgfpathlineto{\pgfqpoint{1.972831in}{2.171507in}}%
\pgfpathlineto{\pgfqpoint{1.974101in}{2.183961in}}%
\pgfpathlineto{\pgfqpoint{1.974418in}{2.181119in}}%
\pgfpathlineto{\pgfqpoint{1.976323in}{2.166716in}}%
\pgfpathlineto{\pgfqpoint{1.976958in}{2.166179in}}%
\pgfpathlineto{\pgfqpoint{1.977593in}{2.169608in}}%
\pgfpathlineto{\pgfqpoint{1.977910in}{2.167971in}}%
\pgfpathlineto{\pgfqpoint{1.978545in}{2.170036in}}%
\pgfpathlineto{\pgfqpoint{1.979815in}{2.169161in}}%
\pgfpathlineto{\pgfqpoint{1.980132in}{2.168571in}}%
\pgfpathlineto{\pgfqpoint{1.981402in}{2.181316in}}%
\pgfpathlineto{\pgfqpoint{1.981720in}{2.177979in}}%
\pgfpathlineto{\pgfqpoint{1.983624in}{2.163643in}}%
\pgfpathlineto{\pgfqpoint{1.984259in}{2.163140in}}%
\pgfpathlineto{\pgfqpoint{1.984894in}{2.166660in}}%
\pgfpathlineto{\pgfqpoint{1.985212in}{2.164859in}}%
\pgfpathlineto{\pgfqpoint{1.985847in}{2.166995in}}%
\pgfpathlineto{\pgfqpoint{1.986164in}{2.165668in}}%
\pgfpathlineto{\pgfqpoint{1.986799in}{2.166452in}}%
\pgfpathlineto{\pgfqpoint{1.987116in}{2.165945in}}%
\pgfpathlineto{\pgfqpoint{1.987434in}{2.165690in}}%
\pgfpathlineto{\pgfqpoint{1.988704in}{2.178251in}}%
\pgfpathlineto{\pgfqpoint{1.989021in}{2.174705in}}%
\pgfpathlineto{\pgfqpoint{1.990926in}{2.160762in}}%
\pgfpathlineto{\pgfqpoint{1.991561in}{2.160118in}}%
\pgfpathlineto{\pgfqpoint{1.992196in}{2.163559in}}%
\pgfpathlineto{\pgfqpoint{1.992513in}{2.161738in}}%
\pgfpathlineto{\pgfqpoint{1.993148in}{2.163980in}}%
\pgfpathlineto{\pgfqpoint{1.993465in}{2.162506in}}%
\pgfpathlineto{\pgfqpoint{1.993783in}{2.163507in}}%
\pgfpathlineto{\pgfqpoint{1.994418in}{2.162682in}}%
\pgfpathlineto{\pgfqpoint{1.994735in}{2.162796in}}%
\pgfpathlineto{\pgfqpoint{1.996005in}{2.175158in}}%
\pgfpathlineto{\pgfqpoint{1.996323in}{2.171615in}}%
\pgfpathlineto{\pgfqpoint{1.998227in}{2.157737in}}%
\pgfpathlineto{\pgfqpoint{1.998862in}{2.157085in}}%
\pgfpathlineto{\pgfqpoint{1.999497in}{2.160432in}}%
\pgfpathlineto{\pgfqpoint{1.999815in}{2.158759in}}%
\pgfpathlineto{\pgfqpoint{2.000450in}{2.160928in}}%
\pgfpathlineto{\pgfqpoint{2.000767in}{2.159323in}}%
\pgfpathlineto{\pgfqpoint{2.003307in}{2.171246in}}%
\pgfpathlineto{\pgfqpoint{2.003624in}{2.168239in}}%
\pgfpathlineto{\pgfqpoint{2.005529in}{2.154893in}}%
\pgfpathlineto{\pgfqpoint{2.006164in}{2.154056in}}%
\pgfpathlineto{\pgfqpoint{2.006799in}{2.157216in}}%
\pgfpathlineto{\pgfqpoint{2.007116in}{2.155711in}}%
\pgfpathlineto{\pgfqpoint{2.007751in}{2.157890in}}%
\pgfpathlineto{\pgfqpoint{2.008069in}{2.156226in}}%
\pgfpathlineto{\pgfqpoint{2.010608in}{2.167523in}}%
\pgfpathlineto{\pgfqpoint{2.010926in}{2.164893in}}%
\pgfpathlineto{\pgfqpoint{2.013465in}{2.151049in}}%
\pgfpathlineto{\pgfqpoint{2.015053in}{2.154791in}}%
\pgfpathlineto{\pgfqpoint{2.016322in}{2.152988in}}%
\pgfpathlineto{\pgfqpoint{2.017592in}{2.159076in}}%
\pgfpathlineto{\pgfqpoint{2.017910in}{2.164385in}}%
\pgfpathlineto{\pgfqpoint{2.018545in}{2.159342in}}%
\pgfpathlineto{\pgfqpoint{2.018862in}{2.160122in}}%
\pgfpathlineto{\pgfqpoint{2.020767in}{2.148022in}}%
\pgfpathlineto{\pgfqpoint{2.022354in}{2.151700in}}%
\pgfpathlineto{\pgfqpoint{2.023624in}{2.149764in}}%
\pgfpathlineto{\pgfqpoint{2.024894in}{2.156212in}}%
\pgfpathlineto{\pgfqpoint{2.025211in}{2.162339in}}%
\pgfpathlineto{\pgfqpoint{2.025846in}{2.156512in}}%
\pgfpathlineto{\pgfqpoint{2.026164in}{2.156608in}}%
\pgfpathlineto{\pgfqpoint{2.028068in}{2.144951in}}%
\pgfpathlineto{\pgfqpoint{2.029656in}{2.148548in}}%
\pgfpathlineto{\pgfqpoint{2.030925in}{2.146629in}}%
\pgfpathlineto{\pgfqpoint{2.032195in}{2.153322in}}%
\pgfpathlineto{\pgfqpoint{2.032513in}{2.159440in}}%
\pgfpathlineto{\pgfqpoint{2.033148in}{2.153909in}}%
\pgfpathlineto{\pgfqpoint{2.033465in}{2.153492in}}%
\pgfpathlineto{\pgfqpoint{2.035370in}{2.141858in}}%
\pgfpathlineto{\pgfqpoint{2.036957in}{2.145404in}}%
\pgfpathlineto{\pgfqpoint{2.038227in}{2.143477in}}%
\pgfpathlineto{\pgfqpoint{2.039814in}{2.156637in}}%
\pgfpathlineto{\pgfqpoint{2.040767in}{2.150643in}}%
\pgfpathlineto{\pgfqpoint{2.042671in}{2.138856in}}%
\pgfpathlineto{\pgfqpoint{2.044259in}{2.142179in}}%
\pgfpathlineto{\pgfqpoint{2.045529in}{2.140407in}}%
\pgfpathlineto{\pgfqpoint{2.047116in}{2.153329in}}%
\pgfpathlineto{\pgfqpoint{2.048068in}{2.147689in}}%
\pgfpathlineto{\pgfqpoint{2.049973in}{2.135966in}}%
\pgfpathlineto{\pgfqpoint{2.051243in}{2.139085in}}%
\pgfpathlineto{\pgfqpoint{2.051560in}{2.138965in}}%
\pgfpathlineto{\pgfqpoint{2.052830in}{2.137333in}}%
\pgfpathlineto{\pgfqpoint{2.054417in}{2.150213in}}%
\pgfpathlineto{\pgfqpoint{2.055370in}{2.144877in}}%
\pgfpathlineto{\pgfqpoint{2.056957in}{2.132787in}}%
\pgfpathlineto{\pgfqpoint{2.059814in}{2.135814in}}%
\pgfpathlineto{\pgfqpoint{2.060132in}{2.134339in}}%
\pgfpathlineto{\pgfqpoint{2.060449in}{2.135863in}}%
\pgfpathlineto{\pgfqpoint{2.061719in}{2.146637in}}%
\pgfpathlineto{\pgfqpoint{2.061084in}{2.135016in}}%
\pgfpathlineto{\pgfqpoint{2.062671in}{2.141620in}}%
\pgfpathlineto{\pgfqpoint{2.064259in}{2.129655in}}%
\pgfpathlineto{\pgfqpoint{2.067116in}{2.132609in}}%
\pgfpathlineto{\pgfqpoint{2.067433in}{2.131349in}}%
\pgfpathlineto{\pgfqpoint{2.067751in}{2.132749in}}%
\pgfpathlineto{\pgfqpoint{2.068386in}{2.131639in}}%
\pgfpathlineto{\pgfqpoint{2.068703in}{2.139142in}}%
\pgfpathlineto{\pgfqpoint{2.069020in}{2.143301in}}%
\pgfpathlineto{\pgfqpoint{2.069338in}{2.137029in}}%
\pgfpathlineto{\pgfqpoint{2.069655in}{2.141263in}}%
\pgfpathlineto{\pgfqpoint{2.071878in}{2.126672in}}%
\pgfpathlineto{\pgfqpoint{2.072195in}{2.128456in}}%
\pgfpathlineto{\pgfqpoint{2.072512in}{2.127191in}}%
\pgfpathlineto{\pgfqpoint{2.072830in}{2.128757in}}%
\pgfpathlineto{\pgfqpoint{2.074100in}{2.129729in}}%
\pgfpathlineto{\pgfqpoint{2.074735in}{2.128413in}}%
\pgfpathlineto{\pgfqpoint{2.075052in}{2.129607in}}%
\pgfpathlineto{\pgfqpoint{2.075370in}{2.130244in}}%
\pgfpathlineto{\pgfqpoint{2.075687in}{2.128413in}}%
\pgfpathlineto{\pgfqpoint{2.076322in}{2.139630in}}%
\pgfpathlineto{\pgfqpoint{2.076957in}{2.138771in}}%
\pgfpathlineto{\pgfqpoint{2.079179in}{2.123347in}}%
\pgfpathlineto{\pgfqpoint{2.080449in}{2.126662in}}%
\pgfpathlineto{\pgfqpoint{2.082036in}{2.125477in}}%
\pgfpathlineto{\pgfqpoint{2.082671in}{2.126678in}}%
\pgfpathlineto{\pgfqpoint{2.082989in}{2.125334in}}%
\pgfpathlineto{\pgfqpoint{2.083623in}{2.136255in}}%
\pgfpathlineto{\pgfqpoint{2.084258in}{2.136158in}}%
\pgfpathlineto{\pgfqpoint{2.086481in}{2.120189in}}%
\pgfpathlineto{\pgfqpoint{2.087750in}{2.123600in}}%
\pgfpathlineto{\pgfqpoint{2.088068in}{2.122473in}}%
\pgfpathlineto{\pgfqpoint{2.089020in}{2.122959in}}%
\pgfpathlineto{\pgfqpoint{2.090290in}{2.122056in}}%
\pgfpathlineto{\pgfqpoint{2.091560in}{2.133635in}}%
\pgfpathlineto{\pgfqpoint{2.093782in}{2.117023in}}%
\pgfpathlineto{\pgfqpoint{2.095052in}{2.120570in}}%
\pgfpathlineto{\pgfqpoint{2.095369in}{2.119236in}}%
\pgfpathlineto{\pgfqpoint{2.096322in}{2.119673in}}%
\pgfpathlineto{\pgfqpoint{2.097592in}{2.119023in}}%
\pgfpathlineto{\pgfqpoint{2.098861in}{2.130845in}}%
\pgfpathlineto{\pgfqpoint{2.101084in}{2.114012in}}%
\pgfpathlineto{\pgfqpoint{2.102353in}{2.117406in}}%
\pgfpathlineto{\pgfqpoint{2.102671in}{2.115892in}}%
\pgfpathlineto{\pgfqpoint{2.102988in}{2.117548in}}%
\pgfpathlineto{\pgfqpoint{2.103623in}{2.116433in}}%
\pgfpathlineto{\pgfqpoint{2.104576in}{2.116806in}}%
\pgfpathlineto{\pgfqpoint{2.104893in}{2.116060in}}%
\pgfpathlineto{\pgfqpoint{2.106163in}{2.128354in}}%
\pgfpathlineto{\pgfqpoint{2.108385in}{2.110924in}}%
\pgfpathlineto{\pgfqpoint{2.109655in}{2.114354in}}%
\pgfpathlineto{\pgfqpoint{2.109972in}{2.112691in}}%
\pgfpathlineto{\pgfqpoint{2.110290in}{2.114498in}}%
\pgfpathlineto{\pgfqpoint{2.110925in}{2.113158in}}%
\pgfpathlineto{\pgfqpoint{2.111560in}{2.113947in}}%
\pgfpathlineto{\pgfqpoint{2.111877in}{2.113555in}}%
\pgfpathlineto{\pgfqpoint{2.112195in}{2.113143in}}%
\pgfpathlineto{\pgfqpoint{2.113464in}{2.125417in}}%
\pgfpathlineto{\pgfqpoint{2.113782in}{2.121327in}}%
\pgfpathlineto{\pgfqpoint{2.115687in}{2.108012in}}%
\pgfpathlineto{\pgfqpoint{2.116322in}{2.107768in}}%
\pgfpathlineto{\pgfqpoint{2.116957in}{2.111178in}}%
\pgfpathlineto{\pgfqpoint{2.117274in}{2.109380in}}%
\pgfpathlineto{\pgfqpoint{2.117909in}{2.111401in}}%
\pgfpathlineto{\pgfqpoint{2.118226in}{2.109976in}}%
\pgfpathlineto{\pgfqpoint{2.118861in}{2.110843in}}%
\pgfpathlineto{\pgfqpoint{2.119179in}{2.110236in}}%
\pgfpathlineto{\pgfqpoint{2.119496in}{2.110207in}}%
\pgfpathlineto{\pgfqpoint{2.120766in}{2.122637in}}%
\pgfpathlineto{\pgfqpoint{2.121083in}{2.118678in}}%
\pgfpathlineto{\pgfqpoint{2.122988in}{2.105002in}}%
\pgfpathlineto{\pgfqpoint{2.123623in}{2.104661in}}%
\pgfpathlineto{\pgfqpoint{2.124258in}{2.107984in}}%
\pgfpathlineto{\pgfqpoint{2.124576in}{2.106276in}}%
\pgfpathlineto{\pgfqpoint{2.125210in}{2.108333in}}%
\pgfpathlineto{\pgfqpoint{2.125528in}{2.106764in}}%
\pgfpathlineto{\pgfqpoint{2.128068in}{2.118882in}}%
\pgfpathlineto{\pgfqpoint{2.128385in}{2.115598in}}%
\pgfpathlineto{\pgfqpoint{2.130290in}{2.102164in}}%
\pgfpathlineto{\pgfqpoint{2.130925in}{2.101532in}}%
\pgfpathlineto{\pgfqpoint{2.131560in}{2.104667in}}%
\pgfpathlineto{\pgfqpoint{2.131877in}{2.103102in}}%
\pgfpathlineto{\pgfqpoint{2.132512in}{2.105290in}}%
\pgfpathlineto{\pgfqpoint{2.132829in}{2.103640in}}%
\pgfpathlineto{\pgfqpoint{2.135369in}{2.115141in}}%
\pgfpathlineto{\pgfqpoint{2.135687in}{2.112483in}}%
\pgfpathlineto{\pgfqpoint{2.137591in}{2.099243in}}%
\pgfpathlineto{\pgfqpoint{2.137909in}{2.099928in}}%
\pgfpathlineto{\pgfqpoint{2.138226in}{2.098460in}}%
\pgfpathlineto{\pgfqpoint{2.138544in}{2.099894in}}%
\pgfpathlineto{\pgfqpoint{2.139813in}{2.102175in}}%
\pgfpathlineto{\pgfqpoint{2.141083in}{2.100387in}}%
\pgfpathlineto{\pgfqpoint{2.142353in}{2.106063in}}%
\pgfpathlineto{\pgfqpoint{2.142671in}{2.111487in}}%
\pgfpathlineto{\pgfqpoint{2.143306in}{2.106634in}}%
\pgfpathlineto{\pgfqpoint{2.143623in}{2.107898in}}%
\pgfpathlineto{\pgfqpoint{2.145528in}{2.095365in}}%
\pgfpathlineto{\pgfqpoint{2.147115in}{2.099058in}}%
\pgfpathlineto{\pgfqpoint{2.148385in}{2.097117in}}%
\pgfpathlineto{\pgfqpoint{2.149655in}{2.102510in}}%
\pgfpathlineto{\pgfqpoint{2.149972in}{2.109263in}}%
\pgfpathlineto{\pgfqpoint{2.150607in}{2.104027in}}%
\pgfpathlineto{\pgfqpoint{2.150924in}{2.104935in}}%
\pgfpathlineto{\pgfqpoint{2.152829in}{2.092276in}}%
\pgfpathlineto{\pgfqpoint{2.154417in}{2.095870in}}%
\pgfpathlineto{\pgfqpoint{2.155686in}{2.093943in}}%
\pgfpathlineto{\pgfqpoint{2.156956in}{2.099091in}}%
\pgfpathlineto{\pgfqpoint{2.157274in}{2.106424in}}%
\pgfpathlineto{\pgfqpoint{2.157909in}{2.101519in}}%
\pgfpathlineto{\pgfqpoint{2.158226in}{2.102000in}}%
\pgfpathlineto{\pgfqpoint{2.160131in}{2.089138in}}%
\pgfpathlineto{\pgfqpoint{2.161718in}{2.092688in}}%
\pgfpathlineto{\pgfqpoint{2.162988in}{2.090774in}}%
\pgfpathlineto{\pgfqpoint{2.164575in}{2.103292in}}%
\pgfpathlineto{\pgfqpoint{2.165528in}{2.099769in}}%
\pgfpathlineto{\pgfqpoint{2.167432in}{2.086115in}}%
\pgfpathlineto{\pgfqpoint{2.169020in}{2.089412in}}%
\pgfpathlineto{\pgfqpoint{2.170289in}{2.087689in}}%
\pgfpathlineto{\pgfqpoint{2.171877in}{2.099759in}}%
\pgfpathlineto{\pgfqpoint{2.172829in}{2.097384in}}%
\pgfpathlineto{\pgfqpoint{2.174734in}{2.083125in}}%
\pgfpathlineto{\pgfqpoint{2.176321in}{2.086119in}}%
\pgfpathlineto{\pgfqpoint{2.177591in}{2.084625in}}%
\pgfpathlineto{\pgfqpoint{2.179178in}{2.096245in}}%
\pgfpathlineto{\pgfqpoint{2.180131in}{2.095138in}}%
\pgfpathlineto{\pgfqpoint{2.182035in}{2.080384in}}%
\pgfpathlineto{\pgfqpoint{2.184258in}{2.083233in}}%
\pgfpathlineto{\pgfqpoint{2.184575in}{2.083085in}}%
\pgfpathlineto{\pgfqpoint{2.185845in}{2.081406in}}%
\pgfpathlineto{\pgfqpoint{2.187432in}{2.092543in}}%
\pgfpathlineto{\pgfqpoint{2.189019in}{2.077742in}}%
\pgfpathlineto{\pgfqpoint{2.189972in}{2.077166in}}%
\pgfpathlineto{\pgfqpoint{2.190289in}{2.077898in}}%
\pgfpathlineto{\pgfqpoint{2.191559in}{2.080108in}}%
\pgfpathlineto{\pgfqpoint{2.193146in}{2.078275in}}%
\pgfpathlineto{\pgfqpoint{2.194734in}{2.089765in}}%
\pgfpathlineto{\pgfqpoint{2.196321in}{2.075096in}}%
\pgfpathlineto{\pgfqpoint{2.197273in}{2.073808in}}%
\pgfpathlineto{\pgfqpoint{2.198861in}{2.077001in}}%
\pgfpathlineto{\pgfqpoint{2.199178in}{2.076661in}}%
\pgfpathlineto{\pgfqpoint{2.200448in}{2.075148in}}%
\pgfpathlineto{\pgfqpoint{2.202035in}{2.086874in}}%
\pgfpathlineto{\pgfqpoint{2.203622in}{2.072532in}}%
\pgfpathlineto{\pgfqpoint{2.204575in}{2.070455in}}%
\pgfpathlineto{\pgfqpoint{2.206162in}{2.073872in}}%
\pgfpathlineto{\pgfqpoint{2.207749in}{2.072091in}}%
\pgfpathlineto{\pgfqpoint{2.209337in}{2.084018in}}%
\pgfpathlineto{\pgfqpoint{2.210924in}{2.070269in}}%
\pgfpathlineto{\pgfqpoint{2.211876in}{2.067148in}}%
\pgfpathlineto{\pgfqpoint{2.212194in}{2.068254in}}%
\pgfpathlineto{\pgfqpoint{2.213464in}{2.070768in}}%
\pgfpathlineto{\pgfqpoint{2.215051in}{2.069050in}}%
\pgfpathlineto{\pgfqpoint{2.216638in}{2.080983in}}%
\pgfpathlineto{\pgfqpoint{2.217273in}{2.076808in}}%
\pgfpathlineto{\pgfqpoint{2.219178in}{2.063941in}}%
\pgfpathlineto{\pgfqpoint{2.220765in}{2.067616in}}%
\pgfpathlineto{\pgfqpoint{2.221400in}{2.067063in}}%
\pgfpathlineto{\pgfqpoint{2.222035in}{2.067034in}}%
\pgfpathlineto{\pgfqpoint{2.222352in}{2.066043in}}%
\pgfpathlineto{\pgfqpoint{2.222670in}{2.067266in}}%
\pgfpathlineto{\pgfqpoint{2.223940in}{2.077664in}}%
\pgfpathlineto{\pgfqpoint{2.225527in}{2.065759in}}%
\pgfpathlineto{\pgfqpoint{2.226479in}{2.060874in}}%
\pgfpathlineto{\pgfqpoint{2.226797in}{2.061208in}}%
\pgfpathlineto{\pgfqpoint{2.228067in}{2.064528in}}%
\pgfpathlineto{\pgfqpoint{2.229654in}{2.063067in}}%
\pgfpathlineto{\pgfqpoint{2.230289in}{2.066062in}}%
\pgfpathlineto{\pgfqpoint{2.230606in}{2.064332in}}%
\pgfpathlineto{\pgfqpoint{2.231241in}{2.074300in}}%
\pgfpathlineto{\pgfqpoint{2.231876in}{2.072597in}}%
\pgfpathlineto{\pgfqpoint{2.234098in}{2.057858in}}%
\pgfpathlineto{\pgfqpoint{2.235368in}{2.061388in}}%
\pgfpathlineto{\pgfqpoint{2.235686in}{2.060178in}}%
\pgfpathlineto{\pgfqpoint{2.236638in}{2.060461in}}%
\pgfpathlineto{\pgfqpoint{2.236956in}{2.060117in}}%
\pgfpathlineto{\pgfqpoint{2.237273in}{2.060899in}}%
\pgfpathlineto{\pgfqpoint{2.237590in}{2.061351in}}%
\pgfpathlineto{\pgfqpoint{2.237908in}{2.060327in}}%
\pgfpathlineto{\pgfqpoint{2.238543in}{2.070542in}}%
\pgfpathlineto{\pgfqpoint{2.239178in}{2.070454in}}%
\pgfpathlineto{\pgfqpoint{2.241400in}{2.054554in}}%
\pgfpathlineto{\pgfqpoint{2.242670in}{2.058240in}}%
\pgfpathlineto{\pgfqpoint{2.242987in}{2.056874in}}%
\pgfpathlineto{\pgfqpoint{2.243940in}{2.057136in}}%
\pgfpathlineto{\pgfqpoint{2.245209in}{2.056923in}}%
\pgfpathlineto{\pgfqpoint{2.246479in}{2.067804in}}%
\pgfpathlineto{\pgfqpoint{2.248701in}{2.051461in}}%
\pgfpathlineto{\pgfqpoint{2.249971in}{2.055011in}}%
\pgfpathlineto{\pgfqpoint{2.250289in}{2.053498in}}%
\pgfpathlineto{\pgfqpoint{2.251241in}{2.053864in}}%
\pgfpathlineto{\pgfqpoint{2.251876in}{2.054541in}}%
\pgfpathlineto{\pgfqpoint{2.252193in}{2.054290in}}%
\pgfpathlineto{\pgfqpoint{2.252511in}{2.053573in}}%
\pgfpathlineto{\pgfqpoint{2.253781in}{2.065522in}}%
\pgfpathlineto{\pgfqpoint{2.256003in}{2.048324in}}%
\pgfpathlineto{\pgfqpoint{2.257273in}{2.051815in}}%
\pgfpathlineto{\pgfqpoint{2.257590in}{2.050177in}}%
\pgfpathlineto{\pgfqpoint{2.257908in}{2.051970in}}%
\pgfpathlineto{\pgfqpoint{2.258543in}{2.050585in}}%
\pgfpathlineto{\pgfqpoint{2.259178in}{2.051419in}}%
\pgfpathlineto{\pgfqpoint{2.259495in}{2.050991in}}%
\pgfpathlineto{\pgfqpoint{2.259812in}{2.050643in}}%
\pgfpathlineto{\pgfqpoint{2.261082in}{2.062609in}}%
\pgfpathlineto{\pgfqpoint{2.261400in}{2.058896in}}%
\pgfpathlineto{\pgfqpoint{2.263305in}{2.045426in}}%
\pgfpathlineto{\pgfqpoint{2.263939in}{2.045218in}}%
\pgfpathlineto{\pgfqpoint{2.264574in}{2.048546in}}%
\pgfpathlineto{\pgfqpoint{2.264892in}{2.046785in}}%
\pgfpathlineto{\pgfqpoint{2.265527in}{2.048851in}}%
\pgfpathlineto{\pgfqpoint{2.265844in}{2.047384in}}%
\pgfpathlineto{\pgfqpoint{2.267749in}{2.052635in}}%
\pgfpathlineto{\pgfqpoint{2.268066in}{2.052045in}}%
\pgfpathlineto{\pgfqpoint{2.268384in}{2.059745in}}%
\pgfpathlineto{\pgfqpoint{2.269019in}{2.051639in}}%
\pgfpathlineto{\pgfqpoint{2.269336in}{2.053203in}}%
\pgfpathlineto{\pgfqpoint{2.270606in}{2.042484in}}%
\pgfpathlineto{\pgfqpoint{2.270923in}{2.043826in}}%
\pgfpathlineto{\pgfqpoint{2.271241in}{2.042051in}}%
\pgfpathlineto{\pgfqpoint{2.271558in}{2.043313in}}%
\pgfpathlineto{\pgfqpoint{2.272828in}{2.045760in}}%
\pgfpathlineto{\pgfqpoint{2.274098in}{2.044285in}}%
\pgfpathlineto{\pgfqpoint{2.274733in}{2.046789in}}%
\pgfpathlineto{\pgfqpoint{2.275685in}{2.055839in}}%
\pgfpathlineto{\pgfqpoint{2.276638in}{2.051136in}}%
\pgfpathlineto{\pgfqpoint{2.277908in}{2.039722in}}%
\pgfpathlineto{\pgfqpoint{2.278225in}{2.040653in}}%
\pgfpathlineto{\pgfqpoint{2.278542in}{2.038896in}}%
\pgfpathlineto{\pgfqpoint{2.278860in}{2.039902in}}%
\pgfpathlineto{\pgfqpoint{2.280130in}{2.042671in}}%
\pgfpathlineto{\pgfqpoint{2.281400in}{2.040923in}}%
\pgfpathlineto{\pgfqpoint{2.282669in}{2.045481in}}%
\pgfpathlineto{\pgfqpoint{2.282987in}{2.051763in}}%
\pgfpathlineto{\pgfqpoint{2.283939in}{2.049277in}}%
\pgfpathlineto{\pgfqpoint{2.285844in}{2.035847in}}%
\pgfpathlineto{\pgfqpoint{2.288066in}{2.039308in}}%
\pgfpathlineto{\pgfqpoint{2.288701in}{2.037655in}}%
\pgfpathlineto{\pgfqpoint{2.289653in}{2.039381in}}%
\pgfpathlineto{\pgfqpoint{2.290288in}{2.048170in}}%
\pgfpathlineto{\pgfqpoint{2.291241in}{2.046870in}}%
\pgfpathlineto{\pgfqpoint{2.293146in}{2.032742in}}%
\pgfpathlineto{\pgfqpoint{2.295368in}{2.036211in}}%
\pgfpathlineto{\pgfqpoint{2.295685in}{2.035988in}}%
\pgfpathlineto{\pgfqpoint{2.296003in}{2.034406in}}%
\pgfpathlineto{\pgfqpoint{2.296955in}{2.035080in}}%
\pgfpathlineto{\pgfqpoint{2.297590in}{2.045453in}}%
\pgfpathlineto{\pgfqpoint{2.298542in}{2.043942in}}%
\pgfpathlineto{\pgfqpoint{2.300447in}{2.029786in}}%
\pgfpathlineto{\pgfqpoint{2.302987in}{2.032933in}}%
\pgfpathlineto{\pgfqpoint{2.303304in}{2.031254in}}%
\pgfpathlineto{\pgfqpoint{2.304257in}{2.031481in}}%
\pgfpathlineto{\pgfqpoint{2.305844in}{2.041586in}}%
\pgfpathlineto{\pgfqpoint{2.306161in}{2.037307in}}%
\pgfpathlineto{\pgfqpoint{2.307749in}{2.026790in}}%
\pgfpathlineto{\pgfqpoint{2.308066in}{2.027075in}}%
\pgfpathlineto{\pgfqpoint{2.310288in}{2.029814in}}%
\pgfpathlineto{\pgfqpoint{2.311558in}{2.028192in}}%
\pgfpathlineto{\pgfqpoint{2.311876in}{2.029590in}}%
\pgfpathlineto{\pgfqpoint{2.313145in}{2.039553in}}%
\pgfpathlineto{\pgfqpoint{2.315050in}{2.024052in}}%
\pgfpathlineto{\pgfqpoint{2.316002in}{2.023921in}}%
\pgfpathlineto{\pgfqpoint{2.317590in}{2.026696in}}%
\pgfpathlineto{\pgfqpoint{2.318860in}{2.024869in}}%
\pgfpathlineto{\pgfqpoint{2.319177in}{2.026163in}}%
\pgfpathlineto{\pgfqpoint{2.320447in}{2.037606in}}%
\pgfpathlineto{\pgfqpoint{2.322352in}{2.021346in}}%
\pgfpathlineto{\pgfqpoint{2.323304in}{2.020610in}}%
\pgfpathlineto{\pgfqpoint{2.324891in}{2.023500in}}%
\pgfpathlineto{\pgfqpoint{2.326161in}{2.021631in}}%
\pgfpathlineto{\pgfqpoint{2.327431in}{2.029390in}}%
\pgfpathlineto{\pgfqpoint{2.327748in}{2.034955in}}%
\pgfpathlineto{\pgfqpoint{2.328383in}{2.027825in}}%
\pgfpathlineto{\pgfqpoint{2.330606in}{2.017267in}}%
\pgfpathlineto{\pgfqpoint{2.331558in}{2.019113in}}%
\pgfpathlineto{\pgfqpoint{2.332193in}{2.020290in}}%
\pgfpathlineto{\pgfqpoint{2.332828in}{2.020121in}}%
\pgfpathlineto{\pgfqpoint{2.333145in}{2.020092in}}%
\pgfpathlineto{\pgfqpoint{2.333463in}{2.018418in}}%
\pgfpathlineto{\pgfqpoint{2.333780in}{2.020036in}}%
\pgfpathlineto{\pgfqpoint{2.335050in}{2.032068in}}%
\pgfpathlineto{\pgfqpoint{2.335367in}{2.025414in}}%
\pgfpathlineto{\pgfqpoint{2.335685in}{2.026021in}}%
\pgfpathlineto{\pgfqpoint{2.336002in}{2.024975in}}%
\pgfpathlineto{\pgfqpoint{2.337590in}{2.014082in}}%
\pgfpathlineto{\pgfqpoint{2.337907in}{2.014062in}}%
\pgfpathlineto{\pgfqpoint{2.339494in}{2.016982in}}%
\pgfpathlineto{\pgfqpoint{2.340764in}{2.015293in}}%
\pgfpathlineto{\pgfqpoint{2.342351in}{2.028453in}}%
\pgfpathlineto{\pgfqpoint{2.343304in}{2.023379in}}%
\pgfpathlineto{\pgfqpoint{2.344891in}{2.010845in}}%
\pgfpathlineto{\pgfqpoint{2.346161in}{2.012549in}}%
\pgfpathlineto{\pgfqpoint{2.347431in}{2.013838in}}%
\pgfpathlineto{\pgfqpoint{2.347748in}{2.013736in}}%
\pgfpathlineto{\pgfqpoint{2.348701in}{2.014134in}}%
\pgfpathlineto{\pgfqpoint{2.349018in}{2.012336in}}%
\pgfpathlineto{\pgfqpoint{2.349653in}{2.024765in}}%
\pgfpathlineto{\pgfqpoint{2.350605in}{2.021819in}}%
\pgfpathlineto{\pgfqpoint{2.352193in}{2.007846in}}%
\pgfpathlineto{\pgfqpoint{2.353462in}{2.009088in}}%
\pgfpathlineto{\pgfqpoint{2.354732in}{2.010730in}}%
\pgfpathlineto{\pgfqpoint{2.356320in}{2.008916in}}%
\pgfpathlineto{\pgfqpoint{2.356955in}{2.020140in}}%
\pgfpathlineto{\pgfqpoint{2.357907in}{2.019790in}}%
\pgfpathlineto{\pgfqpoint{2.359494in}{2.005093in}}%
\pgfpathlineto{\pgfqpoint{2.360129in}{2.005760in}}%
\pgfpathlineto{\pgfqpoint{2.360447in}{2.004732in}}%
\pgfpathlineto{\pgfqpoint{2.362034in}{2.007585in}}%
\pgfpathlineto{\pgfqpoint{2.363621in}{2.005778in}}%
\pgfpathlineto{\pgfqpoint{2.365208in}{2.017351in}}%
\pgfpathlineto{\pgfqpoint{2.366796in}{2.002769in}}%
\pgfpathlineto{\pgfqpoint{2.367748in}{2.001224in}}%
\pgfpathlineto{\pgfqpoint{2.369335in}{2.004469in}}%
\pgfpathlineto{\pgfqpoint{2.370923in}{2.002647in}}%
\pgfpathlineto{\pgfqpoint{2.372510in}{2.014951in}}%
\pgfpathlineto{\pgfqpoint{2.372827in}{2.007857in}}%
\pgfpathlineto{\pgfqpoint{2.373145in}{2.009580in}}%
\pgfpathlineto{\pgfqpoint{2.373462in}{2.006143in}}%
\pgfpathlineto{\pgfqpoint{2.375050in}{1.997799in}}%
\pgfpathlineto{\pgfqpoint{2.376637in}{2.001308in}}%
\pgfpathlineto{\pgfqpoint{2.376954in}{2.000677in}}%
\pgfpathlineto{\pgfqpoint{2.378224in}{1.999592in}}%
\pgfpathlineto{\pgfqpoint{2.379811in}{2.011931in}}%
\pgfpathlineto{\pgfqpoint{2.380446in}{2.007766in}}%
\pgfpathlineto{\pgfqpoint{2.382351in}{1.994532in}}%
\pgfpathlineto{\pgfqpoint{2.382669in}{1.995112in}}%
\pgfpathlineto{\pgfqpoint{2.383938in}{1.998117in}}%
\pgfpathlineto{\pgfqpoint{2.385526in}{1.996536in}}%
\pgfpathlineto{\pgfqpoint{2.386796in}{2.005297in}}%
\pgfpathlineto{\pgfqpoint{2.387113in}{2.008608in}}%
\pgfpathlineto{\pgfqpoint{2.387430in}{2.001834in}}%
\pgfpathlineto{\pgfqpoint{2.387748in}{2.005884in}}%
\pgfpathlineto{\pgfqpoint{2.389970in}{1.991488in}}%
\pgfpathlineto{\pgfqpoint{2.391240in}{1.994922in}}%
\pgfpathlineto{\pgfqpoint{2.392827in}{1.993557in}}%
\pgfpathlineto{\pgfqpoint{2.393462in}{1.995525in}}%
\pgfpathlineto{\pgfqpoint{2.393780in}{1.993752in}}%
\pgfpathlineto{\pgfqpoint{2.394415in}{2.004787in}}%
\pgfpathlineto{\pgfqpoint{2.395049in}{2.003805in}}%
\pgfpathlineto{\pgfqpoint{2.397272in}{1.987992in}}%
\pgfpathlineto{\pgfqpoint{2.397589in}{1.990406in}}%
\pgfpathlineto{\pgfqpoint{2.397907in}{1.988803in}}%
\pgfpathlineto{\pgfqpoint{2.398224in}{1.990568in}}%
\pgfpathlineto{\pgfqpoint{2.399494in}{1.991575in}}%
\pgfpathlineto{\pgfqpoint{2.401081in}{1.990085in}}%
\pgfpathlineto{\pgfqpoint{2.402351in}{2.001260in}}%
\pgfpathlineto{\pgfqpoint{2.404573in}{1.984845in}}%
\pgfpathlineto{\pgfqpoint{2.405526in}{1.986786in}}%
\pgfpathlineto{\pgfqpoint{2.406795in}{1.988506in}}%
\pgfpathlineto{\pgfqpoint{2.408383in}{1.986997in}}%
\pgfpathlineto{\pgfqpoint{2.409652in}{1.998693in}}%
\pgfpathlineto{\pgfqpoint{2.409970in}{1.996561in}}%
\pgfpathlineto{\pgfqpoint{2.411557in}{1.983902in}}%
\pgfpathlineto{\pgfqpoint{2.412510in}{1.981827in}}%
\pgfpathlineto{\pgfqpoint{2.414097in}{1.985399in}}%
\pgfpathlineto{\pgfqpoint{2.414414in}{1.984256in}}%
\pgfpathlineto{\pgfqpoint{2.415367in}{1.984602in}}%
\pgfpathlineto{\pgfqpoint{2.415684in}{1.983998in}}%
\pgfpathlineto{\pgfqpoint{2.416954in}{1.995476in}}%
\pgfpathlineto{\pgfqpoint{2.417271in}{1.994086in}}%
\pgfpathlineto{\pgfqpoint{2.418859in}{1.981812in}}%
\pgfpathlineto{\pgfqpoint{2.419811in}{1.978551in}}%
\pgfpathlineto{\pgfqpoint{2.420129in}{1.979301in}}%
\pgfpathlineto{\pgfqpoint{2.421398in}{1.982318in}}%
\pgfpathlineto{\pgfqpoint{2.422668in}{1.981192in}}%
\pgfpathlineto{\pgfqpoint{2.422986in}{1.980997in}}%
\pgfpathlineto{\pgfqpoint{2.423621in}{1.985299in}}%
\pgfpathlineto{\pgfqpoint{2.423938in}{1.983852in}}%
\pgfpathlineto{\pgfqpoint{2.424256in}{1.992181in}}%
\pgfpathlineto{\pgfqpoint{2.425208in}{1.988778in}}%
\pgfpathlineto{\pgfqpoint{2.427113in}{1.975499in}}%
\pgfpathlineto{\pgfqpoint{2.427430in}{1.975836in}}%
\pgfpathlineto{\pgfqpoint{2.428700in}{1.979165in}}%
\pgfpathlineto{\pgfqpoint{2.429970in}{1.977822in}}%
\pgfpathlineto{\pgfqpoint{2.430605in}{1.978485in}}%
\pgfpathlineto{\pgfqpoint{2.431875in}{1.987916in}}%
\pgfpathlineto{\pgfqpoint{2.432509in}{1.986945in}}%
\pgfpathlineto{\pgfqpoint{2.434414in}{1.972522in}}%
\pgfpathlineto{\pgfqpoint{2.434732in}{1.972369in}}%
\pgfpathlineto{\pgfqpoint{2.436001in}{1.975954in}}%
\pgfpathlineto{\pgfqpoint{2.437271in}{1.974444in}}%
\pgfpathlineto{\pgfqpoint{2.438541in}{1.975581in}}%
\pgfpathlineto{\pgfqpoint{2.439811in}{1.984916in}}%
\pgfpathlineto{\pgfqpoint{2.442033in}{1.969093in}}%
\pgfpathlineto{\pgfqpoint{2.443303in}{1.972673in}}%
\pgfpathlineto{\pgfqpoint{2.444573in}{1.971159in}}%
\pgfpathlineto{\pgfqpoint{2.446160in}{1.977984in}}%
\pgfpathlineto{\pgfqpoint{2.446795in}{1.977147in}}%
\pgfpathlineto{\pgfqpoint{2.447112in}{1.982711in}}%
\pgfpathlineto{\pgfqpoint{2.449335in}{1.965930in}}%
\pgfpathlineto{\pgfqpoint{2.450605in}{1.969334in}}%
\pgfpathlineto{\pgfqpoint{2.450922in}{1.967637in}}%
\pgfpathlineto{\pgfqpoint{2.451239in}{1.969470in}}%
\pgfpathlineto{\pgfqpoint{2.451874in}{1.967918in}}%
\pgfpathlineto{\pgfqpoint{2.454414in}{1.979755in}}%
\pgfpathlineto{\pgfqpoint{2.454731in}{1.976400in}}%
\pgfpathlineto{\pgfqpoint{2.456636in}{1.963058in}}%
\pgfpathlineto{\pgfqpoint{2.457271in}{1.962814in}}%
\pgfpathlineto{\pgfqpoint{2.457906in}{1.965844in}}%
\pgfpathlineto{\pgfqpoint{2.458224in}{1.964259in}}%
\pgfpathlineto{\pgfqpoint{2.458858in}{1.966376in}}%
\pgfpathlineto{\pgfqpoint{2.459176in}{1.964738in}}%
\pgfpathlineto{\pgfqpoint{2.461716in}{1.976105in}}%
\pgfpathlineto{\pgfqpoint{2.462033in}{1.973644in}}%
\pgfpathlineto{\pgfqpoint{2.463938in}{1.960221in}}%
\pgfpathlineto{\pgfqpoint{2.464573in}{1.959640in}}%
\pgfpathlineto{\pgfqpoint{2.465208in}{1.962283in}}%
\pgfpathlineto{\pgfqpoint{2.465525in}{1.960874in}}%
\pgfpathlineto{\pgfqpoint{2.465842in}{1.962747in}}%
\pgfpathlineto{\pgfqpoint{2.466160in}{1.963212in}}%
\pgfpathlineto{\pgfqpoint{2.467430in}{1.961406in}}%
\pgfpathlineto{\pgfqpoint{2.468700in}{1.966274in}}%
\pgfpathlineto{\pgfqpoint{2.469017in}{1.972144in}}%
\pgfpathlineto{\pgfqpoint{2.469969in}{1.969722in}}%
\pgfpathlineto{\pgfqpoint{2.471874in}{1.956489in}}%
\pgfpathlineto{\pgfqpoint{2.474096in}{1.959853in}}%
\pgfpathlineto{\pgfqpoint{2.474414in}{1.959606in}}%
\pgfpathlineto{\pgfqpoint{2.474731in}{1.958088in}}%
\pgfpathlineto{\pgfqpoint{2.475366in}{1.959502in}}%
\pgfpathlineto{\pgfqpoint{2.475684in}{1.959411in}}%
\pgfpathlineto{\pgfqpoint{2.476319in}{1.969305in}}%
\pgfpathlineto{\pgfqpoint{2.477271in}{1.967221in}}%
\pgfpathlineto{\pgfqpoint{2.479176in}{1.953530in}}%
\pgfpathlineto{\pgfqpoint{2.480446in}{1.955660in}}%
\pgfpathlineto{\pgfqpoint{2.481398in}{1.956711in}}%
\pgfpathlineto{\pgfqpoint{2.481080in}{1.955185in}}%
\pgfpathlineto{\pgfqpoint{2.481715in}{1.956545in}}%
\pgfpathlineto{\pgfqpoint{2.482033in}{1.954887in}}%
\pgfpathlineto{\pgfqpoint{2.482985in}{1.955137in}}%
\pgfpathlineto{\pgfqpoint{2.483620in}{1.965276in}}%
\pgfpathlineto{\pgfqpoint{2.484890in}{1.961154in}}%
\pgfpathlineto{\pgfqpoint{2.486795in}{1.950455in}}%
\pgfpathlineto{\pgfqpoint{2.488699in}{1.953484in}}%
\pgfpathlineto{\pgfqpoint{2.489017in}{1.953427in}}%
\pgfpathlineto{\pgfqpoint{2.490287in}{1.951786in}}%
\pgfpathlineto{\pgfqpoint{2.490604in}{1.952826in}}%
\pgfpathlineto{\pgfqpoint{2.491874in}{1.963217in}}%
\pgfpathlineto{\pgfqpoint{2.493779in}{1.948024in}}%
\pgfpathlineto{\pgfqpoint{2.494731in}{1.947267in}}%
\pgfpathlineto{\pgfqpoint{2.496318in}{1.950284in}}%
\pgfpathlineto{\pgfqpoint{2.497588in}{1.948394in}}%
\pgfpathlineto{\pgfqpoint{2.498858in}{1.955798in}}%
\pgfpathlineto{\pgfqpoint{2.499176in}{1.961422in}}%
\pgfpathlineto{\pgfqpoint{2.499810in}{1.953632in}}%
\pgfpathlineto{\pgfqpoint{2.500128in}{1.953327in}}%
\pgfpathlineto{\pgfqpoint{2.502033in}{1.943901in}}%
\pgfpathlineto{\pgfqpoint{2.503620in}{1.947044in}}%
\pgfpathlineto{\pgfqpoint{2.504890in}{1.945100in}}%
\pgfpathlineto{\pgfqpoint{2.506160in}{1.951007in}}%
\pgfpathlineto{\pgfqpoint{2.506477in}{1.958573in}}%
\pgfpathlineto{\pgfqpoint{2.507112in}{1.952087in}}%
\pgfpathlineto{\pgfqpoint{2.507429in}{1.952014in}}%
\pgfpathlineto{\pgfqpoint{2.509334in}{1.940570in}}%
\pgfpathlineto{\pgfqpoint{2.510921in}{1.943756in}}%
\pgfpathlineto{\pgfqpoint{2.512191in}{1.941894in}}%
\pgfpathlineto{\pgfqpoint{2.513461in}{1.946334in}}%
\pgfpathlineto{\pgfqpoint{2.513779in}{1.954966in}}%
\pgfpathlineto{\pgfqpoint{2.514731in}{1.950793in}}%
\pgfpathlineto{\pgfqpoint{2.516636in}{1.937483in}}%
\pgfpathlineto{\pgfqpoint{2.518858in}{1.940562in}}%
\pgfpathlineto{\pgfqpoint{2.519175in}{1.940438in}}%
\pgfpathlineto{\pgfqpoint{2.520445in}{1.938750in}}%
\pgfpathlineto{\pgfqpoint{2.522032in}{1.950049in}}%
\pgfpathlineto{\pgfqpoint{2.523620in}{1.935115in}}%
\pgfpathlineto{\pgfqpoint{2.524572in}{1.934869in}}%
\pgfpathlineto{\pgfqpoint{2.524890in}{1.935081in}}%
\pgfpathlineto{\pgfqpoint{2.526159in}{1.937345in}}%
\pgfpathlineto{\pgfqpoint{2.526477in}{1.937225in}}%
\pgfpathlineto{\pgfqpoint{2.527747in}{1.935460in}}%
\pgfpathlineto{\pgfqpoint{2.529334in}{1.948286in}}%
\pgfpathlineto{\pgfqpoint{2.529651in}{1.941499in}}%
\pgfpathlineto{\pgfqpoint{2.531874in}{1.931211in}}%
\pgfpathlineto{\pgfqpoint{2.532509in}{1.932934in}}%
\pgfpathlineto{\pgfqpoint{2.534731in}{1.933838in}}%
\pgfpathlineto{\pgfqpoint{2.535048in}{1.932180in}}%
\pgfpathlineto{\pgfqpoint{2.535366in}{1.933788in}}%
\pgfpathlineto{\pgfqpoint{2.536636in}{1.946041in}}%
\pgfpathlineto{\pgfqpoint{2.538223in}{1.930522in}}%
\pgfpathlineto{\pgfqpoint{2.539175in}{1.927756in}}%
\pgfpathlineto{\pgfqpoint{2.539493in}{1.928118in}}%
\pgfpathlineto{\pgfqpoint{2.540762in}{1.930810in}}%
\pgfpathlineto{\pgfqpoint{2.541080in}{1.930658in}}%
\pgfpathlineto{\pgfqpoint{2.542350in}{1.929000in}}%
\pgfpathlineto{\pgfqpoint{2.543620in}{1.935488in}}%
\pgfpathlineto{\pgfqpoint{2.543937in}{1.942571in}}%
\pgfpathlineto{\pgfqpoint{2.544572in}{1.937484in}}%
\pgfpathlineto{\pgfqpoint{2.544889in}{1.936556in}}%
\pgfpathlineto{\pgfqpoint{2.546477in}{1.924439in}}%
\pgfpathlineto{\pgfqpoint{2.549334in}{1.927377in}}%
\pgfpathlineto{\pgfqpoint{2.550604in}{1.925841in}}%
\pgfpathlineto{\pgfqpoint{2.551239in}{1.938358in}}%
\pgfpathlineto{\pgfqpoint{2.552191in}{1.936001in}}%
\pgfpathlineto{\pgfqpoint{2.553778in}{1.921543in}}%
\pgfpathlineto{\pgfqpoint{2.554731in}{1.921878in}}%
\pgfpathlineto{\pgfqpoint{2.556318in}{1.924360in}}%
\pgfpathlineto{\pgfqpoint{2.556635in}{1.924113in}}%
\pgfpathlineto{\pgfqpoint{2.557905in}{1.922529in}}%
\pgfpathlineto{\pgfqpoint{2.559492in}{1.934279in}}%
\pgfpathlineto{\pgfqpoint{2.561080in}{1.918940in}}%
\pgfpathlineto{\pgfqpoint{2.561397in}{1.918014in}}%
\pgfpathlineto{\pgfqpoint{2.561715in}{1.919303in}}%
\pgfpathlineto{\pgfqpoint{2.562032in}{1.918265in}}%
\pgfpathlineto{\pgfqpoint{2.563619in}{1.921183in}}%
\pgfpathlineto{\pgfqpoint{2.565207in}{1.919327in}}%
\pgfpathlineto{\pgfqpoint{2.566794in}{1.932164in}}%
\pgfpathlineto{\pgfqpoint{2.567429in}{1.926006in}}%
\pgfpathlineto{\pgfqpoint{2.569334in}{1.914653in}}%
\pgfpathlineto{\pgfqpoint{2.570921in}{1.917999in}}%
\pgfpathlineto{\pgfqpoint{2.571238in}{1.917480in}}%
\pgfpathlineto{\pgfqpoint{2.572508in}{1.916193in}}%
\pgfpathlineto{\pgfqpoint{2.573778in}{1.925008in}}%
\pgfpathlineto{\pgfqpoint{2.574096in}{1.929293in}}%
\pgfpathlineto{\pgfqpoint{2.574413in}{1.921786in}}%
\pgfpathlineto{\pgfqpoint{2.574730in}{1.924446in}}%
\pgfpathlineto{\pgfqpoint{2.576635in}{1.911317in}}%
\pgfpathlineto{\pgfqpoint{2.576953in}{1.911944in}}%
\pgfpathlineto{\pgfqpoint{2.578222in}{1.914742in}}%
\pgfpathlineto{\pgfqpoint{2.579810in}{1.913129in}}%
\pgfpathlineto{\pgfqpoint{2.581080in}{1.919947in}}%
\pgfpathlineto{\pgfqpoint{2.581397in}{1.925542in}}%
\pgfpathlineto{\pgfqpoint{2.581715in}{1.918705in}}%
\pgfpathlineto{\pgfqpoint{2.582032in}{1.922682in}}%
\pgfpathlineto{\pgfqpoint{2.584254in}{1.908026in}}%
\pgfpathlineto{\pgfqpoint{2.585207in}{1.910678in}}%
\pgfpathlineto{\pgfqpoint{2.586476in}{1.911300in}}%
\pgfpathlineto{\pgfqpoint{2.588064in}{1.909689in}}%
\pgfpathlineto{\pgfqpoint{2.588699in}{1.921175in}}%
\pgfpathlineto{\pgfqpoint{2.589651in}{1.919469in}}%
\pgfpathlineto{\pgfqpoint{2.591238in}{1.905835in}}%
\pgfpathlineto{\pgfqpoint{2.591556in}{1.904561in}}%
\pgfpathlineto{\pgfqpoint{2.591873in}{1.906732in}}%
\pgfpathlineto{\pgfqpoint{2.592191in}{1.905324in}}%
\pgfpathlineto{\pgfqpoint{2.593778in}{1.908208in}}%
\pgfpathlineto{\pgfqpoint{2.595365in}{1.906550in}}%
\pgfpathlineto{\pgfqpoint{2.596952in}{1.917757in}}%
\pgfpathlineto{\pgfqpoint{2.598540in}{1.903746in}}%
\pgfpathlineto{\pgfqpoint{2.598857in}{1.901449in}}%
\pgfpathlineto{\pgfqpoint{2.599492in}{1.901688in}}%
\pgfpathlineto{\pgfqpoint{2.601079in}{1.905071in}}%
\pgfpathlineto{\pgfqpoint{2.602667in}{1.903479in}}%
\pgfpathlineto{\pgfqpoint{2.604254in}{1.915238in}}%
\pgfpathlineto{\pgfqpoint{2.605841in}{1.902175in}}%
\pgfpathlineto{\pgfqpoint{2.606794in}{1.898272in}}%
\pgfpathlineto{\pgfqpoint{2.607111in}{1.898840in}}%
\pgfpathlineto{\pgfqpoint{2.608381in}{1.901928in}}%
\pgfpathlineto{\pgfqpoint{2.609968in}{1.900425in}}%
\pgfpathlineto{\pgfqpoint{2.610603in}{1.904871in}}%
\pgfpathlineto{\pgfqpoint{2.610921in}{1.902122in}}%
\pgfpathlineto{\pgfqpoint{2.611556in}{1.912058in}}%
\pgfpathlineto{\pgfqpoint{2.612190in}{1.909385in}}%
\pgfpathlineto{\pgfqpoint{2.614413in}{1.895023in}}%
\pgfpathlineto{\pgfqpoint{2.615683in}{1.898707in}}%
\pgfpathlineto{\pgfqpoint{2.616952in}{1.897659in}}%
\pgfpathlineto{\pgfqpoint{2.617270in}{1.897401in}}%
\pgfpathlineto{\pgfqpoint{2.617587in}{1.898137in}}%
\pgfpathlineto{\pgfqpoint{2.617905in}{1.899106in}}%
\pgfpathlineto{\pgfqpoint{2.618222in}{1.897731in}}%
\pgfpathlineto{\pgfqpoint{2.618857in}{1.908275in}}%
\pgfpathlineto{\pgfqpoint{2.619492in}{1.907519in}}%
\pgfpathlineto{\pgfqpoint{2.621714in}{1.891434in}}%
\pgfpathlineto{\pgfqpoint{2.622032in}{1.894001in}}%
\pgfpathlineto{\pgfqpoint{2.622349in}{1.892332in}}%
\pgfpathlineto{\pgfqpoint{2.622667in}{1.893977in}}%
\pgfpathlineto{\pgfqpoint{2.623619in}{1.895374in}}%
\pgfpathlineto{\pgfqpoint{2.623936in}{1.895172in}}%
\pgfpathlineto{\pgfqpoint{2.625524in}{1.893849in}}%
\pgfpathlineto{\pgfqpoint{2.626794in}{1.904934in}}%
\pgfpathlineto{\pgfqpoint{2.629016in}{1.888349in}}%
\pgfpathlineto{\pgfqpoint{2.629968in}{1.889996in}}%
\pgfpathlineto{\pgfqpoint{2.630920in}{1.892166in}}%
\pgfpathlineto{\pgfqpoint{2.631238in}{1.892098in}}%
\pgfpathlineto{\pgfqpoint{2.631555in}{1.890904in}}%
\pgfpathlineto{\pgfqpoint{2.632508in}{1.891287in}}%
\pgfpathlineto{\pgfqpoint{2.632825in}{1.890766in}}%
\pgfpathlineto{\pgfqpoint{2.634095in}{1.902089in}}%
\pgfpathlineto{\pgfqpoint{2.634413in}{1.900730in}}%
\pgfpathlineto{\pgfqpoint{2.636000in}{1.888508in}}%
\pgfpathlineto{\pgfqpoint{2.636952in}{1.885367in}}%
\pgfpathlineto{\pgfqpoint{2.637270in}{1.885908in}}%
\pgfpathlineto{\pgfqpoint{2.638539in}{1.888980in}}%
\pgfpathlineto{\pgfqpoint{2.638857in}{1.887598in}}%
\pgfpathlineto{\pgfqpoint{2.639809in}{1.887827in}}%
\pgfpathlineto{\pgfqpoint{2.640127in}{1.887735in}}%
\pgfpathlineto{\pgfqpoint{2.640762in}{1.892135in}}%
\pgfpathlineto{\pgfqpoint{2.641079in}{1.890447in}}%
\pgfpathlineto{\pgfqpoint{2.641397in}{1.898189in}}%
\pgfpathlineto{\pgfqpoint{2.642349in}{1.895749in}}%
\pgfpathlineto{\pgfqpoint{2.644254in}{1.882255in}}%
\pgfpathlineto{\pgfqpoint{2.644571in}{1.882183in}}%
\pgfpathlineto{\pgfqpoint{2.645841in}{1.885804in}}%
\pgfpathlineto{\pgfqpoint{2.647111in}{1.884359in}}%
\pgfpathlineto{\pgfqpoint{2.648063in}{1.886213in}}%
\pgfpathlineto{\pgfqpoint{2.648381in}{1.885902in}}%
\pgfpathlineto{\pgfqpoint{2.649016in}{1.894594in}}%
\pgfpathlineto{\pgfqpoint{2.649650in}{1.894069in}}%
\pgfpathlineto{\pgfqpoint{2.651873in}{1.878634in}}%
\pgfpathlineto{\pgfqpoint{2.653143in}{1.882417in}}%
\pgfpathlineto{\pgfqpoint{2.654412in}{1.880973in}}%
\pgfpathlineto{\pgfqpoint{2.655682in}{1.881558in}}%
\pgfpathlineto{\pgfqpoint{2.656952in}{1.892073in}}%
\pgfpathlineto{\pgfqpoint{2.659174in}{1.875428in}}%
\pgfpathlineto{\pgfqpoint{2.660444in}{1.878992in}}%
\pgfpathlineto{\pgfqpoint{2.660761in}{1.877398in}}%
\pgfpathlineto{\pgfqpoint{2.661079in}{1.879172in}}%
\pgfpathlineto{\pgfqpoint{2.661714in}{1.877666in}}%
\pgfpathlineto{\pgfqpoint{2.663619in}{1.884557in}}%
\pgfpathlineto{\pgfqpoint{2.663936in}{1.882678in}}%
\pgfpathlineto{\pgfqpoint{2.664254in}{1.889197in}}%
\pgfpathlineto{\pgfqpoint{2.664888in}{1.881518in}}%
\pgfpathlineto{\pgfqpoint{2.665206in}{1.883412in}}%
\pgfpathlineto{\pgfqpoint{2.665523in}{1.878602in}}%
\pgfpathlineto{\pgfqpoint{2.667111in}{1.872424in}}%
\pgfpathlineto{\pgfqpoint{2.668698in}{1.876024in}}%
\pgfpathlineto{\pgfqpoint{2.669650in}{1.875427in}}%
\pgfpathlineto{\pgfqpoint{2.669968in}{1.874513in}}%
\pgfpathlineto{\pgfqpoint{2.670285in}{1.875001in}}%
\pgfpathlineto{\pgfqpoint{2.671555in}{1.885534in}}%
\pgfpathlineto{\pgfqpoint{2.672507in}{1.882132in}}%
\pgfpathlineto{\pgfqpoint{2.674412in}{1.869386in}}%
\pgfpathlineto{\pgfqpoint{2.674730in}{1.869384in}}%
\pgfpathlineto{\pgfqpoint{2.675999in}{1.872789in}}%
\pgfpathlineto{\pgfqpoint{2.677269in}{1.871071in}}%
\pgfpathlineto{\pgfqpoint{2.678539in}{1.874434in}}%
\pgfpathlineto{\pgfqpoint{2.679809in}{1.880741in}}%
\pgfpathlineto{\pgfqpoint{2.681714in}{1.866587in}}%
\pgfpathlineto{\pgfqpoint{2.682031in}{1.865808in}}%
\pgfpathlineto{\pgfqpoint{2.682349in}{1.867924in}}%
\pgfpathlineto{\pgfqpoint{2.682666in}{1.866497in}}%
\pgfpathlineto{\pgfqpoint{2.683936in}{1.869444in}}%
\pgfpathlineto{\pgfqpoint{2.684253in}{1.869229in}}%
\pgfpathlineto{\pgfqpoint{2.684571in}{1.867739in}}%
\pgfpathlineto{\pgfqpoint{2.685523in}{1.868254in}}%
\pgfpathlineto{\pgfqpoint{2.685841in}{1.869542in}}%
\pgfpathlineto{\pgfqpoint{2.687110in}{1.878560in}}%
\pgfpathlineto{\pgfqpoint{2.689333in}{1.862686in}}%
\pgfpathlineto{\pgfqpoint{2.690603in}{1.865908in}}%
\pgfpathlineto{\pgfqpoint{2.691237in}{1.866132in}}%
\pgfpathlineto{\pgfqpoint{2.691872in}{1.864514in}}%
\pgfpathlineto{\pgfqpoint{2.694412in}{1.875453in}}%
\pgfpathlineto{\pgfqpoint{2.694729in}{1.872787in}}%
\pgfpathlineto{\pgfqpoint{2.696634in}{1.859790in}}%
\pgfpathlineto{\pgfqpoint{2.697269in}{1.859605in}}%
\pgfpathlineto{\pgfqpoint{2.697904in}{1.862206in}}%
\pgfpathlineto{\pgfqpoint{2.698221in}{1.860880in}}%
\pgfpathlineto{\pgfqpoint{2.698856in}{1.862998in}}%
\pgfpathlineto{\pgfqpoint{2.699174in}{1.861347in}}%
\pgfpathlineto{\pgfqpoint{2.701714in}{1.872546in}}%
\pgfpathlineto{\pgfqpoint{2.702031in}{1.870176in}}%
\pgfpathlineto{\pgfqpoint{2.703936in}{1.857221in}}%
\pgfpathlineto{\pgfqpoint{2.704253in}{1.857547in}}%
\pgfpathlineto{\pgfqpoint{2.704571in}{1.856379in}}%
\pgfpathlineto{\pgfqpoint{2.704888in}{1.856707in}}%
\pgfpathlineto{\pgfqpoint{2.706158in}{1.859734in}}%
\pgfpathlineto{\pgfqpoint{2.707428in}{1.857818in}}%
\pgfpathlineto{\pgfqpoint{2.708698in}{1.862213in}}%
\pgfpathlineto{\pgfqpoint{2.709015in}{1.869579in}}%
\pgfpathlineto{\pgfqpoint{2.709967in}{1.866316in}}%
\pgfpathlineto{\pgfqpoint{2.711872in}{1.853453in}}%
\pgfpathlineto{\pgfqpoint{2.712190in}{1.853258in}}%
\pgfpathlineto{\pgfqpoint{2.713459in}{1.856278in}}%
\pgfpathlineto{\pgfqpoint{2.714729in}{1.854574in}}%
\pgfpathlineto{\pgfqpoint{2.716317in}{1.865229in}}%
\pgfpathlineto{\pgfqpoint{2.717269in}{1.864724in}}%
\pgfpathlineto{\pgfqpoint{2.719491in}{1.850014in}}%
\pgfpathlineto{\pgfqpoint{2.720761in}{1.852723in}}%
\pgfpathlineto{\pgfqpoint{2.721396in}{1.853073in}}%
\pgfpathlineto{\pgfqpoint{2.722031in}{1.851444in}}%
\pgfpathlineto{\pgfqpoint{2.724570in}{1.863232in}}%
\pgfpathlineto{\pgfqpoint{2.724888in}{1.858528in}}%
\pgfpathlineto{\pgfqpoint{2.726793in}{1.847028in}}%
\pgfpathlineto{\pgfqpoint{2.729015in}{1.849886in}}%
\pgfpathlineto{\pgfqpoint{2.730285in}{1.847939in}}%
\pgfpathlineto{\pgfqpoint{2.731555in}{1.854484in}}%
\pgfpathlineto{\pgfqpoint{2.731872in}{1.861260in}}%
\pgfpathlineto{\pgfqpoint{2.732507in}{1.853853in}}%
\pgfpathlineto{\pgfqpoint{2.732824in}{1.854001in}}%
\pgfpathlineto{\pgfqpoint{2.734729in}{1.843494in}}%
\pgfpathlineto{\pgfqpoint{2.736951in}{1.846495in}}%
\pgfpathlineto{\pgfqpoint{2.737269in}{1.846333in}}%
\pgfpathlineto{\pgfqpoint{2.737586in}{1.844647in}}%
\pgfpathlineto{\pgfqpoint{2.738539in}{1.845448in}}%
\pgfpathlineto{\pgfqpoint{2.739174in}{1.857737in}}%
\pgfpathlineto{\pgfqpoint{2.740126in}{1.853278in}}%
\pgfpathlineto{\pgfqpoint{2.742031in}{1.840292in}}%
\pgfpathlineto{\pgfqpoint{2.744570in}{1.843154in}}%
\pgfpathlineto{\pgfqpoint{2.745840in}{1.841494in}}%
\pgfpathlineto{\pgfqpoint{2.746158in}{1.843448in}}%
\pgfpathlineto{\pgfqpoint{2.746475in}{1.852595in}}%
\pgfpathlineto{\pgfqpoint{2.747427in}{1.852553in}}%
\pgfpathlineto{\pgfqpoint{2.749332in}{1.837484in}}%
\pgfpathlineto{\pgfqpoint{2.750285in}{1.837502in}}%
\pgfpathlineto{\pgfqpoint{2.751872in}{1.839954in}}%
\pgfpathlineto{\pgfqpoint{2.753142in}{1.838079in}}%
\pgfpathlineto{\pgfqpoint{2.754411in}{1.845996in}}%
\pgfpathlineto{\pgfqpoint{2.754729in}{1.851546in}}%
\pgfpathlineto{\pgfqpoint{2.755046in}{1.844907in}}%
\pgfpathlineto{\pgfqpoint{2.757586in}{1.833933in}}%
\pgfpathlineto{\pgfqpoint{2.759173in}{1.836648in}}%
\pgfpathlineto{\pgfqpoint{2.760443in}{1.834759in}}%
\pgfpathlineto{\pgfqpoint{2.761713in}{1.840926in}}%
\pgfpathlineto{\pgfqpoint{2.762030in}{1.848899in}}%
\pgfpathlineto{\pgfqpoint{2.762665in}{1.841899in}}%
\pgfpathlineto{\pgfqpoint{2.762983in}{1.841317in}}%
\pgfpathlineto{\pgfqpoint{2.764570in}{1.830528in}}%
\pgfpathlineto{\pgfqpoint{2.765205in}{1.831379in}}%
\pgfpathlineto{\pgfqpoint{2.767427in}{1.833208in}}%
\pgfpathlineto{\pgfqpoint{2.768697in}{1.831593in}}%
\pgfpathlineto{\pgfqpoint{2.769332in}{1.844237in}}%
\pgfpathlineto{\pgfqpoint{2.770284in}{1.841635in}}%
\pgfpathlineto{\pgfqpoint{2.771872in}{1.827518in}}%
\pgfpathlineto{\pgfqpoint{2.773141in}{1.828203in}}%
\pgfpathlineto{\pgfqpoint{2.774411in}{1.830121in}}%
\pgfpathlineto{\pgfqpoint{2.774729in}{1.829977in}}%
\pgfpathlineto{\pgfqpoint{2.775999in}{1.828233in}}%
\pgfpathlineto{\pgfqpoint{2.776316in}{1.830028in}}%
\pgfpathlineto{\pgfqpoint{2.777586in}{1.840821in}}%
\pgfpathlineto{\pgfqpoint{2.779173in}{1.824781in}}%
\pgfpathlineto{\pgfqpoint{2.780126in}{1.824090in}}%
\pgfpathlineto{\pgfqpoint{2.780443in}{1.824573in}}%
\pgfpathlineto{\pgfqpoint{2.781713in}{1.826822in}}%
\pgfpathlineto{\pgfqpoint{2.782030in}{1.826685in}}%
\pgfpathlineto{\pgfqpoint{2.783300in}{1.824900in}}%
\pgfpathlineto{\pgfqpoint{2.784570in}{1.832352in}}%
\pgfpathlineto{\pgfqpoint{2.784887in}{1.839072in}}%
\pgfpathlineto{\pgfqpoint{2.785522in}{1.831587in}}%
\pgfpathlineto{\pgfqpoint{2.787427in}{1.820466in}}%
\pgfpathlineto{\pgfqpoint{2.788379in}{1.822094in}}%
\pgfpathlineto{\pgfqpoint{2.790284in}{1.823270in}}%
\pgfpathlineto{\pgfqpoint{2.790602in}{1.821698in}}%
\pgfpathlineto{\pgfqpoint{2.790919in}{1.823206in}}%
\pgfpathlineto{\pgfqpoint{2.791554in}{1.822408in}}%
\pgfpathlineto{\pgfqpoint{2.791871in}{1.827348in}}%
\pgfpathlineto{\pgfqpoint{2.792189in}{1.835478in}}%
\pgfpathlineto{\pgfqpoint{2.792824in}{1.830158in}}%
\pgfpathlineto{\pgfqpoint{2.793141in}{1.829908in}}%
\pgfpathlineto{\pgfqpoint{2.794729in}{1.817188in}}%
\pgfpathlineto{\pgfqpoint{2.797586in}{1.819992in}}%
\pgfpathlineto{\pgfqpoint{2.798856in}{1.818344in}}%
\pgfpathlineto{\pgfqpoint{2.799490in}{1.830226in}}%
\pgfpathlineto{\pgfqpoint{2.800443in}{1.829942in}}%
\pgfpathlineto{\pgfqpoint{2.802030in}{1.814391in}}%
\pgfpathlineto{\pgfqpoint{2.802665in}{1.814971in}}%
\pgfpathlineto{\pgfqpoint{2.802983in}{1.814341in}}%
\pgfpathlineto{\pgfqpoint{2.804570in}{1.816949in}}%
\pgfpathlineto{\pgfqpoint{2.804887in}{1.816663in}}%
\pgfpathlineto{\pgfqpoint{2.806157in}{1.815064in}}%
\pgfpathlineto{\pgfqpoint{2.807744in}{1.828460in}}%
\pgfpathlineto{\pgfqpoint{2.808379in}{1.820843in}}%
\pgfpathlineto{\pgfqpoint{2.810284in}{1.810613in}}%
\pgfpathlineto{\pgfqpoint{2.811871in}{1.813691in}}%
\pgfpathlineto{\pgfqpoint{2.812189in}{1.813300in}}%
\pgfpathlineto{\pgfqpoint{2.813459in}{1.811843in}}%
\pgfpathlineto{\pgfqpoint{2.814728in}{1.819255in}}%
\pgfpathlineto{\pgfqpoint{2.815046in}{1.825703in}}%
\pgfpathlineto{\pgfqpoint{2.815363in}{1.817841in}}%
\pgfpathlineto{\pgfqpoint{2.815681in}{1.820045in}}%
\pgfpathlineto{\pgfqpoint{2.816633in}{1.811210in}}%
\pgfpathlineto{\pgfqpoint{2.817586in}{1.807147in}}%
\pgfpathlineto{\pgfqpoint{2.817903in}{1.807733in}}%
\pgfpathlineto{\pgfqpoint{2.819173in}{1.810322in}}%
\pgfpathlineto{\pgfqpoint{2.819490in}{1.809836in}}%
\pgfpathlineto{\pgfqpoint{2.821395in}{1.810584in}}%
\pgfpathlineto{\pgfqpoint{2.821713in}{1.808616in}}%
\pgfpathlineto{\pgfqpoint{2.822030in}{1.813582in}}%
\pgfpathlineto{\pgfqpoint{2.822347in}{1.821386in}}%
\pgfpathlineto{\pgfqpoint{2.823300in}{1.818311in}}%
\pgfpathlineto{\pgfqpoint{2.824887in}{1.804113in}}%
\pgfpathlineto{\pgfqpoint{2.825205in}{1.803810in}}%
\pgfpathlineto{\pgfqpoint{2.826474in}{1.806818in}}%
\pgfpathlineto{\pgfqpoint{2.828062in}{1.805790in}}%
\pgfpathlineto{\pgfqpoint{2.829014in}{1.805210in}}%
\pgfpathlineto{\pgfqpoint{2.829332in}{1.807688in}}%
\pgfpathlineto{\pgfqpoint{2.830601in}{1.817537in}}%
\pgfpathlineto{\pgfqpoint{2.832189in}{1.802050in}}%
\pgfpathlineto{\pgfqpoint{2.832506in}{1.800394in}}%
\pgfpathlineto{\pgfqpoint{2.833141in}{1.800789in}}%
\pgfpathlineto{\pgfqpoint{2.834728in}{1.803828in}}%
\pgfpathlineto{\pgfqpoint{2.836316in}{1.802025in}}%
\pgfpathlineto{\pgfqpoint{2.837585in}{1.810947in}}%
\pgfpathlineto{\pgfqpoint{2.837903in}{1.815334in}}%
\pgfpathlineto{\pgfqpoint{2.838220in}{1.807458in}}%
\pgfpathlineto{\pgfqpoint{2.838538in}{1.809594in}}%
\pgfpathlineto{\pgfqpoint{2.840443in}{1.797201in}}%
\pgfpathlineto{\pgfqpoint{2.840760in}{1.797939in}}%
\pgfpathlineto{\pgfqpoint{2.842030in}{1.800541in}}%
\pgfpathlineto{\pgfqpoint{2.843617in}{1.798901in}}%
\pgfpathlineto{\pgfqpoint{2.844887in}{1.805189in}}%
\pgfpathlineto{\pgfqpoint{2.845204in}{1.811733in}}%
\pgfpathlineto{\pgfqpoint{2.845522in}{1.804531in}}%
\pgfpathlineto{\pgfqpoint{2.845839in}{1.808203in}}%
\pgfpathlineto{\pgfqpoint{2.846792in}{1.800100in}}%
\pgfpathlineto{\pgfqpoint{2.848062in}{1.793757in}}%
\pgfpathlineto{\pgfqpoint{2.849331in}{1.797204in}}%
\pgfpathlineto{\pgfqpoint{2.850919in}{1.795907in}}%
\pgfpathlineto{\pgfqpoint{2.851871in}{1.795327in}}%
\pgfpathlineto{\pgfqpoint{2.852188in}{1.799627in}}%
\pgfpathlineto{\pgfqpoint{2.852506in}{1.806547in}}%
\pgfpathlineto{\pgfqpoint{2.853458in}{1.806239in}}%
\pgfpathlineto{\pgfqpoint{2.855046in}{1.791796in}}%
\pgfpathlineto{\pgfqpoint{2.855363in}{1.790109in}}%
\pgfpathlineto{\pgfqpoint{2.855680in}{1.792264in}}%
\pgfpathlineto{\pgfqpoint{2.855998in}{1.790990in}}%
\pgfpathlineto{\pgfqpoint{2.857585in}{1.793890in}}%
\pgfpathlineto{\pgfqpoint{2.859173in}{1.792214in}}%
\pgfpathlineto{\pgfqpoint{2.860760in}{1.804733in}}%
\pgfpathlineto{\pgfqpoint{2.861395in}{1.798998in}}%
\pgfpathlineto{\pgfqpoint{2.863299in}{1.787300in}}%
\pgfpathlineto{\pgfqpoint{2.864887in}{1.790694in}}%
\pgfpathlineto{\pgfqpoint{2.865522in}{1.790074in}}%
\pgfpathlineto{\pgfqpoint{2.866157in}{1.789990in}}%
\pgfpathlineto{\pgfqpoint{2.866474in}{1.789106in}}%
\pgfpathlineto{\pgfqpoint{2.866792in}{1.790201in}}%
\pgfpathlineto{\pgfqpoint{2.868061in}{1.801728in}}%
\pgfpathlineto{\pgfqpoint{2.869649in}{1.789551in}}%
\pgfpathlineto{\pgfqpoint{2.870918in}{1.783777in}}%
\pgfpathlineto{\pgfqpoint{2.872188in}{1.787429in}}%
\pgfpathlineto{\pgfqpoint{2.873458in}{1.786504in}}%
\pgfpathlineto{\pgfqpoint{2.873776in}{1.786059in}}%
\pgfpathlineto{\pgfqpoint{2.874093in}{1.786876in}}%
\pgfpathlineto{\pgfqpoint{2.874728in}{1.785715in}}%
\pgfpathlineto{\pgfqpoint{2.875045in}{1.791500in}}%
\pgfpathlineto{\pgfqpoint{2.875363in}{1.797178in}}%
\pgfpathlineto{\pgfqpoint{2.875680in}{1.790499in}}%
\pgfpathlineto{\pgfqpoint{2.876315in}{1.795189in}}%
\pgfpathlineto{\pgfqpoint{2.877903in}{1.781633in}}%
\pgfpathlineto{\pgfqpoint{2.878220in}{1.780031in}}%
\pgfpathlineto{\pgfqpoint{2.878537in}{1.782532in}}%
\pgfpathlineto{\pgfqpoint{2.878855in}{1.781155in}}%
\pgfpathlineto{\pgfqpoint{2.880125in}{1.784047in}}%
\pgfpathlineto{\pgfqpoint{2.880442in}{1.783904in}}%
\pgfpathlineto{\pgfqpoint{2.882029in}{1.782392in}}%
\pgfpathlineto{\pgfqpoint{2.883617in}{1.793808in}}%
\pgfpathlineto{\pgfqpoint{2.885204in}{1.780006in}}%
\pgfpathlineto{\pgfqpoint{2.885522in}{1.777091in}}%
\pgfpathlineto{\pgfqpoint{2.886474in}{1.778205in}}%
\pgfpathlineto{\pgfqpoint{2.887744in}{1.780757in}}%
\pgfpathlineto{\pgfqpoint{2.888061in}{1.779650in}}%
\pgfpathlineto{\pgfqpoint{2.889014in}{1.779950in}}%
\pgfpathlineto{\pgfqpoint{2.889331in}{1.779290in}}%
\pgfpathlineto{\pgfqpoint{2.889648in}{1.780519in}}%
\pgfpathlineto{\pgfqpoint{2.890918in}{1.791351in}}%
\pgfpathlineto{\pgfqpoint{2.892506in}{1.778902in}}%
\pgfpathlineto{\pgfqpoint{2.893775in}{1.773910in}}%
\pgfpathlineto{\pgfqpoint{2.895045in}{1.777560in}}%
\pgfpathlineto{\pgfqpoint{2.896315in}{1.776429in}}%
\pgfpathlineto{\pgfqpoint{2.896633in}{1.776242in}}%
\pgfpathlineto{\pgfqpoint{2.896950in}{1.776938in}}%
\pgfpathlineto{\pgfqpoint{2.898220in}{1.787451in}}%
\pgfpathlineto{\pgfqpoint{2.899172in}{1.783922in}}%
\pgfpathlineto{\pgfqpoint{2.900759in}{1.771592in}}%
\pgfpathlineto{\pgfqpoint{2.901077in}{1.770043in}}%
\pgfpathlineto{\pgfqpoint{2.901394in}{1.772697in}}%
\pgfpathlineto{\pgfqpoint{2.901712in}{1.771237in}}%
\pgfpathlineto{\pgfqpoint{2.902982in}{1.774141in}}%
\pgfpathlineto{\pgfqpoint{2.903299in}{1.773905in}}%
\pgfpathlineto{\pgfqpoint{2.904886in}{1.772541in}}%
\pgfpathlineto{\pgfqpoint{2.906156in}{1.783292in}}%
\pgfpathlineto{\pgfqpoint{2.906474in}{1.782825in}}%
\pgfpathlineto{\pgfqpoint{2.908061in}{1.769592in}}%
\pgfpathlineto{\pgfqpoint{2.908378in}{1.766928in}}%
\pgfpathlineto{\pgfqpoint{2.909013in}{1.767506in}}%
\pgfpathlineto{\pgfqpoint{2.910601in}{1.770791in}}%
\pgfpathlineto{\pgfqpoint{2.910918in}{1.769541in}}%
\pgfpathlineto{\pgfqpoint{2.911870in}{1.769901in}}%
\pgfpathlineto{\pgfqpoint{2.912188in}{1.769491in}}%
\pgfpathlineto{\pgfqpoint{2.913775in}{1.780535in}}%
\pgfpathlineto{\pgfqpoint{2.915363in}{1.768348in}}%
\pgfpathlineto{\pgfqpoint{2.916632in}{1.764026in}}%
\pgfpathlineto{\pgfqpoint{2.917902in}{1.767628in}}%
\pgfpathlineto{\pgfqpoint{2.919172in}{1.766342in}}%
\pgfpathlineto{\pgfqpoint{2.919807in}{1.766942in}}%
\pgfpathlineto{\pgfqpoint{2.921077in}{1.777217in}}%
\pgfpathlineto{\pgfqpoint{2.921712in}{1.775665in}}%
\pgfpathlineto{\pgfqpoint{2.923934in}{1.760178in}}%
\pgfpathlineto{\pgfqpoint{2.925204in}{1.764276in}}%
\pgfpathlineto{\pgfqpoint{2.926474in}{1.762828in}}%
\pgfpathlineto{\pgfqpoint{2.927426in}{1.763592in}}%
\pgfpathlineto{\pgfqpoint{2.927743in}{1.763239in}}%
\pgfpathlineto{\pgfqpoint{2.929013in}{1.773407in}}%
\pgfpathlineto{\pgfqpoint{2.931235in}{1.756913in}}%
\pgfpathlineto{\pgfqpoint{2.932188in}{1.758553in}}%
\pgfpathlineto{\pgfqpoint{2.933140in}{1.760887in}}%
\pgfpathlineto{\pgfqpoint{2.933458in}{1.760817in}}%
\pgfpathlineto{\pgfqpoint{2.933775in}{1.759442in}}%
\pgfpathlineto{\pgfqpoint{2.934727in}{1.759816in}}%
\pgfpathlineto{\pgfqpoint{2.935045in}{1.759643in}}%
\pgfpathlineto{\pgfqpoint{2.936315in}{1.770199in}}%
\pgfpathlineto{\pgfqpoint{2.936632in}{1.769682in}}%
\pgfpathlineto{\pgfqpoint{2.938219in}{1.757983in}}%
\pgfpathlineto{\pgfqpoint{2.939172in}{1.754196in}}%
\pgfpathlineto{\pgfqpoint{2.939489in}{1.754288in}}%
\pgfpathlineto{\pgfqpoint{2.940759in}{1.757676in}}%
\pgfpathlineto{\pgfqpoint{2.942029in}{1.756249in}}%
\pgfpathlineto{\pgfqpoint{2.942664in}{1.757476in}}%
\pgfpathlineto{\pgfqpoint{2.943934in}{1.766748in}}%
\pgfpathlineto{\pgfqpoint{2.945204in}{1.756666in}}%
\pgfpathlineto{\pgfqpoint{2.945521in}{1.756992in}}%
\pgfpathlineto{\pgfqpoint{2.946791in}{1.750384in}}%
\pgfpathlineto{\pgfqpoint{2.948061in}{1.754334in}}%
\pgfpathlineto{\pgfqpoint{2.949330in}{1.752717in}}%
\pgfpathlineto{\pgfqpoint{2.950600in}{1.754415in}}%
\pgfpathlineto{\pgfqpoint{2.951870in}{1.763135in}}%
\pgfpathlineto{\pgfqpoint{2.954092in}{1.746994in}}%
\pgfpathlineto{\pgfqpoint{2.954410in}{1.749249in}}%
\pgfpathlineto{\pgfqpoint{2.954727in}{1.747777in}}%
\pgfpathlineto{\pgfqpoint{2.955045in}{1.748785in}}%
\pgfpathlineto{\pgfqpoint{2.955997in}{1.750932in}}%
\pgfpathlineto{\pgfqpoint{2.956315in}{1.750837in}}%
\pgfpathlineto{\pgfqpoint{2.956632in}{1.749351in}}%
\pgfpathlineto{\pgfqpoint{2.957584in}{1.749696in}}%
\pgfpathlineto{\pgfqpoint{2.957902in}{1.749954in}}%
\pgfpathlineto{\pgfqpoint{2.959172in}{1.760355in}}%
\pgfpathlineto{\pgfqpoint{2.959489in}{1.758732in}}%
\pgfpathlineto{\pgfqpoint{2.961394in}{1.744373in}}%
\pgfpathlineto{\pgfqpoint{2.962664in}{1.747004in}}%
\pgfpathlineto{\pgfqpoint{2.962981in}{1.745503in}}%
\pgfpathlineto{\pgfqpoint{2.963616in}{1.747701in}}%
\pgfpathlineto{\pgfqpoint{2.963934in}{1.746082in}}%
\pgfpathlineto{\pgfqpoint{2.966473in}{1.756327in}}%
\pgfpathlineto{\pgfqpoint{2.966791in}{1.756094in}}%
\pgfpathlineto{\pgfqpoint{2.968695in}{1.742525in}}%
\pgfpathlineto{\pgfqpoint{2.969013in}{1.742577in}}%
\pgfpathlineto{\pgfqpoint{2.969648in}{1.740549in}}%
\pgfpathlineto{\pgfqpoint{2.969965in}{1.743047in}}%
\pgfpathlineto{\pgfqpoint{2.970283in}{1.741652in}}%
\pgfpathlineto{\pgfqpoint{2.971553in}{1.744266in}}%
\pgfpathlineto{\pgfqpoint{2.971870in}{1.743955in}}%
\pgfpathlineto{\pgfqpoint{2.972187in}{1.742605in}}%
\pgfpathlineto{\pgfqpoint{2.972822in}{1.743705in}}%
\pgfpathlineto{\pgfqpoint{2.973457in}{1.745487in}}%
\pgfpathlineto{\pgfqpoint{2.974727in}{1.752652in}}%
\pgfpathlineto{\pgfqpoint{2.976949in}{1.737122in}}%
\pgfpathlineto{\pgfqpoint{2.978219in}{1.740803in}}%
\pgfpathlineto{\pgfqpoint{2.978854in}{1.740939in}}%
\pgfpathlineto{\pgfqpoint{2.979489in}{1.739281in}}%
\pgfpathlineto{\pgfqpoint{2.982029in}{1.750004in}}%
\pgfpathlineto{\pgfqpoint{2.982346in}{1.747758in}}%
\pgfpathlineto{\pgfqpoint{2.984251in}{1.734376in}}%
\pgfpathlineto{\pgfqpoint{2.985521in}{1.736959in}}%
\pgfpathlineto{\pgfqpoint{2.985838in}{1.735595in}}%
\pgfpathlineto{\pgfqpoint{2.986473in}{1.737704in}}%
\pgfpathlineto{\pgfqpoint{2.986791in}{1.736051in}}%
\pgfpathlineto{\pgfqpoint{2.989330in}{1.746502in}}%
\pgfpathlineto{\pgfqpoint{2.989648in}{1.745474in}}%
\pgfpathlineto{\pgfqpoint{2.991552in}{1.732253in}}%
\pgfpathlineto{\pgfqpoint{2.991870in}{1.732426in}}%
\pgfpathlineto{\pgfqpoint{2.992187in}{1.731374in}}%
\pgfpathlineto{\pgfqpoint{2.992505in}{1.730730in}}%
\pgfpathlineto{\pgfqpoint{2.993775in}{1.734361in}}%
\pgfpathlineto{\pgfqpoint{2.995044in}{1.732499in}}%
\pgfpathlineto{\pgfqpoint{2.996314in}{1.736093in}}%
\pgfpathlineto{\pgfqpoint{2.996632in}{1.742932in}}%
\pgfpathlineto{\pgfqpoint{2.997584in}{1.741768in}}%
\pgfpathlineto{\pgfqpoint{2.999806in}{1.727266in}}%
\pgfpathlineto{\pgfqpoint{3.001076in}{1.730731in}}%
\pgfpathlineto{\pgfqpoint{3.001711in}{1.730912in}}%
\pgfpathlineto{\pgfqpoint{3.002346in}{1.729226in}}%
\pgfpathlineto{\pgfqpoint{3.004886in}{1.739668in}}%
\pgfpathlineto{\pgfqpoint{3.005203in}{1.737162in}}%
\pgfpathlineto{\pgfqpoint{3.007108in}{1.724412in}}%
\pgfpathlineto{\pgfqpoint{3.008378in}{1.726849in}}%
\pgfpathlineto{\pgfqpoint{3.008695in}{1.725710in}}%
\pgfpathlineto{\pgfqpoint{3.009013in}{1.727287in}}%
\pgfpathlineto{\pgfqpoint{3.009330in}{1.727684in}}%
\pgfpathlineto{\pgfqpoint{3.010600in}{1.725863in}}%
\pgfpathlineto{\pgfqpoint{3.011870in}{1.731947in}}%
\pgfpathlineto{\pgfqpoint{3.012187in}{1.737474in}}%
\pgfpathlineto{\pgfqpoint{3.012822in}{1.731029in}}%
\pgfpathlineto{\pgfqpoint{3.013139in}{1.732485in}}%
\pgfpathlineto{\pgfqpoint{3.015044in}{1.721390in}}%
\pgfpathlineto{\pgfqpoint{3.015362in}{1.720891in}}%
\pgfpathlineto{\pgfqpoint{3.016632in}{1.724291in}}%
\pgfpathlineto{\pgfqpoint{3.017901in}{1.722401in}}%
\pgfpathlineto{\pgfqpoint{3.019171in}{1.726285in}}%
\pgfpathlineto{\pgfqpoint{3.019489in}{1.733997in}}%
\pgfpathlineto{\pgfqpoint{3.020441in}{1.731205in}}%
\pgfpathlineto{\pgfqpoint{3.022663in}{1.717440in}}%
\pgfpathlineto{\pgfqpoint{3.023933in}{1.720612in}}%
\pgfpathlineto{\pgfqpoint{3.024568in}{1.720882in}}%
\pgfpathlineto{\pgfqpoint{3.025203in}{1.719191in}}%
\pgfpathlineto{\pgfqpoint{3.027743in}{1.730123in}}%
\pgfpathlineto{\pgfqpoint{3.028060in}{1.726199in}}%
\pgfpathlineto{\pgfqpoint{3.029965in}{1.714463in}}%
\pgfpathlineto{\pgfqpoint{3.032187in}{1.717626in}}%
\pgfpathlineto{\pgfqpoint{3.033457in}{1.715722in}}%
\pgfpathlineto{\pgfqpoint{3.034727in}{1.722210in}}%
\pgfpathlineto{\pgfqpoint{3.035044in}{1.728255in}}%
\pgfpathlineto{\pgfqpoint{3.035679in}{1.721053in}}%
\pgfpathlineto{\pgfqpoint{3.035996in}{1.721835in}}%
\pgfpathlineto{\pgfqpoint{3.037901in}{1.711388in}}%
\pgfpathlineto{\pgfqpoint{3.038219in}{1.711071in}}%
\pgfpathlineto{\pgfqpoint{3.039488in}{1.714175in}}%
\pgfpathlineto{\pgfqpoint{3.040758in}{1.712325in}}%
\pgfpathlineto{\pgfqpoint{3.042028in}{1.716142in}}%
\pgfpathlineto{\pgfqpoint{3.042346in}{1.724746in}}%
\pgfpathlineto{\pgfqpoint{3.043298in}{1.721236in}}%
\pgfpathlineto{\pgfqpoint{3.045203in}{1.708452in}}%
\pgfpathlineto{\pgfqpoint{3.045520in}{1.707685in}}%
\pgfpathlineto{\pgfqpoint{3.045838in}{1.708967in}}%
\pgfpathlineto{\pgfqpoint{3.046155in}{1.708269in}}%
\pgfpathlineto{\pgfqpoint{3.047742in}{1.710811in}}%
\pgfpathlineto{\pgfqpoint{3.049012in}{1.709085in}}%
\pgfpathlineto{\pgfqpoint{3.049330in}{1.710352in}}%
\pgfpathlineto{\pgfqpoint{3.050599in}{1.720587in}}%
\pgfpathlineto{\pgfqpoint{3.052504in}{1.706176in}}%
\pgfpathlineto{\pgfqpoint{3.053457in}{1.704626in}}%
\pgfpathlineto{\pgfqpoint{3.053774in}{1.705086in}}%
\pgfpathlineto{\pgfqpoint{3.055044in}{1.707526in}}%
\pgfpathlineto{\pgfqpoint{3.056314in}{1.705597in}}%
\pgfpathlineto{\pgfqpoint{3.057584in}{1.712138in}}%
\pgfpathlineto{\pgfqpoint{3.057901in}{1.718938in}}%
\pgfpathlineto{\pgfqpoint{3.058536in}{1.711340in}}%
\pgfpathlineto{\pgfqpoint{3.058853in}{1.711601in}}%
\pgfpathlineto{\pgfqpoint{3.060441in}{1.701742in}}%
\pgfpathlineto{\pgfqpoint{3.061076in}{1.701390in}}%
\pgfpathlineto{\pgfqpoint{3.062345in}{1.704036in}}%
\pgfpathlineto{\pgfqpoint{3.063615in}{1.702271in}}%
\pgfpathlineto{\pgfqpoint{3.064885in}{1.705767in}}%
\pgfpathlineto{\pgfqpoint{3.065203in}{1.715279in}}%
\pgfpathlineto{\pgfqpoint{3.066155in}{1.711440in}}%
\pgfpathlineto{\pgfqpoint{3.068060in}{1.698402in}}%
\pgfpathlineto{\pgfqpoint{3.068377in}{1.697836in}}%
\pgfpathlineto{\pgfqpoint{3.069012in}{1.698317in}}%
\pgfpathlineto{\pgfqpoint{3.070599in}{1.700734in}}%
\pgfpathlineto{\pgfqpoint{3.071869in}{1.698926in}}%
\pgfpathlineto{\pgfqpoint{3.072187in}{1.700239in}}%
\pgfpathlineto{\pgfqpoint{3.073456in}{1.711276in}}%
\pgfpathlineto{\pgfqpoint{3.075361in}{1.695854in}}%
\pgfpathlineto{\pgfqpoint{3.076314in}{1.694759in}}%
\pgfpathlineto{\pgfqpoint{3.076631in}{1.695375in}}%
\pgfpathlineto{\pgfqpoint{3.077901in}{1.697391in}}%
\pgfpathlineto{\pgfqpoint{3.079171in}{1.695494in}}%
\pgfpathlineto{\pgfqpoint{3.080441in}{1.701816in}}%
\pgfpathlineto{\pgfqpoint{3.080758in}{1.709570in}}%
\pgfpathlineto{\pgfqpoint{3.081393in}{1.701800in}}%
\pgfpathlineto{\pgfqpoint{3.081710in}{1.701655in}}%
\pgfpathlineto{\pgfqpoint{3.083298in}{1.691509in}}%
\pgfpathlineto{\pgfqpoint{3.083933in}{1.691494in}}%
\pgfpathlineto{\pgfqpoint{3.085837in}{1.694034in}}%
\pgfpathlineto{\pgfqpoint{3.086155in}{1.693880in}}%
\pgfpathlineto{\pgfqpoint{3.087425in}{1.692280in}}%
\pgfpathlineto{\pgfqpoint{3.088059in}{1.705166in}}%
\pgfpathlineto{\pgfqpoint{3.089012in}{1.702152in}}%
\pgfpathlineto{\pgfqpoint{3.090599in}{1.688435in}}%
\pgfpathlineto{\pgfqpoint{3.091869in}{1.688500in}}%
\pgfpathlineto{\pgfqpoint{3.093456in}{1.690612in}}%
\pgfpathlineto{\pgfqpoint{3.094726in}{1.688779in}}%
\pgfpathlineto{\pgfqpoint{3.095044in}{1.690242in}}%
\pgfpathlineto{\pgfqpoint{3.096313in}{1.702072in}}%
\pgfpathlineto{\pgfqpoint{3.097901in}{1.686094in}}%
\pgfpathlineto{\pgfqpoint{3.099171in}{1.684773in}}%
\pgfpathlineto{\pgfqpoint{3.099488in}{1.685393in}}%
\pgfpathlineto{\pgfqpoint{3.100758in}{1.687205in}}%
\pgfpathlineto{\pgfqpoint{3.102028in}{1.685420in}}%
\pgfpathlineto{\pgfqpoint{3.103297in}{1.690996in}}%
\pgfpathlineto{\pgfqpoint{3.103615in}{1.699858in}}%
\pgfpathlineto{\pgfqpoint{3.104250in}{1.692551in}}%
\pgfpathlineto{\pgfqpoint{3.104567in}{1.692454in}}%
\pgfpathlineto{\pgfqpoint{3.106155in}{1.681249in}}%
\pgfpathlineto{\pgfqpoint{3.107424in}{1.682290in}}%
\pgfpathlineto{\pgfqpoint{3.108694in}{1.683929in}}%
\pgfpathlineto{\pgfqpoint{3.109012in}{1.683768in}}%
\pgfpathlineto{\pgfqpoint{3.110282in}{1.682075in}}%
\pgfpathlineto{\pgfqpoint{3.110916in}{1.694573in}}%
\pgfpathlineto{\pgfqpoint{3.111869in}{1.693110in}}%
\pgfpathlineto{\pgfqpoint{3.113456in}{1.678280in}}%
\pgfpathlineto{\pgfqpoint{3.115678in}{1.679543in}}%
\pgfpathlineto{\pgfqpoint{3.116948in}{1.680291in}}%
\pgfpathlineto{\pgfqpoint{3.117266in}{1.680280in}}%
\pgfpathlineto{\pgfqpoint{3.117583in}{1.678644in}}%
\pgfpathlineto{\pgfqpoint{3.117901in}{1.680205in}}%
\pgfpathlineto{\pgfqpoint{3.119170in}{1.692951in}}%
\pgfpathlineto{\pgfqpoint{3.120758in}{1.676371in}}%
\pgfpathlineto{\pgfqpoint{3.121710in}{1.674385in}}%
\pgfpathlineto{\pgfqpoint{3.122027in}{1.674822in}}%
\pgfpathlineto{\pgfqpoint{3.124250in}{1.677066in}}%
\pgfpathlineto{\pgfqpoint{3.124567in}{1.676947in}}%
\pgfpathlineto{\pgfqpoint{3.124885in}{1.675382in}}%
\pgfpathlineto{\pgfqpoint{3.125202in}{1.676831in}}%
\pgfpathlineto{\pgfqpoint{3.125837in}{1.676093in}}%
\pgfpathlineto{\pgfqpoint{3.126154in}{1.680088in}}%
\pgfpathlineto{\pgfqpoint{3.126472in}{1.689642in}}%
\pgfpathlineto{\pgfqpoint{3.127107in}{1.683295in}}%
\pgfpathlineto{\pgfqpoint{3.127424in}{1.683598in}}%
\pgfpathlineto{\pgfqpoint{3.129012in}{1.671068in}}%
\pgfpathlineto{\pgfqpoint{3.131869in}{1.673636in}}%
\pgfpathlineto{\pgfqpoint{3.133138in}{1.671935in}}%
\pgfpathlineto{\pgfqpoint{3.133456in}{1.673697in}}%
\pgfpathlineto{\pgfqpoint{3.134726in}{1.684525in}}%
\pgfpathlineto{\pgfqpoint{3.136313in}{1.668437in}}%
\pgfpathlineto{\pgfqpoint{3.137265in}{1.667856in}}%
\pgfpathlineto{\pgfqpoint{3.137583in}{1.668499in}}%
\pgfpathlineto{\pgfqpoint{3.139170in}{1.670255in}}%
\pgfpathlineto{\pgfqpoint{3.140440in}{1.668556in}}%
\pgfpathlineto{\pgfqpoint{3.141710in}{1.675543in}}%
\pgfpathlineto{\pgfqpoint{3.142027in}{1.683355in}}%
\pgfpathlineto{\pgfqpoint{3.142662in}{1.675162in}}%
\pgfpathlineto{\pgfqpoint{3.144567in}{1.664140in}}%
\pgfpathlineto{\pgfqpoint{3.145837in}{1.665935in}}%
\pgfpathlineto{\pgfqpoint{3.147107in}{1.666977in}}%
\pgfpathlineto{\pgfqpoint{3.148694in}{1.665255in}}%
\pgfpathlineto{\pgfqpoint{3.149329in}{1.678751in}}%
\pgfpathlineto{\pgfqpoint{3.150281in}{1.675346in}}%
\pgfpathlineto{\pgfqpoint{3.151868in}{1.661049in}}%
\pgfpathlineto{\pgfqpoint{3.152821in}{1.661423in}}%
\pgfpathlineto{\pgfqpoint{3.154408in}{1.663683in}}%
\pgfpathlineto{\pgfqpoint{3.154726in}{1.663438in}}%
\pgfpathlineto{\pgfqpoint{3.155995in}{1.661816in}}%
\pgfpathlineto{\pgfqpoint{3.156313in}{1.663382in}}%
\pgfpathlineto{\pgfqpoint{3.157583in}{1.675495in}}%
\pgfpathlineto{\pgfqpoint{3.159170in}{1.658771in}}%
\pgfpathlineto{\pgfqpoint{3.160122in}{1.657548in}}%
\pgfpathlineto{\pgfqpoint{3.161710in}{1.660207in}}%
\pgfpathlineto{\pgfqpoint{3.162027in}{1.660011in}}%
\pgfpathlineto{\pgfqpoint{3.163297in}{1.658492in}}%
\pgfpathlineto{\pgfqpoint{3.164567in}{1.664435in}}%
\pgfpathlineto{\pgfqpoint{3.164884in}{1.673275in}}%
\pgfpathlineto{\pgfqpoint{3.165519in}{1.666278in}}%
\pgfpathlineto{\pgfqpoint{3.165837in}{1.665537in}}%
\pgfpathlineto{\pgfqpoint{3.167424in}{1.653954in}}%
\pgfpathlineto{\pgfqpoint{3.169964in}{1.656854in}}%
\pgfpathlineto{\pgfqpoint{3.170281in}{1.656616in}}%
\pgfpathlineto{\pgfqpoint{3.171551in}{1.655008in}}%
\pgfpathlineto{\pgfqpoint{3.172186in}{1.667711in}}%
\pgfpathlineto{\pgfqpoint{3.173138in}{1.666748in}}%
\pgfpathlineto{\pgfqpoint{3.174725in}{1.651166in}}%
\pgfpathlineto{\pgfqpoint{3.175043in}{1.650615in}}%
\pgfpathlineto{\pgfqpoint{3.175360in}{1.651561in}}%
\pgfpathlineto{\pgfqpoint{3.175678in}{1.651074in}}%
\pgfpathlineto{\pgfqpoint{3.177265in}{1.653550in}}%
\pgfpathlineto{\pgfqpoint{3.178852in}{1.651702in}}%
\pgfpathlineto{\pgfqpoint{3.180122in}{1.659704in}}%
\pgfpathlineto{\pgfqpoint{3.180440in}{1.666099in}}%
\pgfpathlineto{\pgfqpoint{3.180757in}{1.657659in}}%
\pgfpathlineto{\pgfqpoint{3.181075in}{1.657948in}}%
\pgfpathlineto{\pgfqpoint{3.182979in}{1.647246in}}%
\pgfpathlineto{\pgfqpoint{3.183297in}{1.648044in}}%
\pgfpathlineto{\pgfqpoint{3.184567in}{1.650117in}}%
\pgfpathlineto{\pgfqpoint{3.184884in}{1.649718in}}%
\pgfpathlineto{\pgfqpoint{3.186154in}{1.648495in}}%
\pgfpathlineto{\pgfqpoint{3.186471in}{1.649765in}}%
\pgfpathlineto{\pgfqpoint{3.187106in}{1.648926in}}%
\pgfpathlineto{\pgfqpoint{3.187424in}{1.653503in}}%
\pgfpathlineto{\pgfqpoint{3.187741in}{1.662722in}}%
\pgfpathlineto{\pgfqpoint{3.188376in}{1.657060in}}%
\pgfpathlineto{\pgfqpoint{3.188694in}{1.657225in}}%
\pgfpathlineto{\pgfqpoint{3.190281in}{1.643879in}}%
\pgfpathlineto{\pgfqpoint{3.190598in}{1.643899in}}%
\pgfpathlineto{\pgfqpoint{3.191868in}{1.646455in}}%
\pgfpathlineto{\pgfqpoint{3.192186in}{1.646041in}}%
\pgfpathlineto{\pgfqpoint{3.192821in}{1.646742in}}%
\pgfpathlineto{\pgfqpoint{3.193138in}{1.646395in}}%
\pgfpathlineto{\pgfqpoint{3.194408in}{1.644907in}}%
\pgfpathlineto{\pgfqpoint{3.194725in}{1.646734in}}%
\pgfpathlineto{\pgfqpoint{3.195995in}{1.657956in}}%
\pgfpathlineto{\pgfqpoint{3.197582in}{1.641552in}}%
\pgfpathlineto{\pgfqpoint{3.197900in}{1.640269in}}%
\pgfpathlineto{\pgfqpoint{3.198535in}{1.640687in}}%
\pgfpathlineto{\pgfqpoint{3.200122in}{1.643445in}}%
\pgfpathlineto{\pgfqpoint{3.201709in}{1.641654in}}%
\pgfpathlineto{\pgfqpoint{3.202979in}{1.648917in}}%
\pgfpathlineto{\pgfqpoint{3.203297in}{1.656144in}}%
\pgfpathlineto{\pgfqpoint{3.203614in}{1.647502in}}%
\pgfpathlineto{\pgfqpoint{3.203932in}{1.649170in}}%
\pgfpathlineto{\pgfqpoint{3.205836in}{1.637006in}}%
\pgfpathlineto{\pgfqpoint{3.206789in}{1.638311in}}%
\pgfpathlineto{\pgfqpoint{3.207424in}{1.639939in}}%
\pgfpathlineto{\pgfqpoint{3.208058in}{1.639628in}}%
\pgfpathlineto{\pgfqpoint{3.208693in}{1.639554in}}%
\pgfpathlineto{\pgfqpoint{3.209963in}{1.638114in}}%
\pgfpathlineto{\pgfqpoint{3.210598in}{1.651263in}}%
\pgfpathlineto{\pgfqpoint{3.211551in}{1.648833in}}%
\pgfpathlineto{\pgfqpoint{3.213138in}{1.634011in}}%
\pgfpathlineto{\pgfqpoint{3.213455in}{1.633376in}}%
\pgfpathlineto{\pgfqpoint{3.213773in}{1.634977in}}%
\pgfpathlineto{\pgfqpoint{3.214090in}{1.634278in}}%
\pgfpathlineto{\pgfqpoint{3.215677in}{1.636627in}}%
\pgfpathlineto{\pgfqpoint{3.217265in}{1.634824in}}%
\pgfpathlineto{\pgfqpoint{3.217582in}{1.636253in}}%
\pgfpathlineto{\pgfqpoint{3.218852in}{1.648695in}}%
\pgfpathlineto{\pgfqpoint{3.220439in}{1.632297in}}%
\pgfpathlineto{\pgfqpoint{3.220757in}{1.630121in}}%
\pgfpathlineto{\pgfqpoint{3.221709in}{1.631192in}}%
\pgfpathlineto{\pgfqpoint{3.222979in}{1.633293in}}%
\pgfpathlineto{\pgfqpoint{3.224566in}{1.631650in}}%
\pgfpathlineto{\pgfqpoint{3.225836in}{1.637165in}}%
\pgfpathlineto{\pgfqpoint{3.226154in}{1.645606in}}%
\pgfpathlineto{\pgfqpoint{3.226788in}{1.640135in}}%
\pgfpathlineto{\pgfqpoint{3.227106in}{1.639421in}}%
\pgfpathlineto{\pgfqpoint{3.228693in}{1.626912in}}%
\pgfpathlineto{\pgfqpoint{3.229011in}{1.626816in}}%
\pgfpathlineto{\pgfqpoint{3.230281in}{1.629827in}}%
\pgfpathlineto{\pgfqpoint{3.231868in}{1.628639in}}%
\pgfpathlineto{\pgfqpoint{3.232503in}{1.629474in}}%
\pgfpathlineto{\pgfqpoint{3.232820in}{1.628022in}}%
\pgfpathlineto{\pgfqpoint{3.233138in}{1.630363in}}%
\pgfpathlineto{\pgfqpoint{3.234407in}{1.640180in}}%
\pgfpathlineto{\pgfqpoint{3.235995in}{1.624475in}}%
\pgfpathlineto{\pgfqpoint{3.236312in}{1.622956in}}%
\pgfpathlineto{\pgfqpoint{3.236630in}{1.624673in}}%
\pgfpathlineto{\pgfqpoint{3.236947in}{1.623800in}}%
\pgfpathlineto{\pgfqpoint{3.238534in}{1.626508in}}%
\pgfpathlineto{\pgfqpoint{3.240122in}{1.624768in}}%
\pgfpathlineto{\pgfqpoint{3.241392in}{1.632301in}}%
\pgfpathlineto{\pgfqpoint{3.241709in}{1.639127in}}%
\pgfpathlineto{\pgfqpoint{3.242026in}{1.630517in}}%
\pgfpathlineto{\pgfqpoint{3.242344in}{1.632185in}}%
\pgfpathlineto{\pgfqpoint{3.244249in}{1.620014in}}%
\pgfpathlineto{\pgfqpoint{3.244566in}{1.620452in}}%
\pgfpathlineto{\pgfqpoint{3.245836in}{1.623169in}}%
\pgfpathlineto{\pgfqpoint{3.247423in}{1.621702in}}%
\pgfpathlineto{\pgfqpoint{3.248376in}{1.621278in}}%
\pgfpathlineto{\pgfqpoint{3.248693in}{1.625352in}}%
\pgfpathlineto{\pgfqpoint{3.249011in}{1.634208in}}%
\pgfpathlineto{\pgfqpoint{3.249963in}{1.631329in}}%
\pgfpathlineto{\pgfqpoint{3.251550in}{1.617095in}}%
\pgfpathlineto{\pgfqpoint{3.251868in}{1.616147in}}%
\pgfpathlineto{\pgfqpoint{3.252185in}{1.618155in}}%
\pgfpathlineto{\pgfqpoint{3.252503in}{1.617306in}}%
\pgfpathlineto{\pgfqpoint{3.254090in}{1.619648in}}%
\pgfpathlineto{\pgfqpoint{3.255677in}{1.617954in}}%
\pgfpathlineto{\pgfqpoint{3.255995in}{1.619287in}}%
\pgfpathlineto{\pgfqpoint{3.257264in}{1.631454in}}%
\pgfpathlineto{\pgfqpoint{3.258852in}{1.615366in}}%
\pgfpathlineto{\pgfqpoint{3.259169in}{1.612818in}}%
\pgfpathlineto{\pgfqpoint{3.259804in}{1.613370in}}%
\pgfpathlineto{\pgfqpoint{3.261391in}{1.616392in}}%
\pgfpathlineto{\pgfqpoint{3.261709in}{1.615462in}}%
\pgfpathlineto{\pgfqpoint{3.262661in}{1.615659in}}%
\pgfpathlineto{\pgfqpoint{3.262979in}{1.614778in}}%
\pgfpathlineto{\pgfqpoint{3.263296in}{1.615821in}}%
\pgfpathlineto{\pgfqpoint{3.264566in}{1.628591in}}%
\pgfpathlineto{\pgfqpoint{3.263931in}{1.615478in}}%
\pgfpathlineto{\pgfqpoint{3.265201in}{1.623256in}}%
\pgfpathlineto{\pgfqpoint{3.265518in}{1.621967in}}%
\pgfpathlineto{\pgfqpoint{3.267106in}{1.609926in}}%
\pgfpathlineto{\pgfqpoint{3.267423in}{1.609624in}}%
\pgfpathlineto{\pgfqpoint{3.268693in}{1.612985in}}%
\pgfpathlineto{\pgfqpoint{3.269963in}{1.612128in}}%
\pgfpathlineto{\pgfqpoint{3.271233in}{1.611090in}}%
\pgfpathlineto{\pgfqpoint{3.272820in}{1.623293in}}%
\pgfpathlineto{\pgfqpoint{3.274407in}{1.607665in}}%
\pgfpathlineto{\pgfqpoint{3.274725in}{1.605674in}}%
\pgfpathlineto{\pgfqpoint{3.275360in}{1.606786in}}%
\pgfpathlineto{\pgfqpoint{3.276947in}{1.609539in}}%
\pgfpathlineto{\pgfqpoint{3.278534in}{1.607930in}}%
\pgfpathlineto{\pgfqpoint{3.279804in}{1.615584in}}%
\pgfpathlineto{\pgfqpoint{3.280121in}{1.621722in}}%
\pgfpathlineto{\pgfqpoint{3.280439in}{1.613374in}}%
\pgfpathlineto{\pgfqpoint{3.280756in}{1.615309in}}%
\pgfpathlineto{\pgfqpoint{3.282661in}{1.603073in}}%
\pgfpathlineto{\pgfqpoint{3.282978in}{1.603267in}}%
\pgfpathlineto{\pgfqpoint{3.284248in}{1.606247in}}%
\pgfpathlineto{\pgfqpoint{3.285518in}{1.605293in}}%
\pgfpathlineto{\pgfqpoint{3.285836in}{1.604836in}}%
\pgfpathlineto{\pgfqpoint{3.286153in}{1.605555in}}%
\pgfpathlineto{\pgfqpoint{3.286471in}{1.605894in}}%
\pgfpathlineto{\pgfqpoint{3.286788in}{1.604344in}}%
\pgfpathlineto{\pgfqpoint{3.287423in}{1.616972in}}%
\pgfpathlineto{\pgfqpoint{3.288375in}{1.614470in}}%
\pgfpathlineto{\pgfqpoint{3.289963in}{1.600194in}}%
\pgfpathlineto{\pgfqpoint{3.290280in}{1.598896in}}%
\pgfpathlineto{\pgfqpoint{3.290597in}{1.601259in}}%
\pgfpathlineto{\pgfqpoint{3.290915in}{1.600305in}}%
\pgfpathlineto{\pgfqpoint{3.292185in}{1.602778in}}%
\pgfpathlineto{\pgfqpoint{3.292502in}{1.602628in}}%
\pgfpathlineto{\pgfqpoint{3.294090in}{1.601071in}}%
\pgfpathlineto{\pgfqpoint{3.294407in}{1.602364in}}%
\pgfpathlineto{\pgfqpoint{3.295677in}{1.614332in}}%
\pgfpathlineto{\pgfqpoint{3.297264in}{1.598725in}}%
\pgfpathlineto{\pgfqpoint{3.297582in}{1.595586in}}%
\pgfpathlineto{\pgfqpoint{3.298534in}{1.596995in}}%
\pgfpathlineto{\pgfqpoint{3.299804in}{1.599424in}}%
\pgfpathlineto{\pgfqpoint{3.301074in}{1.598506in}}%
\pgfpathlineto{\pgfqpoint{3.301391in}{1.597945in}}%
\pgfpathlineto{\pgfqpoint{3.301708in}{1.598786in}}%
\pgfpathlineto{\pgfqpoint{3.302978in}{1.610997in}}%
\pgfpathlineto{\pgfqpoint{3.302343in}{1.598321in}}%
\pgfpathlineto{\pgfqpoint{3.303931in}{1.605560in}}%
\pgfpathlineto{\pgfqpoint{3.305518in}{1.593068in}}%
\pgfpathlineto{\pgfqpoint{3.305835in}{1.592322in}}%
\pgfpathlineto{\pgfqpoint{3.307105in}{1.596036in}}%
\pgfpathlineto{\pgfqpoint{3.308375in}{1.594905in}}%
\pgfpathlineto{\pgfqpoint{3.309645in}{1.594215in}}%
\pgfpathlineto{\pgfqpoint{3.311232in}{1.606159in}}%
\pgfpathlineto{\pgfqpoint{3.312820in}{1.590948in}}%
\pgfpathlineto{\pgfqpoint{3.313137in}{1.588461in}}%
\pgfpathlineto{\pgfqpoint{3.313772in}{1.589750in}}%
\pgfpathlineto{\pgfqpoint{3.315359in}{1.592538in}}%
\pgfpathlineto{\pgfqpoint{3.315677in}{1.591406in}}%
\pgfpathlineto{\pgfqpoint{3.316629in}{1.591709in}}%
\pgfpathlineto{\pgfqpoint{3.316946in}{1.591055in}}%
\pgfpathlineto{\pgfqpoint{3.317264in}{1.591974in}}%
\pgfpathlineto{\pgfqpoint{3.318534in}{1.604651in}}%
\pgfpathlineto{\pgfqpoint{3.320121in}{1.590300in}}%
\pgfpathlineto{\pgfqpoint{3.321391in}{1.585914in}}%
\pgfpathlineto{\pgfqpoint{3.322661in}{1.589266in}}%
\pgfpathlineto{\pgfqpoint{3.323931in}{1.588098in}}%
\pgfpathlineto{\pgfqpoint{3.324565in}{1.588542in}}%
\pgfpathlineto{\pgfqpoint{3.324883in}{1.589061in}}%
\pgfpathlineto{\pgfqpoint{3.325200in}{1.587445in}}%
\pgfpathlineto{\pgfqpoint{3.325518in}{1.591361in}}%
\pgfpathlineto{\pgfqpoint{3.325835in}{1.599279in}}%
\pgfpathlineto{\pgfqpoint{3.326788in}{1.597786in}}%
\pgfpathlineto{\pgfqpoint{3.328375in}{1.583509in}}%
\pgfpathlineto{\pgfqpoint{3.328692in}{1.581575in}}%
\pgfpathlineto{\pgfqpoint{3.329010in}{1.584151in}}%
\pgfpathlineto{\pgfqpoint{3.329327in}{1.583167in}}%
\pgfpathlineto{\pgfqpoint{3.330597in}{1.585827in}}%
\pgfpathlineto{\pgfqpoint{3.330915in}{1.585615in}}%
\pgfpathlineto{\pgfqpoint{3.331232in}{1.584524in}}%
\pgfpathlineto{\pgfqpoint{3.332184in}{1.584929in}}%
\pgfpathlineto{\pgfqpoint{3.332502in}{1.584203in}}%
\pgfpathlineto{\pgfqpoint{3.332819in}{1.585156in}}%
\pgfpathlineto{\pgfqpoint{3.334089in}{1.597442in}}%
\pgfpathlineto{\pgfqpoint{3.335676in}{1.582325in}}%
\pgfpathlineto{\pgfqpoint{3.335994in}{1.578668in}}%
\pgfpathlineto{\pgfqpoint{3.336946in}{1.579546in}}%
\pgfpathlineto{\pgfqpoint{3.338216in}{1.582427in}}%
\pgfpathlineto{\pgfqpoint{3.339486in}{1.581277in}}%
\pgfpathlineto{\pgfqpoint{3.339803in}{1.581103in}}%
\pgfpathlineto{\pgfqpoint{3.340121in}{1.581725in}}%
\pgfpathlineto{\pgfqpoint{3.341391in}{1.593584in}}%
\pgfpathlineto{\pgfqpoint{3.340756in}{1.581246in}}%
\pgfpathlineto{\pgfqpoint{3.342343in}{1.589180in}}%
\pgfpathlineto{\pgfqpoint{3.343930in}{1.576253in}}%
\pgfpathlineto{\pgfqpoint{3.344248in}{1.574941in}}%
\pgfpathlineto{\pgfqpoint{3.344565in}{1.577620in}}%
\pgfpathlineto{\pgfqpoint{3.344883in}{1.576620in}}%
\pgfpathlineto{\pgfqpoint{3.346153in}{1.578953in}}%
\pgfpathlineto{\pgfqpoint{3.346470in}{1.578687in}}%
\pgfpathlineto{\pgfqpoint{3.348057in}{1.577313in}}%
\pgfpathlineto{\pgfqpoint{3.348375in}{1.578808in}}%
\pgfpathlineto{\pgfqpoint{3.349645in}{1.590085in}}%
\pgfpathlineto{\pgfqpoint{3.351232in}{1.574636in}}%
\pgfpathlineto{\pgfqpoint{3.351549in}{1.571363in}}%
\pgfpathlineto{\pgfqpoint{3.352184in}{1.572580in}}%
\pgfpathlineto{\pgfqpoint{3.353772in}{1.575543in}}%
\pgfpathlineto{\pgfqpoint{3.354406in}{1.575085in}}%
\pgfpathlineto{\pgfqpoint{3.355676in}{1.574899in}}%
\pgfpathlineto{\pgfqpoint{3.356946in}{1.587051in}}%
\pgfpathlineto{\pgfqpoint{3.358533in}{1.574026in}}%
\pgfpathlineto{\pgfqpoint{3.359803in}{1.568388in}}%
\pgfpathlineto{\pgfqpoint{3.361073in}{1.572221in}}%
\pgfpathlineto{\pgfqpoint{3.362343in}{1.570835in}}%
\pgfpathlineto{\pgfqpoint{3.363295in}{1.571367in}}%
\pgfpathlineto{\pgfqpoint{3.363613in}{1.570424in}}%
\pgfpathlineto{\pgfqpoint{3.365200in}{1.582133in}}%
\pgfpathlineto{\pgfqpoint{3.366787in}{1.567057in}}%
\pgfpathlineto{\pgfqpoint{3.367105in}{1.564366in}}%
\pgfpathlineto{\pgfqpoint{3.367740in}{1.565960in}}%
\pgfpathlineto{\pgfqpoint{3.369010in}{1.568752in}}%
\pgfpathlineto{\pgfqpoint{3.369327in}{1.568637in}}%
\pgfpathlineto{\pgfqpoint{3.369644in}{1.567340in}}%
\pgfpathlineto{\pgfqpoint{3.370597in}{1.567675in}}%
\pgfpathlineto{\pgfqpoint{3.370914in}{1.567348in}}%
\pgfpathlineto{\pgfqpoint{3.371232in}{1.568036in}}%
\pgfpathlineto{\pgfqpoint{3.372502in}{1.580414in}}%
\pgfpathlineto{\pgfqpoint{3.374089in}{1.566108in}}%
\pgfpathlineto{\pgfqpoint{3.375359in}{1.561956in}}%
\pgfpathlineto{\pgfqpoint{3.376629in}{1.565395in}}%
\pgfpathlineto{\pgfqpoint{3.377898in}{1.564027in}}%
\pgfpathlineto{\pgfqpoint{3.378851in}{1.565336in}}%
\pgfpathlineto{\pgfqpoint{3.379168in}{1.563724in}}%
\pgfpathlineto{\pgfqpoint{3.379803in}{1.575076in}}%
\pgfpathlineto{\pgfqpoint{3.380755in}{1.574055in}}%
\pgfpathlineto{\pgfqpoint{3.382343in}{1.559711in}}%
\pgfpathlineto{\pgfqpoint{3.382660in}{1.557511in}}%
\pgfpathlineto{\pgfqpoint{3.382978in}{1.560288in}}%
\pgfpathlineto{\pgfqpoint{3.383295in}{1.559333in}}%
\pgfpathlineto{\pgfqpoint{3.384565in}{1.561953in}}%
\pgfpathlineto{\pgfqpoint{3.384882in}{1.561709in}}%
\pgfpathlineto{\pgfqpoint{3.385200in}{1.560448in}}%
\pgfpathlineto{\pgfqpoint{3.386152in}{1.560873in}}%
\pgfpathlineto{\pgfqpoint{3.386470in}{1.560484in}}%
\pgfpathlineto{\pgfqpoint{3.386787in}{1.561614in}}%
\pgfpathlineto{\pgfqpoint{3.388057in}{1.573126in}}%
\pgfpathlineto{\pgfqpoint{3.389644in}{1.558321in}}%
\pgfpathlineto{\pgfqpoint{3.389962in}{1.554544in}}%
\pgfpathlineto{\pgfqpoint{3.390914in}{1.555543in}}%
\pgfpathlineto{\pgfqpoint{3.392184in}{1.558540in}}%
\pgfpathlineto{\pgfqpoint{3.393454in}{1.557197in}}%
\pgfpathlineto{\pgfqpoint{3.394089in}{1.557796in}}%
\pgfpathlineto{\pgfqpoint{3.395359in}{1.569282in}}%
\pgfpathlineto{\pgfqpoint{3.394724in}{1.557448in}}%
\pgfpathlineto{\pgfqpoint{3.396311in}{1.565741in}}%
\pgfpathlineto{\pgfqpoint{3.397898in}{1.552504in}}%
\pgfpathlineto{\pgfqpoint{3.398216in}{1.550876in}}%
\pgfpathlineto{\pgfqpoint{3.398533in}{1.553713in}}%
\pgfpathlineto{\pgfqpoint{3.398851in}{1.552745in}}%
\pgfpathlineto{\pgfqpoint{3.400120in}{1.555080in}}%
\pgfpathlineto{\pgfqpoint{3.400438in}{1.554790in}}%
\pgfpathlineto{\pgfqpoint{3.400755in}{1.553606in}}%
\pgfpathlineto{\pgfqpoint{3.401708in}{1.554027in}}%
\pgfpathlineto{\pgfqpoint{3.402025in}{1.553573in}}%
\pgfpathlineto{\pgfqpoint{3.403612in}{1.565819in}}%
\pgfpathlineto{\pgfqpoint{3.405200in}{1.550759in}}%
\pgfpathlineto{\pgfqpoint{3.405517in}{1.547316in}}%
\pgfpathlineto{\pgfqpoint{3.406152in}{1.548775in}}%
\pgfpathlineto{\pgfqpoint{3.406470in}{1.549136in}}%
\pgfpathlineto{\pgfqpoint{3.407739in}{1.551648in}}%
\pgfpathlineto{\pgfqpoint{3.408057in}{1.550154in}}%
\pgfpathlineto{\pgfqpoint{3.409009in}{1.550400in}}%
\pgfpathlineto{\pgfqpoint{3.409644in}{1.550949in}}%
\pgfpathlineto{\pgfqpoint{3.410914in}{1.562783in}}%
\pgfpathlineto{\pgfqpoint{3.412501in}{1.550701in}}%
\pgfpathlineto{\pgfqpoint{3.413771in}{1.544287in}}%
\pgfpathlineto{\pgfqpoint{3.415041in}{1.548314in}}%
\pgfpathlineto{\pgfqpoint{3.416311in}{1.546741in}}%
\pgfpathlineto{\pgfqpoint{3.417263in}{1.547552in}}%
\pgfpathlineto{\pgfqpoint{3.417581in}{1.546716in}}%
\pgfpathlineto{\pgfqpoint{3.419168in}{1.558159in}}%
\pgfpathlineto{\pgfqpoint{3.420755in}{1.543343in}}%
\pgfpathlineto{\pgfqpoint{3.421073in}{1.540373in}}%
\pgfpathlineto{\pgfqpoint{3.421707in}{1.542102in}}%
\pgfpathlineto{\pgfqpoint{3.423295in}{1.544755in}}%
\pgfpathlineto{\pgfqpoint{3.423612in}{1.543292in}}%
\pgfpathlineto{\pgfqpoint{3.424565in}{1.543568in}}%
\pgfpathlineto{\pgfqpoint{3.425200in}{1.544074in}}%
\pgfpathlineto{\pgfqpoint{3.426469in}{1.556314in}}%
\pgfpathlineto{\pgfqpoint{3.428057in}{1.542948in}}%
\pgfpathlineto{\pgfqpoint{3.429326in}{1.537786in}}%
\pgfpathlineto{\pgfqpoint{3.430596in}{1.541483in}}%
\pgfpathlineto{\pgfqpoint{3.431866in}{1.539921in}}%
\pgfpathlineto{\pgfqpoint{3.433453in}{1.543271in}}%
\pgfpathlineto{\pgfqpoint{3.434723in}{1.550637in}}%
\pgfpathlineto{\pgfqpoint{3.436311in}{1.536114in}}%
\pgfpathlineto{\pgfqpoint{3.436628in}{1.533460in}}%
\pgfpathlineto{\pgfqpoint{3.436945in}{1.536270in}}%
\pgfpathlineto{\pgfqpoint{3.437263in}{1.535397in}}%
\pgfpathlineto{\pgfqpoint{3.438533in}{1.538017in}}%
\pgfpathlineto{\pgfqpoint{3.438850in}{1.537846in}}%
\pgfpathlineto{\pgfqpoint{3.439168in}{1.536395in}}%
\pgfpathlineto{\pgfqpoint{3.440120in}{1.536754in}}%
\pgfpathlineto{\pgfqpoint{3.440755in}{1.537352in}}%
\pgfpathlineto{\pgfqpoint{3.442025in}{1.549359in}}%
\pgfpathlineto{\pgfqpoint{3.443612in}{1.535345in}}%
\pgfpathlineto{\pgfqpoint{3.443930in}{1.531086in}}%
\pgfpathlineto{\pgfqpoint{3.444882in}{1.531252in}}%
\pgfpathlineto{\pgfqpoint{3.446152in}{1.534640in}}%
\pgfpathlineto{\pgfqpoint{3.447422in}{1.533072in}}%
\pgfpathlineto{\pgfqpoint{3.449009in}{1.537269in}}%
\pgfpathlineto{\pgfqpoint{3.449326in}{1.544575in}}%
\pgfpathlineto{\pgfqpoint{3.450279in}{1.542882in}}%
\pgfpathlineto{\pgfqpoint{3.451866in}{1.528958in}}%
\pgfpathlineto{\pgfqpoint{3.452183in}{1.526735in}}%
\pgfpathlineto{\pgfqpoint{3.452501in}{1.529606in}}%
\pgfpathlineto{\pgfqpoint{3.452818in}{1.528743in}}%
\pgfpathlineto{\pgfqpoint{3.454088in}{1.531172in}}%
\pgfpathlineto{\pgfqpoint{3.454406in}{1.530954in}}%
\pgfpathlineto{\pgfqpoint{3.454723in}{1.529547in}}%
\pgfpathlineto{\pgfqpoint{3.455675in}{1.529901in}}%
\pgfpathlineto{\pgfqpoint{3.456310in}{1.530709in}}%
\pgfpathlineto{\pgfqpoint{3.457580in}{1.542533in}}%
\pgfpathlineto{\pgfqpoint{3.459167in}{1.527866in}}%
\pgfpathlineto{\pgfqpoint{3.459485in}{1.523770in}}%
\pgfpathlineto{\pgfqpoint{3.460437in}{1.524749in}}%
\pgfpathlineto{\pgfqpoint{3.461707in}{1.527772in}}%
\pgfpathlineto{\pgfqpoint{3.462977in}{1.526260in}}%
\pgfpathlineto{\pgfqpoint{3.463929in}{1.529281in}}%
\pgfpathlineto{\pgfqpoint{3.464247in}{1.527100in}}%
\pgfpathlineto{\pgfqpoint{3.464882in}{1.538128in}}%
\pgfpathlineto{\pgfqpoint{3.465834in}{1.535139in}}%
\pgfpathlineto{\pgfqpoint{3.467421in}{1.521914in}}%
\pgfpathlineto{\pgfqpoint{3.467739in}{1.520023in}}%
\pgfpathlineto{\pgfqpoint{3.468056in}{1.522899in}}%
\pgfpathlineto{\pgfqpoint{3.468374in}{1.522061in}}%
\pgfpathlineto{\pgfqpoint{3.469644in}{1.524338in}}%
\pgfpathlineto{\pgfqpoint{3.469961in}{1.524050in}}%
\pgfpathlineto{\pgfqpoint{3.470279in}{1.522668in}}%
\pgfpathlineto{\pgfqpoint{3.471231in}{1.523076in}}%
\pgfpathlineto{\pgfqpoint{3.471548in}{1.522996in}}%
\pgfpathlineto{\pgfqpoint{3.471866in}{1.524254in}}%
\pgfpathlineto{\pgfqpoint{3.473136in}{1.535347in}}%
\pgfpathlineto{\pgfqpoint{3.474723in}{1.520476in}}%
\pgfpathlineto{\pgfqpoint{3.475040in}{1.516711in}}%
\pgfpathlineto{\pgfqpoint{3.475993in}{1.518227in}}%
\pgfpathlineto{\pgfqpoint{3.477263in}{1.520915in}}%
\pgfpathlineto{\pgfqpoint{3.478532in}{1.519420in}}%
\pgfpathlineto{\pgfqpoint{3.479167in}{1.520160in}}%
\pgfpathlineto{\pgfqpoint{3.480437in}{1.531869in}}%
\pgfpathlineto{\pgfqpoint{3.481072in}{1.528019in}}%
\pgfpathlineto{\pgfqpoint{3.481390in}{1.527318in}}%
\pgfpathlineto{\pgfqpoint{3.482977in}{1.514940in}}%
\pgfpathlineto{\pgfqpoint{3.483294in}{1.513438in}}%
\pgfpathlineto{\pgfqpoint{3.483612in}{1.516238in}}%
\pgfpathlineto{\pgfqpoint{3.483929in}{1.515411in}}%
\pgfpathlineto{\pgfqpoint{3.485199in}{1.517471in}}%
\pgfpathlineto{\pgfqpoint{3.485516in}{1.517174in}}%
\pgfpathlineto{\pgfqpoint{3.485834in}{1.515841in}}%
\pgfpathlineto{\pgfqpoint{3.486786in}{1.516415in}}%
\pgfpathlineto{\pgfqpoint{3.487104in}{1.516105in}}%
\pgfpathlineto{\pgfqpoint{3.488691in}{1.528323in}}%
\pgfpathlineto{\pgfqpoint{3.489009in}{1.520054in}}%
\pgfpathlineto{\pgfqpoint{3.489326in}{1.521364in}}%
\pgfpathlineto{\pgfqpoint{3.490596in}{1.509638in}}%
\pgfpathlineto{\pgfqpoint{3.491231in}{1.511453in}}%
\pgfpathlineto{\pgfqpoint{3.491548in}{1.511690in}}%
\pgfpathlineto{\pgfqpoint{3.492818in}{1.514042in}}%
\pgfpathlineto{\pgfqpoint{3.494088in}{1.512615in}}%
\pgfpathlineto{\pgfqpoint{3.494723in}{1.513305in}}%
\pgfpathlineto{\pgfqpoint{3.495993in}{1.525166in}}%
\pgfpathlineto{\pgfqpoint{3.497262in}{1.512784in}}%
\pgfpathlineto{\pgfqpoint{3.497580in}{1.512916in}}%
\pgfpathlineto{\pgfqpoint{3.498850in}{1.506810in}}%
\pgfpathlineto{\pgfqpoint{3.500120in}{1.510705in}}%
\pgfpathlineto{\pgfqpoint{3.501389in}{1.508985in}}%
\pgfpathlineto{\pgfqpoint{3.502977in}{1.511157in}}%
\pgfpathlineto{\pgfqpoint{3.504246in}{1.521129in}}%
\pgfpathlineto{\pgfqpoint{3.505834in}{1.506049in}}%
\pgfpathlineto{\pgfqpoint{3.506151in}{1.502787in}}%
\pgfpathlineto{\pgfqpoint{3.506786in}{1.504747in}}%
\pgfpathlineto{\pgfqpoint{3.509326in}{1.506616in}}%
\pgfpathlineto{\pgfqpoint{3.509643in}{1.505777in}}%
\pgfpathlineto{\pgfqpoint{3.510278in}{1.506459in}}%
\pgfpathlineto{\pgfqpoint{3.511548in}{1.518689in}}%
\pgfpathlineto{\pgfqpoint{3.512818in}{1.505518in}}%
\pgfpathlineto{\pgfqpoint{3.513135in}{1.505563in}}%
\pgfpathlineto{\pgfqpoint{3.514405in}{1.500224in}}%
\pgfpathlineto{\pgfqpoint{3.515675in}{1.503878in}}%
\pgfpathlineto{\pgfqpoint{3.516945in}{1.502176in}}%
\pgfpathlineto{\pgfqpoint{3.518532in}{1.504549in}}%
\pgfpathlineto{\pgfqpoint{3.519802in}{1.514016in}}%
\pgfpathlineto{\pgfqpoint{3.521389in}{1.498941in}}%
\pgfpathlineto{\pgfqpoint{3.521707in}{1.495904in}}%
\pgfpathlineto{\pgfqpoint{3.522342in}{1.498035in}}%
\pgfpathlineto{\pgfqpoint{3.523929in}{1.500337in}}%
\pgfpathlineto{\pgfqpoint{3.524246in}{1.498758in}}%
\pgfpathlineto{\pgfqpoint{3.525199in}{1.498979in}}%
\pgfpathlineto{\pgfqpoint{3.525834in}{1.499621in}}%
\pgfpathlineto{\pgfqpoint{3.527103in}{1.511895in}}%
\pgfpathlineto{\pgfqpoint{3.528373in}{1.498231in}}%
\pgfpathlineto{\pgfqpoint{3.528691in}{1.498302in}}%
\pgfpathlineto{\pgfqpoint{3.529961in}{1.493594in}}%
\pgfpathlineto{\pgfqpoint{3.531230in}{1.497071in}}%
\pgfpathlineto{\pgfqpoint{3.532500in}{1.495340in}}%
\pgfpathlineto{\pgfqpoint{3.534088in}{1.498125in}}%
\pgfpathlineto{\pgfqpoint{3.535357in}{1.506729in}}%
\pgfpathlineto{\pgfqpoint{3.536627in}{1.491869in}}%
\pgfpathlineto{\pgfqpoint{3.536945in}{1.491909in}}%
\pgfpathlineto{\pgfqpoint{3.537262in}{1.489198in}}%
\pgfpathlineto{\pgfqpoint{3.537897in}{1.491362in}}%
\pgfpathlineto{\pgfqpoint{3.539484in}{1.493516in}}%
\pgfpathlineto{\pgfqpoint{3.539802in}{1.491952in}}%
\pgfpathlineto{\pgfqpoint{3.540754in}{1.492157in}}%
\pgfpathlineto{\pgfqpoint{3.541389in}{1.492801in}}%
\pgfpathlineto{\pgfqpoint{3.542659in}{1.505361in}}%
\pgfpathlineto{\pgfqpoint{3.543929in}{1.491077in}}%
\pgfpathlineto{\pgfqpoint{3.544246in}{1.491108in}}%
\pgfpathlineto{\pgfqpoint{3.544564in}{1.486681in}}%
\pgfpathlineto{\pgfqpoint{3.545516in}{1.486998in}}%
\pgfpathlineto{\pgfqpoint{3.546786in}{1.490255in}}%
\pgfpathlineto{\pgfqpoint{3.548056in}{1.488553in}}%
\pgfpathlineto{\pgfqpoint{3.549643in}{1.491378in}}%
\pgfpathlineto{\pgfqpoint{3.550913in}{1.499667in}}%
\pgfpathlineto{\pgfqpoint{3.552183in}{1.484867in}}%
\pgfpathlineto{\pgfqpoint{3.552500in}{1.484959in}}%
\pgfpathlineto{\pgfqpoint{3.552818in}{1.482430in}}%
\pgfpathlineto{\pgfqpoint{3.553452in}{1.484652in}}%
\pgfpathlineto{\pgfqpoint{3.555040in}{1.486692in}}%
\pgfpathlineto{\pgfqpoint{3.555357in}{1.485125in}}%
\pgfpathlineto{\pgfqpoint{3.556310in}{1.485374in}}%
\pgfpathlineto{\pgfqpoint{3.556944in}{1.485994in}}%
\pgfpathlineto{\pgfqpoint{3.558214in}{1.498579in}}%
\pgfpathlineto{\pgfqpoint{3.559484in}{1.483933in}}%
\pgfpathlineto{\pgfqpoint{3.559802in}{1.484035in}}%
\pgfpathlineto{\pgfqpoint{3.560119in}{1.479812in}}%
\pgfpathlineto{\pgfqpoint{3.561071in}{1.480336in}}%
\pgfpathlineto{\pgfqpoint{3.562341in}{1.483464in}}%
\pgfpathlineto{\pgfqpoint{3.563611in}{1.481743in}}%
\pgfpathlineto{\pgfqpoint{3.565198in}{1.484701in}}%
\pgfpathlineto{\pgfqpoint{3.565516in}{1.493041in}}%
\pgfpathlineto{\pgfqpoint{3.566468in}{1.492619in}}%
\pgfpathlineto{\pgfqpoint{3.567738in}{1.477858in}}%
\pgfpathlineto{\pgfqpoint{3.568055in}{1.478095in}}%
\pgfpathlineto{\pgfqpoint{3.568373in}{1.475789in}}%
\pgfpathlineto{\pgfqpoint{3.569008in}{1.477955in}}%
\pgfpathlineto{\pgfqpoint{3.570595in}{1.479911in}}%
\pgfpathlineto{\pgfqpoint{3.570913in}{1.478358in}}%
\pgfpathlineto{\pgfqpoint{3.571865in}{1.478567in}}%
\pgfpathlineto{\pgfqpoint{3.572500in}{1.479208in}}%
\pgfpathlineto{\pgfqpoint{3.573770in}{1.492019in}}%
\pgfpathlineto{\pgfqpoint{3.575040in}{1.476844in}}%
\pgfpathlineto{\pgfqpoint{3.575357in}{1.476942in}}%
\pgfpathlineto{\pgfqpoint{3.575674in}{1.472849in}}%
\pgfpathlineto{\pgfqpoint{3.576627in}{1.473725in}}%
\pgfpathlineto{\pgfqpoint{3.577897in}{1.476678in}}%
\pgfpathlineto{\pgfqpoint{3.579166in}{1.474987in}}%
\pgfpathlineto{\pgfqpoint{3.580754in}{1.477785in}}%
\pgfpathlineto{\pgfqpoint{3.581071in}{1.486361in}}%
\pgfpathlineto{\pgfqpoint{3.582024in}{1.485729in}}%
\pgfpathlineto{\pgfqpoint{3.583293in}{1.470980in}}%
\pgfpathlineto{\pgfqpoint{3.583611in}{1.471274in}}%
\pgfpathlineto{\pgfqpoint{3.583928in}{1.469075in}}%
\pgfpathlineto{\pgfqpoint{3.584563in}{1.471240in}}%
\pgfpathlineto{\pgfqpoint{3.586151in}{1.473135in}}%
\pgfpathlineto{\pgfqpoint{3.586468in}{1.471577in}}%
\pgfpathlineto{\pgfqpoint{3.587420in}{1.471809in}}%
\pgfpathlineto{\pgfqpoint{3.588055in}{1.472441in}}%
\pgfpathlineto{\pgfqpoint{3.589325in}{1.485248in}}%
\pgfpathlineto{\pgfqpoint{3.590595in}{1.469770in}}%
\pgfpathlineto{\pgfqpoint{3.590912in}{1.469976in}}%
\pgfpathlineto{\pgfqpoint{3.591230in}{1.466133in}}%
\pgfpathlineto{\pgfqpoint{3.592182in}{1.467065in}}%
\pgfpathlineto{\pgfqpoint{3.593452in}{1.469926in}}%
\pgfpathlineto{\pgfqpoint{3.594722in}{1.468215in}}%
\pgfpathlineto{\pgfqpoint{3.596309in}{1.470993in}}%
\pgfpathlineto{\pgfqpoint{3.596627in}{1.479961in}}%
\pgfpathlineto{\pgfqpoint{3.597579in}{1.478863in}}%
\pgfpathlineto{\pgfqpoint{3.598849in}{1.464096in}}%
\pgfpathlineto{\pgfqpoint{3.599166in}{1.464531in}}%
\pgfpathlineto{\pgfqpoint{3.599484in}{1.462468in}}%
\pgfpathlineto{\pgfqpoint{3.600119in}{1.464537in}}%
\pgfpathlineto{\pgfqpoint{3.601706in}{1.466409in}}%
\pgfpathlineto{\pgfqpoint{3.602976in}{1.465032in}}%
\pgfpathlineto{\pgfqpoint{3.603611in}{1.465702in}}%
\pgfpathlineto{\pgfqpoint{3.604881in}{1.478797in}}%
\pgfpathlineto{\pgfqpoint{3.606150in}{1.462918in}}%
\pgfpathlineto{\pgfqpoint{3.606468in}{1.463127in}}%
\pgfpathlineto{\pgfqpoint{3.606785in}{1.459379in}}%
\pgfpathlineto{\pgfqpoint{3.607738in}{1.460378in}}%
\pgfpathlineto{\pgfqpoint{3.609008in}{1.463167in}}%
\pgfpathlineto{\pgfqpoint{3.610277in}{1.461490in}}%
\pgfpathlineto{\pgfqpoint{3.611865in}{1.464110in}}%
\pgfpathlineto{\pgfqpoint{3.612182in}{1.473312in}}%
\pgfpathlineto{\pgfqpoint{3.613134in}{1.472023in}}%
\pgfpathlineto{\pgfqpoint{3.614404in}{1.457322in}}%
\pgfpathlineto{\pgfqpoint{3.614722in}{1.457815in}}%
\pgfpathlineto{\pgfqpoint{3.615039in}{1.455797in}}%
\pgfpathlineto{\pgfqpoint{3.615674in}{1.457838in}}%
\pgfpathlineto{\pgfqpoint{3.617261in}{1.459693in}}%
\pgfpathlineto{\pgfqpoint{3.618531in}{1.458311in}}%
\pgfpathlineto{\pgfqpoint{3.619166in}{1.458985in}}%
\pgfpathlineto{\pgfqpoint{3.620436in}{1.472156in}}%
\pgfpathlineto{\pgfqpoint{3.621706in}{1.456089in}}%
\pgfpathlineto{\pgfqpoint{3.622023in}{1.456404in}}%
\pgfpathlineto{\pgfqpoint{3.622341in}{1.452855in}}%
\pgfpathlineto{\pgfqpoint{3.623293in}{1.453650in}}%
\pgfpathlineto{\pgfqpoint{3.624563in}{1.456441in}}%
\pgfpathlineto{\pgfqpoint{3.625833in}{1.454761in}}%
\pgfpathlineto{\pgfqpoint{3.627420in}{1.457319in}}%
\pgfpathlineto{\pgfqpoint{3.627738in}{1.466912in}}%
\pgfpathlineto{\pgfqpoint{3.628690in}{1.465273in}}%
\pgfpathlineto{\pgfqpoint{3.629960in}{1.450551in}}%
\pgfpathlineto{\pgfqpoint{3.630277in}{1.451182in}}%
\pgfpathlineto{\pgfqpoint{3.630595in}{1.449233in}}%
\pgfpathlineto{\pgfqpoint{3.631230in}{1.451154in}}%
\pgfpathlineto{\pgfqpoint{3.632817in}{1.453027in}}%
\pgfpathlineto{\pgfqpoint{3.634087in}{1.451588in}}%
\pgfpathlineto{\pgfqpoint{3.634722in}{1.452298in}}%
\pgfpathlineto{\pgfqpoint{3.635991in}{1.465744in}}%
\pgfpathlineto{\pgfqpoint{3.637261in}{1.449422in}}%
\pgfpathlineto{\pgfqpoint{3.637579in}{1.449736in}}%
\pgfpathlineto{\pgfqpoint{3.637896in}{1.446260in}}%
\pgfpathlineto{\pgfqpoint{3.638849in}{1.446956in}}%
\pgfpathlineto{\pgfqpoint{3.640118in}{1.449711in}}%
\pgfpathlineto{\pgfqpoint{3.641388in}{1.448104in}}%
\pgfpathlineto{\pgfqpoint{3.642975in}{1.450254in}}%
\pgfpathlineto{\pgfqpoint{3.643293in}{1.460078in}}%
\pgfpathlineto{\pgfqpoint{3.644245in}{1.458791in}}%
\pgfpathlineto{\pgfqpoint{3.645515in}{1.443935in}}%
\pgfpathlineto{\pgfqpoint{3.645833in}{1.444600in}}%
\pgfpathlineto{\pgfqpoint{3.646150in}{1.442602in}}%
\pgfpathlineto{\pgfqpoint{3.646785in}{1.444448in}}%
\pgfpathlineto{\pgfqpoint{3.648372in}{1.446366in}}%
\pgfpathlineto{\pgfqpoint{3.649642in}{1.444933in}}%
\pgfpathlineto{\pgfqpoint{3.650277in}{1.445645in}}%
\pgfpathlineto{\pgfqpoint{3.651547in}{1.459097in}}%
\pgfpathlineto{\pgfqpoint{3.652817in}{1.442772in}}%
\pgfpathlineto{\pgfqpoint{3.653134in}{1.443187in}}%
\pgfpathlineto{\pgfqpoint{3.653452in}{1.439890in}}%
\pgfpathlineto{\pgfqpoint{3.654404in}{1.440244in}}%
\pgfpathlineto{\pgfqpoint{3.655674in}{1.442999in}}%
\pgfpathlineto{\pgfqpoint{3.656944in}{1.441456in}}%
\pgfpathlineto{\pgfqpoint{3.658531in}{1.443385in}}%
\pgfpathlineto{\pgfqpoint{3.658848in}{1.453521in}}%
\pgfpathlineto{\pgfqpoint{3.659801in}{1.452297in}}%
\pgfpathlineto{\pgfqpoint{3.661071in}{1.437357in}}%
\pgfpathlineto{\pgfqpoint{3.661388in}{1.438131in}}%
\pgfpathlineto{\pgfqpoint{3.661705in}{1.436121in}}%
\pgfpathlineto{\pgfqpoint{3.662340in}{1.437766in}}%
\pgfpathlineto{\pgfqpoint{3.665832in}{1.439021in}}%
\pgfpathlineto{\pgfqpoint{3.667102in}{1.452725in}}%
\pgfpathlineto{\pgfqpoint{3.668372in}{1.436402in}}%
\pgfpathlineto{\pgfqpoint{3.668690in}{1.436779in}}%
\pgfpathlineto{\pgfqpoint{3.669959in}{1.433547in}}%
\pgfpathlineto{\pgfqpoint{3.671229in}{1.436223in}}%
\pgfpathlineto{\pgfqpoint{3.671864in}{1.436462in}}%
\pgfpathlineto{\pgfqpoint{3.672499in}{1.434886in}}%
\pgfpathlineto{\pgfqpoint{3.674086in}{1.436525in}}%
\pgfpathlineto{\pgfqpoint{3.674404in}{1.446695in}}%
\pgfpathlineto{\pgfqpoint{3.675356in}{1.445818in}}%
\pgfpathlineto{\pgfqpoint{3.676626in}{1.430921in}}%
\pgfpathlineto{\pgfqpoint{3.676943in}{1.431706in}}%
\pgfpathlineto{\pgfqpoint{3.677261in}{1.429618in}}%
\pgfpathlineto{\pgfqpoint{3.677896in}{1.431128in}}%
\pgfpathlineto{\pgfqpoint{3.678213in}{1.431133in}}%
\pgfpathlineto{\pgfqpoint{3.679483in}{1.433061in}}%
\pgfpathlineto{\pgfqpoint{3.680753in}{1.431698in}}%
\pgfpathlineto{\pgfqpoint{3.681388in}{1.432432in}}%
\pgfpathlineto{\pgfqpoint{3.682658in}{1.446007in}}%
\pgfpathlineto{\pgfqpoint{3.682975in}{1.437884in}}%
\pgfpathlineto{\pgfqpoint{3.683293in}{1.437892in}}%
\pgfpathlineto{\pgfqpoint{3.685515in}{1.426821in}}%
\pgfpathlineto{\pgfqpoint{3.686150in}{1.428026in}}%
\pgfpathlineto{\pgfqpoint{3.687737in}{1.429746in}}%
\pgfpathlineto{\pgfqpoint{3.688054in}{1.428336in}}%
\pgfpathlineto{\pgfqpoint{3.689007in}{1.428705in}}%
\pgfpathlineto{\pgfqpoint{3.689642in}{1.429724in}}%
\pgfpathlineto{\pgfqpoint{3.689959in}{1.439886in}}%
\pgfpathlineto{\pgfqpoint{3.690912in}{1.439503in}}%
\pgfpathlineto{\pgfqpoint{3.692181in}{1.424625in}}%
\pgfpathlineto{\pgfqpoint{3.692499in}{1.425473in}}%
\pgfpathlineto{\pgfqpoint{3.692816in}{1.423358in}}%
\pgfpathlineto{\pgfqpoint{3.693769in}{1.424273in}}%
\pgfpathlineto{\pgfqpoint{3.695039in}{1.426436in}}%
\pgfpathlineto{\pgfqpoint{3.696308in}{1.425142in}}%
\pgfpathlineto{\pgfqpoint{3.697578in}{1.429378in}}%
\pgfpathlineto{\pgfqpoint{3.697896in}{1.430857in}}%
\pgfpathlineto{\pgfqpoint{3.698213in}{1.439325in}}%
\pgfpathlineto{\pgfqpoint{3.698848in}{1.431426in}}%
\pgfpathlineto{\pgfqpoint{3.699165in}{1.430402in}}%
\pgfpathlineto{\pgfqpoint{3.700435in}{1.420402in}}%
\pgfpathlineto{\pgfqpoint{3.701705in}{1.421424in}}%
\pgfpathlineto{\pgfqpoint{3.704245in}{1.422655in}}%
\pgfpathlineto{\pgfqpoint{3.704880in}{1.422238in}}%
\pgfpathlineto{\pgfqpoint{3.705197in}{1.422974in}}%
\pgfpathlineto{\pgfqpoint{3.706467in}{1.433338in}}%
\pgfpathlineto{\pgfqpoint{3.707737in}{1.418557in}}%
\pgfpathlineto{\pgfqpoint{3.708054in}{1.419339in}}%
\pgfpathlineto{\pgfqpoint{3.708372in}{1.417162in}}%
\pgfpathlineto{\pgfqpoint{3.709324in}{1.417522in}}%
\pgfpathlineto{\pgfqpoint{3.711229in}{1.420109in}}%
\pgfpathlineto{\pgfqpoint{3.711864in}{1.418671in}}%
\pgfpathlineto{\pgfqpoint{3.712499in}{1.419430in}}%
\pgfpathlineto{\pgfqpoint{3.713769in}{1.432342in}}%
\pgfpathlineto{\pgfqpoint{3.714086in}{1.425613in}}%
\pgfpathlineto{\pgfqpoint{3.714721in}{1.424685in}}%
\pgfpathlineto{\pgfqpoint{3.715991in}{1.414076in}}%
\pgfpathlineto{\pgfqpoint{3.716308in}{1.415128in}}%
\pgfpathlineto{\pgfqpoint{3.716626in}{1.413805in}}%
\pgfpathlineto{\pgfqpoint{3.717261in}{1.414758in}}%
\pgfpathlineto{\pgfqpoint{3.719800in}{1.416242in}}%
\pgfpathlineto{\pgfqpoint{3.720435in}{1.415863in}}%
\pgfpathlineto{\pgfqpoint{3.720753in}{1.416565in}}%
\pgfpathlineto{\pgfqpoint{3.722022in}{1.426880in}}%
\pgfpathlineto{\pgfqpoint{3.723292in}{1.412427in}}%
\pgfpathlineto{\pgfqpoint{3.723610in}{1.413270in}}%
\pgfpathlineto{\pgfqpoint{3.724245in}{1.410792in}}%
\pgfpathlineto{\pgfqpoint{3.724880in}{1.410944in}}%
\pgfpathlineto{\pgfqpoint{3.726784in}{1.413729in}}%
\pgfpathlineto{\pgfqpoint{3.727102in}{1.413401in}}%
\pgfpathlineto{\pgfqpoint{3.727419in}{1.412226in}}%
\pgfpathlineto{\pgfqpoint{3.728054in}{1.413023in}}%
\pgfpathlineto{\pgfqpoint{3.729324in}{1.425574in}}%
\pgfpathlineto{\pgfqpoint{3.729959in}{1.418886in}}%
\pgfpathlineto{\pgfqpoint{3.730276in}{1.419028in}}%
\pgfpathlineto{\pgfqpoint{3.731546in}{1.407861in}}%
\pgfpathlineto{\pgfqpoint{3.731864in}{1.409013in}}%
\pgfpathlineto{\pgfqpoint{3.732181in}{1.407607in}}%
\pgfpathlineto{\pgfqpoint{3.732816in}{1.408142in}}%
\pgfpathlineto{\pgfqpoint{3.735038in}{1.410177in}}%
\pgfpathlineto{\pgfqpoint{3.735356in}{1.409876in}}%
\pgfpathlineto{\pgfqpoint{3.735673in}{1.409322in}}%
\pgfpathlineto{\pgfqpoint{3.736308in}{1.410127in}}%
\pgfpathlineto{\pgfqpoint{3.737578in}{1.420624in}}%
\pgfpathlineto{\pgfqpoint{3.738848in}{1.406533in}}%
\pgfpathlineto{\pgfqpoint{3.739165in}{1.407293in}}%
\pgfpathlineto{\pgfqpoint{3.739800in}{1.404377in}}%
\pgfpathlineto{\pgfqpoint{3.740435in}{1.404605in}}%
\pgfpathlineto{\pgfqpoint{3.743927in}{1.408543in}}%
\pgfpathlineto{\pgfqpoint{3.744879in}{1.418636in}}%
\pgfpathlineto{\pgfqpoint{3.745832in}{1.413136in}}%
\pgfpathlineto{\pgfqpoint{3.747102in}{1.401762in}}%
\pgfpathlineto{\pgfqpoint{3.747419in}{1.402942in}}%
\pgfpathlineto{\pgfqpoint{3.748054in}{1.401245in}}%
\pgfpathlineto{\pgfqpoint{3.748689in}{1.402083in}}%
\pgfpathlineto{\pgfqpoint{3.750594in}{1.403897in}}%
\pgfpathlineto{\pgfqpoint{3.750911in}{1.403595in}}%
\pgfpathlineto{\pgfqpoint{3.751229in}{1.402995in}}%
\pgfpathlineto{\pgfqpoint{3.751863in}{1.403776in}}%
\pgfpathlineto{\pgfqpoint{3.753133in}{1.414292in}}%
\pgfpathlineto{\pgfqpoint{3.754403in}{1.400806in}}%
\pgfpathlineto{\pgfqpoint{3.754721in}{1.401529in}}%
\pgfpathlineto{\pgfqpoint{3.756308in}{1.398288in}}%
\pgfpathlineto{\pgfqpoint{3.760117in}{1.404152in}}%
\pgfpathlineto{\pgfqpoint{3.760435in}{1.411839in}}%
\pgfpathlineto{\pgfqpoint{3.761070in}{1.406384in}}%
\pgfpathlineto{\pgfqpoint{3.761387in}{1.407533in}}%
\pgfpathlineto{\pgfqpoint{3.762657in}{1.395902in}}%
\pgfpathlineto{\pgfqpoint{3.762974in}{1.396866in}}%
\pgfpathlineto{\pgfqpoint{3.763609in}{1.394859in}}%
\pgfpathlineto{\pgfqpoint{3.764244in}{1.395773in}}%
\pgfpathlineto{\pgfqpoint{3.767419in}{1.397471in}}%
\pgfpathlineto{\pgfqpoint{3.768689in}{1.408067in}}%
\pgfpathlineto{\pgfqpoint{3.769959in}{1.395247in}}%
\pgfpathlineto{\pgfqpoint{3.770276in}{1.395772in}}%
\pgfpathlineto{\pgfqpoint{3.771863in}{1.391795in}}%
\pgfpathlineto{\pgfqpoint{3.773451in}{1.394939in}}%
\pgfpathlineto{\pgfqpoint{3.773768in}{1.394645in}}%
\pgfpathlineto{\pgfqpoint{3.774085in}{1.393333in}}%
\pgfpathlineto{\pgfqpoint{3.774720in}{1.394212in}}%
\pgfpathlineto{\pgfqpoint{3.775038in}{1.394733in}}%
\pgfpathlineto{\pgfqpoint{3.775990in}{1.404624in}}%
\pgfpathlineto{\pgfqpoint{3.776943in}{1.402153in}}%
\pgfpathlineto{\pgfqpoint{3.778212in}{1.390496in}}%
\pgfpathlineto{\pgfqpoint{3.778530in}{1.390953in}}%
\pgfpathlineto{\pgfqpoint{3.779165in}{1.388533in}}%
\pgfpathlineto{\pgfqpoint{3.779800in}{1.389524in}}%
\pgfpathlineto{\pgfqpoint{3.780435in}{1.389831in}}%
\pgfpathlineto{\pgfqpoint{3.781704in}{1.391534in}}%
\pgfpathlineto{\pgfqpoint{3.782339in}{1.390482in}}%
\pgfpathlineto{\pgfqpoint{3.782974in}{1.391173in}}%
\pgfpathlineto{\pgfqpoint{3.784244in}{1.401748in}}%
\pgfpathlineto{\pgfqpoint{3.786149in}{1.388764in}}%
\pgfpathlineto{\pgfqpoint{3.787419in}{1.385414in}}%
\pgfpathlineto{\pgfqpoint{3.789006in}{1.388807in}}%
\pgfpathlineto{\pgfqpoint{3.789323in}{1.388514in}}%
\pgfpathlineto{\pgfqpoint{3.789641in}{1.387145in}}%
\pgfpathlineto{\pgfqpoint{3.790593in}{1.387956in}}%
\pgfpathlineto{\pgfqpoint{3.791546in}{1.397483in}}%
\pgfpathlineto{\pgfqpoint{3.792498in}{1.396665in}}%
\pgfpathlineto{\pgfqpoint{3.793768in}{1.385167in}}%
\pgfpathlineto{\pgfqpoint{3.794085in}{1.385213in}}%
\pgfpathlineto{\pgfqpoint{3.794720in}{1.382381in}}%
\pgfpathlineto{\pgfqpoint{3.795673in}{1.383149in}}%
\pgfpathlineto{\pgfqpoint{3.797260in}{1.385462in}}%
\pgfpathlineto{\pgfqpoint{3.797577in}{1.385136in}}%
\pgfpathlineto{\pgfqpoint{3.797895in}{1.384284in}}%
\pgfpathlineto{\pgfqpoint{3.798530in}{1.385052in}}%
\pgfpathlineto{\pgfqpoint{3.799800in}{1.395347in}}%
\pgfpathlineto{\pgfqpoint{3.800117in}{1.392143in}}%
\pgfpathlineto{\pgfqpoint{3.802974in}{1.379066in}}%
\pgfpathlineto{\pgfqpoint{3.804561in}{1.382722in}}%
\pgfpathlineto{\pgfqpoint{3.804879in}{1.382474in}}%
\pgfpathlineto{\pgfqpoint{3.805196in}{1.381060in}}%
\pgfpathlineto{\pgfqpoint{3.806149in}{1.381525in}}%
\pgfpathlineto{\pgfqpoint{3.808053in}{1.391311in}}%
\pgfpathlineto{\pgfqpoint{3.810276in}{1.376349in}}%
\pgfpathlineto{\pgfqpoint{3.811545in}{1.377398in}}%
\pgfpathlineto{\pgfqpoint{3.812815in}{1.379459in}}%
\pgfpathlineto{\pgfqpoint{3.813450in}{1.378164in}}%
\pgfpathlineto{\pgfqpoint{3.814085in}{1.378972in}}%
\pgfpathlineto{\pgfqpoint{3.815355in}{1.388202in}}%
\pgfpathlineto{\pgfqpoint{3.816307in}{1.386094in}}%
\pgfpathlineto{\pgfqpoint{3.818212in}{1.374287in}}%
\pgfpathlineto{\pgfqpoint{3.818530in}{1.372850in}}%
\pgfpathlineto{\pgfqpoint{3.818847in}{1.373564in}}%
\pgfpathlineto{\pgfqpoint{3.820117in}{1.376685in}}%
\pgfpathlineto{\pgfqpoint{3.820434in}{1.376460in}}%
\pgfpathlineto{\pgfqpoint{3.820752in}{1.375018in}}%
\pgfpathlineto{\pgfqpoint{3.821704in}{1.375448in}}%
\pgfpathlineto{\pgfqpoint{3.822657in}{1.380950in}}%
\pgfpathlineto{\pgfqpoint{3.823609in}{1.385237in}}%
\pgfpathlineto{\pgfqpoint{3.823926in}{1.383017in}}%
\pgfpathlineto{\pgfqpoint{3.826149in}{1.370481in}}%
\pgfpathlineto{\pgfqpoint{3.826466in}{1.371213in}}%
\pgfpathlineto{\pgfqpoint{3.826783in}{1.370548in}}%
\pgfpathlineto{\pgfqpoint{3.827101in}{1.371133in}}%
\pgfpathlineto{\pgfqpoint{3.828371in}{1.373521in}}%
\pgfpathlineto{\pgfqpoint{3.829958in}{1.372757in}}%
\pgfpathlineto{\pgfqpoint{3.831228in}{1.381945in}}%
\pgfpathlineto{\pgfqpoint{3.831545in}{1.379186in}}%
\pgfpathlineto{\pgfqpoint{3.831863in}{1.380992in}}%
\pgfpathlineto{\pgfqpoint{3.832180in}{1.377428in}}%
\pgfpathlineto{\pgfqpoint{3.834085in}{1.366791in}}%
\pgfpathlineto{\pgfqpoint{3.834402in}{1.367322in}}%
\pgfpathlineto{\pgfqpoint{3.835672in}{1.370656in}}%
\pgfpathlineto{\pgfqpoint{3.835990in}{1.370482in}}%
\pgfpathlineto{\pgfqpoint{3.836307in}{1.369080in}}%
\pgfpathlineto{\pgfqpoint{3.837260in}{1.369331in}}%
\pgfpathlineto{\pgfqpoint{3.837894in}{1.369943in}}%
\pgfpathlineto{\pgfqpoint{3.839164in}{1.378692in}}%
\pgfpathlineto{\pgfqpoint{3.839799in}{1.375469in}}%
\pgfpathlineto{\pgfqpoint{3.840117in}{1.375188in}}%
\pgfpathlineto{\pgfqpoint{3.841704in}{1.364550in}}%
\pgfpathlineto{\pgfqpoint{3.843291in}{1.366982in}}%
\pgfpathlineto{\pgfqpoint{3.843926in}{1.367623in}}%
\pgfpathlineto{\pgfqpoint{3.844561in}{1.366112in}}%
\pgfpathlineto{\pgfqpoint{3.846466in}{1.371750in}}%
\pgfpathlineto{\pgfqpoint{3.846783in}{1.375980in}}%
\pgfpathlineto{\pgfqpoint{3.847418in}{1.375397in}}%
\pgfpathlineto{\pgfqpoint{3.849640in}{1.360980in}}%
\pgfpathlineto{\pgfqpoint{3.849958in}{1.361184in}}%
\pgfpathlineto{\pgfqpoint{3.851228in}{1.364637in}}%
\pgfpathlineto{\pgfqpoint{3.851545in}{1.364471in}}%
\pgfpathlineto{\pgfqpoint{3.852815in}{1.363299in}}%
\pgfpathlineto{\pgfqpoint{3.853767in}{1.363952in}}%
\pgfpathlineto{\pgfqpoint{3.855037in}{1.373465in}}%
\pgfpathlineto{\pgfqpoint{3.857259in}{1.358691in}}%
\pgfpathlineto{\pgfqpoint{3.857577in}{1.359421in}}%
\pgfpathlineto{\pgfqpoint{3.857894in}{1.358369in}}%
\pgfpathlineto{\pgfqpoint{3.858212in}{1.358588in}}%
\pgfpathlineto{\pgfqpoint{3.859482in}{1.361738in}}%
\pgfpathlineto{\pgfqpoint{3.861386in}{1.360545in}}%
\pgfpathlineto{\pgfqpoint{3.861704in}{1.361143in}}%
\pgfpathlineto{\pgfqpoint{3.862974in}{1.369383in}}%
\pgfpathlineto{\pgfqpoint{3.863291in}{1.368487in}}%
\pgfpathlineto{\pgfqpoint{3.865513in}{1.355083in}}%
\pgfpathlineto{\pgfqpoint{3.866783in}{1.358608in}}%
\pgfpathlineto{\pgfqpoint{3.867101in}{1.358438in}}%
\pgfpathlineto{\pgfqpoint{3.868370in}{1.357325in}}%
\pgfpathlineto{\pgfqpoint{3.869323in}{1.357887in}}%
\pgfpathlineto{\pgfqpoint{3.870593in}{1.368009in}}%
\pgfpathlineto{\pgfqpoint{3.870910in}{1.365091in}}%
\pgfpathlineto{\pgfqpoint{3.871228in}{1.364618in}}%
\pgfpathlineto{\pgfqpoint{3.873450in}{1.352471in}}%
\pgfpathlineto{\pgfqpoint{3.873767in}{1.352361in}}%
\pgfpathlineto{\pgfqpoint{3.875037in}{1.355849in}}%
\pgfpathlineto{\pgfqpoint{3.876942in}{1.354448in}}%
\pgfpathlineto{\pgfqpoint{3.877259in}{1.355128in}}%
\pgfpathlineto{\pgfqpoint{3.877577in}{1.356276in}}%
\pgfpathlineto{\pgfqpoint{3.878847in}{1.363385in}}%
\pgfpathlineto{\pgfqpoint{3.881069in}{1.349135in}}%
\pgfpathlineto{\pgfqpoint{3.882339in}{1.352609in}}%
\pgfpathlineto{\pgfqpoint{3.882656in}{1.352393in}}%
\pgfpathlineto{\pgfqpoint{3.883291in}{1.352730in}}%
\pgfpathlineto{\pgfqpoint{3.883926in}{1.351454in}}%
\pgfpathlineto{\pgfqpoint{3.885513in}{1.354710in}}%
\pgfpathlineto{\pgfqpoint{3.886148in}{1.362529in}}%
\pgfpathlineto{\pgfqpoint{3.886783in}{1.358977in}}%
\pgfpathlineto{\pgfqpoint{3.889005in}{1.346645in}}%
\pgfpathlineto{\pgfqpoint{3.889323in}{1.346264in}}%
\pgfpathlineto{\pgfqpoint{3.890592in}{1.349961in}}%
\pgfpathlineto{\pgfqpoint{3.892497in}{1.348651in}}%
\pgfpathlineto{\pgfqpoint{3.892815in}{1.349261in}}%
\pgfpathlineto{\pgfqpoint{3.893132in}{1.349798in}}%
\pgfpathlineto{\pgfqpoint{3.894402in}{1.358017in}}%
\pgfpathlineto{\pgfqpoint{3.896624in}{1.343223in}}%
\pgfpathlineto{\pgfqpoint{3.897894in}{1.346667in}}%
\pgfpathlineto{\pgfqpoint{3.898211in}{1.346415in}}%
\pgfpathlineto{\pgfqpoint{3.898846in}{1.346922in}}%
\pgfpathlineto{\pgfqpoint{3.899481in}{1.345557in}}%
\pgfpathlineto{\pgfqpoint{3.901069in}{1.349004in}}%
\pgfpathlineto{\pgfqpoint{3.901703in}{1.356931in}}%
\pgfpathlineto{\pgfqpoint{3.902338in}{1.352945in}}%
\pgfpathlineto{\pgfqpoint{3.904878in}{1.340390in}}%
\pgfpathlineto{\pgfqpoint{3.906148in}{1.344124in}}%
\pgfpathlineto{\pgfqpoint{3.908053in}{1.342883in}}%
\pgfpathlineto{\pgfqpoint{3.908370in}{1.343427in}}%
\pgfpathlineto{\pgfqpoint{3.908688in}{1.343971in}}%
\pgfpathlineto{\pgfqpoint{3.909957in}{1.351971in}}%
\pgfpathlineto{\pgfqpoint{3.912180in}{1.337433in}}%
\pgfpathlineto{\pgfqpoint{3.913449in}{1.340815in}}%
\pgfpathlineto{\pgfqpoint{3.913767in}{1.340540in}}%
\pgfpathlineto{\pgfqpoint{3.914402in}{1.341117in}}%
\pgfpathlineto{\pgfqpoint{3.915037in}{1.339680in}}%
\pgfpathlineto{\pgfqpoint{3.916941in}{1.345810in}}%
\pgfpathlineto{\pgfqpoint{3.917259in}{1.351520in}}%
\pgfpathlineto{\pgfqpoint{3.917894in}{1.346733in}}%
\pgfpathlineto{\pgfqpoint{3.920116in}{1.335239in}}%
\pgfpathlineto{\pgfqpoint{3.920433in}{1.334656in}}%
\pgfpathlineto{\pgfqpoint{3.921703in}{1.338303in}}%
\pgfpathlineto{\pgfqpoint{3.923608in}{1.337119in}}%
\pgfpathlineto{\pgfqpoint{3.923926in}{1.337595in}}%
\pgfpathlineto{\pgfqpoint{3.924243in}{1.338683in}}%
\pgfpathlineto{\pgfqpoint{3.924560in}{1.346818in}}%
\pgfpathlineto{\pgfqpoint{3.925513in}{1.345485in}}%
\pgfpathlineto{\pgfqpoint{3.927418in}{1.332290in}}%
\pgfpathlineto{\pgfqpoint{3.927735in}{1.331678in}}%
\pgfpathlineto{\pgfqpoint{3.929005in}{1.335095in}}%
\pgfpathlineto{\pgfqpoint{3.929322in}{1.334855in}}%
\pgfpathlineto{\pgfqpoint{3.929957in}{1.335291in}}%
\pgfpathlineto{\pgfqpoint{3.930592in}{1.333748in}}%
\pgfpathlineto{\pgfqpoint{3.932497in}{1.340929in}}%
\pgfpathlineto{\pgfqpoint{3.932814in}{1.345729in}}%
\pgfpathlineto{\pgfqpoint{3.933449in}{1.340026in}}%
\pgfpathlineto{\pgfqpoint{3.935037in}{1.329346in}}%
\pgfpathlineto{\pgfqpoint{3.935354in}{1.330434in}}%
\pgfpathlineto{\pgfqpoint{3.935989in}{1.329149in}}%
\pgfpathlineto{\pgfqpoint{3.937259in}{1.332518in}}%
\pgfpathlineto{\pgfqpoint{3.938846in}{1.330988in}}%
\pgfpathlineto{\pgfqpoint{3.939481in}{1.332218in}}%
\pgfpathlineto{\pgfqpoint{3.939798in}{1.333949in}}%
\pgfpathlineto{\pgfqpoint{3.940116in}{1.342566in}}%
\pgfpathlineto{\pgfqpoint{3.941068in}{1.338346in}}%
\pgfpathlineto{\pgfqpoint{3.942973in}{1.326156in}}%
\pgfpathlineto{\pgfqpoint{3.943290in}{1.326167in}}%
\pgfpathlineto{\pgfqpoint{3.944560in}{1.329418in}}%
\pgfpathlineto{\pgfqpoint{3.944878in}{1.329220in}}%
\pgfpathlineto{\pgfqpoint{3.945513in}{1.329443in}}%
\pgfpathlineto{\pgfqpoint{3.946148in}{1.327833in}}%
\pgfpathlineto{\pgfqpoint{3.946782in}{1.328871in}}%
\pgfpathlineto{\pgfqpoint{3.947100in}{1.328418in}}%
\pgfpathlineto{\pgfqpoint{3.948370in}{1.339489in}}%
\pgfpathlineto{\pgfqpoint{3.950274in}{1.323071in}}%
\pgfpathlineto{\pgfqpoint{3.953132in}{1.326507in}}%
\pgfpathlineto{\pgfqpoint{3.953449in}{1.324887in}}%
\pgfpathlineto{\pgfqpoint{3.954401in}{1.325167in}}%
\pgfpathlineto{\pgfqpoint{3.955671in}{1.337833in}}%
\pgfpathlineto{\pgfqpoint{3.956306in}{1.332797in}}%
\pgfpathlineto{\pgfqpoint{3.958528in}{1.320240in}}%
\pgfpathlineto{\pgfqpoint{3.958846in}{1.320948in}}%
\pgfpathlineto{\pgfqpoint{3.960116in}{1.323729in}}%
\pgfpathlineto{\pgfqpoint{3.960433in}{1.323581in}}%
\pgfpathlineto{\pgfqpoint{3.961703in}{1.321896in}}%
\pgfpathlineto{\pgfqpoint{3.962655in}{1.323314in}}%
\pgfpathlineto{\pgfqpoint{3.962973in}{1.332811in}}%
\pgfpathlineto{\pgfqpoint{3.963925in}{1.332134in}}%
\pgfpathlineto{\pgfqpoint{3.965830in}{1.316669in}}%
\pgfpathlineto{\pgfqpoint{3.967417in}{1.320575in}}%
\pgfpathlineto{\pgfqpoint{3.967735in}{1.320453in}}%
\pgfpathlineto{\pgfqpoint{3.968370in}{1.320830in}}%
\pgfpathlineto{\pgfqpoint{3.969004in}{1.318931in}}%
\pgfpathlineto{\pgfqpoint{3.971227in}{1.332048in}}%
\pgfpathlineto{\pgfqpoint{3.971544in}{1.326359in}}%
\pgfpathlineto{\pgfqpoint{3.971862in}{1.325686in}}%
\pgfpathlineto{\pgfqpoint{3.973131in}{1.313973in}}%
\pgfpathlineto{\pgfqpoint{3.973766in}{1.315915in}}%
\pgfpathlineto{\pgfqpoint{3.974084in}{1.314587in}}%
\pgfpathlineto{\pgfqpoint{3.974401in}{1.315838in}}%
\pgfpathlineto{\pgfqpoint{3.975671in}{1.317951in}}%
\pgfpathlineto{\pgfqpoint{3.975989in}{1.317823in}}%
\pgfpathlineto{\pgfqpoint{3.977258in}{1.316033in}}%
\pgfpathlineto{\pgfqpoint{3.978211in}{1.319480in}}%
\pgfpathlineto{\pgfqpoint{3.978528in}{1.328949in}}%
\pgfpathlineto{\pgfqpoint{3.979163in}{1.324567in}}%
\pgfpathlineto{\pgfqpoint{3.980116in}{1.317728in}}%
\pgfpathlineto{\pgfqpoint{3.981385in}{1.310638in}}%
\pgfpathlineto{\pgfqpoint{3.982973in}{1.314968in}}%
\pgfpathlineto{\pgfqpoint{3.983290in}{1.314848in}}%
\pgfpathlineto{\pgfqpoint{3.984560in}{1.312960in}}%
\pgfpathlineto{\pgfqpoint{3.986782in}{1.324837in}}%
\pgfpathlineto{\pgfqpoint{3.988687in}{1.307216in}}%
\pgfpathlineto{\pgfqpoint{3.990274in}{1.311834in}}%
\pgfpathlineto{\pgfqpoint{3.991861in}{1.310166in}}%
\pgfpathlineto{\pgfqpoint{3.992179in}{1.311710in}}%
\pgfpathlineto{\pgfqpoint{3.992814in}{1.310189in}}%
\pgfpathlineto{\pgfqpoint{3.994084in}{1.323920in}}%
\pgfpathlineto{\pgfqpoint{3.994401in}{1.318401in}}%
\pgfpathlineto{\pgfqpoint{3.995036in}{1.314235in}}%
\pgfpathlineto{\pgfqpoint{3.995988in}{1.304757in}}%
\pgfpathlineto{\pgfqpoint{3.996941in}{1.305102in}}%
\pgfpathlineto{\pgfqpoint{3.998528in}{1.309110in}}%
\pgfpathlineto{\pgfqpoint{4.000115in}{1.307086in}}%
\pgfpathlineto{\pgfqpoint{4.000750in}{1.308759in}}%
\pgfpathlineto{\pgfqpoint{4.001068in}{1.309394in}}%
\pgfpathlineto{\pgfqpoint{4.001385in}{1.319372in}}%
\pgfpathlineto{\pgfqpoint{4.002338in}{1.316343in}}%
\pgfpathlineto{\pgfqpoint{4.004242in}{1.301146in}}%
\pgfpathlineto{\pgfqpoint{4.005512in}{1.305628in}}%
\pgfpathlineto{\pgfqpoint{4.005830in}{1.306173in}}%
\pgfpathlineto{\pgfqpoint{4.006147in}{1.305820in}}%
\pgfpathlineto{\pgfqpoint{4.007417in}{1.304213in}}%
\pgfpathlineto{\pgfqpoint{4.009639in}{1.316852in}}%
\pgfpathlineto{\pgfqpoint{4.011544in}{1.297806in}}%
\pgfpathlineto{\pgfqpoint{4.012814in}{1.302720in}}%
\pgfpathlineto{\pgfqpoint{4.014083in}{1.303077in}}%
\pgfpathlineto{\pgfqpoint{4.015671in}{1.301277in}}%
\pgfpathlineto{\pgfqpoint{4.016941in}{1.314867in}}%
\pgfpathlineto{\pgfqpoint{4.017258in}{1.309726in}}%
\pgfpathlineto{\pgfqpoint{4.017893in}{1.307135in}}%
\pgfpathlineto{\pgfqpoint{4.019798in}{1.295796in}}%
\pgfpathlineto{\pgfqpoint{4.021068in}{1.300024in}}%
\pgfpathlineto{\pgfqpoint{4.021702in}{1.299737in}}%
\pgfpathlineto{\pgfqpoint{4.022972in}{1.298441in}}%
\pgfpathlineto{\pgfqpoint{4.024242in}{1.309788in}}%
\pgfpathlineto{\pgfqpoint{4.025194in}{1.308869in}}%
\pgfpathlineto{\pgfqpoint{4.027099in}{1.291780in}}%
\pgfpathlineto{\pgfqpoint{4.028369in}{1.297289in}}%
\pgfpathlineto{\pgfqpoint{4.029956in}{1.296700in}}%
\pgfpathlineto{\pgfqpoint{4.030274in}{1.295715in}}%
\pgfpathlineto{\pgfqpoint{4.030591in}{1.297012in}}%
\pgfpathlineto{\pgfqpoint{4.030909in}{1.296910in}}%
\pgfpathlineto{\pgfqpoint{4.031226in}{1.296810in}}%
\pgfpathlineto{\pgfqpoint{4.032496in}{1.307904in}}%
\pgfpathlineto{\pgfqpoint{4.034401in}{1.289094in}}%
\pgfpathlineto{\pgfqpoint{4.035988in}{1.294298in}}%
\pgfpathlineto{\pgfqpoint{4.036306in}{1.293185in}}%
\pgfpathlineto{\pgfqpoint{4.037258in}{1.293538in}}%
\pgfpathlineto{\pgfqpoint{4.038528in}{1.293426in}}%
\pgfpathlineto{\pgfqpoint{4.039798in}{1.305250in}}%
\pgfpathlineto{\pgfqpoint{4.041702in}{1.288076in}}%
\pgfpathlineto{\pgfqpoint{4.042337in}{1.289275in}}%
\pgfpathlineto{\pgfqpoint{4.042655in}{1.286402in}}%
\pgfpathlineto{\pgfqpoint{4.042972in}{1.290575in}}%
\pgfpathlineto{\pgfqpoint{4.043607in}{1.289613in}}%
\pgfpathlineto{\pgfqpoint{4.044242in}{1.291319in}}%
\pgfpathlineto{\pgfqpoint{4.044559in}{1.290329in}}%
\pgfpathlineto{\pgfqpoint{4.045512in}{1.290709in}}%
\pgfpathlineto{\pgfqpoint{4.045829in}{1.290031in}}%
\pgfpathlineto{\pgfqpoint{4.047099in}{1.301505in}}%
\pgfpathlineto{\pgfqpoint{4.047417in}{1.296987in}}%
\pgfpathlineto{\pgfqpoint{4.047734in}{1.296783in}}%
\pgfpathlineto{\pgfqpoint{4.048051in}{1.298560in}}%
\pgfpathlineto{\pgfqpoint{4.049956in}{1.282932in}}%
\pgfpathlineto{\pgfqpoint{4.051226in}{1.288552in}}%
\pgfpathlineto{\pgfqpoint{4.051543in}{1.288494in}}%
\pgfpathlineto{\pgfqpoint{4.052813in}{1.287375in}}%
\pgfpathlineto{\pgfqpoint{4.053131in}{1.287285in}}%
\pgfpathlineto{\pgfqpoint{4.054401in}{1.297415in}}%
\pgfpathlineto{\pgfqpoint{4.055353in}{1.296405in}}%
\pgfpathlineto{\pgfqpoint{4.057258in}{1.280861in}}%
\pgfpathlineto{\pgfqpoint{4.058845in}{1.285346in}}%
\pgfpathlineto{\pgfqpoint{4.060115in}{1.284071in}}%
\pgfpathlineto{\pgfqpoint{4.061385in}{1.288507in}}%
\pgfpathlineto{\pgfqpoint{4.062655in}{1.294096in}}%
\pgfpathlineto{\pgfqpoint{4.064559in}{1.279995in}}%
\pgfpathlineto{\pgfqpoint{4.064877in}{1.280432in}}%
\pgfpathlineto{\pgfqpoint{4.065194in}{1.280140in}}%
\pgfpathlineto{\pgfqpoint{4.065512in}{1.277401in}}%
\pgfpathlineto{\pgfqpoint{4.065829in}{1.281161in}}%
\pgfpathlineto{\pgfqpoint{4.066464in}{1.280163in}}%
\pgfpathlineto{\pgfqpoint{4.067099in}{1.282406in}}%
\pgfpathlineto{\pgfqpoint{4.067416in}{1.280910in}}%
\pgfpathlineto{\pgfqpoint{4.068369in}{1.281340in}}%
\pgfpathlineto{\pgfqpoint{4.069956in}{1.291202in}}%
\pgfpathlineto{\pgfqpoint{4.070273in}{1.287890in}}%
\pgfpathlineto{\pgfqpoint{4.070908in}{1.286533in}}%
\pgfpathlineto{\pgfqpoint{4.072813in}{1.274566in}}%
\pgfpathlineto{\pgfqpoint{4.074400in}{1.279457in}}%
\pgfpathlineto{\pgfqpoint{4.075670in}{1.277878in}}%
\pgfpathlineto{\pgfqpoint{4.076623in}{1.280468in}}%
\pgfpathlineto{\pgfqpoint{4.076940in}{1.287441in}}%
\pgfpathlineto{\pgfqpoint{4.077892in}{1.286301in}}%
\pgfpathlineto{\pgfqpoint{4.081067in}{1.272336in}}%
\pgfpathlineto{\pgfqpoint{4.082337in}{1.276468in}}%
\pgfpathlineto{\pgfqpoint{4.082654in}{1.276238in}}%
\pgfpathlineto{\pgfqpoint{4.082972in}{1.274632in}}%
\pgfpathlineto{\pgfqpoint{4.083924in}{1.275233in}}%
\pgfpathlineto{\pgfqpoint{4.085194in}{1.285148in}}%
\pgfpathlineto{\pgfqpoint{4.087416in}{1.270571in}}%
\pgfpathlineto{\pgfqpoint{4.088051in}{1.270711in}}%
\pgfpathlineto{\pgfqpoint{4.088369in}{1.269188in}}%
\pgfpathlineto{\pgfqpoint{4.088686in}{1.272039in}}%
\pgfpathlineto{\pgfqpoint{4.089003in}{1.272318in}}%
\pgfpathlineto{\pgfqpoint{4.089321in}{1.271015in}}%
\pgfpathlineto{\pgfqpoint{4.089638in}{1.273439in}}%
\pgfpathlineto{\pgfqpoint{4.089956in}{1.273386in}}%
\pgfpathlineto{\pgfqpoint{4.091226in}{1.271779in}}%
\pgfpathlineto{\pgfqpoint{4.092496in}{1.283685in}}%
\pgfpathlineto{\pgfqpoint{4.092813in}{1.280637in}}%
\pgfpathlineto{\pgfqpoint{4.095670in}{1.266479in}}%
\pgfpathlineto{\pgfqpoint{4.097257in}{1.270220in}}%
\pgfpathlineto{\pgfqpoint{4.098527in}{1.268372in}}%
\pgfpathlineto{\pgfqpoint{4.099480in}{1.270719in}}%
\pgfpathlineto{\pgfqpoint{4.099797in}{1.280506in}}%
\pgfpathlineto{\pgfqpoint{4.100749in}{1.276936in}}%
\pgfpathlineto{\pgfqpoint{4.102654in}{1.264029in}}%
\pgfpathlineto{\pgfqpoint{4.102972in}{1.264050in}}%
\pgfpathlineto{\pgfqpoint{4.104559in}{1.266721in}}%
\pgfpathlineto{\pgfqpoint{4.105194in}{1.267305in}}%
\pgfpathlineto{\pgfqpoint{4.105829in}{1.265374in}}%
\pgfpathlineto{\pgfqpoint{4.108051in}{1.277475in}}%
\pgfpathlineto{\pgfqpoint{4.108368in}{1.272739in}}%
\pgfpathlineto{\pgfqpoint{4.110908in}{1.260709in}}%
\pgfpathlineto{\pgfqpoint{4.113765in}{1.263808in}}%
\pgfpathlineto{\pgfqpoint{4.114083in}{1.262180in}}%
\pgfpathlineto{\pgfqpoint{4.114400in}{1.264435in}}%
\pgfpathlineto{\pgfqpoint{4.115352in}{1.275866in}}%
\pgfpathlineto{\pgfqpoint{4.115987in}{1.269244in}}%
\pgfpathlineto{\pgfqpoint{4.116305in}{1.268449in}}%
\pgfpathlineto{\pgfqpoint{4.118210in}{1.257085in}}%
\pgfpathlineto{\pgfqpoint{4.119797in}{1.260883in}}%
\pgfpathlineto{\pgfqpoint{4.120114in}{1.260743in}}%
\pgfpathlineto{\pgfqpoint{4.120749in}{1.260998in}}%
\pgfpathlineto{\pgfqpoint{4.121384in}{1.259092in}}%
\pgfpathlineto{\pgfqpoint{4.123606in}{1.270131in}}%
\pgfpathlineto{\pgfqpoint{4.125511in}{1.253772in}}%
\pgfpathlineto{\pgfqpoint{4.127098in}{1.257617in}}%
\pgfpathlineto{\pgfqpoint{4.127733in}{1.256964in}}%
\pgfpathlineto{\pgfqpoint{4.128051in}{1.257997in}}%
\pgfpathlineto{\pgfqpoint{4.128368in}{1.257859in}}%
\pgfpathlineto{\pgfqpoint{4.129638in}{1.256015in}}%
\pgfpathlineto{\pgfqpoint{4.130908in}{1.270410in}}%
\pgfpathlineto{\pgfqpoint{4.131225in}{1.264681in}}%
\pgfpathlineto{\pgfqpoint{4.133765in}{1.250993in}}%
\pgfpathlineto{\pgfqpoint{4.135352in}{1.254902in}}%
\pgfpathlineto{\pgfqpoint{4.136940in}{1.252868in}}%
\pgfpathlineto{\pgfqpoint{4.137257in}{1.254567in}}%
\pgfpathlineto{\pgfqpoint{4.137575in}{1.255207in}}%
\pgfpathlineto{\pgfqpoint{4.137892in}{1.254446in}}%
\pgfpathlineto{\pgfqpoint{4.138209in}{1.266304in}}%
\pgfpathlineto{\pgfqpoint{4.139162in}{1.261985in}}%
\pgfpathlineto{\pgfqpoint{4.141067in}{1.247242in}}%
\pgfpathlineto{\pgfqpoint{4.142654in}{1.251797in}}%
\pgfpathlineto{\pgfqpoint{4.144241in}{1.250035in}}%
\pgfpathlineto{\pgfqpoint{4.146463in}{1.263661in}}%
\pgfpathlineto{\pgfqpoint{4.145193in}{1.249822in}}%
\pgfpathlineto{\pgfqpoint{4.146781in}{1.257054in}}%
\pgfpathlineto{\pgfqpoint{4.148368in}{1.244119in}}%
\pgfpathlineto{\pgfqpoint{4.148686in}{1.246400in}}%
\pgfpathlineto{\pgfqpoint{4.149320in}{1.245302in}}%
\pgfpathlineto{\pgfqpoint{4.149955in}{1.248505in}}%
\pgfpathlineto{\pgfqpoint{4.150273in}{1.247348in}}%
\pgfpathlineto{\pgfqpoint{4.150908in}{1.248656in}}%
\pgfpathlineto{\pgfqpoint{4.152495in}{1.246712in}}%
\pgfpathlineto{\pgfqpoint{4.153765in}{1.261032in}}%
\pgfpathlineto{\pgfqpoint{4.154082in}{1.256345in}}%
\pgfpathlineto{\pgfqpoint{4.156622in}{1.241141in}}%
\pgfpathlineto{\pgfqpoint{4.157892in}{1.245478in}}%
\pgfpathlineto{\pgfqpoint{4.159479in}{1.245003in}}%
\pgfpathlineto{\pgfqpoint{4.160114in}{1.245184in}}%
\pgfpathlineto{\pgfqpoint{4.160749in}{1.243874in}}%
\pgfpathlineto{\pgfqpoint{4.162019in}{1.256398in}}%
\pgfpathlineto{\pgfqpoint{4.163923in}{1.237545in}}%
\pgfpathlineto{\pgfqpoint{4.165511in}{1.242527in}}%
\pgfpathlineto{\pgfqpoint{4.165828in}{1.241309in}}%
\pgfpathlineto{\pgfqpoint{4.166781in}{1.241710in}}%
\pgfpathlineto{\pgfqpoint{4.168050in}{1.240556in}}%
\pgfpathlineto{\pgfqpoint{4.169320in}{1.254816in}}%
\pgfpathlineto{\pgfqpoint{4.169638in}{1.249193in}}%
\pgfpathlineto{\pgfqpoint{4.172177in}{1.235196in}}%
\pgfpathlineto{\pgfqpoint{4.173447in}{1.239397in}}%
\pgfpathlineto{\pgfqpoint{4.175035in}{1.238540in}}%
\pgfpathlineto{\pgfqpoint{4.175352in}{1.237668in}}%
\pgfpathlineto{\pgfqpoint{4.175669in}{1.239227in}}%
\pgfpathlineto{\pgfqpoint{4.176304in}{1.239631in}}%
\pgfpathlineto{\pgfqpoint{4.176622in}{1.249998in}}%
\pgfpathlineto{\pgfqpoint{4.177574in}{1.248248in}}%
\pgfpathlineto{\pgfqpoint{4.179479in}{1.231244in}}%
\pgfpathlineto{\pgfqpoint{4.181066in}{1.236298in}}%
\pgfpathlineto{\pgfqpoint{4.182336in}{1.235167in}}%
\pgfpathlineto{\pgfqpoint{4.183606in}{1.234522in}}%
\pgfpathlineto{\pgfqpoint{4.184876in}{1.247806in}}%
\pgfpathlineto{\pgfqpoint{4.186780in}{1.228681in}}%
\pgfpathlineto{\pgfqpoint{4.188368in}{1.233014in}}%
\pgfpathlineto{\pgfqpoint{4.188685in}{1.231434in}}%
\pgfpathlineto{\pgfqpoint{4.189638in}{1.231775in}}%
\pgfpathlineto{\pgfqpoint{4.190272in}{1.232512in}}%
\pgfpathlineto{\pgfqpoint{4.190590in}{1.232116in}}%
\pgfpathlineto{\pgfqpoint{4.190907in}{1.231488in}}%
\pgfpathlineto{\pgfqpoint{4.192177in}{1.243975in}}%
\pgfpathlineto{\pgfqpoint{4.192495in}{1.239825in}}%
\pgfpathlineto{\pgfqpoint{4.193130in}{1.240245in}}%
\pgfpathlineto{\pgfqpoint{4.195034in}{1.224928in}}%
\pgfpathlineto{\pgfqpoint{4.196622in}{1.229960in}}%
\pgfpathlineto{\pgfqpoint{4.197891in}{1.228626in}}%
\pgfpathlineto{\pgfqpoint{4.199161in}{1.229794in}}%
\pgfpathlineto{\pgfqpoint{4.200431in}{1.240227in}}%
\pgfpathlineto{\pgfqpoint{4.202336in}{1.222105in}}%
\pgfpathlineto{\pgfqpoint{4.203923in}{1.226725in}}%
\pgfpathlineto{\pgfqpoint{4.205193in}{1.225208in}}%
\pgfpathlineto{\pgfqpoint{4.207415in}{1.231055in}}%
\pgfpathlineto{\pgfqpoint{4.207733in}{1.237545in}}%
\pgfpathlineto{\pgfqpoint{4.208368in}{1.230874in}}%
\pgfpathlineto{\pgfqpoint{4.208685in}{1.231704in}}%
\pgfpathlineto{\pgfqpoint{4.210590in}{1.218768in}}%
\pgfpathlineto{\pgfqpoint{4.211860in}{1.223643in}}%
\pgfpathlineto{\pgfqpoint{4.212177in}{1.223539in}}%
\pgfpathlineto{\pgfqpoint{4.212495in}{1.221950in}}%
\pgfpathlineto{\pgfqpoint{4.213447in}{1.222146in}}%
\pgfpathlineto{\pgfqpoint{4.214082in}{1.224307in}}%
\pgfpathlineto{\pgfqpoint{4.214399in}{1.223289in}}%
\pgfpathlineto{\pgfqpoint{4.215987in}{1.232338in}}%
\pgfpathlineto{\pgfqpoint{4.217891in}{1.215565in}}%
\pgfpathlineto{\pgfqpoint{4.219479in}{1.220364in}}%
\pgfpathlineto{\pgfqpoint{4.220748in}{1.218641in}}%
\pgfpathlineto{\pgfqpoint{4.222971in}{1.225857in}}%
\pgfpathlineto{\pgfqpoint{4.223288in}{1.230360in}}%
\pgfpathlineto{\pgfqpoint{4.223923in}{1.224171in}}%
\pgfpathlineto{\pgfqpoint{4.224240in}{1.224107in}}%
\pgfpathlineto{\pgfqpoint{4.226145in}{1.212432in}}%
\pgfpathlineto{\pgfqpoint{4.227415in}{1.217217in}}%
\pgfpathlineto{\pgfqpoint{4.227732in}{1.217115in}}%
\pgfpathlineto{\pgfqpoint{4.229002in}{1.215603in}}%
\pgfpathlineto{\pgfqpoint{4.229637in}{1.219604in}}%
\pgfpathlineto{\pgfqpoint{4.229955in}{1.218048in}}%
\pgfpathlineto{\pgfqpoint{4.230590in}{1.225986in}}%
\pgfpathlineto{\pgfqpoint{4.231542in}{1.224461in}}%
\pgfpathlineto{\pgfqpoint{4.233447in}{1.209246in}}%
\pgfpathlineto{\pgfqpoint{4.235034in}{1.213920in}}%
\pgfpathlineto{\pgfqpoint{4.236304in}{1.212088in}}%
\pgfpathlineto{\pgfqpoint{4.238843in}{1.222732in}}%
\pgfpathlineto{\pgfqpoint{4.240748in}{1.207680in}}%
\pgfpathlineto{\pgfqpoint{4.241066in}{1.208224in}}%
\pgfpathlineto{\pgfqpoint{4.241383in}{1.208360in}}%
\pgfpathlineto{\pgfqpoint{4.241701in}{1.206301in}}%
\pgfpathlineto{\pgfqpoint{4.242018in}{1.209467in}}%
\pgfpathlineto{\pgfqpoint{4.242653in}{1.208700in}}%
\pgfpathlineto{\pgfqpoint{4.243288in}{1.210627in}}%
\pgfpathlineto{\pgfqpoint{4.243605in}{1.208893in}}%
\pgfpathlineto{\pgfqpoint{4.244558in}{1.209073in}}%
\pgfpathlineto{\pgfqpoint{4.246145in}{1.218952in}}%
\pgfpathlineto{\pgfqpoint{4.246780in}{1.216491in}}%
\pgfpathlineto{\pgfqpoint{4.247097in}{1.216744in}}%
\pgfpathlineto{\pgfqpoint{4.249002in}{1.202892in}}%
\pgfpathlineto{\pgfqpoint{4.250589in}{1.207422in}}%
\pgfpathlineto{\pgfqpoint{4.251859in}{1.205518in}}%
\pgfpathlineto{\pgfqpoint{4.253447in}{1.215467in}}%
\pgfpathlineto{\pgfqpoint{4.254399in}{1.215088in}}%
\pgfpathlineto{\pgfqpoint{4.256304in}{1.201335in}}%
\pgfpathlineto{\pgfqpoint{4.256621in}{1.201850in}}%
\pgfpathlineto{\pgfqpoint{4.256939in}{1.201659in}}%
\pgfpathlineto{\pgfqpoint{4.257256in}{1.200001in}}%
\pgfpathlineto{\pgfqpoint{4.257574in}{1.203006in}}%
\pgfpathlineto{\pgfqpoint{4.258208in}{1.202240in}}%
\pgfpathlineto{\pgfqpoint{4.258843in}{1.204150in}}%
\pgfpathlineto{\pgfqpoint{4.260113in}{1.202461in}}%
\pgfpathlineto{\pgfqpoint{4.261700in}{1.212942in}}%
\pgfpathlineto{\pgfqpoint{4.262018in}{1.208917in}}%
\pgfpathlineto{\pgfqpoint{4.262335in}{1.209690in}}%
\pgfpathlineto{\pgfqpoint{4.262653in}{1.208915in}}%
\pgfpathlineto{\pgfqpoint{4.264558in}{1.196664in}}%
\pgfpathlineto{\pgfqpoint{4.266145in}{1.200884in}}%
\pgfpathlineto{\pgfqpoint{4.267415in}{1.198948in}}%
\pgfpathlineto{\pgfqpoint{4.269002in}{1.209627in}}%
\pgfpathlineto{\pgfqpoint{4.269954in}{1.207474in}}%
\pgfpathlineto{\pgfqpoint{4.271859in}{1.194809in}}%
\pgfpathlineto{\pgfqpoint{4.272177in}{1.195432in}}%
\pgfpathlineto{\pgfqpoint{4.272811in}{1.193767in}}%
\pgfpathlineto{\pgfqpoint{4.274081in}{1.197685in}}%
\pgfpathlineto{\pgfqpoint{4.274399in}{1.197613in}}%
\pgfpathlineto{\pgfqpoint{4.275669in}{1.195853in}}%
\pgfpathlineto{\pgfqpoint{4.277256in}{1.206483in}}%
\pgfpathlineto{\pgfqpoint{4.279161in}{1.194503in}}%
\pgfpathlineto{\pgfqpoint{4.280113in}{1.190321in}}%
\pgfpathlineto{\pgfqpoint{4.281065in}{1.191649in}}%
\pgfpathlineto{\pgfqpoint{4.282335in}{1.194254in}}%
\pgfpathlineto{\pgfqpoint{4.282653in}{1.194105in}}%
\pgfpathlineto{\pgfqpoint{4.282970in}{1.192374in}}%
\pgfpathlineto{\pgfqpoint{4.283922in}{1.193201in}}%
\pgfpathlineto{\pgfqpoint{4.284557in}{1.203165in}}%
\pgfpathlineto{\pgfqpoint{4.285192in}{1.202323in}}%
\pgfpathlineto{\pgfqpoint{4.288367in}{1.187337in}}%
\pgfpathlineto{\pgfqpoint{4.288684in}{1.189913in}}%
\pgfpathlineto{\pgfqpoint{4.289954in}{1.191067in}}%
\pgfpathlineto{\pgfqpoint{4.291224in}{1.189208in}}%
\pgfpathlineto{\pgfqpoint{4.292494in}{1.200127in}}%
\pgfpathlineto{\pgfqpoint{4.292811in}{1.200040in}}%
\pgfpathlineto{\pgfqpoint{4.294716in}{1.187789in}}%
\pgfpathlineto{\pgfqpoint{4.295668in}{1.184055in}}%
\pgfpathlineto{\pgfqpoint{4.296621in}{1.185177in}}%
\pgfpathlineto{\pgfqpoint{4.297891in}{1.187705in}}%
\pgfpathlineto{\pgfqpoint{4.298208in}{1.187574in}}%
\pgfpathlineto{\pgfqpoint{4.298526in}{1.185803in}}%
\pgfpathlineto{\pgfqpoint{4.299478in}{1.186377in}}%
\pgfpathlineto{\pgfqpoint{4.300113in}{1.196585in}}%
\pgfpathlineto{\pgfqpoint{4.300748in}{1.196181in}}%
\pgfpathlineto{\pgfqpoint{4.303922in}{1.180972in}}%
\pgfpathlineto{\pgfqpoint{4.305192in}{1.184524in}}%
\pgfpathlineto{\pgfqpoint{4.305510in}{1.184470in}}%
\pgfpathlineto{\pgfqpoint{4.306779in}{1.182551in}}%
\pgfpathlineto{\pgfqpoint{4.308367in}{1.193668in}}%
\pgfpathlineto{\pgfqpoint{4.308684in}{1.189108in}}%
\pgfpathlineto{\pgfqpoint{4.309002in}{1.190008in}}%
\pgfpathlineto{\pgfqpoint{4.311224in}{1.177661in}}%
\pgfpathlineto{\pgfqpoint{4.312811in}{1.180902in}}%
\pgfpathlineto{\pgfqpoint{4.313446in}{1.181117in}}%
\pgfpathlineto{\pgfqpoint{4.314081in}{1.179249in}}%
\pgfpathlineto{\pgfqpoint{4.316303in}{1.190059in}}%
\pgfpathlineto{\pgfqpoint{4.316621in}{1.187937in}}%
\pgfpathlineto{\pgfqpoint{4.318525in}{1.175819in}}%
\pgfpathlineto{\pgfqpoint{4.318843in}{1.176206in}}%
\pgfpathlineto{\pgfqpoint{4.319478in}{1.174450in}}%
\pgfpathlineto{\pgfqpoint{4.319795in}{1.176590in}}%
\pgfpathlineto{\pgfqpoint{4.320430in}{1.176250in}}%
\pgfpathlineto{\pgfqpoint{4.321065in}{1.177841in}}%
\pgfpathlineto{\pgfqpoint{4.321700in}{1.177607in}}%
\pgfpathlineto{\pgfqpoint{4.322017in}{1.177567in}}%
\pgfpathlineto{\pgfqpoint{4.322335in}{1.175896in}}%
\pgfpathlineto{\pgfqpoint{4.322652in}{1.177788in}}%
\pgfpathlineto{\pgfqpoint{4.323922in}{1.187203in}}%
\pgfpathlineto{\pgfqpoint{4.325827in}{1.174771in}}%
\pgfpathlineto{\pgfqpoint{4.326144in}{1.174384in}}%
\pgfpathlineto{\pgfqpoint{4.326779in}{1.171365in}}%
\pgfpathlineto{\pgfqpoint{4.327414in}{1.171888in}}%
\pgfpathlineto{\pgfqpoint{4.327732in}{1.171930in}}%
\pgfpathlineto{\pgfqpoint{4.329001in}{1.174541in}}%
\pgfpathlineto{\pgfqpoint{4.329319in}{1.174422in}}%
\pgfpathlineto{\pgfqpoint{4.330589in}{1.172716in}}%
\pgfpathlineto{\pgfqpoint{4.331859in}{1.184081in}}%
\pgfpathlineto{\pgfqpoint{4.332176in}{1.181994in}}%
\pgfpathlineto{\pgfqpoint{4.334081in}{1.169517in}}%
\pgfpathlineto{\pgfqpoint{4.334398in}{1.169825in}}%
\pgfpathlineto{\pgfqpoint{4.335033in}{1.167914in}}%
\pgfpathlineto{\pgfqpoint{4.335351in}{1.169819in}}%
\pgfpathlineto{\pgfqpoint{4.335986in}{1.169532in}}%
\pgfpathlineto{\pgfqpoint{4.337255in}{1.171020in}}%
\pgfpathlineto{\pgfqpoint{4.337573in}{1.170967in}}%
\pgfpathlineto{\pgfqpoint{4.337890in}{1.169238in}}%
\pgfpathlineto{\pgfqpoint{4.338208in}{1.170921in}}%
\pgfpathlineto{\pgfqpoint{4.339478in}{1.180696in}}%
\pgfpathlineto{\pgfqpoint{4.340112in}{1.178288in}}%
\pgfpathlineto{\pgfqpoint{4.342335in}{1.164974in}}%
\pgfpathlineto{\pgfqpoint{4.343922in}{1.167211in}}%
\pgfpathlineto{\pgfqpoint{4.344557in}{1.167917in}}%
\pgfpathlineto{\pgfqpoint{4.345192in}{1.166207in}}%
\pgfpathlineto{\pgfqpoint{4.347414in}{1.177771in}}%
\pgfpathlineto{\pgfqpoint{4.347731in}{1.175906in}}%
\pgfpathlineto{\pgfqpoint{4.349636in}{1.163276in}}%
\pgfpathlineto{\pgfqpoint{4.349954in}{1.163461in}}%
\pgfpathlineto{\pgfqpoint{4.350271in}{1.161191in}}%
\pgfpathlineto{\pgfqpoint{4.350906in}{1.163039in}}%
\pgfpathlineto{\pgfqpoint{4.351224in}{1.162536in}}%
\pgfpathlineto{\pgfqpoint{4.351541in}{1.162764in}}%
\pgfpathlineto{\pgfqpoint{4.352811in}{1.164397in}}%
\pgfpathlineto{\pgfqpoint{4.353128in}{1.164330in}}%
\pgfpathlineto{\pgfqpoint{4.353446in}{1.162610in}}%
\pgfpathlineto{\pgfqpoint{4.354081in}{1.165114in}}%
\pgfpathlineto{\pgfqpoint{4.354398in}{1.163589in}}%
\pgfpathlineto{\pgfqpoint{4.355033in}{1.173857in}}%
\pgfpathlineto{\pgfqpoint{4.355668in}{1.172740in}}%
\pgfpathlineto{\pgfqpoint{4.357890in}{1.158696in}}%
\pgfpathlineto{\pgfqpoint{4.359160in}{1.160369in}}%
\pgfpathlineto{\pgfqpoint{4.358525in}{1.158011in}}%
\pgfpathlineto{\pgfqpoint{4.359477in}{1.160187in}}%
\pgfpathlineto{\pgfqpoint{4.359795in}{1.160078in}}%
\pgfpathlineto{\pgfqpoint{4.360112in}{1.161287in}}%
\pgfpathlineto{\pgfqpoint{4.360747in}{1.159716in}}%
\pgfpathlineto{\pgfqpoint{4.361065in}{1.160940in}}%
\pgfpathlineto{\pgfqpoint{4.361382in}{1.160953in}}%
\pgfpathlineto{\pgfqpoint{4.361700in}{1.159304in}}%
\pgfpathlineto{\pgfqpoint{4.362017in}{1.162543in}}%
\pgfpathlineto{\pgfqpoint{4.362969in}{1.171307in}}%
\pgfpathlineto{\pgfqpoint{4.363287in}{1.170075in}}%
\pgfpathlineto{\pgfqpoint{4.365192in}{1.156896in}}%
\pgfpathlineto{\pgfqpoint{4.365509in}{1.157111in}}%
\pgfpathlineto{\pgfqpoint{4.365827in}{1.154582in}}%
\pgfpathlineto{\pgfqpoint{4.366779in}{1.155458in}}%
\pgfpathlineto{\pgfqpoint{4.368684in}{1.157684in}}%
\pgfpathlineto{\pgfqpoint{4.369954in}{1.156161in}}%
\pgfpathlineto{\pgfqpoint{4.371223in}{1.167075in}}%
\pgfpathlineto{\pgfqpoint{4.373446in}{1.152393in}}%
\pgfpathlineto{\pgfqpoint{4.373763in}{1.153031in}}%
\pgfpathlineto{\pgfqpoint{4.374080in}{1.151141in}}%
\pgfpathlineto{\pgfqpoint{4.374715in}{1.153438in}}%
\pgfpathlineto{\pgfqpoint{4.375350in}{1.153257in}}%
\pgfpathlineto{\pgfqpoint{4.375668in}{1.154571in}}%
\pgfpathlineto{\pgfqpoint{4.376620in}{1.154287in}}%
\pgfpathlineto{\pgfqpoint{4.376938in}{1.154279in}}%
\pgfpathlineto{\pgfqpoint{4.377255in}{1.152569in}}%
\pgfpathlineto{\pgfqpoint{4.377572in}{1.154420in}}%
\pgfpathlineto{\pgfqpoint{4.378842in}{1.164009in}}%
\pgfpathlineto{\pgfqpoint{4.379160in}{1.159107in}}%
\pgfpathlineto{\pgfqpoint{4.379477in}{1.161033in}}%
\pgfpathlineto{\pgfqpoint{4.379795in}{1.157510in}}%
\pgfpathlineto{\pgfqpoint{4.381382in}{1.148058in}}%
\pgfpathlineto{\pgfqpoint{4.381699in}{1.148317in}}%
\pgfpathlineto{\pgfqpoint{4.382969in}{1.150720in}}%
\pgfpathlineto{\pgfqpoint{4.383287in}{1.150547in}}%
\pgfpathlineto{\pgfqpoint{4.384557in}{1.149486in}}%
\pgfpathlineto{\pgfqpoint{4.386779in}{1.161065in}}%
\pgfpathlineto{\pgfqpoint{4.387096in}{1.158812in}}%
\pgfpathlineto{\pgfqpoint{4.389001in}{1.146205in}}%
\pgfpathlineto{\pgfqpoint{4.389318in}{1.146610in}}%
\pgfpathlineto{\pgfqpoint{4.389636in}{1.144336in}}%
\pgfpathlineto{\pgfqpoint{4.390271in}{1.146449in}}%
\pgfpathlineto{\pgfqpoint{4.390588in}{1.145857in}}%
\pgfpathlineto{\pgfqpoint{4.390906in}{1.146283in}}%
\pgfpathlineto{\pgfqpoint{4.392176in}{1.147624in}}%
\pgfpathlineto{\pgfqpoint{4.392493in}{1.147568in}}%
\pgfpathlineto{\pgfqpoint{4.392810in}{1.145889in}}%
\pgfpathlineto{\pgfqpoint{4.393128in}{1.147585in}}%
\pgfpathlineto{\pgfqpoint{4.394398in}{1.157276in}}%
\pgfpathlineto{\pgfqpoint{4.393763in}{1.147121in}}%
\pgfpathlineto{\pgfqpoint{4.395350in}{1.152483in}}%
\pgfpathlineto{\pgfqpoint{4.396937in}{1.141785in}}%
\pgfpathlineto{\pgfqpoint{4.397255in}{1.141955in}}%
\pgfpathlineto{\pgfqpoint{4.398525in}{1.143690in}}%
\pgfpathlineto{\pgfqpoint{4.397890in}{1.141240in}}%
\pgfpathlineto{\pgfqpoint{4.398842in}{1.143417in}}%
\pgfpathlineto{\pgfqpoint{4.400747in}{1.144197in}}%
\pgfpathlineto{\pgfqpoint{4.401064in}{1.142532in}}%
\pgfpathlineto{\pgfqpoint{4.401382in}{1.145661in}}%
\pgfpathlineto{\pgfqpoint{4.402334in}{1.154452in}}%
\pgfpathlineto{\pgfqpoint{4.402652in}{1.153465in}}%
\pgfpathlineto{\pgfqpoint{4.404556in}{1.140023in}}%
\pgfpathlineto{\pgfqpoint{4.404874in}{1.140302in}}%
\pgfpathlineto{\pgfqpoint{4.405191in}{1.137749in}}%
\pgfpathlineto{\pgfqpoint{4.406144in}{1.138608in}}%
\pgfpathlineto{\pgfqpoint{4.408048in}{1.140874in}}%
\pgfpathlineto{\pgfqpoint{4.409318in}{1.139280in}}%
\pgfpathlineto{\pgfqpoint{4.410588in}{1.150436in}}%
\pgfpathlineto{\pgfqpoint{4.413445in}{1.134295in}}%
\pgfpathlineto{\pgfqpoint{4.416302in}{1.137456in}}%
\pgfpathlineto{\pgfqpoint{4.416620in}{1.135756in}}%
\pgfpathlineto{\pgfqpoint{4.416937in}{1.137438in}}%
\pgfpathlineto{\pgfqpoint{4.418207in}{1.147366in}}%
\pgfpathlineto{\pgfqpoint{4.418842in}{1.144831in}}%
\pgfpathlineto{\pgfqpoint{4.420747in}{1.131389in}}%
\pgfpathlineto{\pgfqpoint{4.421064in}{1.131585in}}%
\pgfpathlineto{\pgfqpoint{4.422334in}{1.133764in}}%
\pgfpathlineto{\pgfqpoint{4.421699in}{1.131355in}}%
\pgfpathlineto{\pgfqpoint{4.422651in}{1.133563in}}%
\pgfpathlineto{\pgfqpoint{4.423921in}{1.132739in}}%
\pgfpathlineto{\pgfqpoint{4.426144in}{1.144162in}}%
\pgfpathlineto{\pgfqpoint{4.424874in}{1.132439in}}%
\pgfpathlineto{\pgfqpoint{4.426461in}{1.142848in}}%
\pgfpathlineto{\pgfqpoint{4.428366in}{1.129737in}}%
\pgfpathlineto{\pgfqpoint{4.428683in}{1.129927in}}%
\pgfpathlineto{\pgfqpoint{4.429001in}{1.127559in}}%
\pgfpathlineto{\pgfqpoint{4.429953in}{1.128739in}}%
\pgfpathlineto{\pgfqpoint{4.431858in}{1.130712in}}%
\pgfpathlineto{\pgfqpoint{4.432175in}{1.129068in}}%
\pgfpathlineto{\pgfqpoint{4.432810in}{1.131151in}}%
\pgfpathlineto{\pgfqpoint{4.433128in}{1.129487in}}%
\pgfpathlineto{\pgfqpoint{4.433763in}{1.140144in}}%
\pgfpathlineto{\pgfqpoint{4.434397in}{1.139666in}}%
\pgfpathlineto{\pgfqpoint{4.437255in}{1.124225in}}%
\pgfpathlineto{\pgfqpoint{4.440112in}{1.127300in}}%
\pgfpathlineto{\pgfqpoint{4.440429in}{1.125606in}}%
\pgfpathlineto{\pgfqpoint{4.440747in}{1.127471in}}%
\pgfpathlineto{\pgfqpoint{4.442016in}{1.137235in}}%
\pgfpathlineto{\pgfqpoint{4.443604in}{1.125923in}}%
\pgfpathlineto{\pgfqpoint{4.445508in}{1.121363in}}%
\pgfpathlineto{\pgfqpoint{4.447413in}{1.124001in}}%
\pgfpathlineto{\pgfqpoint{4.448683in}{1.122285in}}%
\pgfpathlineto{\pgfqpoint{4.449318in}{1.131585in}}%
\pgfpathlineto{\pgfqpoint{4.449635in}{1.127540in}}%
\pgfpathlineto{\pgfqpoint{4.449953in}{1.122907in}}%
\pgfpathlineto{\pgfqpoint{4.449953in}{1.122907in}}%
\pgfusepath{stroke}%
\end{pgfscope}%
\begin{pgfscope}%
\pgfsetrectcap%
\pgfsetmiterjoin%
\pgfsetlinewidth{0.803000pt}%
\definecolor{currentstroke}{rgb}{0.000000,0.000000,0.000000}%
\pgfsetstrokecolor{currentstroke}%
\pgfsetdash{}{0pt}%
\pgfpathmoveto{\pgfqpoint{0.450320in}{0.472202in}}%
\pgfpathlineto{\pgfqpoint{0.450320in}{3.352990in}}%
\pgfusepath{stroke}%
\end{pgfscope}%
\begin{pgfscope}%
\pgfsetrectcap%
\pgfsetmiterjoin%
\pgfsetlinewidth{0.803000pt}%
\definecolor{currentstroke}{rgb}{0.000000,0.000000,0.000000}%
\pgfsetstrokecolor{currentstroke}%
\pgfsetdash{}{0pt}%
\pgfpathmoveto{\pgfqpoint{4.640412in}{0.472202in}}%
\pgfpathlineto{\pgfqpoint{4.640412in}{3.352990in}}%
\pgfusepath{stroke}%
\end{pgfscope}%
\begin{pgfscope}%
\pgfsetrectcap%
\pgfsetmiterjoin%
\pgfsetlinewidth{0.803000pt}%
\definecolor{currentstroke}{rgb}{0.000000,0.000000,0.000000}%
\pgfsetstrokecolor{currentstroke}%
\pgfsetdash{}{0pt}%
\pgfpathmoveto{\pgfqpoint{0.450320in}{0.472202in}}%
\pgfpathlineto{\pgfqpoint{4.640412in}{0.472202in}}%
\pgfusepath{stroke}%
\end{pgfscope}%
\begin{pgfscope}%
\pgfsetrectcap%
\pgfsetmiterjoin%
\pgfsetlinewidth{0.803000pt}%
\definecolor{currentstroke}{rgb}{0.000000,0.000000,0.000000}%
\pgfsetstrokecolor{currentstroke}%
\pgfsetdash{}{0pt}%
\pgfpathmoveto{\pgfqpoint{0.450320in}{3.352990in}}%
\pgfpathlineto{\pgfqpoint{4.640412in}{3.352990in}}%
\pgfusepath{stroke}%
\end{pgfscope}%
\begin{pgfscope}%
\pgfsetbuttcap%
\pgfsetmiterjoin%
\definecolor{currentfill}{rgb}{1.000000,1.000000,1.000000}%
\pgfsetfillcolor{currentfill}%
\pgfsetfillopacity{0.800000}%
\pgfsetlinewidth{1.003750pt}%
\definecolor{currentstroke}{rgb}{0.800000,0.800000,0.800000}%
\pgfsetstrokecolor{currentstroke}%
\pgfsetstrokeopacity{0.800000}%
\pgfsetdash{}{0pt}%
\pgfpathmoveto{\pgfqpoint{2.811006in}{2.754843in}}%
\pgfpathlineto{\pgfqpoint{4.523745in}{2.754843in}}%
\pgfpathquadraticcurveto{\pgfqpoint{4.557078in}{2.754843in}}{\pgfqpoint{4.557078in}{2.788176in}}%
\pgfpathlineto{\pgfqpoint{4.557078in}{3.236324in}}%
\pgfpathquadraticcurveto{\pgfqpoint{4.557078in}{3.269657in}}{\pgfqpoint{4.523745in}{3.269657in}}%
\pgfpathlineto{\pgfqpoint{2.811006in}{3.269657in}}%
\pgfpathquadraticcurveto{\pgfqpoint{2.777673in}{3.269657in}}{\pgfqpoint{2.777673in}{3.236324in}}%
\pgfpathlineto{\pgfqpoint{2.777673in}{2.788176in}}%
\pgfpathquadraticcurveto{\pgfqpoint{2.777673in}{2.754843in}}{\pgfqpoint{2.811006in}{2.754843in}}%
\pgfpathlineto{\pgfqpoint{2.811006in}{2.754843in}}%
\pgfpathclose%
\pgfusepath{stroke,fill}%
\end{pgfscope}%
\begin{pgfscope}%
\pgfsetrectcap%
\pgfsetroundjoin%
\pgfsetlinewidth{1.505625pt}%
\definecolor{currentstroke}{rgb}{0.121569,0.466667,0.705882}%
\pgfsetstrokecolor{currentstroke}%
\pgfsetdash{}{0pt}%
\pgfpathmoveto{\pgfqpoint{2.844340in}{3.144657in}}%
\pgfpathlineto{\pgfqpoint{3.011006in}{3.144657in}}%
\pgfpathlineto{\pgfqpoint{3.177673in}{3.144657in}}%
\pgfusepath{stroke}%
\end{pgfscope}%
\begin{pgfscope}%
\definecolor{textcolor}{rgb}{0.000000,0.000000,0.000000}%
\pgfsetstrokecolor{textcolor}%
\pgfsetfillcolor{textcolor}%
\pgftext[x=3.311006in,y=3.086324in,left,base]{\color{textcolor}{\rmfamily\fontsize{12.000000}{14.400000}\selectfont\catcode`\^=\active\def^{\ifmmode\sp\else\^{}\fi}\catcode`\%=\active\def%{\%}Osculating SMA}}%
\end{pgfscope}%
\begin{pgfscope}%
\pgfsetrectcap%
\pgfsetroundjoin%
\pgfsetlinewidth{1.505625pt}%
\definecolor{currentstroke}{rgb}{1.000000,0.498039,0.054902}%
\pgfsetstrokecolor{currentstroke}%
\pgfsetdash{}{0pt}%
\pgfpathmoveto{\pgfqpoint{2.844340in}{2.912250in}}%
\pgfpathlineto{\pgfqpoint{3.011006in}{2.912250in}}%
\pgfpathlineto{\pgfqpoint{3.177673in}{2.912250in}}%
\pgfusepath{stroke}%
\end{pgfscope}%
\begin{pgfscope}%
\definecolor{textcolor}{rgb}{0.000000,0.000000,0.000000}%
\pgfsetstrokecolor{textcolor}%
\pgfsetfillcolor{textcolor}%
\pgftext[x=3.311006in,y=2.853916in,left,base]{\color{textcolor}{\rmfamily\fontsize{12.000000}{14.400000}\selectfont\catcode`\^=\active\def^{\ifmmode\sp\else\^{}\fi}\catcode`\%=\active\def%{\%}Mean SMA}}%
\end{pgfscope}%
\end{pgfpicture}%
\makeatother%
\endgroup%

    \end{center}
    \caption{Osculating vs. mean semi-major axis for a LEO satellite under atmospheric drag and $J_2$ perturbations}
\end{figure}

Several perturbation techniques have been proposed to perform the conversion between osculating and mean elements. 
Brouwer and Kozai works were ones of the first methods appeared in literature, both consisting of first order solutions which take into account the Earth's asphericity but neglect drag effects \cite{arnas2022analytic}.
This thesis uses a refinement of Brouwer's approach, suggested by Lyddane, which solves zero eccentricity and inclination singularities that are involved in the original theory.
Moreover, the algorithm will account only first order $J_2$ terms \cite{schaub2002analytical}.

\subsection{Sun-Synchronous Orbit} \label{sso_paragraph}
Sun-synchronous orbits are specialized orbits characterized by a constant geometry with the Sun over time \cite{vallado2013fundamentals}. 
They are used for several reasons, from technical needs like that ones deriving from thermal and electric power subsystems of the spacecraft to application requirements such as remote sensing.
Indeed, a constant Sun angle is very precious for missions working with electro-optical sensors \cite{brown1998spacecraft}.
This is the case, for instance, of hyperspectral technology. 
An SSO can be obtained by matching the regression of nodes to the solar motion projected on the equator.
This will allow the satellite's line of nodes to keep a constant angular separation with respect to the Sun.
This separation can be achieved with very good approximation only considering the dominant cause of the secular motion of RAAN: $J_2$ perturbation.  

\section{Repetitive ground tracks}


\section{Orbit Maintenance}


\section{Satellite Constellations}
\subsection{Walker Delta Constellation}
\subsection{Constellation Design}
\subsection{Constellation Maintenance}


\section{Differential Drag Method}


