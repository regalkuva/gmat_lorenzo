\chapter{Methodology}

Exclusively open source tools have been used to carry out the thesis work. 
In particular, the orbital scenarios under examination have been simulated in Python environment.
The scripts produced by this research take advantage of existing free Python libraries.   
Nevertheless, several functions have been written to achieve the purposes of the thesis.
Results are always compared with the General Mission Analysis Tool (GMAT). 
The following paragraphs present a detailed description of the tools mentioned before.

\section{Python for Astrodynamics Application}

Due to the computationally intensive nature of astrodynamics tasks, astrodynamicists have relied on compiled programming languages such as Fortran for the development of astrodynamics software.
Interpreted languages such as Python on the other hand offer higher flexibility and development speed thereby increasing the productivity of the programmer.
While interpreted languages are generally slower than compiled languages recent developments such as JIT (just-in-time) compilers or transpilers have been able to close this speed gap
significantly. Another important factor for the usefulness of a
programming language is its wider ecosystem which consists
of the available open-source packages and development tools
such as integrated development environments or debuggers. \cite{eichhorn2018comparative}

\subsection{poliastro Library}

\section{General Mission Analysis Tool}